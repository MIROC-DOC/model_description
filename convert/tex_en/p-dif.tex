\hypertarget{vertical-diffusion.}{%
\subsection{Vertical Diffusion.}\label{vertical-diffusion.}}

\hypertarget{vertical-diffusion-scheme-overview.}{%
\subsubsection{Vertical Diffusion Scheme
Overview.}\label{vertical-diffusion-scheme-overview.}}

The vertical diffusion scheme, due to sub-grid scale turbulent
diffusion. Evaluating the vertical flux of physical quantities. The main
input data are wind speed, \(u, v\), \(u, v\), and temperature \(T\) The
specific humidity \(q\), the cloud cover \(l\), is The output data are
the vertical fluxes of momentum, heat, water vapor, cloud water and It
is the differential value for obtaining an implicit solution.

To estimate the vertical diffusion coefficient, the Mellor and Yamada
(1974, 1982). The turbulent closure model. Using level 2
parameterization.

The outline of the calculation procedure is as follows.

\begin{enumerate}
\def\labelenumi{\arabic{enumi}.}
\item
  as the stability of the atmosphere. Richardson numbers.
\item
  calculate the diffusion coefficient from Richardson number
  \texttt{MODULE:{[}VDFCOF{]}}.
\item
  calculate the flux and its derivative from the diffusion coefficient.
\end{enumerate}

\hypertarget{basic-formula-for-flux-calculations}{%
\subsubsection{Basic Formula for Flux
Calculations}\label{basic-formula-for-flux-calculations}}

The vertical diffuse flux in the atmosphere is , Using the diffusion
coefficient \(K\), it is evaluated as follows.

\begin{eqnarray}
  F{u} = K_{M} \frac{\partial u}{\partial \sigma} 
\end{eqnarray}

\begin{eqnarray}
  F{v} = K_M \frac{\partial v}{\partial \sigma} 
\end{eqnarray}

\begin{eqnarray}
  F{\theta} = K_H \frac{\partial \theta}{\partial \sigma} 
\end{eqnarray}

\begin{eqnarray}
  F{q} = K_q \frac{\partial q}{\partial \sigma} 
\end{eqnarray}

\hypertarget{richardson-number.}{%
\subsubsection{Richardson Number.}\label{richardson-number.}}

The standard for atmospheric stratospheric stability, Bulk Richardson
number \(R_{iB}\) is

\begin{eqnarray}
R_{iB} = \frac{\displaystyle 
               \frac{g}{\theta_s} \frac{\Delta \theta}{\Delta z} }
              {\displaystyle
                  \left( \frac{\Delta u}{\Delta z} \right)^2 
                + \left( \frac{\Delta v}{\Delta z} \right)^2      }
\end{eqnarray}

. defined by . Here, \((\Delta A)_{k-1/2}\) represents
\(A_{k} - A_{k-1}\). The \((\Delta z)_{k-1/2}\) is based on the
hydrostatic pressure equation,

\begin{eqnarray}
(\Delta z)_{k-1/2} = \frac{R Tv_{k}}{g} 
                     \frac{(\Delta \sigma)_{k-1/2}}{\sigma_{k-1/2}}
\end{eqnarray}

The flux Ricahrdson number \(R_{if}\) is ,

\begin{eqnarray}
R_{if} = \frac{1}{2 \beta_2}
      \left[ \beta_1 + \beta_4 R_{iB}
              - \sqrt{ ( \beta_1 + \beta_4 R_{iB} )^2 
                       - 4 \beta_2 \beta_3 R_{iB} }
              \right] ,
\end{eqnarray}

However,

\begin{eqnarray}
\alpha_1  =  3 A_2 \gamma_1  \\
\alpha_2  =  3 A_2 (\gamma_1+\gamma_2) \\
\beta_1   =  A_1 B_1 ( \gamma_1 - C_1 ) \\
\beta_2   =  A_1 [ B_1 ( \gamma_1 - C_1 ) + 6 A_1 + 3 A_2 ] \\
\beta_3   =  A_2 B_1 \gamma_1 \\
\beta_4   =  A_2 [ B_1 ( \gamma_1 + \gamma_2 ) - 3 A_1 ] ,
\end{eqnarray}

\begin{eqnarray}
(A_1, B_1, A_2, B_2, C_1 ) = ( 0.92, 16.6, 0.74, 10.1, 0.08 ) ,
\end{eqnarray}

\begin{eqnarray}
\gamma_1 = \frac{1}{3} - \frac{2 A_1}{B_1}\, , \, \, \, 
\gamma_2 = \frac{B_2}{B_1} + 6\frac{A_1}{B_1} .
\end{eqnarray}

The relationship between the \(R_{iB}\) and the \(R_{if}\) is
illustrated in this figure, Figure {[}p-dif:rib-rif{]}{]}
(\#p-dif:rib-rif).

\hypertarget{diffusion-coefficient.}{%
\subsubsection{Diffusion Coefficient.}\label{diffusion-coefficient.}}

The diffusion coefficient is , For each layer boundary (\(k-1/2\) level)
, It is given as follows.

\begin{eqnarray}
K_M        =  l^2 \frac{\Delta |\mathbf{v}|}{\Delta z} S_M  \\
K_H = K_q  =  l^2 \frac{\Delta |\mathbf{v}|}{\Delta z} S_H 
\end{eqnarray}

Here, \(S_M, S_H\) are,

\begin{eqnarray}
\widetilde{S_H} = \frac{ \alpha_1-\alpha_2 R_{if} }{ 1-R_{if} }
\end{eqnarray}

\begin{eqnarray}
\widetilde{S_M} = \frac{ \beta_1-\beta_2 R_{if} }{ \beta_3-\beta_4 R_{if} } 
                  \widetilde{S_H} ,
\end{eqnarray}

with ,

\begin{eqnarray}
S_M  =  B_1^{1/2} ( 1- R_{if} )^{1/2} 
          \widetilde{S_M}^{3/2} \\
S_H  =  B_1^{1/2} ( 1- R_{if} )^{1/2} 
          \widetilde{S_M}^{1/2} \widetilde{S_H} .
\end{eqnarray}

\(l\) is a mixing distance, according to Blakadar (1962),

\begin{eqnarray}
l = \frac{kz}{1+kz/l_0}
\end{eqnarray}

Take. \(k\) is a Kárman constant. The current standard value is
\(l_0=200\) m.

If \(S_H, S_M\) are shown as functions of \(R_{if}\),
Figure\protect\hyperlink{p-dif:smsh-rif}{p-dif:smsh-rif{]}} Yes.

\hypertarget{calculating-flux.}{%
\subsubsection{Calculating Flux.}\label{calculating-flux.}}

Using the above, we calculate the fluxes and flux derivatives.

\begin{eqnarray}
  Fu_{k-1/2} = K_{M,k-1/2}(u_{k-1}-u_{k})/(\sigma_{k-1}-\sigma_{k})
\end{eqnarray}

\begin{eqnarray}
  Fv_{k-1/2} = K_{M,k-1/2}(v_{k-1}-v_{k})/(\sigma_{k-1}-\sigma_{k})
\end{eqnarray}

\begin{eqnarray}
  F\theta_{k-1/2} 
  = K_{H,k-1/2}(\theta_{k-1}-\theta_{k})/(\sigma_{k-1}-\sigma_{k})
\end{eqnarray}

\begin{eqnarray}
  Fq_{k-1/2} = K_{q,k-1/2}(q_{k-1}-q_{k})/(\sigma_{k-1}-\sigma_{k})
\end{eqnarray}

\begin{eqnarray}
     \frac{\partial Fu_{k-1/2}}{\partial u_{k-1}} =   \frac{\partial Fv_{k-1/2}}{\partial v_{k-1}} 
  = -\frac{\partial Fu_{k-1/2}}{\partial u_{k}} = - \frac{\partial Fv_{k-1/2}}{\partial v_{k}}  
  = K_{M,k-1/2}/(\sigma_{k-1}-\sigma_{k})
\end{eqnarray}

\begin{eqnarray}
  \frac{\partial F\theta_{k-1/2}}{\partial T_{k-1}}
  = \sigma_{k-1}^{-\kappa} K_{H,k-1/2}/(\sigma_{k-1}-\sigma_{k})
\end{eqnarray}

\begin{eqnarray}
  \frac{\partial F\theta_{k-1/2}}{\partial T_{k}}
 = \sigma_{k}^{-\kappa} K_{H,k-1/2}/(\sigma_{k-1}-\sigma_{k})
\end{eqnarray}

\begin{eqnarray}
  \frac{\partial Fq_{k-1/2}}{\partial u_{k-1}}
 = - \frac{\partial Fq_{k-1/2}}{\partial u_{k}}
 = K_{q,k-1/2}/(\sigma_{k-1}-\sigma_{k})
\end{eqnarray}

\hypertarget{minimum-diffusion-coefficient.}{%
\subsubsection{Minimum Diffusion
Coefficient.}\label{minimum-diffusion-coefficient.}}

In the very stable case, the above estimate gives zero as the diffusion
coefficient. As it is, the model's behavior can be modified in various
ways Set a suitable minimum value as it will have a negative effect. The
current standard values are the same for all fluxes and \(K_{min}=\)
0.15 m\(^{2}\)/s

\hypertarget{other-notes.}{%
\subsubsection{Other Notes.}\label{other-notes.}}

I'm calling the shallow cumulus convection \texttt{MODULE:{[}SHLCOF{]}},
By default, this is a dummy.
