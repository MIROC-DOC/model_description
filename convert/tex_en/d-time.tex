\hypertarget{time-integration.}{%
\subsection{Time integration.}\label{time-integration.}}

The time difference scheme is essentially a leap frog. However, the
diffusion terms and physical process terms are backward or forward
differences. A time filter (Asselin, 1972) is used to suppress the
computational mode. In order to increase the value of \(\Delta t\), we
use a time filter (Asselin, 1972), Applying the semi-implicit method to
the gravitational wave term (Bourke, 1988).

\hypertarget{time-integration-and-time-filtering-by-leap-frog}{%
\subsubsection{Time integration and time filtering by leap
frog}\label{time-integration-and-time-filtering-by-leap-frog}}

We use leap frog as a time integration scheme for advection terms and so
on. The backward difference of \(2 \Delta t\) is used for the horizontal
diffusion term. The pseudo \(p\) surface correction of the diffusion
term and the frictional heat by horizontal diffusion term are combined
with treated as a correction and becomes a forward difference in
\(2 \Delta t\). The physical process terms
(\({\mathcal F}_\lambda, {\mathcal F}_\varphi, Q, S_q\)) are treated as
We still use the forward difference of \(2 \Delta t\). (However, for the
calculation of the time-varying term of vertical diffusion, we treat it
as a backward difference. Please refer to the chapter on physical
processes for details.)

Representing each of the forecast variables as \({X}\), we obtain

\begin{eqnarray}
  \hat{X}^{t+\Delta t} 
    =  \bar{X}^{t-\Delta t}
    + 2 \Delta t 
      \dot{X}_{adv}\left( {X}^{t} \right)
    + 2 \Delta t 
      \dot{X}_{dif}\left( \hat{X}^{t+\Delta t} \right)
\end{eqnarray}

\$ \dot{X}\emph{\{adv\} \$ is an advection term etc, \$ \dot{X}}\{dif\}
\$ is a horizontal diffusion term.

In \$ \hat{X}\^{}\{t+\Delta t\} \$, there is a horizontal diffusion
term, Pseudo, etc. \(p\) Correction of frictional heat (\$
\dot{X}\emph{\{dis\}
\() by surface and horizontal diffusion and physical processes (\)
\dot{X}}\{phy\} \$) have been added, \$ \{X\}\^{}\{t+\Delta t\} \$.

\begin{eqnarray}
  {X}^{t+\Delta t} 
    =  \hat{X}^{t+\Delta t}
    + 2 \Delta t 
      \dot{X}_{dis}\left( \hat{X}^{t+\Delta t} \right)
    + 2 \Delta t 
      \dot{X}_{phy}\left( \hat{X}^{t+\Delta t} \right)
\end{eqnarray}

To remove the calculation mode in leap frog The time filter of Asselin
(1972) is applied at every step. I.e. ,

\begin{eqnarray}
  \bar{X}^{t}
    = ( 1-2 \epsilon_f ) {X}^{t}
    +  \epsilon_f 
        \left( \bar{X}^{t-\Delta t} + {X}^{t+\Delta t} \right)
\end{eqnarray}

and \(\bar{X}\). Normally 0.05 is used as the \(\epsilon_f\).

\hypertarget{semi-implicit-time-integration}{%
\subsubsection{semi-implicit time
integration}\label{semi-implicit-time-integration}}

For mechanics calculations, the leap frog is basically used, We treat
some terms as implicit. Here, the implicit considers the trapezoidal
implicit. For the vector quantity \({\mathbf q}\), The value in \(t\) is
converted to \({\mathbf q}\), The value in \(t+\Delta t\) was converted
to \({\mathbf q}^+\), If you write the value of \(t-\Delta t\) as
\({\mathbf q}^-\), What is trapezoidal implicit?
\(({\mathbf q}^+ + {\mathbf q}^- )/2\). We use the time-varying terms
evaluated by using Now, as a time-varying term in {q}, The term A is
treated in the leap forg method and the term B is treated in the
trapezoidal implicit method. Assume that A is nonlinear with respect to
{q}, while B is linear. In other words,

\begin{eqnarray}
  {\mathbf q}^+ 
      = {\mathbf q}^- 
      + 2 \Delta t {\mathcal A}( {\mathbf q}  )
      + 2 \Delta t B (   {\mathbf q}^+ 
                       + {\mathbf q}^-   )/2
\end{eqnarray}

where \(B\) is a square matrix. Then,
\(\Delta {\mathbf q} \equiv {\mathbf q}^+ - {\mathbf q}\) And then you
can write,

\begin{eqnarray}
  ( I - \Delta t B ) \Delta {\mathbf q} 
      = 2 \Delta t \left( {\mathcal A}({\mathbf q})
                         + B {\mathbf q} \right) 
\end{eqnarray}

This can be easily solved by matrix operations.

\hypertarget{semi-implicit-time-integration-applied}{%
\subsubsection{semi-implicit time integration
applied}\label{semi-implicit-time-integration-applied}}

Then, we apply this method and treat the term of linear gravity waves as
implicit. This allows us to reduce the time step (\(\Delta t\)).

In the system of equations, the basic field is such that \(T=\bar{T}_k\)
Separation of the linear gravity wave term and the other terms (with the
index \(NG\)). Vertical Vector Representation Using
\(\mathbf{D}=\{ D_{k} \}\), \(\mathbf{T}=\{ T_{k} \}\),

\begin{eqnarray}
   \frac{\partial \pi}{\partial t} = 
          \left( \frac{\partial \pi}{\partial t} \right)_{NG}  
     - \mathbf{C} \cdot \mathbf{D}  ,
\end{eqnarray}

\begin{eqnarray}
  \frac{\partial \mathbf{D}}{\partial t} = 
          \left( \frac{\partial \mathbf{D}}{\partial t} \right)_{NG}  
          - \nabla^{2}_{\sigma} ( \mathbf{\Phi}_{S} 
                                  + \underline{W} \mathbf{T}
                                  + \mathbf{G} \pi )  
          - {\mathcal D}_M \mathbf{D} ,
\end{eqnarray}

\begin{eqnarray}
  \frac{\partial \mathbf{T}}{\partial t} 
      =   \left( \frac{\partial \mathbf{T}}
                        {\partial t}       \right)_{NG}  
         - \underline{h} \mathbf{D}
         - {\mathcal D}_H \mathbf{T} ,
\end{eqnarray}

where the non-gravitational wave term is,

\begin{eqnarray}
  \left( \frac{\partial \pi}{\partial t} \right)^{NG}
   =   - \sum_{k=1}^{K} \mathbf{v}_{k} \cdot \nabla \pi  
       \Delta  \sigma_{k}  \\
   =   Z_{k}
\end{eqnarray}

\begin{quote}
\textless span id=``Section Z'' label=``Section Z'' label=``Section Z''
label="\textgreater\textbackslash\textbackslash.\}
\end{quote}

\begin{eqnarray}
  \dot{\sigma}^{NG}_{k-1/2}
 = - \sigma_{k-1/2} \left( \frac{\partial \pi}{\partial t} \right)^{NG}
   - \sum_{l=k}^{K} \mathbf{v}_{l} \cdot \nabla \pi
       \Delta  \sigma_{l}
\end{eqnarray}

\begin{eqnarray}
  \left( \frac{\partial D}{\partial t} \right)^{NG}
       =   \frac{1}{a\cos\varphi}
            \frac{\partial (A_u)_{k}}{\partial \lambda}
          + \frac{1}{a\cos\varphi}
            \frac{\partial }{\partial \varphi} (A_v \cos\varphi)_k
          - \nabla^{2}_{\sigma} \hat{E}_{k} 
          - {\mathcal D}(D_{k}) 
\end{eqnarray}

\begin{eqnarray}
  \left( \frac{\partial T_{k}}{\partial t} \right)^{NG} 
      =   - \frac{1}{a\cos\varphi} 
               \frac{\partial u_k T'_k}{\partial \lambda}
          - \frac{1}{a\cos\varphi}
               \frac{\partial }{\partial \varphi} (v_k T'_k \cos\varphi)
          + \hat{H}_{k} 
          - {\mathcal D}(T_{k}) 
\end{eqnarray}

\begin{eqnarray}
 \hat{H}_k  =  T_{k}^{\prime} D_{k}  \\
         - \frac{1}{\Delta \sigma_{k}} 
             [   \dot{\sigma}_{k-1/2} ( \hat{T^{\prime}}_{k-1/2} 
                                         - T^{\prime}_{k}   )
               + \dot{\sigma}_{k+1/2} ( T^{\prime}_{k}  
                                         - \hat{T^{\prime}}_{k+1/2} ) ]
                \\
         - \frac{1}{\Delta \sigma_{k}} 
             [   \dot{\sigma}^{NG}_{k-1/2} ( \hat{\bar{T}}_{k-1/2} 
                                         - \bar{T}_{k}   )
               + \dot{\sigma}^{NG}_{k+1/2} ( \bar{T}_{k}  
                                         - \hat{\bar{T}}_{k+1/2} ) ]
                \\
         + \hat{\kappa}_{k} T_{v,k} \mathbf{v}_{k} \cdot \nabla \pi
                \\
         - \frac{\alpha_{k}}{\Delta \sigma_{k} } T_{v,k}
             \sum_{l=k}^{K} \mathbf{v}_{l} \cdot \nabla \pi 
               \Delta \sigma_{l}
           - \frac{\beta_{k}}{\Delta \sigma_{k} } T_{v,k}
             \sum_{l=k+1}^{K} \mathbf{v}_{l} \cdot \nabla \pi 
               \Delta \sigma_{l}
                \\
         - \frac{\alpha_{k}}{\Delta \sigma_{k} } T'_{v,k}
             \sum_{l=k}^{K} D_l  \Delta \sigma_{l}
           - \frac{\beta_{k}}{\Delta \sigma_{k} } T'_{v,k}
             \sum_{l=k+1}^{K} D_l  \Delta \sigma_{l}
                \\
         + \frac{Q_k + (Q_{diff})_k}{C_p}
\end{eqnarray}

\begin{eqnarray}
  \hat{E}_k = E_{k} 
            + \sum_{k=1}^{K} W_{kl} ( T_{v,l}-T_{l} )
\end{eqnarray}

where the vector and matrix of the gravitational wave terms (underlined)
are,

\begin{eqnarray}
  C_{k} = \Delta \sigma_{k}
\end{eqnarray}

\begin{quote}
\protect\hypertarget{Coefficientux20C}{}{\textbackslash c{[}Coefficient
C{]}}
\end{quote}

\begin{eqnarray}
  W_{kl} = C_{p} \alpha_{l} \delta_{k \geq l}
         + C_{p} \beta_{l} \delta_{k-1 \geq l}
\end{eqnarray}

\begin{eqnarray}
  G_{k} = \hat{\kappa}_{k} C_{p} \bar{T}_{k}
\end{eqnarray}

\begin{eqnarray}
\underline{h} = \underline{Q}\underline{S} - \underline{R}
\end{eqnarray}

\begin{eqnarray}
  Q_{kl} = \frac{1}{\Delta \sigma_{k}} 
             ( \hat{\bar{T}}_{k-1/2} - \bar{T}_{k} ) \delta_{k=l} 
         + \frac{1}{\Delta \sigma_{k}} 
             ( \bar{T}_{k} - \hat{\bar{T}}_{k+1/2}  ) \delta_{k+1=l} 
\end{eqnarray}

\begin{eqnarray}
  S_{kl} = \sigma_{k-1/2} \Delta \sigma_{l} 
           - \Delta \sigma_{l} \delta_{k \leq l } 
\end{eqnarray}

\begin{eqnarray}
  R_{kl} = - \left(  \frac{ \alpha_{k} }{ \Delta \sigma_{k} } 
                     \Delta \sigma_{l} \delta_{k \leq l} 
                   + \frac{ \beta_{k} }{ \Delta \sigma_{k} } 
                     \Delta \sigma_{l} \delta_{k+1 \leq l}  
             \right) \bar{T}_{k} .
\end{eqnarray}

\begin{quote}
\protect\hypertarget{Coefficientux20R}{}{\R{[}Coefficient R{]}}.
\end{quote}

Here, for example, \(\delta_{k \leq l}\) is the same as It is a function
that is 1 if \$ k \leq l\$ is true and 0 otherwise.

Using the following expression,

\begin{eqnarray}
  \delta_{t} {X} \equiv \frac{1}{2 \Delta t} 
        \left( {X}^{t+\Delta t} - {X}^{t-\Delta t} \right)
\end{eqnarray}

\begin{quote}
\protect\hypertarget{Shemiinp}{}{\textbackslash centric=``Shemiinp\centric''\textgreater{}
}.
\end{quote}

\begin{eqnarray}
    \overline{X}^{t}
   \equiv  \frac{1}{2} \left( {X}^{t+\Delta t} 
                              + {X}^{t-\Delta t} \right)
         \\ 
   =  {X}^{t-\Delta t} + \delta_{t} {X} \Delta t   ,
\end{eqnarray}

Applying the semi-implicit method to a system of equations,

\begin{eqnarray}
  \delta_{t} \pi =
          \left( \frac{\partial \pi}{\partial t} \right)_{NG}  
     - \mathbf{C} \cdot \overline{ \mathbf{D} }^{t}
\end{eqnarray}

\begin{quote}
\protect\hypertarget{semi-impux20pi}{}{{[}semi-imp pi{]}}
\end{quote}

\begin{eqnarray}
  \delta_{t} \mathbf{D} =
          \left( \frac{\partial \mathbf{D}}{\partial t} \right)_{NG}  
          - \nabla^{2}_{\sigma} ( \mathbf{\Phi}_{S} 
                                  + \underline{W} 
                                     \overline{ \mathbf{T} }^{t}
                                  + \mathbf{G}
                                  \overline{\pi}^{t} ) 
          - {\mathcal D}_M ( \mathbf{D}^{t-\Delta t} 
                         + 2 \Delta t \delta_{t} \mathbf{D} )
\end{eqnarray}

\begin{quote}
\protect\hypertarget{semi-impux20D}{}{{[}semi-imp
D\textbackslash{[}semi-imp D{]}}
\end{quote}

\begin{eqnarray}
  \delta_{t} \mathbf{T} =
        \left( \frac{\partial \mathbf{T}}{\partial t} \right)_{NG}  
         - \underline{h} \overline{ \mathbf{D} }^{t} 
         - {\mathcal D}_H ( \mathbf{T}^{t-\Delta t}
                        + 2 \Delta t \delta_{t} \mathbf{T} )
\end{eqnarray}

\begin{quote}
\protect\hypertarget{semi-impux20T}{}{{[}semi-imp T{]}}
\end{quote}

So..,

\begin{quote}
\protect\hypertarget{semi-impux20barD}{}{\textbackslash brachos{[}semi-imp
barD\brachos{]}} \begin{eqnarray}
\left\{ ( 1+2\Delta t {\mathcal D}_H )( 1+2\Delta t {\mathcal D}_M )
\underline{I}  
- ( \Delta t )^{2}  ( \underline{W} \ \underline{h} 
+ (1+2\Delta t {\mathcal D}_M)
\mathbf{G} \mathbf{C}^{T} ) \nabla^{2}_{\sigma}
\right\}
\overline{ \mathbf{D} }^{t} 
\\
= ( 1+2\Delta t {\mathcal D}_H )( 1+\Delta t {\mathcal D}_M ) 
\mathbf{D}^{t-\Delta t}
+ \Delta t 
\left( \frac{\partial \mathbf{D}}{\partial t} \right)_{NG}  
\\
-  \Delta t \nabla^{2}_{\sigma}     
\left\{  ( 1+2\Delta t {\mathcal D}_H ) \mathbf{\Phi}_{S} 
+ \underline{W} 
\left[ ( 1-2\Delta t {\mathcal D}_H ) 
\mathbf{T}^{t-\Delta t}
+ \Delta t 
\left( \frac{\partial \mathbf{T}}
{\partial t}     
\right)_{NG} \right]
\right.
\\
\left.  \hspace*{20mm} 
+ ( 1+2\Delta t {\mathcal D}_H ) \mathbf{G} 
\left[ \pi^{t-\Delta t} 
+ \Delta t
\left( \frac{\partial \pi}
{\partial t} 
\right)_{NG}  \right]
\right\} . 
\end{eqnarray}
\end{quote}

Since we use the spherical harmonic expansion, we can use it,

and the above formula can be solved for
\(\overline{ \mathbf{D}_n^m }^{t}\). After that,

\begin{eqnarray}
   D^{t+\Delta t} = 2\overline{ \mathbf{D} }^{t} - D^{t-\Delta t}
\end{eqnarray}

and (\protect\hyperlink{semi-impux5cux2520pi}{semi-imp pi{]}},
(\protect\hyperlink{semi-impux5cux2520T}{semi-imp T\textbackslash{}})
The value in \(t+\Delta t\) according to \(\hat{X}^{t+\Delta t}\) is
required .

\hypertarget{time-scheme-properties-and-time-step-estimates}{%
\subsubsection{Time scheme properties and time step
estimates}\label{time-scheme-properties-and-time-step-estimates}}

advectional equation

\begin{eqnarray}
  \frac{\partial X}{\partial t} = c \frac{\partial X}{\partial x}
\end{eqnarray}

Consider the stability of the leap frog discretization in Now,

If we place the difference between

\begin{eqnarray}
  X^{n+1} = X^{n-1} + 2 i k \Delta t X^n
\end{eqnarray}

where Here , \textbackslash lambda = X\textsuperscript{\{n+1\}/X}n =
X\textsuperscript{n/X}\{n-1\}\}\textbackslash\textbackslash bars\}{]}
So,

\begin{eqnarray}
  \lambda^2 = 1 + 2 i kc \Delta t \lambda \; .
\end{eqnarray}

The solution is labeled \(kc \Delta t = p\),

\begin{eqnarray}
 \lambda = -i p \pm \sqrt{1-p^2}
\end{eqnarray}

This absolute value is

\begin{eqnarray}
  |\lambda| = \left\{ 
             \begin{array}{ll}
               1                      |p| \le 1 \\
               p \pm \sqrt{p^2-1} \;\;    |p| > 1
             \end{array}
             \right.
\end{eqnarray}

In the case of \(|p|>1\), it would be \(|\lambda| > 1\), The solution
becomes exponentially larger in absolute value with time. This indicates
that the computation is unstable.

On the other hand, in the case of \(|p| \le 1\), the value is
\(|\lambda| = 1\), The calculation is neutral. However, there are two
solutions for \(\lambda\), One of them, when \(\Delta t \rightarrow 1\)
is set to This is a \(\lambda \rightarrow 1\), but.., The other is
\(\lambda \rightarrow -1\). This indicates a time-varying solution. This
mode is called ``calculation mode'', One of the problems with the leap
frog method. This mode can be applied by applying a time filter to the
It can be attenuated.

The terms of the \(|p|=kc \Delta t \le 1\) are, Given the horizontal
discretization grid spacing \(\Delta x\), the This will cause the
maximum value of \(k\) to be \More than one person can be in a position
to do so. From becoming ,

\begin{eqnarray}
   \Delta t \le \frac{\Delta x}{\pi c}
\end{eqnarray}

In the case of the spectral model, the maximum wavenumber of For the
spectral model, the maximum wavenumber is determined by \(N\), Earth
radius is set to \(a\),

\begin{eqnarray}
   \Delta t \le \frac{a}{N c}  
\end{eqnarray}

This is the stability condition.

To guarantee the stability of the integral, As for \(c\), it has the
fastest advection and propagation speed, You may use a time step smaller
than \(\Delta t\) which is determined by the semi-implicit method. If
semi-implicit is not used, the propagation speed of the gravitational
wave (\(c \sim 300m/s\)) is the criterion for stability, When
semi-implicit is used, advection by the east-west wind is usually This
is a limiting factor. Therefore, the \(\Delta t\) sets \(U_{max}\) as
the maximum value of the east-west wind,

\begin{eqnarray}
   \Delta t \le \frac{a}{N U_{max}}  
\end{eqnarray}

In practice, this is multiplied by a safety factor. In practice, this
should be multiplied by a safety factor.

\hypertarget{treatment-at-the-beginning-of-time-integration.}{%
\subsubsection{Treatment at the beginning of time
integration.}\label{treatment-at-the-beginning-of-time-integration.}}

Not calculated by AGCM, If you start with an appropriate initial value,
you can use a model-consistent You cannot give the physical quantities
for two times of \(t\) and \(t-\Delta t\). However, if you give an
inconsistent value for \(t-\Delta t\), then you should not give an
inconsistent value for \(t-\Delta t\), A large calculation mode is
generated.

So, first, as \(X^{\Delta t/4} = X^0\), in the time step of \(1/4\)
\{X\textsuperscript{\{\{X\}}\{D\Delta t/2\} = X\^{}0 + X\^{}0 +
\{\{X\}\^{}\{D\Delta t/2\} = X\^{}0 + t/2
\textbackslash Dentro\{X\}\^{}0\textbackslash0\}{]} and furthermore, in
the time step of \(1/2\), \textbackslash\textbackslash lopen\}t =
X\textsuperscript{\{X\}}\{X\}\^{}\textsuperscript{\{X\}}\{\Delta t/2\}\}\lopen\}t
= X\^{}0 + \lopen\}t\lopen\}t\lopen\}t/2\lopen\}t And, in the original
time step, \textbackslash\textbackslash ltraLabella t\} =
X\^{}\{2\Delta t\} = X\^{}0 +
2\labella t\labella t\labella t\}\textsuperscript{\{X\}}\{\labella t\}{]}
and then perform the calculation with leap frog as usual, The occurrence
of computation modes is reduced.
