\hypertarget{time-integration.}{%
\subsection{Time integration.}\label{time-integration.}}

The time difference scheme is basically a leap frog. However, the terms
of the diffusion terms and physical processes are backward or forward
differences. A time filter (Asselin, 1972) is used to suppress the
computational mode. And to make the \(\Delta t\) larger, Applying the
semi-implicit method to the term gravitational wave (Bourke, 1988).

\hypertarget{time-integration-and-time-filtering-with-leap-frog}{%
\subsubsection{Time integration and time filtering with leap
frog}\label{time-integration-and-time-filtering-with-leap-frog}}

The leap frog is used as a time integration scheme for advection terms
and so on. The backward difference of \(2 \Delta t\) is used for the
horizontal diffusion term. In addition, the pseudo \(p\) surface
correction of the diffusion term and the frictional heat term by
horizontal diffusion are treated as a correction and becomes the forward
difference in \(2 \Delta t\). The physical process section
(\({\mathcal F}_\lambda, {\mathcal F}_\varphi, Q, S_q\)) is , I still
use the forward differential of \(2 \Delta t\). (However, we treat the
calculation of the time varying term of vertical diffusion as a backward
difference. See the chapter on physical processes for details.)

Expressed as \({X}\) on behalf of each forecast variable,

\begin{eqnarray}
  \hat{X}^{t+\Delta t}
    =  \bar{X}^{t-\Delta t}
    + 2 \Delta t
      \dot{X}_{adv}\left( {X}^{t} \right)
    + 2 \Delta t
      \dot{X}_{dif}\left( \hat{X}^{t+\Delta t} \right)
\end{eqnarray}

\(\dot{X}_{adv}\) is an advection term etc, \(\dot{X}_{dif}\) is a
horizontal diffusion term.

\$ \hat{X}\^{}\{t+\Delta t\} \$ has a , Pseudo, etc. \(p\) Correction of
frictional heat (\(\dot{X}_{dis}\)) by surface and horizontal diffusion
and physical processes (\$ \dot{X}\_\{phy\} \$) have been added, \$
\{X\}\^{}\{t+\Delta t\} \$.

\begin{eqnarray}
  {X}^{t+\Delta t}
    =  \hat{X}^{t+\Delta t}
    + 2 \Delta t
      \dot{X}_{dis}\left( \hat{X}^{t+\Delta t} \right)
    + 2 \Delta t
      \dot{X}_{phy}\left( \hat{X}^{t+\Delta t} \right)
\end{eqnarray}

To remove the computation mode in leap frog Apply the time filter of
Asselin (1972) at every step. Namely,

\begin{eqnarray}
  \bar{X}^{t}
    = ( 1-2 \epsilon_f ) {X}^{t}
    +  \epsilon_f
        \left( \bar{X}^{t-\Delta t} + {X}^{t+\Delta t} \right)
\end{eqnarray}

and \(\bar{X}\). For \(\epsilon_f\) it is standard to use 0.05.

\hypertarget{semi-implicit-time-integration}{%
\subsubsection{semi-implicit time
integration}\label{semi-implicit-time-integration}}

For mechanics calculations, the leap frog is basically used, Compute
some terms as implicit. Here, implicit considers a trapezoidal implicit.
Regarding the vector quantity \({\mathbf q}\), The value in \(t\) is
converted to \({\mathbf q}\), The value in \(t+\Delta t\) was converted
to \({\mathbf q}^+\), If you write the value of \(t-\Delta t\) as
\({\mathbf q}^-\), What is trapezoidal implicit?
\(({\mathbf q}^+ + {\mathbf q}^- )/2\). The solution is done by using
the time-varying terms evaluated by using Now, as a time-varying term in
{q}, The term A is treated in the leap forg method and the term B is
treated in the trapezoidal implicit method. A is nonlinear with respect
to {q}, while B is Suppose it is linear. Namely,

\begin{eqnarray}
  {\mathbf q}^+
      = {\mathbf q}^-
      + 2 \Delta t {\mathcal A}( {\mathbf q}  )
      + 2 \Delta t B (   {\mathbf q}^+
                       + {\mathbf q}^-   )/2
\end{eqnarray}

Note that \(B\) is a square matrix. Then,
\(\Delta {\mathbf q} \equiv {\mathbf q}^+ - {\mathbf q}\) And then you
can write,

\begin{eqnarray}
  ( I - \Delta t B ) \Delta {\mathbf q}
      = 2 \Delta t \left( {\mathcal A}({\mathbf q})
                         + B {\mathbf q} \right)
\end{eqnarray}

This can be easily solved by matrix operations.

\hypertarget{applying-semi-implicit-time-integration}{%
\subsubsection{Applying semi-implicit time
integration}\label{applying-semi-implicit-time-integration}}

So we apply this method and treat the term of linear gravity waves as
implicit. This makes the time step \(\Delta t\) smaller.

In a system of equations, the basic field is such that \(T=\bar{T}_k\)
Separation of the linear gravitational wave term and the other terms
(with the index \(NG\)). Vertical Vector Representation Using
\(\mathbf{D}=\{ D_{k} \}\), \(\mathbf{T}=\{ T_{k} \}\),

\begin{eqnarray}
   \frac{\partial \pi}{\partial t} =
          \left( \frac{\partial \pi}{\partial t} \right)_{NG}  
     - \mathbf{C} \cdot \mathbf{D}  ,
\end{eqnarray}

\begin{eqnarray}
  \frac{\partial \mathbf{D}}{\partial t} =
          \left( \frac{\partial \mathbf{D}}{\partial t} \right)_{NG}  
          - \nabla^{2}_{\sigma} ( \mathbf{\Phi}_{S}
                                  + \underline{W} \mathbf{T}
                                  + \mathbf{G} \pi )  
          - {\mathcal D}_M \mathbf{D} ,
\end{eqnarray}

\begin{eqnarray}
  \frac{\partial \mathbf{T}}{\partial t}
      =   \left( \frac{\partial \mathbf{T}}
                        {\partial t}       \right)_{NG}  
         - \underline{h} \mathbf{D}
         - {\mathcal D}_H \mathbf{T} ,
\end{eqnarray}

Here, the non-gravitational wave term is

\begin{eqnarray}
  \left( \frac{\partial \pi}{\partial t} \right)^{NG}
   =   - \sum_{k=1}^{K} \mathbf{v}_{k} \cdot \nabla \pi  
       \Delta  \sigma_{k}  \\
   =   Z_{k}
\end{eqnarray}

\begin{quote}
\bout[Section Z]\&lt ;/span\textgreater{}
\end{quote}

\begin{eqnarray}
  \dot{\sigma}^{NG}_{k-1/2}
 = - \sigma_{k-1/2} \left( \frac{\partial \pi}{\partial t} \right)^{NG}
   - \sum_{l=k}^{K} \mathbf{v}_{l} \cdot \nabla \pi
       \Delta  \sigma_{l}
\end{eqnarray}

\begin{eqnarray}
  \left( \frac{\partial D}{\partial t} \right)^{NG}
       =   \frac{1}{a\cos\varphi}
            \frac{\partial (A_u)_{k}}{\partial \lambda}
          + \frac{1}{a\cos\varphi}
            \frac{\partial }{\partial \varphi} (A_v \cos\varphi)_k
          - \nabla^{2}_{\sigma} \hat{E}_{k}
          - {\mathcal D}(D_{k})
\end{eqnarray}

\begin{eqnarray}
  \left( \frac{\partial T_{k}}{\partial t} \right)^{NG}
      =   - \frac{1}{a\cos\varphi}
               \frac{\partial u_k T'_k}{\partial \lambda}
          - \frac{1}{a\cos\varphi}
               \frac{\partial }{\partial \varphi} (v_k T'_k \cos\varphi)
          + \hat{H}_{k}
          - {\mathcal D}(T_{k})
\end{eqnarray}

\begin{eqnarray}
 \hat{H}_k  =  T_{k}^{\prime} D_{k}  \\
         - \frac{1}{\Delta \sigma_{k}}
             [   \dot{\sigma}_{k-1/2} ( \hat{T^{\prime}}_{k-1/2}
                                         - T^{\prime}_{k}   )
               + \dot{\sigma}_{k+1/2} ( T^{\prime}_{k}  
                                         - \hat{T^{\prime}}_{k+1/2} ) ]
                \\
         - \frac{1}{\Delta \sigma_{k}}
             [   \dot{\sigma}^{NG}_{k-1/2} ( \hat{\bar{T}}_{k-1/2}
                                         - \bar{T}_{k}   )
               + \dot{\sigma}^{NG}_{k+1/2} ( \bar{T}_{k}  
                                         - \hat{\bar{T}}_{k+1/2} ) ]
                \\
         + \hat{\kappa}_{k} T_{v,k} \mathbf{v}_{k} \cdot \nabla \pi
                \\
         - \frac{\alpha_{k}}{\Delta \sigma_{k} } T_{v,k}
             \sum_{l=k}^{K} \mathbf{v}_{l} \cdot \nabla \pi
               \Delta \sigma_{l}
           - \frac{\beta_{k}}{\Delta \sigma_{k} } T_{v,k}
             \sum_{l=k+1}^{K} \mathbf{v}_{l} \cdot \nabla \pi
               \Delta \sigma_{l}
                \\
         - \frac{\alpha_{k}}{\Delta \sigma_{k} } T'_{v,k}
             \sum_{l=k}^{K} D_l  \Delta \sigma_{l}
           - \frac{\beta_{k}}{\Delta \sigma_{k} } T'_{v,k}
             \sum_{l=k+1}^{K} D_l  \Delta \sigma_{l}
                \\
         + \frac{Q_k + (Q_{diff})_k}{C_p}
\end{eqnarray}

\begin{eqnarray}
  \hat{E}_k = E_{k}
            + \sum_{k=1}^{K} W_{kl} ( T_{v,l}-T_{l} )
\end{eqnarray}

where the vector and matrix of the gravitational wave term (underlined)
are

\begin{eqnarray}
  C_{k} = \Delta \sigma_{k}
\end{eqnarray}

\begin{quote}
Dr./Ing. lt;/span\textgreater{}
\end{quote}

\begin{eqnarray}
  W_{kl} = C_{p} \alpha_{l} \delta_{k \geq l}
         + C_{p} \beta_{l} \delta_{k-1 \geq l}
\end{eqnarray}

\begin{eqnarray}
  G_{k} = \hat{\kappa}_{k} C_{p} \bar{T}_{k}
\end{eqnarray}

\begin{eqnarray}
\underline{h} = \underline{Q}\underline{S} - \underline{R}
\end{eqnarray}

\begin{eqnarray}
  Q_{kl} = \frac{1}{\Delta \sigma_{k}}
             ( \hat{\bar{T}}_{k-1/2} - \bar{T}_{k} ) \delta_{k=l}
         + \frac{1}{\Delta \sigma_{k}}
             ( \bar{T}_{k} - \hat{\bar{T}}_{k+1/2}  ) \delta_{k+1=l}
\end{eqnarray}

\begin{eqnarray}
  S_{kl} = \sigma_{k-1/2} \Delta \sigma_{l}
           - \Delta \sigma_{l} \delta_{k \leq l }
\end{eqnarray}

\begin{eqnarray}
  R_{kl} = - \left(  \frac{ \alpha_{k} }{ \Delta \sigma_{k} }
                     \Delta \sigma_{l} \delta_{k \leq l}
                   + \frac{ \beta_{k} }{ \Delta \sigma_{k} }
                     \Delta \sigma_{l} \delta_{k+1 \leq l}  
             \right) \bar{T}_{k} .
\end{eqnarray}

\begin{quote}
Dr.R.{[}Coefficient R{]}\& lt;/span\textgreater{}
\end{quote}

Here, for example, \(\delta_{k \leq l}\) is the same as A function that
is 1 if the \$ k \leq l\$ is valid and 0 otherwise.

Using the following expression ,

\begin{eqnarray}
  \delta_{t} {X} \equiv \frac{1}{2 \Delta t}
        \left( {X}^{t+\Delta t} - {X}^{t-\Delta t} \right)
\end{eqnarray}

PluginPointPointPoint.com

\begin{eqnarray}
    \overline{X}^{t}
   \equiv  \frac{1}{2} \left( {X}^{t+\Delta t}
                              + {X}^{t-\Delta t} \right)
         \\
   =  {X}^{t-\Delta t} + \delta_{t} {X} \Delta t   ,
\end{eqnarray}

If we apply the semi-implicit method to the system of equations,

\begin{eqnarray}
  \delta_{t} \pi =
          \left( \frac{\partial \pi}{\partial t} \right)_{NG}  
     - \mathbf{C} \cdot \overline{ \mathbf{D} }^{t}
\end{eqnarray}

\begin{quote}
\textless span id=``semi-imp pi'' label="semi-imp
\end{quote}

\begin{eqnarray}
  \delta_{t} \mathbf{D} =
          \left( \frac{\partial \mathbf{D}}{\partial t} \right)_{NG}  
          - \nabla^{2}_{\sigma} ( \mathbf{\Phi}_{S}
                                  + \underline{W}
                                     \overline{ \mathbf{T} }^{t}
                                  + \mathbf{G}
                                  \overline{\pi}^{t} )
          - {\mathcal D}_M ( \mathbf{D}^{t-\Delta t}
                         + 2 \Delta t \delta_{t} \mathbf{D} )
\end{eqnarray}

\begin{quote}
\protect\hypertarget{semi-impux20D}{}{\textbackslash brahammer{[}semi-imp
D\textbackslash{]}}
\end{quote}

\begin{eqnarray}
  \delta_{t} \mathbf{T} =
        \left( \frac{\partial \mathbf{T}}{\partial t} \right)_{NG}  
         - \underline{h} \overline{ \mathbf{D} }^{t}
         - {\mathcal D}_H ( \mathbf{T}^{t-\Delta t}
                        + 2 \Delta t \delta_{t} \mathbf{T} )
\end{eqnarray}

\begin{quote}
\protect\hypertarget{semi-impux20T}{}{ \textgreater Semi-imp
T\textbackslash{[}semi-imp T{]}}.
\end{quote}

So..,

\begin{quote}
\protect\hypertarget{semi-impux20barD}{}{\blank{[}semi-imp
barD\blank{]}}\textless/a
\end{quote}

\begin{eqnarray}
      \left\{ ( 1+2\Delta t {\mathcal D}_H )( 1+2\Delta t {\mathcal D}_M )
           \underline{I}  
      - ( \Delta t )^{2}  ( \underline{W} \ \underline{h}
           + (1+2\Delta t {\mathcal D}_M)
             \mathbf{G} \mathbf{C}^{T} ) \nabla^{2}_{\sigma}
  \right\}
      \overline{ \mathbf{D} }^{t}
       \\
  = ( 1+2\Delta t {\mathcal D}_H )( 1+\Delta t {\mathcal D}_M )
       \mathbf{D}^{t-\Delta t}
  + \Delta t
     \left( \frac{\partial \mathbf{D}}{\partial t} \right)_{NG}  
  \\
  -  \Delta t \nabla^{2}_{\sigma}     
                   \left\{  ( 1+2\Delta t {\mathcal D}_H ) \mathbf{\Phi}_{S}
                          + \underline{W}
                            \left[ ( 1-2\Delta t {\mathcal D}_H )
                                    \mathbf{T}^{t-\Delta t}
                                  + \Delta t
                                      \left( \frac{\partial \mathbf{T}}
                                                  {\partial t}     
                                      \right)_{NG} \right]
                   \right.
  \\
                 \left.   
                          + ( 1+2\Delta t {\mathcal D}_H ) \mathbf{G}
                            \left[ \pi^{t-\Delta t}
                                  + \Delta t
                                     \left( \frac{\partial \pi}
                                                 {\partial t}
                                     \right)_{NG}  \right]
                   \right\} .
\end{eqnarray}

Since the spherical harmonic expansion is used, \begin{eqnarray}
    \nabla^{2}_{\sigma} = - \frac{n(n+1)}{a^{2}}
\end{eqnarray} com{]} and the above equation can be solved for
\(\overline{ \mathbf{D}_n^m }^{t}\). And then..,

\begin{eqnarray}
   D^{t+\Delta t} = 2\overline{ \mathbf{D} }^{t} - D^{t-\Delta t}
\end{eqnarray}

and, (\protect\hyperlink{semi-impux5cux2520T}{{[}semi-imp
pi{]}{]}(\#semi-imp\%20pi)),
({[}\textbackslash blink\blink\}{]}(\#semi-imp\%20pi) semi-imp T{]}})
The value in \(t+\Delta t\) according to \(\hat{X}^{t+\Delta t}\) is
required.

\hypertarget{time-scheme-properties-and-time-step-estimates}{%
\subsubsection{Time scheme properties and time step
estimates}\label{time-scheme-properties-and-time-step-estimates}}

advectional equation

\begin{eqnarray}
  \frac{\partial X}{\partial t} = c \frac{\partial X}{\partial x}
\end{eqnarray}

Considering the stability of the discretization in the leap frog in Now,

If we place the difference between

\begin{eqnarray}
  X^{n+1} = X^{n-1} + 2 i k \Delta t X^n
\end{eqnarray}

That would be. Here, \textbackslash lambda =
X\textsuperscript{\{n+1\}/X}n =
X\textsuperscript{n/X}\{n-1\}\}\textbackslash\textbackslash bars\}{]}
So,

\begin{eqnarray}
  \lambda^2 = 1 + 2 i kc \Delta t \lambda \; .
\end{eqnarray}

The solution is called \(kc \Delta t = p\),

\begin{eqnarray}
 \lambda = -i p \pm \sqrt{1-p^2}
\end{eqnarray}

This absolute value is

\begin{eqnarray}
  |\lambda| = \left\{
             \begin{array}{ll}
               1                      |p| \le 1 \\
               p \pm \sqrt{p^2-1} \;\;    |p| > 1
             \end{array}
             \right.
\end{eqnarray}

In the case of \(|p|>1\), it is \(|\lambda| > 1\), It is a solution
whose absolute value increases exponentially with time. This indicates
that the computation is unstable.

On the other hand, for \(|p| \le 1\), the number is \(|\lambda| = 1\),
The calculation is neutral. However, there are two solutions for
\(\lambda\), One of them, when \(\Delta t \rightarrow 1\) is set to This
is a \(\lambda \rightarrow 1\), but.., The other is
\(\lambda \rightarrow -1\). This indicates that the solution oscillates
strongly in time. This mode is called calculation mode, One of the
problems with the leap frog method. This mode can be used by applying a
time filter to the It can be attenuated.

The terms of the \(|p|=kc \Delta t \le 1\) are , Given the horizontal
discretization grid spacing \(\Delta x\), the This will cause the
maximum value of \(k\) to be \More than one person can be in a position
to do so. From becoming ,

\begin{eqnarray}
   \Delta t \le \frac{\Delta x}{\pi c}
\end{eqnarray}

That would be. In the case of a spectral model, the maximum wavenumber
depends on the \(N\), Earth radius is set to \(a\),

\begin{eqnarray}
   \Delta t \le \frac{a}{N c}  
\end{eqnarray}

This is a condition for stability.

To guarantee the stability of the integration, As for \(c\), it has the
fastest advection and propagation speed, You can use a time step smaller
than \(\Delta t\) determined by that. When semi-implicit is not used,
the propagation speed of gravity wave (\(c \sim 300m/s\)) is the
criterion for stability, When semi-implicit is used, advection by the
east-west wind is usually Limiting factors. Therefore, \(\Delta t\) sets
\(U_{max}\) as the maximum value of the east-west wind,

\begin{eqnarray}
   \Delta t \le \frac{a}{N U_{max}}  
\end{eqnarray}

Take to meet the . In practice, this is multiplied by a safety factor.

\hypertarget{handling-of-the-initiation-of-time-integration}{%
\subsubsection{Handling of the Initiation of Time
Integration}\label{handling-of-the-initiation-of-time-integration}}

Not calculated by AGCM, If you start with an appropriate initial value,
you can use a model-consistent You cannot give two physical quantities
of time in \(t\) and \(t-\Delta t\). However, if you give an
inconsistent value for \(t-\Delta t\) A large calculation mode occurs.

So, first, as \(X^{\Delta t/4} = X^0\), in the time step of \(1/4\)

\begin{eqnarray}
X^{\frac{\Delta t}{2}} = X^0 + \frac{\Delta t}{2}\dot{X}^{ \frac{\Delta t}{4}}  = X^0 + \frac{\Delta t}{2}\dot{X}^0
\end{eqnarray}

and furthermore, in the time step of \(1/2\),

\begin{eqnarray}
 X^{\Delta t} = X^0 + \Delta t \dot{X}^{\frac{\Delta t}{2}}
 \end{eqnarray}

And, in the original time step,

\begin{eqnarray}
 X^{2\Delta t} = X^0 + 2\Delta t \dot{X}^{\Delta t}
 \end{eqnarray}

and then perform the calculation with leap frog as usual, The occurrence
of computation mo
