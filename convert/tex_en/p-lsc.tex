\hypertarget{massive-coagulation}{%
\subsection{Massive Coagulation}\label{massive-coagulation}}

\hypertarget{overview-of-large-scale-condensation-schemes.}{%
\subsubsection{Overview of Large Scale Condensation
Schemes.}\label{overview-of-large-scale-condensation-schemes.}}

Large-scale condensation schemes are , This is a representation of the
condensation processes involved in clouds other than cumulus convection,
Calculating latent heat release and water vapor reduction,
precipitation. We also calculate the cloud water content and cloud
coverage involved in the radiation. The main input data are temperature
\(T\), specific humidity \(q\), and cloud cover \(l\), and is, The
output data is the time rate of change of temperature, specific humidity
and cloud water content,
\(\partial T/\partial t, \partial q/\partial t, \partial l/\partial t\),
The cloud cover is \(C\).

In the CCSR/NIES AGCM, in addition to the water-vapor mixture ratio
(specific humidity \(q\)), the Cloud water content (\(l\)) is also a
forecast variable in the model. In fact, in this large-scale
condensation routine First, calculate the sum of the two, the total
amount of water (\(q^t = q+l\)), We are dividing it again into cloud
water and water vapor, In effect, the forecast variable is a single
total water volume (\(q^t\)). By assuming the distribution of the
variation of \(q^t\) in the grid, Diagnosis of the cloud cover and cloud
water content in each grid. The conversion of cloud water into
precipitation and the evaporation of precipitation during its fall are
also considered.

The outline of the calculation procedure is as follows.

\begin{enumerate}
\def\labelenumi{\arabic{enumi}.}
\item
  add the amount of water vapor (\(q\)) and the amount of cloud water
  (\(l\)) Total water volume \(q^t\) The temperature has evaporated the
  cloud water, Set the liquid water temperature \(T_l\).
\item
  assuming the distribution of the variation in \(q^t\), Find the cloud
  cover and separate it again into cloud water and water vapor.
\item
  considering the temperature change due to condensation, By successive
  approximation Determine the cloud cover, cloud water content, and
  water vapor distribution.
\item
  evaluate the conversion of cloud water to precipitation.
\item
  evaluate the ice fall.
\item
  evaluate precipitation and evaporation of falling ice.
\end{enumerate}

\hypertarget{diagnosis-of-cloud-water-levels}{%
\subsubsection{Diagnosis of cloud water
levels}\label{diagnosis-of-cloud-water-levels}}

When the grid-averaged total water volume
\(\bar{q}^t = \bar{q} + \bar{l}\) is given, Distribution of the total
water volume \(q^t\) in the grid, Between \((1-b)\bar{q}^t\) and
\((1+b)\bar{q}^t\) It is assumed to be uniformly distributed. That is,
the probability density function is ,

\begin{eqnarray}
  F(q^t) = \left\{
           \begin{array}{ll}
             (2b\bar{q}^t)^{-1} \; \;
                 (1-b)\bar{q}^t < q^t <  (1+b)\bar{q}^t \\
             0                         (その他)
           \end{array}
           \right. \; .
\end{eqnarray}

We consider this distribution to be a horizontal distribution. On the
other hand, the saturation specific humidity is based on the grid
average of \(\bar{q}^*\).

In the grid, Consider the existence of clouds in the region of
\(q^t > q^*\) (Figure {[}lsc:fig-cloud
\protect\hyperlink{lsc:fig-cloud}{{]}}).

Then, as shown by the shading in the figure, the The horizontal ratio of
the portion of the total water volume exceeding saturation \(C\) is ,

\begin{eqnarray}
  C = \left\{
           \begin{array}{ll}
             0 \; \;  (1+b)\bar{q}^t \leq \bar{q}^* \\
             \displaystyle
             \frac{(1+b)\bar{q}^t - \bar{q}^*}
                  {2b\bar{q}^t}                    
               \; \;  (1-b)\bar{q}^t < \bar{q}^* < (1+b)\bar{q}^t \\
             1 \; \;  (1-b)\bar{q}^t \leq \bar{q}^*
           \end{array}
        \right.
\end{eqnarray}

and this is the cloud cover (horizontal cloud coverage).

In addition, the cloud cover of \(l\) is in the region of \(q^t > q^*\)
This is an integral of \(q^t - q^*\),

\begin{eqnarray}
  l = \left\{
           \begin{array}{ll} \displaystyle
             0 \; \;  (1+b)\bar{q}^t \leq \bar{q}^* \\
            \displaystyle
             \frac{\left[(1+b)\bar{q}^t - \bar{q}^*\right]^2}
                  {4b\bar{q}^t}
               \; \;  (1-b)\bar{q}^t \leq \bar{q}^* \leq (1+b)\bar{q}^t  \\
            \displaystyle
             \bar{q}^t - \bar{q}^*
                \; \;  (1-b)\bar{q}^t \geq \bar{q}^*
           \end{array}
        \right.
\end{eqnarray}

\begin{quote}
\textless span id=``p-lsc:l'' label=``p-lsc:l''\&gt
;p-lsc:l\textbackslash.com;
\end{quote}

\hypertarget{determination-by-successive-approximation}{%
\subsubsection{Determination by successive
approximation}\label{determination-by-successive-approximation}}

First, from the Water Vapor \(q\) and Cloud Water \(l\) and the
Temperature \(T\), Total water volume \(q^t\) and liquid water
temperature TERM Ask for 00029.

\begin{eqnarray}
  q^t   =  q + l \; , \\
  T_l   =  T - \frac{L}{C_P} l \; .
\end{eqnarray}

The \(T_l\) corresponds to the temperature at which all cloud water is
evaporated. \(T^{(0)} = T_l\), \(l^{(0)} = 0\)

By saturation specific humidity relative to temperature \(T_l\),
Assuming that the cloud water content evaluated by the aforementioned
method is \(l^{(1)}\), It changes the temperature,

\begin{eqnarray}
  T^{(1)} = T_l +  \frac{L}{C_P} l^{(1)} \; .
\end{eqnarray}

\begin{quote}
{[}p-lsc:itereate1{[}p-lsc:itereate1{]}\textless/ span\textgreater{}
\end{quote}

The cloud water content evaluated using the saturation specific humidity
versus temperature was estimated from \(l^{(2)}\), The resulting
temperature change is solved by successive approximation as \(T^{(2)}\)
\ldots{} In order to speed up this sequential convergence, we use the
Newton method. I.e.,
(\protect\hyperlink{p-lsc:itereateux5cux25201}{{[}p-lsc:itereate1{]}}))
instead of

\begin{eqnarray}
  T^{(1)} = T_l +  \frac{L}{C_P} l^{(1)}
                   \left( 1 - \frac{L}{C_P} \frac{dl}{dT} \right)^{-1}
\end{eqnarray}

. \(dl/dT\) uses the analytical You can ask for.

\hypertarget{precipitation-process.}{%
\subsubsection{precipitation process.}\label{precipitation-process.}}

Precipitation occurs dependent on the amount of cloud water diagnosed.
If the precipitation rate (in 1/s) is set to \(P\),

\begin{eqnarray}
  P = l / \tau_P \; .
\end{eqnarray}

\(\tau_P\) is the time scale of precipitation,

\begin{eqnarray}
  \tau_P  = \tau_0 \left\{ 1 - \exp\left[ - \left(\frac{l}{l_C}\right)^2  
                                   \right]  \right\}^{-1} \; .
\end{eqnarray}

where \(l_C\) is the critical cloud water content, In view of the
Bergeron-Findeisen effect ,

\begin{eqnarray}
  l_C = \left\{
        \begin{array}{ll}
          l_C^0 \; \;  T \ge T_0 \\
          l_C^0 \left\{ 1+\alpha \exp\left[ - \beta(T-Tc)^2 \right]
                \right\}^{-1}\; \;
                       T_0 > T >  T_c \\
          l_C^0 ( 1+\alpha )^{-1}
                       T \le T_c
        \end{array}
        \right. \; .
\end{eqnarray}

\(l_C^0=10^{-4}\), \(\alpha=50\), \(\beta=0.03\), \(T_0=273.15\) K,
\(T_c=258.15\) K

Precipitation results in a decrease in \(l\).

\begin{eqnarray}
  P          =  l / \tau_P \; , \\
  \frac{\partial l}{\partial t}  =  -P \; .
\end{eqnarray}

Integrating this during \(\Delta t\),

\begin{eqnarray}
  P \Delta t  =  \left\{ 1- \exp(- \Delta t/\tau_P) \right\} l \; .
\end{eqnarray}

Precipitation flux at a certain height, \(p\) If (unit kg m\(^{-2}\)
s\(^{-1}\)) is set to \(F_P\)

\begin{eqnarray}
  F_P(p) = \int_0^p P \frac{dp}{g} \; .
\end{eqnarray}

\hypertarget{ice-falling-process.}{%
\subsubsection{Ice Falling Process.}\label{ice-falling-process.}}

Cloud water is divided into ice and water clouds depending on the
temperature. The ice cloud ratio is

\begin{eqnarray}
   f_I = \frac{ T_0 - T }{ T_0 - T_1 }
\end{eqnarray}

(but with a maximum value of 1 and a minimum value of 0). Also,
\(T_0 = 273.15{K}, T_1 = 258.15{K}\). The ice cloud will fall at a slow
speed, Consider the effect. Rate of descent \(V_S\) is,

\begin{eqnarray}
  V_S = V_S^0 ( \rho_a f_I l )^\gamma \; .
\end{eqnarray}

However, \(V_S^0=3\) m/s, \(\gamma=0.17\). So..,

\begin{eqnarray}
  \tau_S = \frac{\Delta p}{\rho g V_S}
\end{eqnarray}

as well as precipitation.

\hypertarget{evaporation-process-of-precipitation.}{%
\subsubsection{Evaporation process of
precipitation.}\label{evaporation-process-of-precipitation.}}

Evaporation of precipitation The evaporation of precipitation, \(E\), is
estimated as follows.

\begin{eqnarray}
E = k_E (q^w - q) \frac{F_P}{V_T} \; .
\end{eqnarray}

However, if \(q^w < q\) is set, this should be zero. The \(q_w\) is the
saturation specific humidity corresponding to the wet bulb temperature,

\begin{eqnarray}
  q^w = q + \frac{q^* - q}{1+ \frac{L}{C_P}\frac{\partial q^*}{\partial T}} \; .
\end{eqnarray}

This means that precipitation is

\begin{eqnarray}
  F_P(p) = \int_0^p (P - E) \frac{dp}{g}
\end{eqnarray}

The temperature drop due to evaporation is estimated to be We also
estimate the temperature drop due to evaporation.

\begin{eqnarray}
  \frac{\partial T}{\partial t} = - \frac{L}{C_P} E \; .
\end{eqnarray}

\hypertarget{other-notes.}{%
\subsubsection{Other Notes.}\label{other-notes.}}

Calculations are made from the topmost layer down. For convenience, the
calculation is based on the precipitation from the upper layers of the
We start by evaluating evaporation in that layer.

\begin{enumerate}
\def\labelenumi{\arabic{enumi}.}
\setcounter{enumi}{1}
\tightlist
\item
  fallen ice in the layer just below. It will be treated the same as the
  cloud water that already exists in that layer, incorporated into the
  total water volume.
\end{enumerate}
