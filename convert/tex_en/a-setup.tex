\hypertarget{basic-settings.}{%
\subsection{Basic Settings.}\label{basic-settings.}}

Here we present the basic setup of the model.

\hypertarget{coordinate-system.}{%
\subsubsection{Coordinate System.}\label{coordinate-system.}}

The coordinate system is basically, Longitude \(\lambda\), Latitude
\(\varphi\), Normalized Pressure \(\sigma = p/p_S\)
(\(p_S(\lambda,\varphi)\) are surface pressure.) and treat each as
orthogonal. However, \(z\) is used as the vertical coordinates.

Longitude is discretized at equal intervals \texttt{MODULE:{[}ASETL{]}}.

\begin{eqnarray}
  \lambda_i = 2 \pi \frac{i-1}{I}  \;\;\; i = 1, \ldots I-1
\end{eqnarray}

Latitude is the Gauss latitude \(\varphi_j\) described in Mechanics, and
\texttt{MODULE:{[}\ ASETL{]}}, Gauss-Legendre derived from the integral
formula. This takes \(\mu = \sin \varphi\) as its argument J The zero
point of the next Legendre polynomial \texttt{MODULE:{[}GAUSS{]}}.

If J is large, we can approximate

\begin{eqnarray}
  \varphi_j =  \pi ( \frac{1}{2}- \frac{j-1/2}{J} ) \;\;\; j = 1, \ldots J-1
\end{eqnarray}

Normally, the grid spacing of longitude and latitude is taken as
\(J = I/2\) almost equally. This is based on the triangular truncation
of the spectral method.

Normalized atmospheric pressure (\(\sigma\)) is designed to give a good
representation of the vertical structure of the atmosphere, suitably
discretized at unequal intervals \texttt{MODULE:{[}ASETS{]}}. As we will
discuss later in Mechanics, the value of the layer boundaries Define the
\(\sigma_{k-1/2}\) in \(k = 1 \ldots K+1\) and then ,

\begin{eqnarray}
 \sigma_k = \left\{ \frac{1}{1+\kappa}
                     \left( \frac{  \sigma^{\kappa +1}_{k-1/2}
                                  - \sigma^{\kappa +1}_{k+1/2}      }
                                  { \sigma_{k-1/2} - \sigma_{k+1/2} }
                     \right)
              \right\}^{1/\kappa}
\end{eqnarray}

Find the \(\sigma\) representing the layer by Figure
{[}a-setup:level{]}{]} (\#a-setup:level), with the standard Indicates
the level of the 20 layers used.

Each forecast variable is all, \((\lambda_i, \varphi_j, \sigma_k)\) or
defined on the grid of \((\lambda_i, \varphi_j, z_l)\). (The underground
level, \(z_l\), is discussed in the Physical Processes section.)

In the time direction, they are discretized at equally spaced
\(\Delta t\), The time integration of the forecasting equation is
performed. However, if the stability of the time integration may be
compromised \(\Delta t\) can change.

\hypertarget{physical-constants.}{%
\subsubsection{Physical Constants.}\label{physical-constants.}}

The basic physical constants are shown below
\texttt{MODULE:{[}APCON{]}}.

\begin{itemize}
\item
  TAB00000:0.0 earth radius
\item
  TAB00000:0.1 \(a\)
\item
  TAB00000:0.2 m
\item
  TAB00000:0.3 6.37 \(\times 10^6\)
\item
  TAB00000:1.0 acceleration of gravity
\item
  TAB00000:1.1 \(g\)
\item
  TAB00000:1.2 ms\(^-2\)
\item
  TAB00000:1.3 9.8
\item
  TAB00000:2.0 atmospheric pressure specific heat
\item
  TAB00000:2.1 \(C_p\)
\item
  TAB00000:2.2 J kg\(^{-1}\) K\(^{-1}\)
\item
  TAB00000:2.3 1004.6
\item
  TAB00000:3.0 Atmospheric gas constant
\item
  TAB00000:3.1 \(R\)
\item
  TAB00000:3.2 J kg\(^{-1}\) K\(^{-1}\)
\item
  TAB00000:3.3 287.04
\item
  TAB00000:4.0 Latent heat of water evaporation
\item
  TAB00000:4.1 \(L\)
\item
  TAB00000:4.2 J kg\(^{-1}\)
\item
  TAB00000:4.3 2.5 \(\times 10^6\)
\item
  TAB00000:5.0 Water vapor constant pressure specific heat
\item
  TAB00000:5.1 \(C_v\)
\item
  TAB00000:5.2 J kg\(^{-1}\) K\(^{-1}\)
\item
  TAB00000:5.3 1810\bsp.
\item
  TAB00000:6.0 Gas constant of water
\item
  TAB00000:6.1 \(R_v\)
\item
  TAB00000:6.2 J kg\(^{-1}\) K\(^{-1}\)
\item
  TAB00000:6.3 461.
\item
  TAB00000:7.0 Density of liquid water
\item
  TAB00000:7.1 \(d_{H_2O}\)
\item
  TAB00000:7.2 J kg\(^{-1}\) K\(^{-1}\)
\item
  TAB00000:7.3

  \begin{enumerate}
  \def\labelenumi{\arabic{enumi}.}
  \setcounter{enumi}{999}
  \tightlist
  \item
  \end{enumerate}
\item
  TAB00000: 8.0 0 in \(^{\circ}\). saturation vapor
\item
  TAB00000:8.1 \(e^*\)(273K)
\item
  TAB00000:8.2 Pa.
\item
  TAB00000:8.3 611
\item
  TAB00000:9.0 Stefan Bolzman constant
\item
  TAB00000:9.1 \(\sigma_{SB}\)
\item
  TAB00000:9.2 W m\(^{-2}\) K\(^{-4}\)
\item
  TAB00000:9.3 5.67 \(\times 10^{-8}\)
\item
  TAB00000:10.0 Kárman Constant
\item
  TAB00000:10.1 \(k\)
\item
  TAB00000:10.2
\item
  TAB00000:10.3 0.4
\item
  TAB00000:11.0 Latent heat of ice melting
\item
  TAB00000:11.1 \(L_M\)
\item
  TAB00000:11.2 J kg\(^{-1}\)
\item
  TAB00000:11.3 3.4 \(\times 10^5\)
\item
  TAB00000:12.0 Water Freezing Point
\item
  TAB00000:12.1 \(T_M\)
\item
  TAB00000:12.2 K
\item
  TAB00000:12.3 273.15
\item
  TAB00000:13.0 Constant pressure specific heat of water
\item
  TAB00000:13.1 \(C_w\)
\item
  TAB00000:13.2 J kg\(^{-1}\)
\item
  TAB00000:13.3 4,200.
\item
  TAB00000:14.0 The freezing point of seawater
\item
  TAB00000:14.1 \(T_I\)
\item
  TAB00000:14.2 K
\item
  TAB00000:14.3 271.35
\item
  TAB00000:15.0 Specific heat ratio of ice at constant pressure
\item
  TAB00000:15.1 \(C_I = C_w - L_M/T_M\)
\item
  TAB00000:15.2
\item
  TAB00000:15.3 2397.
\item
  TAB00000:16.0 water vapor molecular weight ratio
\item
  TAB00000:16.1 \(\epsilon = R/R_v\)
\item
  TAB00000:16.2
\item
  TAB00000:16.3 0.622
\item
  TAB00000:17.0 coefficient of provisional temperature
\item
  TAB00000:17.1 \(\epsilon_v = \epsilon^{-1} - 1\)
\item
  TAB00000:17.2
\item
  TAB00000:17.3 0.606
\item
  TAB00000:18.0 Ratio of specific heat to gas constant
\item
  TAB00000:18.1 \(\kappa = R/C_p\)
\item
  TAB00000:18.2
\item
  TAB00000:18.3 0.286
\end{itemize}
