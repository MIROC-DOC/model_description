\hypertarget{basic-settings.}{%
\subsection{Basic Settings.}\label{basic-settings.}}

Here we present the basic setup of the model.

\hypertarget{coordinate-system.}{%
\subsubsection{Coordinate System.}\label{coordinate-system.}}

The coordinate system is basically, Longitude \(\lambda\), Latitude
\(\varphi\), Normalized Pressure \(\sigma = p/p_S\)
(\(p_S(\lambda,\varphi)\) are surface pressure.) and treat each as
orthogonal. However, \(z\) is used as the vertical coordinates.

Longitude is discretized at equal intervals \texttt{MODULE:{[}ASETL{]}}.

\begin{eqnarray}
  \lambda_i = 2 \pi \frac{i-1}{I}  \;\;\; i = 1, \ldots I-1
\end{eqnarray}

Latitude is the Gauss latitude \(\varphi_j\) described in Mechanics, and
\texttt{MODULE:{[}ASETL{]}}, Gauss-Legendre derived from the integral
formula. This takes \(\mu = \sin \varphi\) as its argument J The zero
point of the next Legendre polynomial \texttt{MODULE:{[}GAUSS{]}}.

If J is large, we can approximate

\begin{eqnarray}
  \varphi_j =  \pi ( \frac{1}{2}- \frac{j-1/2}{J} ) \;\;\; j = 1, \ldots J-1
\end{eqnarray}

Normally, the grid spacing of longitude and latitude is taken as
\(J = I/2\) almost equally. This is based on the triangular truncation
of the spectral method.

Normalized atmospheric pressure (\(\sigma\)) is designed to give a good
representation of the vertical structure of the atmosphere, suitably
discretized at unequal intervals \texttt{MODULE:{[}ASETS{]}}. As we will
discuss later in Mechanics, the value of the layer boundaries Define the
\(\sigma_{k-1/2}\) in \(k = 1 \ldots K+1\) and then ,

\begin{eqnarray}
 \sigma_k = \left\{ \frac{1}{1+\kappa}
                     \left( \frac{  \sigma^{\kappa +1}_{k-1/2}
                                  - \sigma^{\kappa +1}_{k+1/2}      }
                                  { \sigma_{k-1/2} - \sigma_{k+1/2} }
                     \right)
              \right\}^{1/\kappa}
\end{eqnarray}

Find the \(\sigma\) representing the layer by Figure
{[}a-setup:level{]}{]} (\#a-setup:level) shows the 20 levels of the
standard.

Each forecast variable is all, \((\lambda_i, \varphi_j, \sigma_k)\) or
defined on the grid of \((\lambda_i, \varphi_j, z_l)\). (The underground
level, \(z_l\), is discussed in the Physical Processes section.)

In the time direction, they are discretized at equally spaced
\(\Delta t\), The time integration of the forecasting equation is
performed. However, if the stability of the time integration may be
compromised \(\Delta t\) can change.

\hypertarget{physical-constants.}{%
\subsubsection{Physical Constants.}\label{physical-constants.}}

The basic physical constants are shown below
\texttt{MODULE:{[}APCON{]}}.

\begin{longtable}[]{@{}llll@{}}
\toprule
\begin{minipage}[b]{0.22\columnwidth}\raggedright
Header0\strut
\end{minipage} & \begin{minipage}[b]{0.22\columnwidth}\raggedright
Header1\strut
\end{minipage} & \begin{minipage}[b]{0.22\columnwidth}\raggedright
Header2\strut
\end{minipage} & \begin{minipage}[b]{0.22\columnwidth}\raggedright
Header3\strut
\end{minipage}\tabularnewline
\midrule
\endhead
\begin{minipage}[t]{0.22\columnwidth}\raggedright
earth radius\strut
\end{minipage} & \begin{minipage}[t]{0.22\columnwidth}\raggedright
\(a\)\strut
\end{minipage} & \begin{minipage}[t]{0.22\columnwidth}\raggedright
m\strut
\end{minipage} & \begin{minipage}[t]{0.22\columnwidth}\raggedright
6.37 \(\times 10^6\)\strut
\end{minipage}\tabularnewline
\begin{minipage}[t]{0.22\columnwidth}\raggedright
acceleration of gravity\strut
\end{minipage} & \begin{minipage}[t]{0.22\columnwidth}\raggedright
\(g\)\strut
\end{minipage} & \begin{minipage}[t]{0.22\columnwidth}\raggedright
ms\(^-2\)\strut
\end{minipage} & \begin{minipage}[t]{0.22\columnwidth}\raggedright
9.8\strut
\end{minipage}\tabularnewline
\begin{minipage}[t]{0.22\columnwidth}\raggedright
atmospheric pressure specific heat\strut
\end{minipage} & \begin{minipage}[t]{0.22\columnwidth}\raggedright
\(C_p\)\strut
\end{minipage} & \begin{minipage}[t]{0.22\columnwidth}\raggedright
J kg\(^{-1}\) K\(^{-1}\)\strut
\end{minipage} & \begin{minipage}[t]{0.22\columnwidth}\raggedright
1004.6\strut
\end{minipage}\tabularnewline
\begin{minipage}[t]{0.22\columnwidth}\raggedright
Atmospheric gas constant\strut
\end{minipage} & \begin{minipage}[t]{0.22\columnwidth}\raggedright
\(R\)\strut
\end{minipage} & \begin{minipage}[t]{0.22\columnwidth}\raggedright
J kg\(^{-1}\) K\(^{-1}\)\strut
\end{minipage} & \begin{minipage}[t]{0.22\columnwidth}\raggedright
287.04\strut
\end{minipage}\tabularnewline
\begin{minipage}[t]{0.22\columnwidth}\raggedright
Latent heat of water evaporation\strut
\end{minipage} & \begin{minipage}[t]{0.22\columnwidth}\raggedright
\(L\)\strut
\end{minipage} & \begin{minipage}[t]{0.22\columnwidth}\raggedright
J kg\(^{-1}\)\strut
\end{minipage} & \begin{minipage}[t]{0.22\columnwidth}\raggedright
2.5 \(\times 10^6\)\strut
\end{minipage}\tabularnewline
\begin{minipage}[t]{0.22\columnwidth}\raggedright
Water vapor constant pressure specific heat\strut
\end{minipage} & \begin{minipage}[t]{0.22\columnwidth}\raggedright
\(C_v\)\strut
\end{minipage} & \begin{minipage}[t]{0.22\columnwidth}\raggedright
J kg\(^{-1}\) K\(^{-1}\)\strut
\end{minipage} & \begin{minipage}[t]{0.22\columnwidth}\raggedright
1810\bsp.\strut
\end{minipage}\tabularnewline
\begin{minipage}[t]{0.22\columnwidth}\raggedright
Gas constant of water\strut
\end{minipage} & \begin{minipage}[t]{0.22\columnwidth}\raggedright
\(R_v\)\strut
\end{minipage} & \begin{minipage}[t]{0.22\columnwidth}\raggedright
J kg\(^{-1}\) K\(^{-1}\)\strut
\end{minipage} & \begin{minipage}[t]{0.22\columnwidth}\raggedright
461.\strut
\end{minipage}\tabularnewline
\begin{minipage}[t]{0.22\columnwidth}\raggedright
Density of liquid water\strut
\end{minipage} & \begin{minipage}[t]{0.22\columnwidth}\raggedright
\(d_{H_2O}\)\strut
\end{minipage} & \begin{minipage}[t]{0.22\columnwidth}\raggedright
J kg\(^{-1}\) K\(^{-1}\)\strut
\end{minipage} & \begin{minipage}[t]{0.22\columnwidth}\raggedright
1000.\strut
\end{minipage}\tabularnewline
\begin{minipage}[t]{0.22\columnwidth}\raggedright
\strut
\end{minipage} & \begin{minipage}[t]{0.22\columnwidth}\raggedright
\(e^*\)(273K)\strut
\end{minipage} & \begin{minipage}[t]{0.22\columnwidth}\raggedright
Pa.\strut
\end{minipage} & \begin{minipage}[t]{0.22\columnwidth}\raggedright
611\strut
\end{minipage}\tabularnewline
\begin{minipage}[t]{0.22\columnwidth}\raggedright
Stefan Bolzman\strut
\end{minipage} & \begin{minipage}[t]{0.22\columnwidth}\raggedright
\(\sigma_{SB}\)\strut
\end{minipage} & \begin{minipage}[t]{0.22\columnwidth}\raggedright
W m\(^{-2}\) K\(^{-4}\)\strut
\end{minipage} & \begin{minipage}[t]{0.22\columnwidth}\raggedright
5.67\strut
\end{minipage}\tabularnewline
\begin{minipage}[t]{0.22\columnwidth}\raggedright
Kárman Constant\strut
\end{minipage} & \begin{minipage}[t]{0.22\columnwidth}\raggedright
\(k\)\strut
\end{minipage} & \begin{minipage}[t]{0.22\columnwidth}\raggedright
\strut
\end{minipage} & \begin{minipage}[t]{0.22\columnwidth}\raggedright
0.4\strut
\end{minipage}\tabularnewline
\begin{minipage}[t]{0.22\columnwidth}\raggedright
Latent heat of ice melting\strut
\end{minipage} & \begin{minipage}[t]{0.22\columnwidth}\raggedright
\(L_M\)\strut
\end{minipage} & \begin{minipage}[t]{0.22\columnwidth}\raggedright
J kg\(^{-1}\)\strut
\end{minipage} & \begin{minipage}[t]{0.22\columnwidth}\raggedright
3.4 \(\times 10^5\)\strut
\end{minipage}\tabularnewline
\begin{minipage}[t]{0.22\columnwidth}\raggedright
Water Freezing Point\strut
\end{minipage} & \begin{minipage}[t]{0.22\columnwidth}\raggedright
\(T_M\)\strut
\end{minipage} & \begin{minipage}[t]{0.22\columnwidth}\raggedright
K\strut
\end{minipage} & \begin{minipage}[t]{0.22\columnwidth}\raggedright
273.15\strut
\end{minipage}\tabularnewline
\begin{minipage}[t]{0.22\columnwidth}\raggedright
Constant pressure specific heat of water\strut
\end{minipage} & \begin{minipage}[t]{0.22\columnwidth}\raggedright
\(C_w\)\strut
\end{minipage} & \begin{minipage}[t]{0.22\columnwidth}\raggedright
J kg\(^{-1}\)\strut
\end{minipage} & \begin{minipage}[t]{0.22\columnwidth}\raggedright
4,200.\strut
\end{minipage}\tabularnewline
\begin{minipage}[t]{0.22\columnwidth}\raggedright
The freezing point of seawater\strut
\end{minipage} & \begin{minipage}[t]{0.22\columnwidth}\raggedright
\(T_I\)\strut
\end{minipage} & \begin{minipage}[t]{0.22\columnwidth}\raggedright
K\strut
\end{minipage} & \begin{minipage}[t]{0.22\columnwidth}\raggedright
271.35\strut
\end{minipage}\tabularnewline
\begin{minipage}[t]{0.22\columnwidth}\raggedright
Specific heat ratio of ice at constant pressure\strut
\end{minipage} & \begin{minipage}[t]{0.22\columnwidth}\raggedright
\(C_I = C_w - L_M/T_M\)\strut
\end{minipage} & \begin{minipage}[t]{0.22\columnwidth}\raggedright
\strut
\end{minipage} & \begin{minipage}[t]{0.22\columnwidth}\raggedright
2397.\strut
\end{minipage}\tabularnewline
\begin{minipage}[t]{0.22\columnwidth}\raggedright
water vapor molecular weight ratio\strut
\end{minipage} & \begin{minipage}[t]{0.22\columnwidth}\raggedright
\(\epsilon = R/R_v\)\strut
\end{minipage} & \begin{minipage}[t]{0.22\columnwidth}\raggedright
\strut
\end{minipage} & \begin{minipage}[t]{0.22\columnwidth}\raggedright
0.622\strut
\end{minipage}\tabularnewline
\begin{minipage}[t]{0.22\columnwidth}\raggedright
coefficient of provisional temperature\strut
\end{minipage} & \begin{minipage}[t]{0.22\columnwidth}\raggedright
\(\epsilon_v = \epsilon^{-1} - 1\)\strut
\end{minipage} & \begin{minipage}[t]{0.22\columnwidth}\raggedright
\strut
\end{minipage} & \begin{minipage}[t]{0.22\columnwidth}\raggedright
0.606\strut
\end{minipage}\tabularnewline
\begin{minipage}[t]{0.22\columnwidth}\raggedright
Ratio of specific heat to gas constant\strut
\end{minipage} & \begin{minipage}[t]{0.22\columnwidth}\raggedright
\(\kappa = R/C_p\)\strut
\end{minipage} & \begin{minipage}[t]{0.22\columnwidth}\raggedright
\strut
\end{minipage} & \begin{minipage}[t]{0.22\columnwidth}\raggedright
0.286\strut
\end{minipage}\tabularnewline
\bottomrule
\end{longtable}
