\hypertarget{gravitational-wave-resistance}{%
\subsection{Gravitational wave
resistance}\label{gravitational-wave-resistance}}

\hypertarget{gravitational-wave-resistance-scheme-overview}{%
\subsubsection{Gravitational Wave Resistance Scheme
Overview}\label{gravitational-wave-resistance-scheme-overview}}

The gravitational wave resistance scheme is, Excited by sub-grid scale
terrain The upward momentum flux of the gravitational wave is
represented, The horizontal wind deceleration associated with its
convergence is calculated. The main input data are: east-west wind
\(u\), north-south wind \(v\), temperature \(T\), and is, The output
data is the time rate of change for the east-west and north-south winds,
\(\partial u/\partial t, \partial v/\partial t\), is.

The outline of the calculation procedure is as follows.

\begin{enumerate}
\def\labelenumi{\arabic{enumi}.}
\item
  the momentum flux at the ground surface. Dispersion of surface
  altitude, The horizontal wind speed at the lowest level,
  stratification stability, etc.
\item
  consider the upward propagation of gravitational waves with momentum
  fluxes. Momentum fluxes are determined from the critical fluid number
  If the critical flux is exceeded, Suppose a breaking wave occurs and
  the flux is at its critical value.
\item
  according to the convergence of the momentum flux at each layer.
  Calculate the time variation of the horizontal wind.
\end{enumerate}

\hypertarget{relationship-between-local-fluid-number-and-momentum-flux}{%
\subsubsection{Relationship between local fluid number and momentum
flux}\label{relationship-between-local-fluid-number-and-momentum-flux}}

The gravitational waves of surface origin Given the vertical flux of
horizontal momentum, Fluxes at a certain altitude \(\tau\) and Local
Fluid Number \(F_L = NH/U\) and ,

\begin{eqnarray}
   F_L = \left(
            \frac{\tau N}{E_f \rho U^3}
           \right)^{1/2} \; ,
\end{eqnarray}

\begin{quote}
\protect\hypertarget{p-grav:tau-fl}{}{\textbackslash fl.com{[}p-grav:tau-fl{]}}
\end{quote}

The relationship between the two is valid. Here,
\(N = g/\theta \partial \theta/\partial z\) is Brandt Vaisala Frequency,
\(\rho\) is the density of the atmosphere, \(U\) corresponds to the wind
speed and \(E_f\) corresponds to the horizontal scale of the ripples on
the ground surface It is a proportional constant . Now..,

\begin{eqnarray}
  \tau = \frac{E_f F_L^2 \rho U^3}{N}
\end{eqnarray}

\begin{quote}
\protect\hypertarget{p-grav:fl-tau}{}{fl-tau{[}p-grav:fl-tau{]}}
\end{quote}

Local Fluid Number \(F_L\) is , Assume that a certain value, the
critical fluid number \(F_{c}\), cannot be exceeded. Calculated from
(\protect\hyperlink{p-grav:tau-fl}{£l;p-grav:tau-fl{]}} used as an
example of If the local fluid number exceeds the critical fluid number
\(F_{c}\) The gravitational waves are supersaturated, Up to the momentum
flux corresponding to the critical fluid number Flux decreases.

\hypertarget{momentum-fluxes-at-the-surface.}{%
\subsubsection{Momentum fluxes at the
surface.}\label{momentum-fluxes-at-the-surface.}}

due to gravitational waves excited at the earth's surface. The magnitude
of the vertical flux of horizontal momentum \(\tau_{1/2}\) is , except
for the local fluid number at the surface \((F_L)_{1/2} = N_1 h/U_1\).
Substitute into (\protect\hyperlink{p-grav:fl-tau}{p-grav:fl-tau{]}} By
,

\begin{eqnarray}
  \tau_{1/2} = E_f h^2 \rho_1 N_1 U_1 \; ,
\end{eqnarray}

It is estimated that . Here, The
\(U_1 = |{\mathbf v}_1| = (u_1^2 + v_1^2)^{1/2}\) has a surface wind
speed of , \(N_1, \rho_1\) are the two most common types of data in the
atmosphere near the earth's surface. It is stability and density. \(h\)
is an indicator of the sub-grid surface elevation change, Assume that
the standard deviation of the surface elevation is equal to \(Z_{SD}\).

Here, the local fluid number at the surface If
\((F_L)_{1/2} = N_1 Z_{SD}/U_1\) is the critical fluid number When you
exceed the \(F_c\), Momentum fluxes are defined by \(F_c\)
({[}p-grav\protect\hyperlink{p-ux5cux2520grav:fl-tau}{p-grav:fl-tau{]}}))
to the value substituted for Let's say it can be contained. Namely,

\begin{eqnarray}
  \tau_{1/2} = \min \left(
                   E_f Z_{SD}^{2} \rho_1 N_1 U_1, \; 
                  \frac{E_f F_c^{2} \rho_1 U_1^3}{N_1}
               \right)
\end{eqnarray}

\hypertarget{momentum-fluxes-in-the-upper-levels.}{%
\subsubsection{Momentum fluxes in the upper
levels.}\label{momentum-fluxes-in-the-upper-levels.}}

The momentum flux at level \(k-1/2\) is Suppose we are required to.
\(\tau_{k+1/2}\), when no saturation occurs Equal to \(\tau_{k-1/2}\).
This momentum flux, \(\tau_{k-1/2}\), is , Momentum fluxes calculated
from the critical fluid number at the \(k+1/2\) level In the case of a
wave breaking event that exceeds The momentum flux decreases to the flux
corresponding to the criticality.

\begin{eqnarray}
  \tau_{k+1/2} = \min \left( 
               \tau_{k-1/2}, \;
               \frac{E_f F_c^2 \rho_{k+1/2} U_{k+1/2}^3}{N_{k+1/2}}
                      \right),
\end{eqnarray}

However, the \(U_{k+1/2}\), . of the wind velocity vector at each layer,
It is the magnitude of the projective component of the lowest level with
respect to the direction of the horizontal wind,

\begin{eqnarray}
  U_{k+1/2} = \frac{{\mathbf v}_{k+1/2} 
                      \cdot {\mathbf v}_{1}}
                   {|{\mathbf v}_{1}|       }
\end{eqnarray}

\hypertarget{the-magnitude-of-the-time-variation-of-horizontal-wind-due-to-momentum-convergence.}{%
\subsubsection{The magnitude of the time variation of horizontal wind
due to momentum
convergence.}\label{the-magnitude-of-the-time-variation-of-horizontal-wind-due-to-momentum-convergence.}}

The temporal rate of change of the projective component of the
horizontal wind, \(U_{k}\), is

\begin{eqnarray}
  \frac{\partial U}{\partial t} 
        = - \frac{1}{\rho} \frac{\partial \tau}{\partial z}
        = g  \frac{\partial \tau}{\partial p}
\end{eqnarray}

as determined by i.e.~,

\begin{eqnarray}
  \frac{\partial U_{k}}{\partial t} 
        =  g  \frac{\tau_{k+1/2} - \tau{k-1/2}}{\Delta p}.
\end{eqnarray}

With this , The rate of change of the east-west and north-south winds
over time is calculated as follows.

\begin{eqnarray}
  \frac{\partial u_{k}}{\partial t}  = 
           \frac{\partial U_{k}}{\partial t} \frac{u_{1}}{U_{1}} \\
  \frac{\partial v_{k}}{\partial t}  = 
           \frac{\partial U_{k}}{\partial t} \frac{v_{1}}{U_{1}}
\end{eqnarray}

\hypertarget{other-notes.}{%
\subsubsection{Other Notes.}\label{other-notes.}}

\begin{enumerate}
\def\labelenumi{\arabic{enumi}.}
\tightlist
\item
  when the wind speed at the lowest level is small
  (\(U_{1} \le v_{min}\)) and In the case of small undulations in the
  earth's surface (\(Z_{SD} \le (Z_{SD})_{min}\)), Assuming no
  gravitational waves are excited at the earth's surface.
\end{enumerate}
