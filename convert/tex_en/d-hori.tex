\hypertarget{horizontal-discretization}{%
\subsection{Horizontal discretization}\label{horizontal-discretization}}

The horizontal discretization of the The spectral transformation method
is used (Bourke, 1988). The differential terms for longitude and
latitude are evaluated by orthogonal function expansion, On the other
hand, non-linear terms are computed on the grid points.

\hypertarget{spectral-expansion.}{%
\subsubsection{Spectral Expansion.}\label{spectral-expansion.}}

As an expansion function system, it is a Laplacian eigenfunction system
on a sphere The spherical harmonic function \(Y_n^m(\lambda,\mu)\) are
used. with \(\mu \equiv \sin\varphi\). \(Y_n^m\) satisfies the following
equation,

\begin{eqnarray}
\nabla^{2}_{\sigma} Y_n^m(\lambda,\mu) 
= - \frac{n(n+1)}{a^{2}} Y_n^m(\lambda,\mu) 
\end{eqnarray}

Using the Legendre junction number \(P_n^m\) it is written as follows.

\begin{eqnarray}
Y_n^m(\lambda,\mu) = P_n^m (\mu) e^{im \lambda}
\end{eqnarray}

but \$ n \geq \textbar{} m \textbar{} \$.

The expansion by the spherical harmonic function is written as ,

\begin{eqnarray}
   {Y_n^m}_{ij} \equiv Y_n^m ( \lambda_i, \mu_j )
\end{eqnarray}

If you write ,

\begin{eqnarray}
  X_{ij} \equiv X ( \lambda_i, \mu_j )
  =  {\mathcal R}\mathbf{e} \sum_{m=-N}^{N} \sum_{n=|m|}^{N} 
        X_n^m {Y_n^m}_{ij} ,
\end{eqnarray}

\begin{quote}
\protect\hypertarget{Sphericalux20Expansion}{}{\textbackslash{}\blur[Spherical Expansion]}
\end{quote}

The inverse of that is ,

\begin{eqnarray}
  X_n^m 
         =  \frac{1}{4 \pi} 
             \int_{-1}^{1} d \mu \int_{0}^{\pi} d \lambda 
               X( \lambda, \mu ) Y_n^{m *} ( \lambda, \mu ) \\
         =  \frac{1}{I} \sum_{i=1}^{I} \sum_{j=1}^{J}  
               X_{ij} {Y_n^{m*}}_{ij} w_j 
\end{eqnarray}

\begin{quote}
\protect\hypertarget{Deploymentux20Factor}{}{\textbackslash.com{[}Deployment
Factor{]}}
\end{quote}

. When evaluating by replacing the integral with the sum, See Gauss's
trapezoidal formula for the \(\lambda\) integral, We use the
Gauss-Legendre integral formula for the \(\mu\) integral. \(\mu_j\) is
the Gauss latitude and \(w_j\) is the Gauss load. Also, \(\lambda_i\) is
a grid of evenly spaced Gauss loads.

Using the spectral expansion, we can obtain a new formula for
Gauss-Legendre integration, The grid point values for the terms
containing the derivatives are found as follows.

\begin{eqnarray}
        \left(  \frac{\partial X}{\partial \lambda} \right)_{ij}
     =  
        {\mathcal R}\mathbf{e} \sum_{m=-N}^{N} \sum_{n=|m|}^{N} 
       im X_n^m {Y_n^m}_{ij}
\end{eqnarray}

\begin{quote}
\protect\hypertarget{barometricux20pressureux20x}{}{\a[barometric pressure x]}.
\end{quote}

\begin{eqnarray}
   \left( \cos\varphi \frac{\partial X}{\partial \varphi} \right)_{ij}
     =  {\mathcal R}\mathbf{e} \sum_{m=-N}^{N} \sum_{n=|m|}^{N} 
       X_n^m 
       ( 1-\mu^{2} ) \frac{\partial }{\partial \mu} {Y_n^m}_{ij}
\end{eqnarray}

\begin{quote}
\protect\hypertarget{barometricux20y}{}{\blaze[barometric y]}.
\end{quote}

Furthermore, From the spectral components of \(\zeta\) and \(D\), The
grid point values for \(u,v\) are obtained as follows.

\begin{eqnarray}
  u_{ij}
  = \frac{1}{\cos\varphi}
     {\mathcal R}\mathbf{e} \sum_{m=-N}^{N} 
                       \sum_{\stackrel{n=|m|}{n \neq 0}}^{N} 
    \left\{
             \frac{a}{n(n+1)} \zeta_n^m 
            (1-\mu^{2}) \frac{\partial }{\partial \mu} {Y_n^m}_{ij}
          -  \frac{im a}{n(n+1)} D_n^m {Y_n^m}_{ij}
    \right\}
\end{eqnarray}

\begin{quote}
\protect\hypertarget{Seekingux20U}{}{\textbackslash{[}Seeking U{]}}.
\end{quote}

\begin{eqnarray}
  v_{ij}
  = \frac{1}{\cos\varphi}
   {\mathcal R}\mathbf{e} \sum_{m=-N}^{N}
                     \sum_{\stackrel{n=|m|}{n \neq 0}}^{N}
    \left\{
          -  \frac{im a}{n(n+1)} \zeta_n^m {Y_n^m}_{ij}
          -  \frac{a}{n(n+1)} D_n^m 
            (1-\mu^{2}) \frac{\partial }{\partial \mu} {Y_n^m}_{ij}
    \right\}
\end{eqnarray}

\begin{quote}
\protect\hypertarget{Seekingux20V}{}{\textbackslash{[}V seeking{]}}
\end{quote}

The derivative that appears in the advection term of the equation is, It
is required as follows.

\begin{quote}
\protect\hypertarget{Aux20integral}{}{{[}A integral{]}} \begin{eqnarray}
\left( \frac{1}{a\cos\varphi} \frac{\partial A}{\partial \lambda} \right)_n^m 
=  \frac{1}{4 \pi} 
\int_{-1}^{1} d \mu \int_{0}^{\pi} d \lambda 
\frac{1}{a\cos\varphi} \frac{\partial A}{\partial \lambda} Y_n^{m *} \\
=  \frac{1}{4 \pi} 
\int_{-1}^{1} d \mu \int_{0}^{\pi} d \lambda \,
im A \cos\varphi \frac{1}{a(1-\mu^{2})} Y_n^{m *} \\
=  \frac{1}{I} \sum_{i=1}^{I} \sum_{j=1}^{J}  
im A_{ij} \cos\varphi_j
{Y_n^{m *}}_{ij} \frac{w_j}{a(1-\mu_j^{2})} 
\end{eqnarray}
\end{quote}

\begin{quote}
\protect\hypertarget{Bux20integral}{}{B integral\blazer\blazer{]}}. \begin{eqnarray}
\left( \frac{1}{a\cos\varphi} 
\frac{\partial }{\partial \varphi} (A\cos\varphi) \right)_n^m 
=  \frac{1}{4 \pi a} 
\int_{-1}^{1} d \mu \int_{0}^{\pi} d \lambda 
\frac{\partial }{\partial \mu} (A\cos\varphi) Y_n^{m *}  \\
=  - \frac{1}{4 \pi a} 
\int_{-1}^{1} d \mu \int_{0}^{\pi} d \lambda 
A \cos\varphi \frac{\partial }{\partial \mu} Y_n^{m *}
\\
=  - \frac{1}{I} \sum_{i=1}^{I} \sum_{j=1}^{J}  
A_{ij}  \cos\varphi_j
(1-\mu_j^2)  \frac{\partial }{\partial \mu} 
{Y_n^{m *}}_{ij} \frac{w_j}{a(1-\mu_j^{2})} 
\end{eqnarray}
\end{quote}

Further ,

\begin{eqnarray}
     \left( \nabla^{2}_{\sigma} X \right)_n^m
       =  - \frac{n(n+1)}{a^{2}} X_n^m
\end{eqnarray}

to evaluate the term \(\nabla^2\).

\hypertarget{horizontal-diffusion-term.}{%
\subsubsection{Horizontal Diffusion
Term.}\label{horizontal-diffusion-term.}}

The horizontal diffusion term is entered in the form \(\nabla^{N_D}\) as
follows.

\begin{eqnarray}
  {\mathcal D}(\zeta) = K_{MH} 
                      \left[ (-1)^{N_D/2} \nabla^{N_D}
                              - \left( \frac{2}{a^2} \right)^{N_D/2} 
                      \right]
                    \zeta ,
\end{eqnarray}

\begin{quote}
\protect\hypertarget{Horizontalux20Diffusion}{}{Regular{[}Horizontal
Diffusion{]}}.
\end{quote}

\begin{eqnarray}
     {\mathcal D}(D) = K_{MH} 
                      \left[ (-1)^{N_D/2} \nabla^{N_D}
                              - \left( \frac{2}{a^2} \right)^{N_D/2} 
                      \right]
                    D ,
\end{eqnarray}

\begin{eqnarray}
    {\mathcal D}(T) = (-1)^{N_D/2} K_{HH} \nabla^{N_D} T ,
\end{eqnarray}

\begin{eqnarray}
    {\mathcal D}(q) = (-1)^{N_D/2} K_{EH} \nabla^{N_D} q .
\end{eqnarray}

This horizontal diffusion term has strong implications for computational
stability. In order to represent selective horizontal diffusion on small
scales, For \(N_D\), 4 \(\sim\) 16 is used. The extra terms on the
diffusion of vorticity and divergence are It represents that the term
for rigid body rotation in \(n=1\) does not decay.

\hypertarget{spectral-representation-of-the-equation}{%
\subsubsection{Spectral representation of the
equation}\label{spectral-representation-of-the-equation}}

\begin{enumerate}
\def\labelenumi{\arabic{enumi}.}
\tightlist
\item
  a series of equations
\end{enumerate}

\begin{eqnarray}
  \frac{\partial \pi_m^m}{\partial t}
  =  - \sum_{k=1}^{K} (D_n^m)_k \Delta  \sigma_k  \\
     + \frac{1}{I} \sum_{i=1}^{I} \sum_{j=1}^{J}  
               Z_{ij} {Y_n^{m *}}_{ij} w_j  ,
\end{eqnarray}

\begin{verbatim}
Here,
\end{verbatim}

\begin{eqnarray}
Z \equiv - \sum_{k=1}^{K} \mathbf{v}_k \cdot \nabla \pi .
\end{eqnarray}

\begin{enumerate}
\def\labelenumi{\arabic{enumi}.}
\setcounter{enumi}{1}
\tightlist
\item
  equation of motion
\end{enumerate}

\begin{eqnarray}
  \frac{\partial \zeta_n^m}{\partial t} 
   =  \frac{1}{I} \sum_{i=1}^{I} \sum_{j=1}^{J}  
          im (A_v)_{ij} \cos\varphi_j
          {Y_n^{m *}}_{ij}
         \frac{w_j}{a(1-\mu_j^{2})} 
          \\
   +    \frac{1}{I} \sum_{i=1}^{I} \sum_{j=1}^{J}  
          (A_u)_{ij} \cos\varphi_j
          (1-\mu_j^2) 
          \frac{\partial }{\partial \mu} {Y_n^{m *}}_{ij}
          \frac{w_j}{a(1-\mu_j^{2})} 
          \\ 
   -   ({\mathcal D}_M)_n^m \zeta_n^m  \; ,
\end{eqnarray}

\begin{eqnarray}
  \frac{\partial \tilde{D}_n^m}{\partial t} 
   =  \frac{1}{I} \sum_{i=1}^{I} \sum_{j=1}^{J}  
          im (A_u)_{ij} \cos\varphi_j
          {Y_n^{m *}}_{ij}
         \frac{w_j}{a(1-\mu_j^{2})} 
          \\
   -    \frac{1}{I} \sum_{i=1}^{I} \sum_{j=1}^{J}  
          (A_v)_{ij} \cos\varphi_j
          (1-\mu_j^2) 
          \frac{\partial }{\partial \mu} {Y_n^{m *}}_{ij}
          \frac{w_j}{a(1-\mu_j^{2})} 
          \\
   -   \frac{n(n+1)}{a^{2}} 
         \frac{1}{I} \sum_{i=1}^{I} \sum_{j=1}^{J}  
          E_{ij} {Y_n^{m *}}_{ij} w_j
          \\ 
   +   \frac{n(n+1)}{a^{2}} 
          ( \Phi_n^m + C_{p} \hat{\kappa}_k \bar{T}_k \pi_n^m ) 
          -  ({\mathcal D}_M)_n^m D_n^m  ,
\end{eqnarray}

\begin{verbatim}
However,
\end{verbatim}

\begin{eqnarray}
({\mathcal D}_M)_n^m = K_{MH} \left[ 
                            \left( \frac{n(n+1)}{a^{2}} \right)^{N_D/2}
                            - \left( \frac{2}{a^2} \right)^{N_D/2}
                            \right]  .
\end{eqnarray}

\begin{enumerate}
\def\labelenumi{\arabic{enumi}.}
\setcounter{enumi}{2}
\tightlist
\item
  thermodynamic equation
\end{enumerate}

\begin{eqnarray}
  \frac{\partial T_n^m}{\partial t}
   =  - \frac{1}{I} \sum_{i=1}^{I} \sum_{j=1}^{J}  
          im u_{ij} T'_{ij} \cos\varphi_j
          {Y_n^{m *}}_{ij}
         \frac{w_j}{a(1-\mu_j^{2})} 
          \\
     + \frac{1}{I} \sum_{i=1}^{I} \sum_{j=1}^{J}  
          v_{ij} T'_{ij} \cos\varphi_j
          (1-\mu_j^2) 
          \frac{\partial }{\partial \mu} {Y_n^{m *}}_{ij}
          \frac{w_j}{a(1-\mu_j^{2})} 
          \\
     + \frac{1}{I} \sum_{i=1}^{I} \sum_{j=1}^{J}  
          \left( H_{ij} + \frac{Q_{ij}+Q_{diff}}{C_{p}} \right)
          {Y_n^{m *}}_{ij} w_j
          \\ 
     - (\tilde{\mathcal D}_H)_n^m T_n^m \; ,
\end{eqnarray}

\begin{verbatim}
However,
\end{verbatim}

\begin{eqnarray}
({\mathcal D}_H)_n^m 
   =  K_{HH} \left( \frac{n(n+1)}{a^{2}} \right)^{N_D/2} .
\end{eqnarray}

\begin{enumerate}
\def\labelenumi{\arabic{enumi}.}
\setcounter{enumi}{3}
\tightlist
\item
  water vapor formula
\end{enumerate}

\begin{eqnarray}
  \frac{\partial q_n^m}{\partial t}
   =  - \frac{1}{I} \sum_{i=1}^{I} \sum_{j=1}^{J}  
          im u_{ij} q_{ij} \cos\varphi_j
          {Y_n^{m *}}_{ij} \frac{w_j}{a(1-\mu_j^{2})} 
          \\
     + \frac{1}{I} \sum_{i=1}^{I} \sum_{j=1}^{J}  
          v_{ij} q_{ij} \cos\varphi_j
          (1-\mu_j^2) 
          \frac{\partial }{\partial \mu} {Y_n^{m *}}_{ij}
          \frac{w_j}{a(1-\mu_j^{2})} 
          \\
     + \frac{1}{I} \sum_{i=1}^{I} \sum_{j=1}^{J}  
          \left( \hat{R}_{ij} + S_{q,ij} \right)
          {Y_n^{m *}}_{ij} w_j
          \\ 
     + ({\mathcal D}_H)_n^m q_n^m
\end{eqnarray}

\begin{verbatim}
However,
\end{verbatim}

\begin{eqnarray}
({\mathcal D}_E)_n^m 
   =  K_{EH} \left( \frac{n(n+1)}{a^{2}} \right)^{N_D/2} .
\end{eqnarray}
