\hypertarget{summary-of-the-mechanics-part}{%
\subsection{Summary of the mechanics
part}\label{summary-of-the-mechanics-part}}

Here, we duplicate the previous description, Enumerate the calculations
performed in the mechanical part.

\hypertarget{summary-of-calculations-in-the-mechanics-part.}{%
\subsubsection{Summary of calculations in the mechanics
part.}\label{summary-of-calculations-in-the-mechanics-part.}}

The mechanical processes are calculated in the following order.

\begin{enumerate}
\def\labelenumi{\arabic{enumi}.}
\item
  the transformation of horizontal wind into vorticity and divergence
  \texttt{MODULE:{[}UV2VDG(dvect){]}}
\item
  calculation of pseudotemperature \texttt{MODULE:{[}VIRTMD(dvtmp){]}}
\item
  calculation of the barometric gradient term
  \texttt{MODULE:{[}HGRAD(dvect){]}}
\item
  diagnostic calculation of vertical flow
  \texttt{MODULE:{[}GRDDYN/PSDOT(dgdyn){]}}
\item
  time change term due to advection \texttt{MODULE:{[}GRDDYN(dgdyn){]}}
\item
  convert the predictive variable to a spectrum
  \texttt{MODULE:{[}GD2WD(dg2wd){]}}
\item
  convert the time-varying term into a spectrum
  \texttt{MODULE:{[}TENG2W(dg2wd){]}}
\item
  time integration of spectral values
  \texttt{MODULE:{[}TINTGR(dintg){]}}
\item
  convert the predictive variables to grid values
  \texttt{MODULE:{[}GENGD(dgeng){]}}
\item
  pseudo etc. \(p\) plane spreading correction
  \texttt{MODULE:{[}CORDIF(ddifc){]}}
\item
  consideration of frictional heat by diffusion
  \texttt{MODULE:{[}CORFRC(ddifc){]}}
\item
  correction for conservation of mass
  \texttt{MODULE:{[}MASFIX(dmfix){]}}
\item
  (physical process) \texttt{MODULE:{[}PHYSCS(padmn){]}}
\item
  (time filter) \texttt{MODULE:{[}TFILT(aadvn){]}}
\end{enumerate}

\hypertarget{conversion-of-horizontal-wind-to-vorticity-and-divergence}{%
\subsubsection{Conversion of Horizontal Wind to Vorticity and
Divergence}\label{conversion-of-horizontal-wind-to-vorticity-and-divergence}}

Grid point values for horizontal wind \(u_{ij}, v_{ij}\) from the grid
point values of vorticity and divergence \(\zeta_{ij}, D_{ij}\). First,
the spectra of vorticity and divergence Ask for \(\zeta_n^m, D_n^m\),

\begin{eqnarray}
\zeta_n^m  =  \frac{1}{I} \sum_{i=1}^{I} \sum_{j=1}^{J}  
                  im v_{ij} \cos\varphi_j {Y_n^{m*}}_{ij}
                \frac{w_j}{a(1-\mu_j^{2})} 
                 \\
           +    \frac{1}{I} \sum_{i=1}^{I} \sum_{j=1}^{J}  
                     u_{ij} \cos\varphi_j (1-\mu_j^2) 
                  \frac{\partial }{\partial \mu} {Y_n^{m*}}_{ij}
                 \frac{w_j}{a(1-\mu_j^{2})} \; ,
\end{eqnarray}

\begin{quote}
\protect\hypertarget{d-summ:uv-zeta}{}{ }
\end{quote}

\begin{eqnarray}
    D_n^m  =  \frac{1}{I} \sum_{i=1}^{I} \sum_{j=1}^{J}  
                  im u_{ij} \cos\varphi_j {Y_n^{m*}}_{ij}
                \frac{w_j}{a(1-\mu_j^{2})} 
                 \\
           -    \frac{1}{I} \sum_{i=1}^{I} \sum_{j=1}^{J}  
                  v_{ij} \cos\varphi_j  (1-\mu_j^2) 
                  \frac{\partial }{\partial \mu} {Y_n^{m*}}_{ij}
                 \frac{w_j}{a(1-\mu_j^{2})} ; .
\end{eqnarray}

\begin{quote}
\protect\hypertarget{d-summ:uv-D}{}{{[}d-summ:uv-D\textbackslash{[}d-summ:uv-D{]}}.
\end{quote}

And more,

\begin{eqnarray}
  \zeta_{ij} 
   =  {\mathcal R}\mathbf{e} \sum_{m=-N}^{N} \sum_{n=|m|}^{N} 
      \zeta_n^m  {Y_n^m}_{ij} \; ,
\end{eqnarray}

and so on.

\hypertarget{calculating-pseudotemperature}{%
\subsubsection{Calculating
Pseudotemperature}\label{calculating-pseudotemperature}}

Provisional Temperature \(T_v\) is ,

\begin{eqnarray}
  T_v = T ( 1 + \epsilon_v q - l ) \; ,
\end{eqnarray}

However, it is \(\epsilon_v = R_v/R - 1\), \(R_v\) is a gas constant for
water vapor (461 Jkg\(^{-1}\)K\(^{-1}\)) \(R\) is a gas constant of air
(287.04 Jkg\(^{-1}\)K\(^{-1}\)) .

\hypertarget{calculating-the-barometric-gradient-term}{%
\subsubsection{Calculating the Barometric gradient
term}\label{calculating-the-barometric-gradient-term}}

The barometric gradient term \(\nabla \pi = \frac{1}{p_S} \nabla p_S\)
is , First, we need to get the \(\pi_n^m\)

\begin{eqnarray}
  \pi_n^m  =  \frac{1}{I} \sum_{i=1}^{I} \sum_{j=1}^{J}  
               (\ln {p_S})_{ij} {Y_n^{m *}}_{ij}  w_j \; ,
\end{eqnarray}

to a spectral representation and then ,

\begin{eqnarray}
   \frac{1}{a \cos \varphi} 
   \left( \frac{\partial \pi}{\partial \lambda} \right)_{ij}
     = 
   \frac{1}{a \cos \varphi} 
        {\mathcal R}\mathbf{e} \sum_{m=-N}^{N} \sum_{n=|m|}^{N} 
       im \tilde{X}_n^m {Y_n^m}_{ij}  \; ,
\end{eqnarray}

\begin{eqnarray}
   \frac{1}{a}
   \left( \frac{\partial \pi}{\partial \varphi} \right)_{ij}
     =  
   \frac{1}{a \cos \varphi} 
       {\mathcal R}\mathbf{e} \sum_{m=-N}^{N} \sum_{n=|m|}^{N} 
       \pi_n^m 
       ( 1-\mu^{2} ) \frac{\partial }{\partial \mu} {Y_n^m}_{ij}  \; .
\end{eqnarray}

\hypertarget{diagnostic-calculations-of-vertical-flow.}{%
\subsubsection{Diagnostic calculations of vertical
flow.}\label{diagnostic-calculations-of-vertical-flow.}}

Barometric pressure change term, and lead DC,

\begin{eqnarray}
  \frac{\partial \pi}{\partial t}
 = - \sum_{k=1}^{K} ( D_k + \mathbf{v}_k \cdot \nabla \pi ) 
       \Delta  \sigma_k
\end{eqnarray}

\begin{eqnarray}
  \dot{\sigma}_{k-1/2}
 = - \sigma_{k-1/2} \frac{\partial \pi}{\partial t}
   - \sum_{l=k}^{K} ( D_l + \mathbf{v}_l \cdot \nabla \pi )          
       \Delta  \sigma_l
\end{eqnarray}

and its non-gravity component.

\begin{eqnarray}
  \left( \frac{\partial \pi}{\partial t} \right)^{NG}
   =   - \sum_{k=1}^{K} \mathbf{v}_{k} \cdot \nabla \pi  
       \Delta  \sigma_{k}  \\
\end{eqnarray}

\begin{eqnarray}
  \dot{\sigma}^{NG}_{k-1/2}
 = - \sigma_{k-1/2} \left( \frac{\partial \pi}{\partial t} \right)^{NG}
   - \sum_{l=k}^{K} \mathbf{v}_{l} \cdot \nabla \pi
       \Delta  \sigma_{l}
\end{eqnarray}

\hypertarget{the-time-varying-term-due-to-advection.}{%
\subsubsection{The time-varying term due to
advection.}\label{the-time-varying-term-due-to-advection.}}

Momentum advection term:

\begin{eqnarray}
  (A_u)_k 
    =  ( \zeta_k + f ) v_k 
             - \frac{1}{2 \Delta \sigma_k} 
             [   \dot{\sigma}_{k-1/2} ( u_{k-1} - u_k   )
               + \dot{\sigma}_{k+1/2} ( u_k   - u_{k+1} ) ]
            \\
           - \frac{C_{p} \hat{\kappa}_k T_{v,k}'}{ a \cos \varphi} 
                  \frac{\partial \pi}{\partial \lambda} 
\end{eqnarray}

\begin{eqnarray}
  (A_v)_k 
    =  - ( \zeta_k + f ) u_k 
             - \frac{1}{2 \Delta \sigma_k} 
             [   \dot{\sigma}_{k-1/2} ( v_{k-1} - v_k   )
               + \dot{\sigma}_{k+1/2} ( v_k   - v_{k+1} ) ]
            \\
           - \frac{C_{p} \hat{\kappa}_k T_{v,k}'}{a} 
             \frac{\partial \pi}{\partial \varphi} 
\end{eqnarray}

\begin{eqnarray}
 \hat{E}_k    
  = \frac{1}{2} ( u^2 + v^2 ) 
    +  \sum_{k'=1}^{k} \left[  C_p \alpha_k ( T_v - T )_{k'}
                              + C_p \beta_k ( T_v - T )_{k'-1} \right]
\end{eqnarray}

Temperature Advection Term:

\begin{eqnarray}
 (u T')_k  = u_k (T_k - \bar{T} )
\end{eqnarray}

\begin{eqnarray}
 (v T')_k  = v_k (T_k - \bar{T} )
\end{eqnarray}

\begin{eqnarray}
 \hat{H}_k  =  T_{k}^{\prime} D_{k}  \\
         - \frac{1}{\Delta \sigma_{k}} 
             [   \dot{\sigma}_{k-1/2} ( \hat{T^{\prime}}_{k-1/2} 
                                         - T^{\prime}_{k}   )
               + \dot{\sigma}_{k+1/2} ( T^{\prime}_{k}  
                                         - \hat{T^{\prime}}_{k+1/2} ) ]
                \\
         - \frac{1}{\Delta \sigma_{k}} 
             [   \dot{\sigma}^{NG}_{k-1/2} ( \hat{\bar{T}}_{k-1/2} 
                                         - \bar{T}_{k}   )
               + \dot{\sigma}^{NG}_{k+1/2} ( \bar{T}_{k}  
                                         - \hat{\bar{T}}_{k+1/2} ) ]
                \\
         + \hat{\kappa}_{k} T_{v,k} \mathbf{v}_{k} \cdot \nabla \pi
                \\
         - \frac{\alpha_{k}}{\Delta \sigma_{k} } T_{v,k}
             \sum_{l=k}^{K} \mathbf{v}_{l} \cdot \nabla \pi 
               \Delta \sigma_{l}
           - \frac{\beta_{k}}{\Delta \sigma_{k} } T_{v,k}
             \sum_{l=k+1}^{K} \mathbf{v}_{l} \cdot \nabla \pi 
               \Delta \sigma_{l}
                \\
         - \frac{\alpha_{k}}{\Delta \sigma_{k} } T'_{v,k}
             \sum_{l=k}^{K} D_l  \Delta \sigma_{l}
           - \frac{\beta_{k}}{\Delta \sigma_{k} } T'_{v,k}
             \sum_{l=k+1}^{K} D_l  \Delta \sigma_{l}
\end{eqnarray}

Water Vapor Advection Term:

\begin{eqnarray}
 (u q)_k  = u_k q_k
\end{eqnarray}

\begin{eqnarray}
 (v q)_k  = v_k q_k
\end{eqnarray}

\begin{eqnarray}
R_k  =  q_k D_k 
       - \frac{1}{2 \Delta \sigma_k} 
             [   \dot{\sigma}_{k-1/2} ( q_{k-1} - q_k   )
               + \dot{\sigma}_{k+1/2} ( q_k   - q_{k+1} ) ]
\end{eqnarray}

\hypertarget{conversion-of-predictive-variables-to-spectra.}{%
\subsubsection{Conversion of Predictive Variables to
Spectra.}\label{conversion-of-predictive-variables-to-spectra.}}

(\protect\hyperlink{d-summ:uv-D}{\textbackslash{}\d\begin{eqnarray}d-summ:uv-D\end{eqnarray}}) to

\(u_{ij}^{t-\Delta t}, v_{ij}^{t-\Delta t}\). Spectral representation of
vorticity and divergence Convert to \(\zeta_n^m, D_n^m\). Furthermore ,
Temperature \(T^{t-\Delta t}\), Specific Humidity \(q^{t-\Delta t}\),
\(\pi = \ln p_S^{t-\Delta t}\).

\begin{eqnarray}
  X_n^m  =  \frac{1}{I} \sum_{i=1}^{I} \sum_{j=1}^{J}  
               X_{ij} {Y_n^{m *}}_{ij}  w_j \; ,
\end{eqnarray}

to a spectral representation.

\hypertarget{conversion-of-time-varying-terms-to-spectra.}{%
\subsubsection{Conversion of time-varying terms to
spectra.}\label{conversion-of-time-varying-terms-to-spectra.}}

Time Variation Term of Vorticity

\begin{eqnarray}
  \frac{\partial \zeta_n^m}{\partial t} 
   =  \frac{1}{I} \sum_{i=1}^{I} \sum_{j=1}^{J}  
          im (A_v)_{ij} \cos \varphi_j
          {Y_n^{m *}}_{ij}
         \frac{w_j}{a(1-\mu_j^{2})} 
          \\
   +    \frac{1}{I} \sum_{i=1}^{I} \sum_{j=1}^{J}  
          (A_u)_{ij} \cos \varphi_j
          (1-\mu_j^2) 
          \frac{\partial }{\partial \mu} {Y_n^{m *}}_{ij}
          \frac{w_j}{a(1-\mu_j^{2})} 
          \\ 
\end{eqnarray}

The non-gravity wave component of the time-varying term of the
divergence

\begin{eqnarray}
  \left( \frac{\partial D_n^m}{\partial t} \right)^{NG}
   =  \frac{1}{I} \sum_{i=1}^{I} \sum_{j=1}^{J}  
          im (A_u)_{ij} \cos \varphi_j
          {Y_n^{m *}}_{ij}
         \frac{w_j}{a(1-\mu_j^{2})} 
          \\
   -    \frac{1}{I} \sum_{i=1}^{I} \sum_{j=1}^{J}  
          (A_v)_{ij} \cos \varphi_j
          (1-\mu_j^2) 
          \frac{\partial }{\partial \mu} {Y_n^{m *}}_{ij}
          \frac{w_j}{a(1-\mu_j^{2})} 
          \\
   -   \frac{n(n+1)}{a^{2}} 
         \frac{1}{I} \sum_{i=1}^{I} \sum_{j=1}^{J}  
          \hat{E}_{ij}  {Y_n^{m *}}_{ij} w_j
          \\ 
\end{eqnarray}

The non-gravitational component of the time-varying term of temperature

\begin{eqnarray}
  \left( \frac{\partial T_n^m}{\partial t} \right)^{NG}
   =  - \frac{1}{I} \sum_{i=1}^{I} \sum_{j=1}^{J}  
          im (u T')_{ij} \cos \varphi_j
          {Y_n^{m *}}_{ij}
         \frac{w_j}{a(1-\mu_j^{2})} 
          \\
     + \frac{1}{I} \sum_{i=1}^{I} \sum_{j=1}^{J}  
          (v T')_{ij} \cos \varphi_j
          (1-\mu_j^2) 
          \frac{\partial }{\partial \mu} {Y_n^{m *}}_{ij}
          \frac{w_j}{a(1-\mu_j^{2})} 
          \\
     + \frac{1}{I} \sum_{i=1}^{I} \sum_{j=1}^{J}  
          \hat{H}_{ij} 
          {Y_n^{m *}}_{ij} w_j
\end{eqnarray}

Time-varying terms for water vapor

\begin{eqnarray}
  \frac{\partial q_n^m}{\partial t}
   =  - \frac{1}{I} \sum_{i=1}^{I} \sum_{j=1}^{J}  
          im (uq)_{ij} \cos \varphi_j
          {Y_n^{m *}}_{ij}
         \frac{w_j}{a(1-\mu_j^{2})} 
          \\
     + \frac{1}{I} \sum_{i=1}^{I} \sum_{j=1}^{J}  
          (vq)_{ij} \cos \varphi_j
          (1-\mu_j^2) 
          \frac{\partial }{\partial \mu} {Y_n^{m *}}_{ij}
          \frac{w_j}{a(1-\mu_j^{2})} 
          \\
     + \frac{1}{I} \sum_{i=1}^{I} \sum_{j=1}^{J}  
          R_{ij} 
          {Y_n^{m *}}_{ij} w_j
\end{eqnarray}

\hypertarget{spectral-value-time-integration}{%
\subsubsection{Spectral Value Time
Integration}\label{spectral-value-time-integration}}

Equations in matrix form

\begin{eqnarray}
      \left\{ ( 1+2\Delta t {\mathcal D}_H )( 1+2\Delta t {\mathcal D}_M )
           \underline{I}  
      - ( \Delta t )^{2}  ( \underline{W} \ \underline{h} 
           + (1+2\Delta t {\mathcal D}_M)
             \mathbf{G} \mathbf{C}^{T} ) \nabla^{2}_{\sigma}
  \right\}
      \overline{ \mathbf{D} }^{t} 
       \\
  = ( 1+2\Delta t {\mathcal D}_H )( 1-\Delta t {\mathcal D}_M ) 
       \mathbf{D}^{t-\Delta t}
  + \Delta t 
         \left( \frac{\partial \mathbf{D}}{\partial t} \right)_{NG}  
  \\
  -  \Delta t \nabla^{2}_{\sigma}     
                   \left\{  ( 1+2\Delta t {\mathcal D}_H ) \mathbf{\Phi}_{S} 
                          + \underline{W} 
                            \left[ ( 1-2\Delta t {\mathcal D}_H ) 
                                    \mathbf{T}^{t-\Delta t}
                                  + \Delta t 
                                      \left( \frac{\partial \mathbf{T}}
                                                  {\partial t}     
                                      \right)_{NG} \right]
                   \right.
  \\
                 \left.  \hspace*{20mm} 
                          + ( 1+2\Delta t {\mathcal D}_H ) \mathbf{G} 
                            \left[ \pi^{t-\Delta t} 
                                  + \Delta t
                                     \left( \frac{\partial \pi}
                                                 {\partial t} 
                                     \right)_{NG}  \right]
                   \right\} . 
\end{eqnarray}

By using LU decomposition to solve for Ask for \(\bar{D}\),

\begin{eqnarray}
  \frac{\partial \mathbf{T}}{\partial t} 
      =   \left( \frac{\partial \mathbf{T}}
                        {\partial t}       \right)_{NG}  
         - \underline{h} \mathbf{D}
\end{eqnarray}

\begin{eqnarray}
  \frac{\partial \pi}{\partial t} 
      =   \left( \frac{\partial \pi}
                        {\partial t}       \right)_{NG}  
         - \mathbf{C} \cdot \mathbf{D}
\end{eqnarray}

by. \(\partial \mathbf{T}/\partial t\), \(\partial \pi/\partial t\) and
calculate the value of the spectrum in \(t+\Delta t\) by finding

\begin{eqnarray}
  \zeta^{t+\Delta t}  =  \left( \zeta^{t-\Delta t}
                                +   2 \Delta t \frac{\partial \zeta}{\partial t} \right)
                          ( 1 + 2 \Delta t {\mathcal D}_M )^{-1} \\
  D^{t+\Delta t}  =  2 \bar{D} - D^{t-\Delta t}\\
  T^{t+\Delta t}  =  \left( T^{t-\Delta t}
                                +  2 \Delta t  \frac{\partial T}{\partial t} \right)
                          ( 1 + 2 \Delta t {\mathcal D}_H )^{-1} \\
  q^{t+\Delta t}  =  \left( q^{t-\Delta t}
                                +  2 \Delta t \frac{\partial q}{\partial t} \right)
                          ( 1 + 2 \Delta t {\mathcal D}_E )^{-1} \\
\pi^{t+\Delta t}  =  \pi^{t-\Delta t}
                                +  2 \Delta t \frac{\partial \pi}{\partial t}
\end{eqnarray}

\hypertarget{conversion-of-prediction-variables-to-grid-values}{%
\subsubsection{Conversion of Prediction Variables to Grid
Values}\label{conversion-of-prediction-variables-to-grid-values}}

Spectral values of vorticity and divergence from \(\zeta_n^m, D_n^m\)
Find the grid values for the horizontal wind speed \(u_{ij}, v_{ij}\).

\begin{eqnarray}
  u_{ij}
  =  \frac{1}{\cos \varphi_j}
     {\mathcal R}\mathbf{e} \sum_{m=-N}^{N} 
                       \sum_{\stackrel{n=|m|}{n \neq 0}}^{N} 
    \left\{
             \frac{a}{n(n+1)} \zeta_n^m 
            (1-\mu^{2}) \frac{\partial }{\partial \mu} {Y_n^m}_{ij}
          -  \frac{im a}{n(n+1)} D_n^m {Y_n^m}_{ij}
    \right\}
\end{eqnarray}

\begin{eqnarray}
  v_{ij}
  =  \frac{1}{\cos \varphi_j}
     {\mathcal R}\mathbf{e} \sum_{m=-N}^{N}
                       \sum_{\stackrel{n=|m|}{n \neq 0}}^{N}
    \left\{
          -  \frac{im a}{n(n+1)} \zeta_n^m  {Y_n^m}_{ij}
          -  \frac{a}{n(n+1)} \tilde{D}_n^m 
            (1-\mu^{2}) \frac{\partial }{\partial \mu} {Y_n^m}_{ij}
    \right\}
\end{eqnarray}

Furthermore ,

\begin{eqnarray}
  T_{ij} 
   =  {\mathcal R}\mathbf{e} \sum_{m=-N}^{N} \sum_{n=|m|}^{N} 
      T_n^m  {Y_n^m}_{ij} \; ,
\end{eqnarray}

\(T_{ij}, \pi_{ij}, q_{ij}\) are obtained by such methods as

\begin{eqnarray}
  {p_S}_{ij} = \exp \pi_{ij} 
\end{eqnarray}

Calculate the .

\hypertarget{pseudo-etc.-p-surface-diffusion-correction}{%
\subsubsection{\texorpdfstring{Pseudo etc. \(p\) Surface Diffusion
Correction}{Pseudo etc. p Surface Diffusion Correction}}\label{pseudo-etc.-p-surface-diffusion-correction}}

Horizontal diffusion is applied on the surface of \(\sigma\) and so on,
In large areas of mountain slopes, water vapor is transported uphill,
causing problems such as bringing false precipitation at the top of the
mountain. To mitigate it, such as close to the diffusion of the \(p\)
surface Insert a correction for \(T,q,l\).

\begin{eqnarray}
  {\mathcal D}_p (T) = (-1)^{N_D/2} K \nabla^{N_D}_p T  
                \simeq  (-1)^{N_D/2} K \nabla^{N_D}_{\sigma} T  
                      - \frac{\partial \sigma}{\partial p} 
                      (-1)^{N_D/2} K \nabla^{N_D}_{\sigma} p
                      \cdot \frac{\partial T}{\partial \sigma}                   \\
                =      (-1)^{N_D/2} K \nabla^{N_D}_{\sigma} T  
                    -  (-1)^{N_D/2} K \nabla^{N_D}_{\sigma} \pi
                          \cdot \sigma \frac{\partial T}{\partial \sigma}  \\
                =    {\mathcal D} (T) 
                    -  {\mathcal D} (\pi) 
                       \sigma \frac{\partial T}{\partial \sigma}
\end{eqnarray}

so,

\begin{eqnarray}
  T_k \leftarrow  T_k 
       -  2 \Delta t
        \sigma_{k} \frac{T_{k+1}-T_{k-1}}{\sigma_{k+1} - \sigma_{k-1}}
        {\mathcal D}(\pi)
\end{eqnarray}

And so on. \({\mathcal D}(\pi)\) is equal to the spectral value of
\(pi\) in \(\pi_n^m\) The spectral representation of the diffusion
coefficient multiplied by Converted into a grid value.

\hypertarget{consideration-of-frictional-heat-by-diffusion.}{%
\subsubsection{Consideration of frictional heat by
diffusion.}\label{consideration-of-frictional-heat-by-diffusion.}}

Frictional heat from diffusion is ,

\begin{eqnarray}
  Q_{DIF} = - \left( u_{ij} {\mathcal D}(u)_{ij} 
                   + v_{ij} {\mathcal D}(v)_{ij} \right)
\end{eqnarray}

It is estimated that Therefore,

\begin{eqnarray}
  T_k \leftarrow  T_k 
       -  \frac{2 \Delta t}{C_p}
           \left( u_{ij} {\mathcal D}(u)_{ij} 
                 + v_{ij} {\mathcal D}(v)_{ij} \right)
\end{eqnarray}

\hypertarget{correction-for-conservation-of-mass.}{%
\subsubsection{Correction for conservation of
mass.}\label{correction-for-conservation-of-mass.}}

The spectral method is not used, The global integration of the
\(\pi = \ln p_S\) is preserved except for rounding errors, The
conservation of the mass, i.e., the global integration of \(p_S\), is
not guaranteed. Also, as the spectral wavenumber expires, it is not
possible to preserve the global integration of \(p_S\), Negative values
of the grid points of water vapor are sometimes observed. For these
reasons, Let the mass of dry air and water vapor, the mass of cloud
water be preserved, Furthermore, corrections are made to remove the
negative water vapor content in the region.

First, at the beginning of the dynamics calculation,
\texttt{MODULE:{[}FIXMAS{]}} is added, Global integrals of water vapor
and cloud water are calculated for \(M_q, M_l\).

\begin{eqnarray}
  M_q^0  =  \sum_{ijk} q p_S  \Delta\lambda_i w_j \Delta\sigma_k  \\
  M_l^0  =  \sum_{ijk} l p_S  \Delta\lambda_i w_j \Delta\sigma_k 
\end{eqnarray}

In the first step of the calculation, the global integrals of Calculate
and memorize dry mass \(M_d\).

\begin{eqnarray}
  M_d^0 = \sum_{ijk} (1-q-l) p_S \Delta\lambda_i w_j \Delta\sigma_k 
\end{eqnarray}

At the end of the dynamics calculation, \texttt{MODULE:{[}MASFIX{]}},
The correction is performed as follows.

First, for the grid points with negative water vapor content, The water
vapor is distributed from the grid points directly below, Remove
negative water vapor. If this is \$q\_k \textless{} 0 \$,

\begin{eqnarray}
        q_k'      =  0          \\
        q_{k-1}'  =  q_{k-1} + \frac{\Delta p_k}{\Delta p_{k-1}} q_k
\end{eqnarray}

\begin{verbatim}
However, this is only done for $q_{k-1}' \ge 0 $.
\end{verbatim}

The value is then set to zero for the grid points not removed by the
above procedure.

\begin{enumerate}
\def\labelenumi{\arabic{enumi}.}
\setcounter{enumi}{2}
\tightlist
\item
  calculate the global integration value \(M_q\) and Make sure this
  matches the \(M_q^0\), Multiply the global water vapor content by a
  constant percentage.
\end{enumerate}

\begin{eqnarray}
        q'' = \frac{M_q^0}{M_q} q' 
\end{eqnarray}

\begin{enumerate}
\def\labelenumi{\arabic{enumi}.}
\setcounter{enumi}{3}
\tightlist
\item
  correcting the dry air mass. Similarly, calculate \(M_d\) and
\end{enumerate}

\begin{eqnarray}
        p_S'' = \frac{M_d^0}{M_d} p_S
\end{eqnarray}

\hypertarget{horizontal-diffusion-and-rayleigh-friction}{%
\subsubsection{Horizontal Diffusion and Rayleigh
Friction}\label{horizontal-diffusion-and-rayleigh-friction}}

The coefficients of horizontal diffusion can be expressed spectrally,

\begin{eqnarray}
 {{\mathcal D}_M}_n^m = K_M 
                      \left[ \left( \frac{n(n+1)}{a^2} \right)^{N_D/2} 
                                - \left( \frac{2}{a^2} \right)^{N_D/2} 
                      \right]
                  + K_R
\end{eqnarray}

\begin{eqnarray}
  {{\mathcal D}_H}_n^m = K_M \left( \frac{n(n+1)}{a^2} \right)^{N_D/2} 
\end{eqnarray}

\begin{eqnarray}
  {{\mathcal D}_E}_n^m = K_E \left( \frac{n(n+1)}{a^2} \right)^{N_D/2} 
\end{eqnarray}

The \(K_R\) is the Rayleigh coefficient of friction. The Rayleigh
coefficient of friction is

\begin{eqnarray}
  K_R = K_R^0 \left[ 1+\tanh \left( \frac{z-z_R}{H_R} \right) \right]
\end{eqnarray}

The profile is given as However,

\begin{eqnarray}
  z = - H \ln \sigma 
\end{eqnarray}

It is approximated as follows. Standard values are,
\(K_R^0 = {(30day)}^{-1}\), \(z_R = -H \ln \sigma_{top}\)
(\(\sigma_{top}\) : the top level of the model), \(H = 8000\) m,
\(H_R = 7000\) m. .

\hypertarget{time-filter.}{%
\subsubsection{Time Filter.}\label{time-filter.}}

To remove the computation mode in leap frog Applying Asselin's (1972)
time filter at every step.

\begin{eqnarray}
  \bar{T}^{t}
    = ( 1-2 \epsilon_f ) T^{t}
    +  \epsilon_f 
        \left( \bar{T}^{t-\Delta t} + T^{t+\Delta t} \right)
\end{eqnarray}

and \(\bar{T}\) are obtained. \(T^{t-\Delta t}\), which is used in the
next step of the mechanical process, is Use this \(\bar{T}^t\). For a
\(\epsilon_f\) it is standard to use 0.05.

In practice, you should use First, in the \texttt{MODULE:{[}GENGD{]}}
conversion of the predictor to a grid of values, the following variables
are used,

\begin{eqnarray}
  \bar{T}^{t*}
    = ( 1 -\epsilon_f )^{-1} 
     \left[ ( 1-2 \epsilon_f ) T^{t} + \epsilon_f \bar{T}^{t-\Delta t}
     \right]
\end{eqnarray}

and when the physical process is complete, the After determining the
value of \(T^{t+\Delta t}\), you can use \texttt{MODULE:{[}TFILT{]}} to
determine the value of the \(T^{t+\Delta t}\),

\begin{eqnarray}
 \bar{T}^{t}
    = ( 1 -\epsilon_f ) \bar{T}^{t*}  
       +  \epsilon_f \bar{T}^{t+\Delta t}
\end{eqnarray}
