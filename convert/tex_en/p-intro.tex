\hypertarget{physical-processes.}{%
\section{Physical Processes.}\label{physical-processes.}}

\hypertarget{overview-of-physical-processes.}{%
\subsection{Overview of Physical
Processes.}\label{overview-of-physical-processes.}}

As a physical process, we can consider the following

\begin{itemize}
\item
  cumulus convection process
\item
  large-scale condensation process
\item
  radiation process
\item
  vertical diffusion process
\item
  surface flux
\item
  Surface and underground processes
\item
  gravitational wave resistance
\end{itemize}

The time-varying terms of the forecast variables due to these processes
Calculate \(F_x, F_y, Q, M, S\), \(F_x, F_y, Q, M, S\) and do time
integration. In addition, in order to evaluate the atmospheric and
surface fluxes Using the surface sub-model. In the surface sub-model,
Ground Temperature \(T_g\), Ground Moisture \(W_g\), Snowpack \(W_y\),
etc. It is used as a predictor variable.

\hypertarget{fundamental-equations.}{%
\subsubsection{Fundamental Equations.}\label{fundamental-equations.}}

Equation of motion of the atmosphere in the \(\sigma\) coordinate
system, thermodynamic equation, Consider the equation for a sequence of
substances such as water vapor. The vertical fluxes of momentum, heat,
water vapor, etc. are considered, Find the time variation due to its
convergence. All vertical fluxes are positive upward.

\begin{enumerate}
\def\labelenumi{\arabic{enumi}.}
\tightlist
\item
  equation of motion
\end{enumerate}

\begin{eqnarray}
  \rho \frac{du}{dt} = \frac{\partial Fu}{\partial \sigma}
\end{eqnarray}

\begin{verbatim}
 > <span id="u-eq.orig" label="u-eq.orig" label="u-eq.orig">\\centric</span>
\end{verbatim}

\begin{eqnarray}
  \rho \frac{dv}{dt} = \frac{\partial Fv}{\partial \sigma}
\end{eqnarray}

\begin{verbatim}
 $u, v$: East-West, North-South Wind;
 $Fu, Fv$: Their vertical flux.
\end{verbatim}

\begin{enumerate}
\def\labelenumi{\arabic{enumi}.}
\setcounter{enumi}{1}
\tightlist
\item
  thermodynamic equation
\end{enumerate}

\begin{eqnarray}
  \rho \frac{dc_p T}{dt} = \frac{T}{\theta} \frac{\partial F{\theta}}{\partial \sigma} 
                     + \frac{\partial F{R}}{\partial \sigma} 
\end{eqnarray}

\begin{verbatim}
 $T$: Temperature ;
 $c_p$: Constant Pressure Specific Heat;
 $\theta=T(p/p_0)^{-R/c_p}=T(p/p_0)^{-\kappa}$: Hot Position;
 $F\theta$: Vertical Sensible Heat Flux;
 $FR$: Vertical Radiation Flux.
\end{verbatim}

If we write \(\theta'=T(p/p_s)^{-\kappa}=T\sigma^{-\kappa}\), then this
is

\begin{eqnarray}
  \rho \frac{dc_p T}{dt} = \sigma^\kappa \frac{\partial F{\theta'}}{\partial \sigma} 
                     + \frac{\partial F{R}}{\partial \sigma} 
\end{eqnarray}

As far as vertical 1D processes are concerned, Instead of the
\(\theta\), consider the \(\theta'\). For the sake of simplicity, unless
there is a risk of confusion, Write \(\theta'\) as \(\theta\).

\begin{enumerate}
\def\labelenumi{\arabic{enumi}.}
\setcounter{enumi}{2}
\tightlist
\item
  water vapor continuity formula
\end{enumerate}

\begin{eqnarray}
  \rho \frac{dq}{dt} = \frac{\partial Fq}{\partial \sigma} 
\end{eqnarray}

\begin{verbatim}
 $q$: Specific Humidity;
 $F{q}$: Vertical Steam Flux.

 ### Fundamental Equations in the Ground
\end{verbatim}

Considered in terms of \(z\) coordinates with the downward direction
positive. After all, all vertical fluxes are positive upward.

\begin{enumerate}
\def\labelenumi{\arabic{enumi}.}
\setcounter{enumi}{3}
\tightlist
\item
  thermal formula
\end{enumerate}

\begin{eqnarray}
  \frac{\partial C_g G}{\partial t} = \frac{\partial Fg}{\partial z} + Sg
\end{eqnarray}

\begin{verbatim}
 $G$: Ground Temperature; $C_g$: Constant Pressure Specific Heat;
 $F{g}$: Vertical Heat Flux;
 $Sg$; Heating term (due to phase change, etc.).
\end{verbatim}

\begin{enumerate}
\def\labelenumi{\arabic{enumi}.}
\setcounter{enumi}{4}
\tightlist
\item
  formula for ground moisture
\end{enumerate}

\begin{eqnarray}
  C_w \frac{\partial w}{\partial t} = \frac{\partial Fw}{\partial z} + Sw
\end{eqnarray}

\begin{verbatim}
 $w$: Ground Moisture;
 $F{w}$: Lead Water Flux;
 $Sw$; Sources of water (spills, etc.).
\end{verbatim}

\begin{enumerate}
\def\labelenumi{\arabic{enumi}.}
\setcounter{enumi}{5}
\tightlist
\item
  energy balance equation
\end{enumerate}

At the surface, an energy balance is established.

\begin{eqnarray}
    F{\theta} + L F{q} + F{R} - F{g} = \Delta s \; \; (\sigma=1, z=0)
\end{eqnarray}

\begin{verbatim}
 $L$: Latent Heat of Evaporation;
 $\Delta s$: Surface energy balance (due to phase change, etc.).
\end{verbatim}

\begin{enumerate}
\def\labelenumi{\arabic{enumi}.}
\setcounter{enumi}{6}
\tightlist
\item
  surface water balance
\end{enumerate}

\begin{eqnarray}
  Pg + Fw - Rg = 0
\end{eqnarray}

\begin{verbatim}
 $Pg$: Precipitation;
 $Rg$: Surface Runoff.
\end{verbatim}

\begin{enumerate}
\def\labelenumi{\arabic{enumi}.}
\setcounter{enumi}{7}
\tightlist
\item
  the snow balance
\end{enumerate}

\begin{eqnarray}
  \frac{\partial Wy}{\partial t} = Py - Fy - My
\end{eqnarray}

\begin{verbatim}
 $Wy$: Snow cover(kg/m$^2$);
 $Py$: Snowfall;
 $Fy$: Sublimation;
 $My$: Snowmelt.
\end{verbatim}

\hypertarget{time-integration-of-physical-processes.}{%
\subsubsection{Time integration of physical
processes.}\label{time-integration-of-physical-processes.}}

Classifying physical processes in terms of the time integration of
predictor variables, The order of execution can be divided into the
following three categories.

\begin{enumerate}
\def\labelenumi{\arabic{enumi}.}
\item
  cumulus convection and large-scale condensation
\item
  radiation, vertical diffusion, ground boundary layer and surface
  processes
\item
  gravitational wave resistance, mass regulation, dry convection
  regulation
\end{enumerate}

Cumulus convection and large-scale condensation,

\begin{eqnarray}
  \hat{T}^{t+\Delta t,(1)} = \hat{T}^{t+\Delta t} 
                         +  2 \Delta t Q_{CUM}(\hat{T}^{t+\Delta t})
\end{eqnarray}

\begin{eqnarray}
  \hat{T}^{t+\Delta t,(2)} = \hat{T}^{t+\Delta t,(1)} 
                         +  2 \Delta t Q_{LSC}(\hat{T}^{t+\Delta t,(1)})
\end{eqnarray}

by the usual Euler difference. Large-scale condensation schemes include
, Note that the updated values are passed on by the cumulus convection
scheme. In practice, the output of the heating rate and so on are used
in the routines for cumulus convection and large-scale condensation,
Time integration is done by the immediately following
\texttt{MODULE:{[}GDINTG{]}}.

Radiation in the following groups, vertical diffusion, ground boundary
layer and surface processes calculations are essentially all of these
updated values ( \(\hat{T}^{t+\Delta t,(1)}, \hat{q}^{t+\Delta t,(2)}\),
\(\hat{T}^{t+\Delta t,(1)}, \hat{q}^{t+\Delta t,(2)}\), etc. ) This is
done by using However, in order to calculate some of the terms as
implicit, the Calculate all of these terms together and calculate the
heating rate, etc, Finally, we do time integration. In other words,
symbolically,

\begin{eqnarray}
  \hat{T}^{t+\Delta t,(3)} = \hat{T}^{t+\Delta t,(2)} 
              + 2 \Delta t Q_{RAD,DIF,SFC}
               (\hat{T}^{t+\Delta t,(2)},\hat{T}^{t+\Delta t,(3)})
\end{eqnarray}

That would be.

As for gravitational wave resistance, mass regulation and dry convection
regulation, It is similar to cumulus convection and large-scale
condensation.

\begin{eqnarray}
  \hat{T}^{t+\Delta t,(4)} = \hat{T}^{t+\Delta t,(3)} 
              +  2 \Delta t Q_{ADJ}(\hat{T}^{t+\Delta t,(3)})
\end{eqnarray}

\hypertarget{various-physical-quantities.}{%
\subsubsection{Various physical
quantities.}\label{various-physical-quantities.}}

A simple calculation from the predictive variables can be used to find
Definitions of various physical quantities. Some of these are ,
Calculated with \texttt{MODULE:{[}PSETUP{]}}.

\begin{enumerate}
\def\labelenumi{\arabic{enumi}.}
\tightlist
\item
  temporary temperature
\end{enumerate}

Provisional Temperature \(T_v\) is ,

\begin{eqnarray}
  T_v = T ( 1 + \epsilon_v q - l )
\end{eqnarray}

\begin{enumerate}
\def\labelenumi{\arabic{enumi}.}
\setcounter{enumi}{1}
\tightlist
\item
  air density
\end{enumerate}

The atmospheric density, \(\rho\), is calculated as follows

\begin{eqnarray}
  \rho = \frac{p}{RT_v}
\end{eqnarray}

\begin{enumerate}
\def\labelenumi{\arabic{enumi}.}
\setcounter{enumi}{2}
\tightlist
\item
  high degree
\end{enumerate}

The high degree \(z\) is a mechanical process The same method is used to
calculate the geopotential.

\begin{eqnarray}
  z = \frac{\Phi}{g} 
\end{eqnarray}

\begin{eqnarray}
 \Phi_{1}  =  \Phi_{s} + C_{p} ( \sigma_{1}^{-\kappa} - 1  ) T_{v,1}
\end{eqnarray}

\begin{eqnarray}
 \Phi_k - \Phi_{k-1} 
   =  C_{p}
   \left[ \left( \frac{ \sigma_{k-1/2} }{ \sigma_k } \right)^{\kappa}
          - 1 \right] T_{v,k} 
       + C_{p}
   \left[ 1- 
         \left( \frac{ \sigma_{k-1/2} }{ \sigma_{k-1} } \right)^{\kappa}
              \right] T_{v,k-1}
\end{eqnarray}

\begin{enumerate}
\def\labelenumi{\arabic{enumi}.}
\setcounter{enumi}{3}
\tightlist
\item
  layer boundary temperature
\end{enumerate}

The temperature of the boundary of the layer is determined by the
temperature of the \(\ln p\), i.e., the temperature of the boundary
relative to the \(\ln \sigma\) Perform a linear interpolation and
calculate.

\begin{eqnarray}
  T_{k-1/2} = \frac{\ln \sigma_{k-1} - \ln \sigma_{k-1/2}}
                   {\ln \sigma_{k-1} - \ln \sigma_k      } T_k
            + \frac{\ln \sigma_{k-1/2} - \ln \sigma_k}
                   {\ln \sigma_{k-1} - \ln \sigma_k      } T_{k-1}
\end{eqnarray}

\begin{enumerate}
\def\labelenumi{\arabic{enumi}.}
\setcounter{enumi}{4}
\tightlist
\item
  saturated specific humidity
\end{enumerate}

Saturation Specific Humidity \(q^*(T,p)\) is approximated using the
saturation vapor pressure \(e^*(T)\),

\begin{eqnarray}
q^*(T,p) = \frac{\epsilon e^*(T)}{p} .
\end{eqnarray}

Here, it is \(\epsilon=0.622\),

\begin{eqnarray}
\frac{1}{e^*_v} \frac{\partial e^*_v}{\partial T} = \frac{L}{R_v T^2}
\end{eqnarray}

\begin{verbatim}
 > <span id="e-sat" label="e-sat">\\blade[e-sat]< /span>
\end{verbatim}

Therefore, if the latent heat of evaporation (\(L\)) and the gas
constant (\(R_v\)) of the water vapor are held constant, then the number
of vapor particles will be reduced,

\begin{eqnarray}
  e^*(T) = e^*(T=273{K}) 
                      \exp \left[ \frac{L}{R_v} 
                            \left( \frac{1}{273} - \frac{1}{T} \right)
                       \right] ,
\end{eqnarray}

\begin{verbatim}
 $e^*(T=273{K}) = 611$ is a \\blank\blank\blank\blank\.com.

 (From [\\\begin{eqnarray}e-sat\end{eqnarray}] (#e-sat)),
\end{verbatim}

\begin{eqnarray}
\frac{\partial q^*}{\partial T} = \frac{L}{R_v T^2} q^*(T,p) .
\end{eqnarray}

Here, if the temperature is lower than the freezing point 273.15 K Use
the sublimation latent heat \(L+L_M\) as the latent heat \(L\).

\begin{enumerate}
\def\labelenumi{\arabic{enumi}.}
\setcounter{enumi}{5}
\tightlist
\item
  dry static energy, wet static energy
\end{enumerate}

Dry static energy \(s\) is

\begin{eqnarray}
  s = C_p T + g z \; ,
\end{eqnarray}

Wet Static Energy \(h\) is

\begin{eqnarray}
  h = C_p T + g z + L q \; ,
\end{eqnarray}

. defined by .
