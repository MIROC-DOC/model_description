\hypertarget{model-configuration.}{%
\section{Model Configuration.}\label{model-configuration.}}

\hypertarget{composition-overview.}{%
\subsection{Composition Overview.}\label{composition-overview.}}

The AGCM program body has a hierarchical structure, The files are
maintained in multiple directories, each of which is divided into
multiple files. A single file (package) can contain , In addition, there
are several program modules (subroutines and functions) in the package,
In some cases, there may be multiple entries in a module.

Example:

** Directory\textbf{} dynamics: \textbf{ {\textbf{File}} }dadmn.F:
\textbf{dadmn.F: \blanket* Package DADMN . {\textbf{File}} }dshpe.F:
\textbf{ \blanket* Package DSPHE } Module DSETNM: \textbf{̄
{\textbf{Module W2G}} SUBROUTINE W2G ENTRY G2W ENTRY SPSTUP } Module
DSETNM:** SUBROUTINE DSETNM.

Currently, there are 10 directories as follows

\begin{itemize}
\item
  TAB00000:0.0 admin
\item
  TAB00000:0.1 Modules related to the structure of the entire model
  (coordinates, time, constants, etc.)
\item
  TAB00000:1.0 dynamics
\item
  TAB00000:1.1 Modules related to mechanical processes
\item
  TAB00000:2.0 physics
\item
  TAB00000:2.1 Modules involved in physical processes
\item
  TAB00000:3.0 io
\item
  TAB00000:3.1 Modules for data input/output
\item
  TAB00000:4.0 util
\item
  TAB00000:4.1 General-purpose operation libraries
\item
  TAB00000:5.0 sysdep
\item
  TAB00000:5.1 system dependent module
\item
  TAB00000:6.0 include
\item
  TAB00000:6.1 {Included by \texttt{\#include}} header types
\item
  TAB00000:7.0 nonstd
\item
  TAB00000:7.1 Non-standard plug-in modules
\item
  TAB00000: 8.0 special
\item
  TAB00000:8.1 test module
\item
  TAB00000:9.0 shalo
\item
  TAB00000:9.1 Module for single layer barotropic shallow water models
  (under test)
\end{itemize}

Note that the files containing the main routines are {\texttt{src/}}and
immediately below.

These dependencies are as follows.

\begin{itemize}
\item
  TAB00001:0.0 MAIN
\item
  TAB00001:0.1 - Replica Hermes Handbags
\item
  TAB00001:0.2
\item
  TAB00001:1.0
\item
  TAB00001:1.1 - Dynamics
\item
  TAB00001:1.2
\item
  TAB00001:2.0
\item
  TAB00001:2.1 - Physical properties
\item
  TAB00001:2.2
\item
  TAB00001:3.0
\item
  TAB00001:3.1
\item
  TAB00001:3.2 - I'm not sure how to describe it.
\item
  TAB00001:3.3
\item
  TAB00001:4.0
\item
  TAB00001:4.1
\item
  TAB00001:4.2
\item
  TAB00001:4.3 - You can't even tell me how to do it.
\item
  TAB00001:4.4
\item
  TAB00001:5.0
\item
  TAB00001:5.1
\item
  TAB00001:5.2
\item
  TAB00001:5.3
\item
  TAB00001:5.4 - I'm sure you'll be able to find out more about it in
  the next few days.
\end{itemize}

That is, each of them is independent of the others in the same row, The
one on the left is used to call the one on the right, but the reverse is
not allowed.

Several closely related routines in one file (package). Particularly in
the physical process, the replacement of one or a few files with The use
of parameterization is possible.

\hypertarget{special-note-on-the-program.}{%
\subsection{Special note on the
program.}\label{special-note-on-the-program.}}

There are several modules with multiple entries using the ENTRY
statement. Its main purpose is the local storage of data. For example,
in the case of the module W2G above, the variables PNM and DPNM are It
is stored as a local variable in this module, Commonly used in W2G, G2W
and SPSTUP. W2G and G2W are used in many places, but this structure
makes it possible for them to be used in various ways, This avoids the
complexity of having to use PNM and DPNM as arguments. The COMMON
variable is usually used in such cases. Here, the COMMON variable is
used as an inconvenience for management and debugging. We avoid this
type of encapsulation structure as much as possible and instead use such
an encapsulated structure.

Only two COMMONs are in use.

\begin{itemize}
\item
  TAB00002:0.0 COMMON /COMCON/
\item
  TAB00002:0.1 Standard physical constants (earth radius, gas constant,
  etc.)
\item
  TAB00002:1.0 COMMON /COMWRK/
\item
  TAB00002:1.1 work area
\end{itemize}

COMCON contains the standard physical constants. This COMMON definition
is \texttt{include/zccom.F}\&lt It is in ;/span\textgreater, It is used
to include as necessary. A set of values can be set by the subroutine
PCONST ({\texttt{admin/\ apcon.F}}). COMWRK is used as a work area by
many modules. It is used to reduce the overall memory consumption, It
doesn't matter if you delete all applicable COMMON statements.
\textbackslash0.1{[}1{]}.

For include file inclusion and conditional compilation It uses the C
preprocessor instruction. So, instead of the file being named
{\texttt{.f}}, it is named , {\texttt{.F}}. As a conditional
compilation, {\texttt{\#ifdef}} and \textless{}
span\textgreater{}\texttt{\#ifndef} selection. Importing files is
\texttt{inlcude}\textless/span\&gt ; I'm doing this from the directory,
It is as follows.

\begin{itemize}
\item
  TAB00003:0.0 Array size parameter statements
\item
  TAB00003:0.1 zcdim.F
\item
  TAB00003:1.0
\item
  TAB00003:1.1 zpdim.F
\item
  TAB00003:2.0
\item
  TAB00003:2.1 zidim.F
\item
  TAB00003:3.0
\item
  TAB00003:3.1 zsdim.F
\item
  TAB00003:4.0
\item
  TAB00003:4.1 zhdim.F
\item
  TAB00003:5.0
\item
  TAB00003:5.1 zradim.F
\item
  TAB00003:6.0
\item
  TAB00003:6.1 zwdim.F
\item
  TAB00003:7.0 COMMON definition (physical constants)
\item
  TAB00003:7.1 zccom.F
\item
  TAB00003:8.0 Statement function definition (saturation ratio humidity)
\item
  TAB00003:8.1 zqsat.F
\end{itemize}

FORTRAN 77 As a non-standard specification , I'm using NAMELIST reading,
It seems to be able to be used in many processing systems without any
problem. For the specifications of NAMELIST, please refer to the manual
of each system.

\hypertarget{program-writing.}{%
\subsection{Program Writing.}\label{program-writing.}}

End-of-line comments are used in various explanations. \texttt{!"} The
end of the line below is a comment \textbackslash0.2{[}2{]}.

The variables are all declared. IMPLICIT NONE (e.g.~Sun's {\texttt{-u}}
option) to It is a prerequisite for use.

Each entry's argument is accompanied by a continuation line column to
explain the function.

\begin{itemize}
\item
  TAB00004:0.0 symbol
\item
  TAB00004:0.1 meaning
\item
  TAB00004:0.2 input
\item
  TAB00004:0.3 Outputs
\item
  TAB00004:0.4 function
\item
  TAB00004:1.0 O
\item
  TAB00004:1.1 output
\item
  TAB00004:1.2 ×impossibility
\item
  TAB00004:1.3 circle (e.g.~of friends)
\item
  TAB00004:1.4 Generate values
\item
  TAB00004:2.0 M
\item
  TAB00004:2.1 modify
\item
  TAB00004:2.2 circle (e.g.~of friends)
\item
  TAB00004:2.3 circle (e.g.~of friends)
\item
  TAB00004:2.4 Processing the input values and outputting them
\item
  TAB00004:3.0 I
\item
  TAB00004:3.1 input
\item
  TAB00004:3.2 circle (e.g.~of friends)
\item
  TAB00004:3.3

  \begin{itemize}
  \tightlist
  \item
  \end{itemize}
\item
  TAB00004:3.4 Input value (`variable')
\item
  TAB00004:4.0 C
\item
  TAB00004:4.1 constant
\item
  TAB00004:4.2 circle (e.g.~of friends)
\item
  TAB00004:4.3

  \begin{itemize}
  \tightlist
  \item
  \end{itemize}
\item
  TAB00004:4.4 Input value (`constant')
\item
  TAB00004:5.0 D
\item
  TAB00004:5.1 dimension
\item
  TAB00004:5.2 circle (e.g.~of friends)
\item
  TAB00004:5.3

  \begin{itemize}
  \tightlist
  \item
  \end{itemize}
\item
  TAB00004:5.4 Variables that determine the size of the matching array
\item
  TAB00004:6.0 W
\item
  TAB00004:6.1 work
\item
  TAB00004:6.2 ×impossibility
\item
  TAB00004:6.3 ×impossibility
\item
  TAB00004:6.4 work area
\item
  TAB00004:7.0 U
\item
  TAB00004:7.1 undefined
\item
  TAB00004:7.2 ×impossibility
\item
  TAB00004:7.3 ×impossibility
\item
  TAB00004:7.4 dummy
\end{itemize}

Here, the meaning of the input and output columns is as follows \In the
meantime, I'm going to be in a position to take a look at some of the
things I've done in the past. \textbackslash begin\{array\}\{ll\}

× x \& whatever is in it \{array\}

\The output is a lot more than just
a\textbackslash lopenopenPointPointPoint.com
\textbackslash begin\{array\}\{ll\}\\
O O \& may be subject to change in the course of the event - \& the
value will not change × x \& I can't guarantee what you'll find
\{array\}

The important ones are M,O,I, where M,O and I are important, and C,D are
a type of I. The use of C, D, and I is not so neat.

The contents of each file are as follows.

\texttt{*"\ PACKAGE\ PSAVE\ save/load\ data\ (real\ memory\ version)} :̄
Package Name \texttt{*"\ {[}HIS{]}\ 93/11/10(numaguti)\ AGCM5.3} Change
log \texttt{SUBROUTINE\ PGSAVE\ \ \ \ \ !"\ Internal\ Data\ Save} :
Module Declaration \texttt{*\ \ \ \ {[}PARAM{]}} The following,
parameter statements are (included) followed by \texttt{*\ {[}MODIFY{]}}
: Declarations of input and output variables \texttt{*\ {[}OUTPUT{]}} :
the following, output variable declarations \texttt{*\ {[}INPUT{]}} :
The following, Declarations of Input Variables
\texttt{*\ {[}ENTRY\ OUTPUT{]}} : Declare output variables in the
entry\ldots{} \texttt{*\ {[}INTERNAL\ WORK{]}} : Declarations of
internal work variables \texttt{*\ \ \ \ {[}INTERNAL\ SAVE{]}}
Declarations of internal variables (which should be kept after RETURN)
\texttt{*\ \ \ \ {[}INTERNAL\ PARAM{]}} Declarations of internal
parameters (to be read by NAMELIST, etc.) \texttt{*\ \ \ \ {[}ONCE{]}}
The following is the part to be done only once on the first call

Sentence numbers are assigned to each block in the thousands, I'm
guessing as structurally as possible.

\hypertarget{naming-rules.}{%
\subsection{Naming Rules.}\label{naming-rules.}}

The names of variables, entry names, etc. must be six characters or
less.

Variable name and type mapping

\begin{itemize}
\item
  TAB00005:0.0 A-G, P-Z
\item
  TAB00005:0.1 Floating point number ({\texttt{REAL*8}})
\item
  TAB00005:1.0 H
\item
  TAB00005:1.1 String ({\texttt{CHARACTER}})
\item
  TAB00005:2.0 I-N.
\item
  TAB00005:2.1 Integer ({\texttt{INTEGER}})
\item
  TAB00005:3.0 O
\item
  TAB00005:3.1 Logical type ({\texttt{LOGICAL}})
\end{itemize}

However, in the variables read by NAMELIST, This may not be met.

Conventions on the Correspondence between Variable Names and Contents

\begin{itemize}
\item
  TAB00006:0.0 Prefix:
\item
  TAB00006:0.1 GA
\item
  TAB00006:0.2 The grid point state quantity (\(t\))
\item
  TAB00006:1.0
\item
  TAB00006:1.1 GB
\item
  TAB00006:1.2 Grid point state quantity (\(t-\Delta t\))
\item
  TAB00006:2.0
\item
  TAB00006:2.1 GD
\item
  TAB00006:2.2 Grid State Quantity (for common use)
\item
  TAB00006:3.0
\item
  TAB00006:3.1 GT
\item
  TAB00006:3.2 Time differential term of the grid state quantity
\item
  TAB00006:4.0
\item
  TAB00006:4.1 WD
\item
  TAB00006:4.2 Spectral representation of state quantities
\item
  TAB00006:5.0
\item
  TAB00006:5.1 WT
\item
  TAB00006:5.2 Spectral representation of the time differential term of
  the state quantity
\item
  TAB00006:6.0
\item
  TAB00006:6.1 I
\item
  TAB00006:6.2 Longitude index
\item
  TAB00006:7.0
\item
  TAB00006:7.1 J
\item
  TAB00006:7.2 Index of latitude
\item
  TAB00006:8.0
\item
  TAB00006:8.1 K
\item
  TAB00006:8.2 Index for vertical level
\item
  TAB00006:9.0
\item
  TAB00006:9.1 IJ
\item
  TAB00006:9.2 An index of all the latitudes and longitudes in one place
\item
  TAB00006:10.0
\item
  TAB00006:10.1 NM.
\item
  TAB00006:10.2 Index of the spectrum
\item
  TAB00006:11.0
\item
  TAB00006:11.1 NM.
\item
  TAB00006:11.2 NAMELIST Name
\item
  TAB00006:12.0
\item
  TAB00006:12.1 COM
\item
  TAB00006:12.2 COMMON Name
\item
  TAB00006:13.0 Tangent:
\item
  TAB00006:13.1 U
\item
  TAB00006:13.2 east-west wind
\item
  TAB00006:14.0
\item
  TAB00006:14.1 V
\item
  TAB00006:14.2 north-south wind
\item
  TAB00006:15.0
\item
  TAB00006:15.1 T
\item
  TAB00006:15.2 temperature
\item
  TAB00006:16.0
\item
  TAB00006:16.1 PS
\item
  TAB00006:16.2 surface pressure
\item
  TAB00006:17.0
\item
  TAB00006:17.1 Q
\item
  TAB00006:17.2 Specific humidity, various tracers
\item
  TAB00006:18.0
\item
  TAB00006:18.1 QL
\item
  TAB00006:18.2 cloud liquidity
\item
  TAB00006:19.0
\item
  TAB00006:19.1 FLX,FLUX
\item
  TAB00006:19.2 flux density
\item
  TAB00006:20.0
\item
  TAB00006:20.1 MTX
\item
  TAB00006:20.2 Matrices for implicit solutions
\item
  TAB00006:21.0
\item
  TAB00006:21.1 MAX
\item
  TAB00006:21.2 data length
\item
  TAB00006:22.0
\item
  TAB00006:22.1 DIM
\item
  TAB00006:22.2 Size of the array region
\end{itemize}

For file names, see , The first letter is unified to the first letter of
the directory. (However, {\texttt{include}} can be used in \&lt
;span\textgreater{}\texttt{z}). Also, {\texttt{-admn}} ( administer)
indicates the main module in it.

\begin{enumerate}
\def\labelenumi{\arabic{enumi}.}
\tightlist
\item
  this COMMON is actually a grammatical violation. The size of the
  common block due to the COMMON statement is It has to be the same. (An
  unnamed common block is acceptable.) . This area will be changed in
  the near future.
\end{enumerate}

The reason for the use of two letters here, as well as
\texttt{!\ I\ use\ two\ letters\ as\ well\ as}! Systems that use other
end-of-line comment formats (e.g.~HITAC VOS3) to ensure substitution
for, and The reason for this is that Sun's CPP will malfunction if you
use only
\texttt{!\ is\ because\ Sun\textquotesingle{}s\ CPP\ will\ malfunction\ if\ there\ is\ only}!
