\hypertarget{description-of-the-program-code}{%
\section{Description of the program
code}\label{description-of-the-program-code}}

\hypertarget{the-basics-of-reading-programs.}{%
\subsection{The Basics of Reading
Programs.}\label{the-basics-of-reading-programs.}}

\hypertarget{configuration-overview.}{%
\subsubsection{Configuration Overview.}\label{configuration-overview.}}

The AGCM program body has a hierarchical structure, The files are
maintained in multiple directories, each of which is divided into
multiple files. A single file (package) can contain , In addition, there
are several program modules (subroutines and functions) in the package,
In some cases, there may be multiple entries in a module.

Example:

{\textbf{directory}}\textbf{dynamics: } {\textbf{File}} \textbf{dadmn.F:
} \blanket* Package DADMN . {\textbf{File}} \textbf{dshpe.F: }
\blanket* Package DSPHE ** Module DSETNM: \textbf{̄ {\textbf{Module
W2G}} SUBROUTINE W2G ENTRY G2W ENTRY SPSTUP } Module DSETNM:**
SUBROUTINE DSETNM

\hypertarget{directory-structure.}{%
\subsubsection{Directory Structure.}\label{directory-structure.}}

Currently, there are 9 directories as follows

\begin{longtable}[]{@{}ll@{}}
\toprule
\begin{minipage}[b]{0.47\columnwidth}\raggedright
Header0\strut
\end{minipage} & \begin{minipage}[b]{0.47\columnwidth}\raggedright
Header1\strut
\end{minipage}\tabularnewline
\midrule
\endhead
\begin{minipage}[t]{0.47\columnwidth}\raggedright
admin\strut
\end{minipage} & \begin{minipage}[t]{0.47\columnwidth}\raggedright
Modules related to the structure of the entire model (coordinates, time,
constants, etc.)\strut
\end{minipage}\tabularnewline
\begin{minipage}[t]{0.47\columnwidth}\raggedright
dynamics\strut
\end{minipage} & \begin{minipage}[t]{0.47\columnwidth}\raggedright
Modules related to mechanical processes\strut
\end{minipage}\tabularnewline
\begin{minipage}[t]{0.47\columnwidth}\raggedright
physics\strut
\end{minipage} & \begin{minipage}[t]{0.47\columnwidth}\raggedright
Modules involved in physical processes\strut
\end{minipage}\tabularnewline
\begin{minipage}[t]{0.47\columnwidth}\raggedright
io\strut
\end{minipage} & \begin{minipage}[t]{0.47\columnwidth}\raggedright
Modules for data input/output\strut
\end{minipage}\tabularnewline
\begin{minipage}[t]{0.47\columnwidth}\raggedright
util\strut
\end{minipage} & \begin{minipage}[t]{0.47\columnwidth}\raggedright
General-purpose operation libraries\strut
\end{minipage}\tabularnewline
\begin{minipage}[t]{0.47\columnwidth}\raggedright
sysdep\strut
\end{minipage} & \begin{minipage}[t]{0.47\columnwidth}\raggedright
system dependent module\strut
\end{minipage}\tabularnewline
\begin{minipage}[t]{0.47\columnwidth}\raggedright
include\strut
\end{minipage} & \begin{minipage}[t]{0.47\columnwidth}\raggedright
Headers included by {\texttt{\#include}}\strut
\end{minipage}\tabularnewline
\begin{minipage}[t]{0.47\columnwidth}\raggedright
nonstd\strut
\end{minipage} & \begin{minipage}[t]{0.47\columnwidth}\raggedright
Non-standard plug-in modules\strut
\end{minipage}\tabularnewline
\begin{minipage}[t]{0.47\columnwidth}\raggedright
\strut
\end{minipage} & \begin{minipage}[t]{0.47\columnwidth}\raggedright
test module\strut
\end{minipage}\tabularnewline
\bottomrule
\end{longtable}

Note that the files containing the main routines are located directly
under {\texttt{src/}}.

These dependencies are as follows.

\begin{longtable}[]{@{}lllll@{}}
\toprule
\begin{minipage}[b]{0.17\columnwidth}\raggedright
Header0\strut
\end{minipage} & \begin{minipage}[b]{0.17\columnwidth}\raggedright
Header1\strut
\end{minipage} & \begin{minipage}[b]{0.17\columnwidth}\raggedright
Header2\strut
\end{minipage} & \begin{minipage}[b]{0.17\columnwidth}\raggedright
Header3\strut
\end{minipage} & \begin{minipage}[b]{0.17\columnwidth}\raggedright
Header4\strut
\end{minipage}\tabularnewline
\midrule
\endhead
\begin{minipage}[t]{0.17\columnwidth}\raggedright
MAIN\strut
\end{minipage} & \begin{minipage}[t]{0.17\columnwidth}\raggedright
- Replica Hermes Handbags\strut
\end{minipage} & \begin{minipage}[t]{0.17\columnwidth}\raggedright
\strut
\end{minipage} & \begin{minipage}[t]{0.17\columnwidth}\raggedright
\strut
\end{minipage} & \begin{minipage}[t]{0.17\columnwidth}\raggedright
\strut
\end{minipage}\tabularnewline
\begin{minipage}[t]{0.17\columnwidth}\raggedright
\strut
\end{minipage} & \begin{minipage}[t]{0.17\columnwidth}\raggedright
- Dynamics\strut
\end{minipage} & \begin{minipage}[t]{0.17\columnwidth}\raggedright
\strut
\end{minipage} & \begin{minipage}[t]{0.17\columnwidth}\raggedright
\strut
\end{minipage} & \begin{minipage}[t]{0.17\columnwidth}\raggedright
\strut
\end{minipage}\tabularnewline
\begin{minipage}[t]{0.17\columnwidth}\raggedright
\strut
\end{minipage} & \begin{minipage}[t]{0.17\columnwidth}\raggedright
- Physical properties\strut
\end{minipage} & \begin{minipage}[t]{0.17\columnwidth}\raggedright
\strut
\end{minipage} & \begin{minipage}[t]{0.17\columnwidth}\raggedright
\strut
\end{minipage} & \begin{minipage}[t]{0.17\columnwidth}\raggedright
\strut
\end{minipage}\tabularnewline
\begin{minipage}[t]{0.17\columnwidth}\raggedright
\strut
\end{minipage} & \begin{minipage}[t]{0.17\columnwidth}\raggedright
\strut
\end{minipage} & \begin{minipage}[t]{0.17\columnwidth}\raggedright
- I'm not sure how to describe it.\strut
\end{minipage} & \begin{minipage}[t]{0.17\columnwidth}\raggedright
\strut
\end{minipage} & \begin{minipage}[t]{0.17\columnwidth}\raggedright
\strut
\end{minipage}\tabularnewline
\begin{minipage}[t]{0.17\columnwidth}\raggedright
\strut
\end{minipage} & \begin{minipage}[t]{0.17\columnwidth}\raggedright
\strut
\end{minipage} & \begin{minipage}[t]{0.17\columnwidth}\raggedright
\strut
\end{minipage} & \begin{minipage}[t]{0.17\columnwidth}\raggedright
- You can't even tell me how to do it.\strut
\end{minipage} & \begin{minipage}[t]{0.17\columnwidth}\raggedright
\strut
\end{minipage}\tabularnewline
\begin{minipage}[t]{0.17\columnwidth}\raggedright
\strut
\end{minipage} & \begin{minipage}[t]{0.17\columnwidth}\raggedright
\strut
\end{minipage} & \begin{minipage}[t]{0.17\columnwidth}\raggedright
\strut
\end{minipage} & \begin{minipage}[t]{0.17\columnwidth}\raggedright
\strut
\end{minipage} & \begin{minipage}[t]{0.17\columnwidth}\raggedright
- I'm sure you'll be able to find out more about it in the next few
days.\strut
\end{minipage}\tabularnewline
\bottomrule
\end{longtable}

That is, each of them is independent of the others in the same row, The
one on the left is used to call the one on the right, but the reverse is
not allowed.

Several closely related routines in one file (package). Particularly in
the physical process, the replacement of one or a few files with The use
of parameterization is possible.

\hypertarget{special-notes-on-the-program.}{%
\subsubsection{Special Notes on the
Program.}\label{special-notes-on-the-program.}}

There are several modules that have multiple entries using the ENTRY
statement. Its main purpose is the local storage of data. For example,
in the case of the module W2G above, the variables PNM and DPNM are It
is stored as a local variable in this module, Commonly used in W2G, G2W
and SPSTUP. W2G and G2W are used in many places, but this structure
makes it possible for them to be used in various ways, This avoids the
complexity of having to use PNM and DPNM as arguments. The COMMON
variable is usually used in such cases. Here, the COMMON variable is
used as an inconvenience for management and debugging. We avoid this
type of encapsulation structure as much as possible and instead use such
an encapsulation structure.

Only two COMMONs are in use.

\begin{longtable}[]{@{}ll@{}}
\toprule
\begin{minipage}[b]{0.47\columnwidth}\raggedright
Header0\strut
\end{minipage} & \begin{minipage}[b]{0.47\columnwidth}\raggedright
Header1\strut
\end{minipage}\tabularnewline
\midrule
\endhead
\begin{minipage}[t]{0.47\columnwidth}\raggedright
COMMON /COMCON/\strut
\end{minipage} & \begin{minipage}[t]{0.47\columnwidth}\raggedright
Standard physical constants (earth radius, gas constant, etc.)\strut
\end{minipage}\tabularnewline
\begin{minipage}[t]{0.47\columnwidth}\raggedright
Anonymous COMMON\strut
\end{minipage} & \begin{minipage}[t]{0.47\columnwidth}\raggedright
work area\strut
\end{minipage}\tabularnewline
\bottomrule
\end{longtable}

\begin{verbatim}
 COMCON contains the standard physical constants.
\end{verbatim}

This COMMON definition is in {\texttt{include/zccom.F}}, It is used to
include as necessary. The value is set by calling the subroutine PCONST
({\texttt{admin/apcon.F}}). An unnamed COMMON block is used as a work
area from many modules. It is used to reduce the overall memory
consumption. Deleting all the relevant COMMON statements does not have a
problem as it only affects the amount of memory.

For include file inclusion and conditional compilation It uses the C
preprocessor instruction.
F\texttt{\textless{}/span\textgreater{}\ instead\ of\ \textless{}span\textgreater{}}.f\texttt{\textless{}/span\textgreater{},\ so\ the\ file\ name\ is\ \textless{}span\textgreater{}}.
As a conditional compilation, Using selection by {\texttt{\#ifdef}} and
{\texttt{\#ifndef}}. Files are imported from the {\texttt{inlcude}}
directory, It is as follows.

\begin{longtable}[]{@{}ll@{}}
\toprule
Header0 & Header1\tabularnewline
\midrule
\endhead
Array size parameter statements & zcdim.F\tabularnewline
& zpdim.F\tabularnewline
& zidim.F\tabularnewline
& zsdim.F\tabularnewline
& zhdim.F\tabularnewline
& zradim.F\tabularnewline
& zwdim.F\tabularnewline
COMMON definition (physical constants) & zccom.F\tabularnewline
Statement function definition (saturation ratio humidity) &
zqsat.F\tabularnewline
\bottomrule
\end{longtable}

FORTRAN 77 As a non-standard specification , I'm using NAMELIST reading,
It seems to be able to be used in many processing systems without any
problem. For the specifications of NAMELIST, please refer to the manual
of each system.

\hypertarget{program-writing.}{%
\subsubsection{Program Writing.}\label{program-writing.}}

\begin{enumerate}
\def\labelenumi{\arabic{enumi}.}
\tightlist
\item
  end-of-line comments are used in various explanations. \texttt{!"} The
  end of the line below is a comment \textbackslash0.1{[}1{]}.
\end{enumerate}

All variables are declared. 2. IMPLICIT NONE (e.g., the {\texttt{u}}
option for Sun) is set to It is a prerequisite for use.

Each entry's argument is accompanied by a continuation line column to
explain the function.

\begin{longtable}[]{@{}lllll@{}}
\toprule
\begin{minipage}[b]{0.17\columnwidth}\raggedright
Header0\strut
\end{minipage} & \begin{minipage}[b]{0.17\columnwidth}\raggedright
Header1\strut
\end{minipage} & \begin{minipage}[b]{0.17\columnwidth}\raggedright
Header2\strut
\end{minipage} & \begin{minipage}[b]{0.17\columnwidth}\raggedright
Header3\strut
\end{minipage} & \begin{minipage}[b]{0.17\columnwidth}\raggedright
Header4\strut
\end{minipage}\tabularnewline
\midrule
\endhead
\begin{minipage}[t]{0.17\columnwidth}\raggedright
symbol\strut
\end{minipage} & \begin{minipage}[t]{0.17\columnwidth}\raggedright
meaning\strut
\end{minipage} & \begin{minipage}[t]{0.17\columnwidth}\raggedright
input\strut
\end{minipage} & \begin{minipage}[t]{0.17\columnwidth}\raggedright
Outputs\strut
\end{minipage} & \begin{minipage}[t]{0.17\columnwidth}\raggedright
function\strut
\end{minipage}\tabularnewline
\begin{minipage}[t]{0.17\columnwidth}\raggedright
O\strut
\end{minipage} & \begin{minipage}[t]{0.17\columnwidth}\raggedright
output\strut
\end{minipage} & \begin{minipage}[t]{0.17\columnwidth}\raggedright
×impossibility\strut
\end{minipage} & \begin{minipage}[t]{0.17\columnwidth}\raggedright
circle (e.g.~of friends)\strut
\end{minipage} & \begin{minipage}[t]{0.17\columnwidth}\raggedright
Generate values\strut
\end{minipage}\tabularnewline
\begin{minipage}[t]{0.17\columnwidth}\raggedright
M\strut
\end{minipage} & \begin{minipage}[t]{0.17\columnwidth}\raggedright
modify\strut
\end{minipage} & \begin{minipage}[t]{0.17\columnwidth}\raggedright
circle (e.g.~of friends)\strut
\end{minipage} & \begin{minipage}[t]{0.17\columnwidth}\raggedright
circle (e.g.~of friends)\strut
\end{minipage} & \begin{minipage}[t]{0.17\columnwidth}\raggedright
Processing the input values and outputting them\strut
\end{minipage}\tabularnewline
\begin{minipage}[t]{0.17\columnwidth}\raggedright
I\strut
\end{minipage} & \begin{minipage}[t]{0.17\columnwidth}\raggedright
input\strut
\end{minipage} & \begin{minipage}[t]{0.17\columnwidth}\raggedright
circle (e.g.~of friends)\strut
\end{minipage} & \begin{minipage}[t]{0.17\columnwidth}\raggedright
-\strut
\end{minipage} & \begin{minipage}[t]{0.17\columnwidth}\raggedright
Input value (`variable')\strut
\end{minipage}\tabularnewline
\begin{minipage}[t]{0.17\columnwidth}\raggedright
C\strut
\end{minipage} & \begin{minipage}[t]{0.17\columnwidth}\raggedright
constant\strut
\end{minipage} & \begin{minipage}[t]{0.17\columnwidth}\raggedright
circle (e.g.~of friends)\strut
\end{minipage} & \begin{minipage}[t]{0.17\columnwidth}\raggedright
-\strut
\end{minipage} & \begin{minipage}[t]{0.17\columnwidth}\raggedright
Input value (`constant')\strut
\end{minipage}\tabularnewline
\begin{minipage}[t]{0.17\columnwidth}\raggedright
D\strut
\end{minipage} & \begin{minipage}[t]{0.17\columnwidth}\raggedright
dimension\strut
\end{minipage} & \begin{minipage}[t]{0.17\columnwidth}\raggedright
circle (e.g.~of friends)\strut
\end{minipage} & \begin{minipage}[t]{0.17\columnwidth}\raggedright
-\strut
\end{minipage} & \begin{minipage}[t]{0.17\columnwidth}\raggedright
Variables that determine the size of the matching array\strut
\end{minipage}\tabularnewline
\begin{minipage}[t]{0.17\columnwidth}\raggedright
W\strut
\end{minipage} & \begin{minipage}[t]{0.17\columnwidth}\raggedright
work\strut
\end{minipage} & \begin{minipage}[t]{0.17\columnwidth}\raggedright
×impossibility\strut
\end{minipage} & \begin{minipage}[t]{0.17\columnwidth}\raggedright
×impossibility\strut
\end{minipage} & \begin{minipage}[t]{0.17\columnwidth}\raggedright
work area\strut
\end{minipage}\tabularnewline
\begin{minipage}[t]{0.17\columnwidth}\raggedright
U\strut
\end{minipage} & \begin{minipage}[t]{0.17\columnwidth}\raggedright
undefined\strut
\end{minipage} & \begin{minipage}[t]{0.17\columnwidth}\raggedright
×impossibility\strut
\end{minipage} & \begin{minipage}[t]{0.17\columnwidth}\raggedright
×impossibility\strut
\end{minipage} & \begin{minipage}[t]{0.17\columnwidth}\raggedright
dummy\strut
\end{minipage}\tabularnewline
\bottomrule
\end{longtable}

Here, the meaning of the input and output columns is as follows \In the
meantime, I'm going to be in a position to take a look at some of the
things I've done in the past. \textbackslash begin\{array\}\{ll\}

× x \& whatever is in it \{array\}

\begin{verbatim}
     \begin{array}{ll}   
\end{verbatim}

O O \& may be subject to change in the course of the event - \& the
value will not change × x \& I can't guarantee what you'll find
\{array\}

The important ones are M,O,I, where M,O and I are important, and C,D are
a type of I. The use of C, D, and I is not so neat.

The contents of each file are as follows.

\begin{verbatim}
 ` *" PACKAGE PSAVE save/load data (real memory version) ` 
 :̄ Package Name 
 `*" [HIS] 93/11/10(numaguti) AGCM5.3` 
 Change log 
 `        SUBROUTINE PGSAVE     !" Internal Data Save ` 
 : Module Declaration 
 `*    [PARAM]` 
 The following, parameter statements are (included) followed by 
 `    * [MODIFY] ` 
 : Declarations of input and output variables 
 `    * [OUTPUT] ` 
 : the following, output variable declarations 
 `    * [INPUT] ` 
 : The following, Declarations of Input Variables 
 `    * [ENTRY OUTPUT] ` 
 : Declare output variables in the entry... 
 `    * [INTERNAL WORK] ` 
 : Declarations of internal work variables 
 `*    [INTERNAL SAVE]` 
 Declarations of internal variables (which should be kept after RETURN) 
 `*    [INTERNAL PARAM]` 
 Declarations of internal parameters (to be read by NAMELIST, etc.) 
 `*    [ONCE]` 
\end{verbatim}

The following is the part to be done only once on the first call

Sentence numbers are assigned to each block in the thousands, I'm
guessing as structurally as possible.

\hypertarget{naming-conventions.}{%
\subsubsection{Naming Conventions.}\label{naming-conventions.}}

The names of variables, entry names, etc. must be six characters or
less.

\begin{enumerate}
\def\labelenumi{\arabic{enumi}.}
\setcounter{enumi}{1}
\tightlist
\item
  variable name and type mapping
\end{enumerate}

\begin{longtable}[]{@{}ll@{}}
\toprule
Header0 & Header1\tabularnewline
\midrule
\endhead
A-G, P-Z & Floating point number ({\texttt{REAL*8}})\tabularnewline
H & String ({\texttt{CHARACTER}})\tabularnewline
I-N. & Integer ({\texttt{INTEGER}})\tabularnewline
O & Logical type ({\texttt{LOGICAL}})\tabularnewline
\bottomrule
\end{longtable}

However, in the variables read by NAMELIST, This may not be met.

Conventions on the correspondence between variable names and contents

\begin{longtable}[]{@{}lll@{}}
\toprule
\begin{minipage}[b]{0.30\columnwidth}\raggedright
Header0\strut
\end{minipage} & \begin{minipage}[b]{0.30\columnwidth}\raggedright
Header1\strut
\end{minipage} & \begin{minipage}[b]{0.30\columnwidth}\raggedright
Header2\strut
\end{minipage}\tabularnewline
\midrule
\endhead
\begin{minipage}[t]{0.30\columnwidth}\raggedright
Prefix:\strut
\end{minipage} & \begin{minipage}[t]{0.30\columnwidth}\raggedright
GA\strut
\end{minipage} & \begin{minipage}[t]{0.30\columnwidth}\raggedright
The grid point state quantity (\(t\))\strut
\end{minipage}\tabularnewline
\begin{minipage}[t]{0.30\columnwidth}\raggedright
\strut
\end{minipage} & \begin{minipage}[t]{0.30\columnwidth}\raggedright
GB\strut
\end{minipage} & \begin{minipage}[t]{0.30\columnwidth}\raggedright
Grid point state quantity (\(t-\Delta t\))\strut
\end{minipage}\tabularnewline
\begin{minipage}[t]{0.30\columnwidth}\raggedright
\strut
\end{minipage} & \begin{minipage}[t]{0.30\columnwidth}\raggedright
GD\strut
\end{minipage} & \begin{minipage}[t]{0.30\columnwidth}\raggedright
Grid State Quantity (for common use)\strut
\end{minipage}\tabularnewline
\begin{minipage}[t]{0.30\columnwidth}\raggedright
\strut
\end{minipage} & \begin{minipage}[t]{0.30\columnwidth}\raggedright
GT\strut
\end{minipage} & \begin{minipage}[t]{0.30\columnwidth}\raggedright
Time differential term of the grid state quantity\strut
\end{minipage}\tabularnewline
\begin{minipage}[t]{0.30\columnwidth}\raggedright
\strut
\end{minipage} & \begin{minipage}[t]{0.30\columnwidth}\raggedright
WD\strut
\end{minipage} & \begin{minipage}[t]{0.30\columnwidth}\raggedright
Spectral representation of state quantities\strut
\end{minipage}\tabularnewline
\begin{minipage}[t]{0.30\columnwidth}\raggedright
\strut
\end{minipage} & \begin{minipage}[t]{0.30\columnwidth}\raggedright
WT\strut
\end{minipage} & \begin{minipage}[t]{0.30\columnwidth}\raggedright
Spectral representation of the time differential term of the state
quantity\strut
\end{minipage}\tabularnewline
\begin{minipage}[t]{0.30\columnwidth}\raggedright
\strut
\end{minipage} & \begin{minipage}[t]{0.30\columnwidth}\raggedright
I\strut
\end{minipage} & \begin{minipage}[t]{0.30\columnwidth}\raggedright
Longitude index\strut
\end{minipage}\tabularnewline
\begin{minipage}[t]{0.30\columnwidth}\raggedright
\strut
\end{minipage} & \begin{minipage}[t]{0.30\columnwidth}\raggedright
J\strut
\end{minipage} & \begin{minipage}[t]{0.30\columnwidth}\raggedright
Index of latitude\strut
\end{minipage}\tabularnewline
\begin{minipage}[t]{0.30\columnwidth}\raggedright
\strut
\end{minipage} & \begin{minipage}[t]{0.30\columnwidth}\raggedright
K\strut
\end{minipage} & \begin{minipage}[t]{0.30\columnwidth}\raggedright
Index for vertical level\strut
\end{minipage}\tabularnewline
\begin{minipage}[t]{0.30\columnwidth}\raggedright
\strut
\end{minipage} & \begin{minipage}[t]{0.30\columnwidth}\raggedright
IJ\strut
\end{minipage} & \begin{minipage}[t]{0.30\columnwidth}\raggedright
An index of all the latitudes and longitudes in one place\strut
\end{minipage}\tabularnewline
\begin{minipage}[t]{0.30\columnwidth}\raggedright
\strut
\end{minipage} & \begin{minipage}[t]{0.30\columnwidth}\raggedright
NM.\strut
\end{minipage} & \begin{minipage}[t]{0.30\columnwidth}\raggedright
Index of the spectrum\strut
\end{minipage}\tabularnewline
\begin{minipage}[t]{0.30\columnwidth}\raggedright
\strut
\end{minipage} & \begin{minipage}[t]{0.30\columnwidth}\raggedright
NM.\strut
\end{minipage} & \begin{minipage}[t]{0.30\columnwidth}\raggedright
NAMELIST Name\strut
\end{minipage}\tabularnewline
\begin{minipage}[t]{0.30\columnwidth}\raggedright
\strut
\end{minipage} & \begin{minipage}[t]{0.30\columnwidth}\raggedright
COM\strut
\end{minipage} & \begin{minipage}[t]{0.30\columnwidth}\raggedright
COMMON Name\strut
\end{minipage}\tabularnewline
\begin{minipage}[t]{0.30\columnwidth}\raggedright
Tangent:\strut
\end{minipage} & \begin{minipage}[t]{0.30\columnwidth}\raggedright
U\strut
\end{minipage} & \begin{minipage}[t]{0.30\columnwidth}\raggedright
east-west wind\strut
\end{minipage}\tabularnewline
\begin{minipage}[t]{0.30\columnwidth}\raggedright
\strut
\end{minipage} & \begin{minipage}[t]{0.30\columnwidth}\raggedright
V\strut
\end{minipage} & \begin{minipage}[t]{0.30\columnwidth}\raggedright
north-south wind\strut
\end{minipage}\tabularnewline
\begin{minipage}[t]{0.30\columnwidth}\raggedright
\strut
\end{minipage} & \begin{minipage}[t]{0.30\columnwidth}\raggedright
T\strut
\end{minipage} & \begin{minipage}[t]{0.30\columnwidth}\raggedright
temperature\strut
\end{minipage}\tabularnewline
\begin{minipage}[t]{0.30\columnwidth}\raggedright
\strut
\end{minipage} & \begin{minipage}[t]{0.30\columnwidth}\raggedright
PS\strut
\end{minipage} & \begin{minipage}[t]{0.30\columnwidth}\raggedright
surface pressure\strut
\end{minipage}\tabularnewline
\begin{minipage}[t]{0.30\columnwidth}\raggedright
\strut
\end{minipage} & \begin{minipage}[t]{0.30\columnwidth}\raggedright
Q\strut
\end{minipage} & \begin{minipage}[t]{0.30\columnwidth}\raggedright
Specific humidity, various tracers\strut
\end{minipage}\tabularnewline
\begin{minipage}[t]{0.30\columnwidth}\raggedright
\strut
\end{minipage} & \begin{minipage}[t]{0.30\columnwidth}\raggedright
QL\strut
\end{minipage} & \begin{minipage}[t]{0.30\columnwidth}\raggedright
cloud liquidity\strut
\end{minipage}\tabularnewline
\begin{minipage}[t]{0.30\columnwidth}\raggedright
\strut
\end{minipage} & \begin{minipage}[t]{0.30\columnwidth}\raggedright
FLX,FLUX\strut
\end{minipage} & \begin{minipage}[t]{0.30\columnwidth}\raggedright
flux density\strut
\end{minipage}\tabularnewline
\begin{minipage}[t]{0.30\columnwidth}\raggedright
\strut
\end{minipage} & \begin{minipage}[t]{0.30\columnwidth}\raggedright
MTX\strut
\end{minipage} & \begin{minipage}[t]{0.30\columnwidth}\raggedright
Matrices for implicit solutions\strut
\end{minipage}\tabularnewline
\begin{minipage}[t]{0.30\columnwidth}\raggedright
\strut
\end{minipage} & \begin{minipage}[t]{0.30\columnwidth}\raggedright
MAX\strut
\end{minipage} & \begin{minipage}[t]{0.30\columnwidth}\raggedright
data length\strut
\end{minipage}\tabularnewline
\begin{minipage}[t]{0.30\columnwidth}\raggedright
\strut
\end{minipage} & \begin{minipage}[t]{0.30\columnwidth}\raggedright
DIM\strut
\end{minipage} & \begin{minipage}[t]{0.30\columnwidth}\raggedright
Size of the array region\strut
\end{minipage}\tabularnewline
\bottomrule
\end{longtable}

For file names, The first letter is unified to the first letter of the
directory. (However, {\texttt{include}} is {\texttt{z}}.) Also,
{\texttt{-admn}}(admin) indicates the main module in it.

\textless! -- end list --\textgreater!

The reason for the use of two characters here, not just
\texttt{!\ I\ use\ two\ letters\ as\ well\ as}! Systems that use other
end-of-line comment formats (e.g.~HITAC VOS3) to ensure substitution
for, and The reason for this is that Sun's CPP will malfunction if you
use only
\texttt{!\ is\ because\ Sun\textquotesingle{}s\ CPP\ will\ malfunction\ if\ there\ is\ only}!
