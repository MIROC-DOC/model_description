\hypertarget{description-of-the-program-code}{%
\section{Description of the program
code}\label{description-of-the-program-code}}

\hypertarget{the-basics-of-reading-programs.}{%
\subsection{The Basics of Reading
Programs.}\label{the-basics-of-reading-programs.}}

\hypertarget{configuration-overview.}{%
\subsubsection{Configuration Overview.}\label{configuration-overview.}}

The AGCM program body has a hierarchical structure, The files are
maintained in multiple directories, each of which is divided into
multiple files. A single file (package) can contain , In addition, there
are several program modules (subroutines and functions) in the package,
In some cases, there may be multiple entries in a module.

Example:

** Directory\textbf{} dynamics: \textbf{ {\textbf{File}} }dadmn.F:
\textbf{dadmn.F: \blanket* Package DADMN . {\textbf{File}} }dshpe.F:
\textbf{ \blanket* Package DSPHE } Module DSETNM: \textbf{̄
{\textbf{Module W2G}} SUBROUTINE W2G ENTRY G2W ENTRY SPSTUP } Module
DSETNM:** SUBROUTINE DSETNM.

\hypertarget{directory-structure.}{%
\subsubsection{Directory Structure.}\label{directory-structure.}}

Currently, there are 9 directories as follows

\begin{itemize}
\item
  TAB00000:0.0 admin
\item
  TAB00000:0.1 Modules related to the structure of the entire model
  (coordinates, time, constants, etc.)
\item
  TAB00000:1.0 dynamics
\item
  TAB00000:1.1 Modules related to mechanical processes
\item
  TAB00000:2.0 physics
\item
  TAB00000:2.1 Modules involved in physical processes
\item
  TAB00000:3.0 io
\item
  TAB00000:3.1 Modules for data input/output
\item
  TAB00000:4.0 util
\item
  TAB00000:4.1 General-purpose operation libraries
\item
  TAB00000:5.0 sysdep
\item
  TAB00000:5.1 system dependent module
\item
  TAB00000:6.0 include
\item
  TAB00000:6.1 {Included by \texttt{\#include}} header types
\item
  TAB00000:7.0 nonstd
\item
  TAB00000:7.1 Non-standard plug-in modules
\item
  TAB00000: 8.0 special
\item
  TAB00000:8.1 test module
\end{itemize}

Note that the files containing the main routines are {\texttt{src/}}and
immediately below.

These dependencies are as follows.

\begin{itemize}
\item
  TAB00001:0.0 MAIN
\item
  TAB00001:0.1 - Replica Hermes Handbags
\item
  TAB00001:0.2
\item
  TAB00001:1.0
\item
  TAB00001:1.1 - Dynamics
\item
  TAB00001:1.2
\item
  TAB00001:2.0
\item
  TAB00001:2.1 - Physical properties
\item
  TAB00001:2.2
\item
  TAB00001:3.0
\item
  TAB00001:3.1
\item
  TAB00001:3.2 - I'm not sure how to describe it.
\item
  TAB00001:3.3
\item
  TAB00001:4.0
\item
  TAB00001:4.1
\item
  TAB00001:4.2
\item
  TAB00001:4.3 - You can't even tell me how to do it.
\item
  TAB00001:4.4
\item
  TAB00001:5.0
\item
  TAB00001:5.1
\item
  TAB00001:5.2
\item
  TAB00001:5.3
\item
  TAB00001:5.4 - I'm sure you'll be able to find out more about it in
  the next few days.
\end{itemize}

That is, each of them is independent of the others in the same row, The
one on the left is used to call the one on the right, but the reverse is
not allowed.

Several closely related routines in one file (package). Particularly in
the physical process, the replacement of one or a few files with The use
of parameterization is possible.

\hypertarget{special-notes-on-the-program.}{%
\subsubsection{Special Notes on the
Program.}\label{special-notes-on-the-program.}}

There are several modules that have multiple entries using the ENTRY
statement. Its main purpose is the local storage of data. For example,
in the case of the module W2G above, the variables PNM and DPNM are It
is stored as a local variable in this module, Commonly used in W2G, G2W
and SPSTUP. W2G and G2W are used in many places, but this structure
makes it possible for them to be used in various ways, This avoids the
complexity of having to use PNM and DPNM as arguments. The COMMON
variable is usually used in such cases. Here, the COMMON variable is
used as an inconvenience for management and debugging. We avoid this
type of encapsulation structure as much as possible and instead use such
an encapsulation structure.

Only two COMMONs are in use.

\begin{verbatim}
   - TAB00002:0.0 
     COMMON /COMCON/
 
   - TAB00002:0.1 
\end{verbatim}

Standard physical constants (earth radius, gas constant, etc.)

\begin{verbatim}
   - TAB00002:1.0 
\end{verbatim}

Anonymous COMMON

\begin{verbatim}
   - TAB00002:1.1 
\end{verbatim}

work area

\begin{verbatim}
 COMCON contains the standard physical constants.
\end{verbatim}

This COMMON definition is \texttt{include/zccom.F}\&lt It is in
;/span\textgreater, It is used to include as necessary. A set of values
can be set by the subroutine PCONST ({\texttt{admin/\ apcon.F}}). An
unnamed COMMON block is used as a work area from many modules. It is
used to reduce the overall memory consumption. Deleting all the relevant
COMMON statements does not have a problem as it only affects the amount
of memory.

For include file inclusion and conditional compilation It uses the C
preprocessor instruction. So, instead of the file being named
{\texttt{.f}}, it is named , {\texttt{.F}}. As a conditional
compilation, {\texttt{\#ifdef}} and \textless{}
span\textgreater{}\texttt{\#ifndef} selection. Importing files is
\texttt{inlcude}\textless/span\&gt ; I'm doing this from the directory,
It is as follows.

\begin{verbatim}
   - TAB00003:0.0 
\end{verbatim}

Array size parameter statements

\begin{verbatim}
   - TAB00003:0.1 
     zcdim.F
 
   - TAB00003:1.0
 
   - TAB00003:1.1 
     zpdim.F
 
   - TAB00003:2.0
 
   - TAB00003:2.1 
     zidim.F
 
   - TAB00003:3.0
 
   - TAB00003:3.1 
     zsdim.F
 
   - TAB00003:4.0
 
   - TAB00003:4.1 
     zhdim.F
 
   - TAB00003:5.0
 
   - TAB00003:5.1 
     zradim.F
 
   - TAB00003:6.0
 
   - TAB00003:6.1 
     zwdim.F
 
   - TAB00003:7.0 
     COMMON definition (physical constants)
 
   - TAB00003:7.1 
     zccom.F
 
   - TAB00003:8.0 
\end{verbatim}

Statement function definition (saturation ratio humidity)

\begin{verbatim}
   - TAB00003:8.1 
     zqsat.F
\end{verbatim}

FORTRAN 77 As a non-standard specification , I'm using NAMELIST reading,
It seems to be able to be used in many processing systems without any
problem. For the specifications of NAMELIST, please refer to the manual
of each system.

\hypertarget{program-writing.}{%
\subsubsection{Program Writing.}\label{program-writing.}}

\begin{enumerate}
\def\labelenumi{\arabic{enumi}.}
\tightlist
\item
  end-of-line comments are used in various explanations. \texttt{!"} The
  end of the line below is a comment \textbackslash0.1{[}1{]}.
\end{enumerate}

All variables are declared. 2. IMPLICIT NONE (e.g.~Sun's {\texttt{-u}}
option) to It is a prerequisite for use.

Each entry's argument is accompanied by a continuation line column to
explain the function.

\begin{verbatim}
   - TAB00004:0.0 
\end{verbatim}

symbol

\begin{verbatim}
   - TAB00004:0.1 
\end{verbatim}

meaning

\begin{verbatim}
   - TAB00004:0.2 
\end{verbatim}

input

\begin{verbatim}
   - TAB00004:0.3 
\end{verbatim}

Outputs

\begin{verbatim}
   - TAB00004:0.4 
\end{verbatim}

function

\begin{verbatim}
   - TAB00004:1.0 
     O
 
   - TAB00004:1.1 
     output
 
   - TAB00004:1.2 
     ×impossibility
 
   - TAB00004:1.3 
\end{verbatim}

circle (e.g.~of friends)

\begin{verbatim}
   - TAB00004:1.4 
\end{verbatim}

Generate values

\begin{verbatim}
   - TAB00004:2.0 
     M
 
   - TAB00004:2.1 
     modify
 
   - TAB00004:2.2 
\end{verbatim}

circle (e.g.~of friends)

\begin{verbatim}
   - TAB00004:2.3 
\end{verbatim}

circle (e.g.~of friends)

\begin{verbatim}
   - TAB00004:2.4 
\end{verbatim}

Processing the input values and outputting them

\begin{verbatim}
   - TAB00004:3.0 
     I
 
   - TAB00004:3.1 
     input
 
   - TAB00004:3.2 
\end{verbatim}

circle (e.g.~of friends)

\begin{verbatim}
   - TAB00004:3.3 
     -
 
   - TAB00004:3.4 
\end{verbatim}

Input value (`variable')

\begin{verbatim}
   - TAB00004:4.0 
     C
 
   - TAB00004:4.1 
     constant
 
   - TAB00004:4.2 
\end{verbatim}

circle (e.g.~of friends)

\begin{verbatim}
   - TAB00004:4.3 
     -
 
   - TAB00004:4.4 
\end{verbatim}

Input value (`constant')

\begin{verbatim}
   - TAB00004:5.0 
     D
 
   - TAB00004:5.1 
     dimension
 
   - TAB00004:5.2 
\end{verbatim}

circle (e.g.~of friends)

\begin{verbatim}
   - TAB00004:5.3 
     -
 
   - TAB00004:5.4 
\end{verbatim}

Variables that determine the size of the matching array

\begin{verbatim}
   - TAB00004:6.0 
     W
 
   - TAB00004:6.1 
     work
 
   - TAB00004:6.2 
     ×impossibility
 
   - TAB00004:6.3 
     ×impossibility
 
   - TAB00004:6.4 
\end{verbatim}

work area

\begin{verbatim}
   - TAB00004:7.0 
     U
 
   - TAB00004:7.1 
     undefined
 
   - TAB00004:7.2 
     ×impossibility
 
   - TAB00004:7.3 
     ×impossibility
 
   - TAB00004:7.4 
\end{verbatim}

dummy

Here, the meaning of the input and output columns is as follows \In the
meantime, I'm going to be in a position to take a look at some of the
things I've done in the past. \textbackslash begin\{array\}\{ll\}

× x \& whatever is in it \{array\}

\begin{verbatim}
     \begin{array}{ll}   
\end{verbatim}

O O \& may be subject to change in the course of the event - \& the
value will not change × x \& I can't guarantee what you'll find
\{array\}

The important ones are M,O,I, where M,O and I are important, and C,D are
a type of I. The use of C, D, and I is not so neat.

The contents of each file are as follows.

\begin{verbatim}
 ` *" PACKAGE PSAVE save/load data (real memory version) ` 
 :̄ Package Name 
 `*" [HIS] 93/11/10(numaguti) AGCM5.3` 
 Change log 
 `        SUBROUTINE PGSAVE     !" Internal Data Save ` 
 : Module Declaration 
 `*    [PARAM]` 
 The following, parameter statements are (included) followed by 
 `    * [MODIFY] ` 
 : Declarations of input and output variables 
 `    * [OUTPUT] ` 
 : the following, output variable declarations 
 `    * [INPUT] ` 
 : The following, Declarations of Input Variables 
 `    * [ENTRY OUTPUT] ` 
 : Declare output variables in the entry... 
 `    * [INTERNAL WORK] ` 
 : Declarations of internal work variables 
 `*    [INTERNAL SAVE]` 
 Declarations of internal variables (which should be kept after RETURN) 
 `*    [INTERNAL PARAM]` 
 Declarations of internal parameters (to be read by NAMELIST, etc.) 
 `*    [ONCE]` 
\end{verbatim}

The following is the part to be done only once on the first call

Sentence numbers are assigned to each block in the thousands, I'm
guessing as structurally as possible.

\hypertarget{naming-conventions.}{%
\subsubsection{Naming Conventions.}\label{naming-conventions.}}

The names of variables, entry names, etc. must be six characters or
less.

\begin{enumerate}
\def\labelenumi{\arabic{enumi}.}
\setcounter{enumi}{1}
\item
  variable name and type mapping

\begin{verbatim}
- TAB00005:0.0 
  A-G, P-Z

- TAB00005:0.1 
\end{verbatim}

  Floating point number ({\texttt{REAL*8}})

\begin{verbatim}
- TAB00005:1.0 
  H

- TAB00005:1.1 
\end{verbatim}

  String ({\texttt{CHARACTER}})

\begin{verbatim}
- TAB00005:2.0 
  I-N.

- TAB00005:2.1 
\end{verbatim}

  Integer ({\texttt{INTEGER}})

\begin{verbatim}
- TAB00005:3.0 
  O

- TAB00005:3.1 
\end{verbatim}

  Logical type ({\texttt{LOGICAL}})
\end{enumerate}

However, in the variables read by NAMELIST, This may not be met.

Conventions on the correspondence between variable names and contents

\begin{verbatim}
   - TAB00006:0.0 
\end{verbatim}

Prefix:

\begin{verbatim}
   - TAB00006:0.1 
     GA
 
   - TAB00006:0.2 
\end{verbatim}

The grid point state quantity (\(t\))

\begin{verbatim}
   - TAB00006:1.0
 
   - TAB00006:1.1 
     GB
 
   - TAB00006:1.2 
\end{verbatim}

Grid point state quantity (\(t-\Delta t\))

\begin{verbatim}
   - TAB00006:2.0
 
   - TAB00006:2.1 
     GD
 
   - TAB00006:2.2 
\end{verbatim}

Grid State Quantity (for common use)

\begin{verbatim}
   - TAB00006:3.0
 
   - TAB00006:3.1 
     GT
 
   - TAB00006:3.2 
\end{verbatim}

Time differential term of the grid state quantity

\begin{verbatim}
   - TAB00006:4.0
 
   - TAB00006:4.1 
     WD
 
   - TAB00006:4.2 
\end{verbatim}

Spectral representation of state quantities

\begin{verbatim}
   - TAB00006:5.0
 
   - TAB00006:5.1 
     WT
 
   - TAB00006:5.2 
\end{verbatim}

Spectral representation of the time differential term of the state
quantity

\begin{verbatim}
   - TAB00006:6.0
 
   - TAB00006:6.1 
     I
 
   - TAB00006:6.2 
\end{verbatim}

Longitude index

\begin{verbatim}
   - TAB00006:7.0
 
   - TAB00006:7.1 
     J
 
   - TAB00006:7.2 
\end{verbatim}

Index of latitude

\begin{verbatim}
   - TAB00006:8.0
 
   - TAB00006:8.1 
     K
 
   - TAB00006:8.2 
\end{verbatim}

Index for vertical level

\begin{verbatim}
   - TAB00006:9.0
 
   - TAB00006:9.1 
     IJ
 
   - TAB00006:9.2 
\end{verbatim}

An index of all the latitudes and longitudes in one place

\begin{verbatim}
   - TAB00006:10.0
 
   - TAB00006:10.1 
     NM.
 
   - TAB00006:10.2 
\end{verbatim}

Index of the spectrum

\begin{verbatim}
   - TAB00006:11.0
 
   - TAB00006:11.1 
     NM.
 
   - TAB00006:11.2 
     NAMELIST Name
 
   - TAB00006:12.0
 
   - TAB00006:12.1 
     COM
 
   - TAB00006:12.2 
     COMMON Name
 
   - TAB00006:13.0 
\end{verbatim}

Tangent:

\begin{verbatim}
   - TAB00006:13.1 
     U
 
   - TAB00006:13.2 
\end{verbatim}

east-west wind

\begin{verbatim}
   - TAB00006:14.0
 
   - TAB00006:14.1 
     V
 
   - TAB00006:14.2 
\end{verbatim}

north-south wind

\begin{verbatim}
   - TAB00006:15.0
 
   - TAB00006:15.1 
     T
 
   - TAB00006:15.2 
\end{verbatim}

temperature

\begin{verbatim}
   - TAB00006:16.0
 
   - TAB00006:16.1 
     PS
 
   - TAB00006:16.2 
\end{verbatim}

surface pressure

\begin{verbatim}
   - TAB00006:17.0
 
   - TAB00006:17.1 
     Q
 
   - TAB00006:17.2 
\end{verbatim}

Specific humidity, various tracers

\begin{verbatim}
   - TAB00006:18.0
 
   - TAB00006:18.1 
     QL
 
   - TAB00006:18.2 
\end{verbatim}

cloud liquidity

\begin{verbatim}
   - TAB00006:19.0
 
   - TAB00006:19.1 
     FLX,FLUX
 
   - TAB00006:19.2 
\end{verbatim}

flux density

\begin{verbatim}
   - TAB00006:20.0
 
   - TAB00006:20.1 
     MTX
 
   - TAB00006:20.2 
\end{verbatim}

Matrices for implicit solutions

\begin{verbatim}
   - TAB00006:21.0
 
   - TAB00006:21.1 
     MAX
 
   - TAB00006:21.2 
\end{verbatim}

data length

\begin{verbatim}
   - TAB00006:22.0
 
   - TAB00006:22.1 
     DIM
 
   - TAB00006:22.2 
\end{verbatim}

Size of the array region

For file names, The first letter is unified to the first letter of the
directory. (However, {\texttt{include}} can be used in \&lt
;span\textgreater{}\texttt{z}). Also, {\texttt{-admn}} ( administer)
indicates the main module in it.

\textless! -- end list --\textgreater!

The reason for the use of two characters here, not just
\texttt{!\ I\ use\ two\ letters\ as\ well\ as}! Systems that use other
end-of-line comment formats (e.g.~HITAC VOS3) to ensure substitution
for, and The reason for this is that Sun's CPP will malfunction if you
use only
\texttt{!\ is\ because\ Sun\textquotesingle{}s\ CPP\ will\ malfunction\ if\ there\ is\ only}!
