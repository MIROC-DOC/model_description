\hypertarget{reference-list}{%
\section{Reference list}\label{reference-list}}

Arakawa, A. and W.H. Schubert, 1974: Interactions of cumulus cloud
ensemble with the large-scale environment. Part I. {\emph{J. Atmos.
Sci.,}} {\textbf{31,}} 671--701.

Arakawa A., Suarez M.J., 1983: Vertical differencing of the primitive
equations in sigma coodinates. {\emph{Mon. Weather Rev.}},
{\textbf{111}}, 34--45.

Bourke, W., 1988: Spectral methods in global climate and weather
prediction models. {\emph{in Physically-Based Modelling and Simulation
of Climate and Climatic Change. Part I.}}, 169--220., Kluwer.

Haltiner, G.J. and R.T. Williams, 1980: Numerical Prediction and Dynamic
Meteorology (2nd ed.), John Wiley \& Sons, 477pp.

Kondo J., 1993: A new bucket model for predicting water content in the
surface soil layer. {\emph{J. Japan Soc. Hydrol. Water Res.}},
{\textbf{6}}, 344-349. (in Japanese)

Le Treut H. and Z.-X. Li, 1991: Sensitivity of an atmospheric general
circulation model to prescribed SST changes: feedback effects associated
with the simulation of cloud optical properties. {\emph{Climate
Dynamics}}, {\textbf{5}}, 175-187.

Louis, J., 1979: A parametric model of vertical eddy fluxes in the
atmosphere. {\emph{Bound. Layer Meteor.}}, {\textbf{17}}, 187--202.

Louis, J., M. Tiedtle, J.-F. Geleyn, 1982: A short history of the PBL
parameterization at ECMWF. {\emph{Workshop on Planetary Boundary layer
Parameterization}}, 59-80, ECMWF, Reading U.K.

Manabe, S., J. Smagorinsky and R.F. Strickler, 1965: Simulated
climatology of a general circulation model with a hydrologic cycle.
{\emph{Mon. Weather Rev.}} , {\textbf{93}}, 769--798.

Miller, M.J., A.C.M. Beljaars and T.N. Palmer, 1992: The sensitivity of
the ECMWF model to the parameterization of evaporation from the tropical
oceans. {\emph{J. Climate}}, {\textbf{5}}, 418-434.

Moorthi S. and M.J. Suarez, 1992: Relaxed Arakawa-Scubert: A
parameterization of moist convection for general circulation models.
{\emph{Mon. Weather Rev.,}} {\textbf{120}} 978--1002.

Mellor, G.L. and T. Yamada, 1974: A hierarchy of turbulence closure
models for planetary boundary layers. {\emph{J. Atmos. Sci.}},
{\textbf{31}}, 1791--1806.

Mellor, G.L. and T. Yamada, 1982: Development of a turbulence closure
model for geophysical fluid problems. {\emph{Rev.~Geophys Atmos.
Phys.}}, {\textbf{20}}, 851--875.

Nakajima T. and M. Tanaka, 1986: Matrix formulation for the transfer of
solar radiation in a plane-parallel scattering atmosphere. {\emph{J.
Quant. Spectrosc. Radiat. Transfer}}, {\textbf{35}}, 13-21.

Randall D.A., Pan D-.M., 1993: Imprementation of the Arakawa-Schubert
cumulus parameterization with a prognostic closure. {Meteorological
Monograph.}
