\hypertarget{radiant-flux.}{%
\subsection{Radiant Flux.}\label{radiant-flux.}}

\hypertarget{summary-of-radiation-flux-calculations}{%
\subsubsection{Summary of Radiation Flux
Calculations}\label{summary-of-radiation-flux-calculations}}

The CCSR/NIES AGCM radiation calculation scheme is , Discrete Ordinate
Method and It was created based on the k-Distribution Method. by gases
and clouds/aerosols Considering the absorption, emission, and scattering
processes of solar and terrestrial radiation, Calculate the value of the
radiation flux at each level. The main input data are temperature \(T\),
specific humidity \(q\), cloud cover \(l\), and cloud cover \(C\), The
output data are upward and downward radiation fluxes, \(F^-, F^+\), and
the differential coefficient of upward radiation flux with respect to
surface temperature This is a \(dF^-/dT_g\).

The calculation is done for several wavelengths. Each wavelength range
is based on the k-distribution method, It is further divided into
several sub-channels. As for gaseous absorption, Band Absorption in
H\(_2\)O, CO\(_2\), O\(_3\), N\(_2\)O, CH\(_4\) and , Continuous
absorption of H\(_2\)O, O\(_2\), O\(_3\) and CFC absorption is
incorporated. As for the scattering, we can use Rayleigh scattering of
gases and Scattering by cloud and aerosol particles is incorporated.

The outline of the calculation procedure is as follows (subroutine names
in parentheses).

\begin{enumerate}
\def\labelenumi{\arabic{enumi}.}
\item
  calculate the Planck function from atmospheric temperature
  \texttt{MODULE:{[}PLANKS{]}}.
\item
  in each subchannel, Calculates the optical thickness due to gas
  absorption \texttt{MODULE:{[}PTFIT{]}}.
\item
  by continuous absorption and CFC absorption calculate the optical
  thickness \texttt{MODULE:{[}CNTCFC{]}}.
\item
  of Rayleigh scattering and particle scattering Calculates the optical
  thickness and scattering moment \texttt{MODULE:{[}SCATMM{]}}.
\item
  from the optical thickness and solar zenith angle of the scattering,
  Seeking sea-level albedo \texttt{MODULE:{[}SSRFC{]}}.
\item
  for each sub-channel, Expand the Planck function by optical thickness
  \texttt{MODULE:{[}PLKEXP{]}}.
\item
  for each sub-channel, Calculates the transmission coefficient,
  reflection coefficient, and source function for each layer
  \texttt{MODULE:{[}TWST{]}}
\item
  by adding method, the calculate the radiation flux
  \texttt{MODULE:{[}ADDING{]}}
\end{enumerate}

To account for the partial coverage of clouds, The transmission
coefficients, reflection coefficients and source functions for each
layer are Calculated separately for cloud cover and cloud-free
conditions, Multiply the cloud cover by the weight of the cloud cover
and take the average. We also consider the cloud cover of the cumulus
clouds. In addition, we also perform several additions and calculate the
clear-sky radiation flux.

\hypertarget{wavelengths-and-subchannels.}{%
\subsubsection{Wavelengths and
Subchannels.}\label{wavelengths-and-subchannels.}}

The basics of radiative flux calculations are , Beer-Lambert's Law

\begin{eqnarray}
  F^\lambda(z) = F^\lambda(0) exp (-k^\lambda z)
\end{eqnarray}

. where \(F^\lambda\) is the radiant flux density at the wavelength of
\(\lambda\). \(k^\lambda\) is the absorption coefficient. In order to
calculate the radiative fluxes related to the heating rate, we need to
calculate the An integration operation with respect to the wavelength is
required.

\begin{eqnarray}
  F(z) = \int F^\lambda(z) d \lambda 
 = \int F^\lambda(0) exp (-k^\lambda z) d \lambda
\end{eqnarray}

\begin{quote}
\protect\hypertarget{p-rad:beer}{}{\blazer[p-rad:beer]}
\end{quote}

However, the absorption and emission of radiation by gas molecules is
not Due to the complicated wavelength dependence of the absorption line
structure of the molecule, It is not easy to evaluate this integration
precisely. The method designed to make the relatively precise
calculation easier is It is a k-distribution method. Within a certain
wavelength range, the absorption coefficient of \(k\) Consider the
density function \(F(k)\) for \(\lambda\),
(\protect\hyperlink{p-rad:beer}{p-rad:beer{]}})

\begin{eqnarray}
 \int F^\lambda(0) exp (-k^\lambda z) d \lambda 
 \simeq \int \bar{F}^k(0) exp (-k z) F(k) dk
\end{eqnarray}

which is approximated by where \(bar{F}^k(0)\) is In the \(z=0\), the
absorption coefficient in this wavelength range has a value of \(k\) It
is a flux averaged over a wavelength. This expression shows that
\(\bar{F}_k, F(k)\) are the same as the \(k\) If you have a relatively
smooth function,

\begin{eqnarray}
 \int F^\lambda(0) exp (-k^\lambda z) d \lambda 
 \simeq \sum \bar{F}^i(0) exp (-k^i z) F^i
\end{eqnarray}

\begin{quote}
\protect\hypertarget{p-rad:beer-kd}{}{ }
\end{quote}

with the addition of a finite number of exponential terms (subchannels),
such that It is possible to calculate relatively precisely. This method
is furthermore , It has the advantage that it is easy to consider
absorption and scattering at the same time.

In the CCSR/NIES AGCM , By changing the radiation parameter data, the
Calculations can be performed at various wavelengths. Not currently used
as a standard, The wavelength range is divided into 18 parts. In
addition, each wavelength range is divided into one to six sub-channels
(corresponding to the \(i\) in the above formula), There will be 37
channels in total. The wavelength range is a wavenumber (cm\(^{-1}\))
50, 250, 400, 550, 770, 990, 1100, 1400, 2000, 2500, 4000, 14500, 31500,
33000, 34500, 36000, 43000, 46000, 50000 Divided by.

\hypertarget{calculating-the-planck-function-moduleplanks}{%
\subsubsection{\texorpdfstring{Calculating the Planck function
\texttt{MODULE:{[}PLANKS{]}}}{Calculating the Planck function MODULE:{[}PLANKS{]}}}\label{calculating-the-planck-function-moduleplanks}}

The Planck function \(\overline{B}^w(T)\) integrated in each wavelength
range is, Evaluate by the following formula.

\begin{eqnarray}
  \overline{B}^w(T) 
   = \lambda^{-2} T \exp \left\{ \sum_{n=0}{4} B^w_n (\bar{\lambda}^w T)^{-n}
                         \right\}
\end{eqnarray}

\(\bar{\lambda}^w\) is the average wavelength of the wavelength range,
\(B^w_n\) is a parameter defined by function fitting. This is the
atmospheric temperature of each layer, \(T_l\), and the boundary
atmospheric temperature of each layer, \(T_{l+1/2}\) and surface
temperature \(T_g\).

In the following, the index \(w\) is basically abbreviated for the
wavelength range.

\hypertarget{calculating-the-optical-thickness-by-gas-absorption-moduleptfit}{%
\subsubsection{\texorpdfstring{Calculating the optical thickness by gas
absorption
\texttt{MODULE:{[}PTFIT{]}}}{Calculating the optical thickness by gas absorption MODULE:{[}PTFIT{]}}}\label{calculating-the-optical-thickness-by-gas-absorption-moduleptfit}}

The optical thickness of the gas absorption is determined by using the
index \(m\) as the type of molecule, It looks like the following.

\begin{eqnarray}
  \tau^g = \sum_{m=1}{N_m} k^{(m)} C^{(m)}
\end{eqnarray}

where \(k^{(m)}\) is the absorption coefficient of the molecule \(m\),
which is different for each subchannel.

\begin{eqnarray}
 k^{(m)} = \exp\left\{ \sum_{i=0}{N_i} \sum_{j=0}{N_j} A^{(m)}_{ij}
                   (\ln p)^{i} (T-T_{STD})^{j}
               \right\}
\end{eqnarray}

as a function of temperature \(T\)(K) and atmospheric pressure
\(p\)(hPa). \(C^{(m)}\) is the amount of gas in the layer represented by
mol cm\(^{-2}\), Volume Mixing Ratio \(r\) (in ppmv) to ,

\begin{eqnarray}
  C = 1\times 10^{-5} \frac{p}{R_u T} \Delta z \cdot r
\end{eqnarray}

And it can be calculated that . Note that \(R_u\) is the gas constant
per mole (8.31 J mol\(_{-1}\) K\(^{-1}\)), The unit of air layer
thickness \(\Delta z\) is in km. The volume mixing ratio \(r\) at ppmv
is Mass Mixture Ratio \(q\) to ,

\begin{eqnarray}
  r = 10^6 R^{(m)}/R^{(air)} q = 10^6 M^{(air)}/M^{(m)}
\end{eqnarray}

This can be converted by . \(R^{(m)},R^{(air)}\) are Gas constant per
target molecule and atmospheric mass, respectively,
\(M^{(m)},M^{(air)}\) is It is the average molecular weight of the
target molecule and the atmosphere, respectively.

This calculation is done for each sub-channel and each layer.

\hypertarget{optical-thickness-by-continuous-absorption-and-cfc-absorption-modulecntcfc}{%
\subsubsection{\texorpdfstring{Optical Thickness by Continuous
Absorption and CFC Absorption
\texttt{MODULE:{[}CNTCFC{]}}}{Optical Thickness by Continuous Absorption and CFC Absorption MODULE:{[}CNTCFC{]}}}\label{optical-thickness-by-continuous-absorption-and-cfc-absorption-modulecntcfc}}

The optical thickness of the H\(_2\)O continuous absorption
\(\tau^{H_2O}\) is , Think of it as a dimer, Basically, it is evaluated
in proportion to the square of the volume mixing ratio of water vapor.

\begin{eqnarray}
\tau^{H_2O} = ( A^{H_2O} + f(T) \hat{A}^{H_2O} ) (r^{H_2O})^2 \rho \Delta z
\end{eqnarray}

The \(f(T)\) for the \(\hat{A}\) section is , The temperature dependence
of the absorption of the dimer. Furthermore, in the wavelength range
where normal gas absorption is ignored, the Incorporate a contribution
proportional to the square of the volume mixing ratio of water vapor.

The continuous absorption of O\(_2\) is assumed to be constant in the
mixing ratio,

\begin{eqnarray}
\tau^{O_2} = A^{O_2} \rho \Delta z
\end{eqnarray}

.

The continuous absorption of O\(_3\) is based on the mixing ratio
\(r^{O_3}\) and incorporates a temperature dependence,

\begin{eqnarray}
\tau^{O_3} = \sum_{n=0}{2} A^{O_3}_n r^{O_3} \frac{T}{T_{STD}}^n \rho \Delta z
\end{eqnarray}

Absorption of CFCs is considered for \(N_m\) types of CFCs,

\begin{eqnarray}
\tau^{CFC} = \sum_{m=1}{N_m} A^{CFC}_m r^{(m)} \rho \Delta z
\end{eqnarray}

The sum of these optical thicknesses is \(\tau^{CON}\).

\begin{eqnarray}
 \tau^{CON} =  \tau^{H_2O} + \tau^{O_2} + \tau^{O_3} + \tau^{CFC} 
\end{eqnarray}

This calculation is performed for each wavelength range and each layer.

\hypertarget{scattering-optical-thickness-and-scattering-moments-modulescatmm}{%
\subsubsection{\texorpdfstring{Scattering optical thickness and
scattering moments
\texttt{MODULE:{[}SCATMM{]}}}{Scattering optical thickness and scattering moments MODULE:{[}SCATMM{]}}}\label{scattering-optical-thickness-and-scattering-moments-modulescatmm}}

The optical thicknesses of Rayleigh scattering and particle dissipation
(including scattering and absorption) are

\begin{eqnarray}
\tau^{s} 
 = \left( e^R + \sum_{p=1}{N_p} e^{(p)}_m r^{(p)}\right) \rho \Delta z
\end{eqnarray}

where \(e^R\) is the dissipation coefficient of Rayleigh scattering, The
\(e^{(p)}\) is the dissipation factor of the particle \(p\), \(r^{(p)}\)
converted to standard conditions It is the volume mixing ratio of the
particle \(p\).

Here, the mass mixing ratio of cloud water from \(l\) The conversion of
cloud grains to standard state-conversion volume mixing ratios (ppmv) is
as follows.

\begin{eqnarray}
  r = 10^6 \frac{p_{STD}}{R T_{STD}}/\rho_w
\end{eqnarray}

However, \(\rho_w\) is the density of cloud particles.

On the other hand, the scattering-induced part of the optical thickness,
\(\tau_s^s\), is

\begin{eqnarray}
\tau_s^{s} 
 = \left( s^R + \sum_{p=1}{N_p} s^{(p)}_m r^{(p)}\right) \rho \Delta z
\end{eqnarray}

where \(s^R\) is the scattering coefficient of Rayleigh scattering,
\(s^{(p)}\) is the scattering coefficient for the particle \(p\).

Also, the standardized scattering moments \(g\) (asymmetry factor) and
\(f\) (forward scattering factor) were not

\begin{eqnarray}
g = \frac{1}{\tau_s} \left[
    \left( g^R + \sum_{p=1}{N_p} g^{(p)}_m r^{(p)}\right) \rho \Delta z
    \right]
\end{eqnarray}

\begin{eqnarray}
f = \frac{1}{\tau_s} \left[ 
    \left( f^R + \sum_{p=1}{N_p} f^{(p)}_m r^{(p)}\right) \rho \Delta z
    \right]
\end{eqnarray}

Here, \(g^R, f^R\) are the scattering moments of Rayleigh scattering,
\(g^{(p)}, f^{(p)}\) is the scattering moment of the particle \(p\).

This calculation is performed for each wavelength range and each layer.

\hypertarget{albedo-at-sea-level-modulessrfc}{%
\subsubsection{\texorpdfstring{Albedo at Sea Level
\texttt{MODULE:{[}SSRFC{]}}}{Albedo at Sea Level MODULE:{[}SSRFC{]}}}\label{albedo-at-sea-level-modulessrfc}}

Albedo \(\alpha_s\) at sea level is the vertical addition of the optical
thickness of the scattering Using \(<\tau^{s}>\) and the solar incidence
angle factor \(\mu_0\),

\begin{eqnarray}
  \alpha_s = \exp\left\{ \sum_{i,j} C_{ij} {\mathcal T}^j {\mu_0}^j \right\}
\end{eqnarray}

expressed as follows. However,

\begin{eqnarray}
 {\mathcal T} = ( 4 <\tau^{s}>/\mu )^{-1}
\end{eqnarray}

It is.

This calculation is done for each wavelength.

\hypertarget{total-optical-thickness.}{%
\subsubsection{Total Optical
Thickness.}\label{total-optical-thickness.}}

Gaseous band absorption, continuous absorption, Rayleigh scattering,
particle scattering and absorption All things considered, the optical
thickness is ,

\begin{eqnarray}
  \tau = \tau^g + \tau^{CON} + \tau^{s}
\end{eqnarray}

where \(\tau^g\) is different for each subchannel. Here, since
\(\tau^g\) is different for each subchannel, The calculation is done for
each sub-channel and each layer.

\hypertarget{planck-function-expansion-moduleplkexp}{%
\subsubsection{\texorpdfstring{Planck function expansion
\texttt{MODULE:{[}PLKEXP{]}}}{Planck function expansion MODULE:{[}PLKEXP{]}}}\label{planck-function-expansion-moduleplkexp}}

In each layer, the Planck function \(B\) is

\begin{eqnarray}
  B(\tau') = b_0 + b_1 \tau' + b_2 \left(\tau'\right)^2
\end{eqnarray}

and obtain the expansion coefficients \(b_0, b_1, b,2\). Here, as
\(B(0)\) \(B\) at the top of each layer (bordering the top layer), As
\(B(\tau)\), \(B\) at the bottom edge of each layer (the boundary with
the layer below), As \(B(\tau/2)\), use the \(B\) at the representative
level of each layer.

\begin{eqnarray}
  b_0  =  B(0)  \\
  b_1  =  ( 4B(\tau/2) - B(\tau) - 3B(0) )/\tau  \\
  b_2  =  2 ( B(\tau) + B(0) - 2B(\tau/2) )/\tau^2  
\end{eqnarray}

This calculation is done for each sub-channel and each layer.

\hypertarget{transmission-and-reflection-coefficients-of-each-layer-the-source-function-moduletwst}{%
\subsubsection{\texorpdfstring{Transmission and reflection coefficients
of each layer, the source function
\texttt{MODULE:{[}TWST{]}}}{Transmission and reflection coefficients of each layer, the source function MODULE:{[}TWST{]}}}\label{transmission-and-reflection-coefficients-of-each-layer-the-source-function-moduletwst}}

So far obtained, optical thickness \(\tau\), optical thickness of
scattering \(\tau^s\), Scattering Moments \(g, f\), Expansion
Coefficient for Planck Function \(b_0, b_1, b_2\), Using the solar
incidence angle factor \(\mu_0\), Assuming a uniform layer, and using
the two-stream approximation Transmission Coefficient \(R\), Reflection
Coefficient \(T\), Downward Radiation Source Function \(\epsilon^+\),
Find the upward radiation source function \(\epsilon^-\).

The single-scattering albedo \(\omega\) is,

\begin{eqnarray}
  \omega = \tau_s^s/\tau
\end{eqnarray}

The contribution from the forward scattering factor \(f\) is Corrected
Optical Thickness \(\tau^*\), The single-scattering albedo \(\omega^*\),
asymmetric factor \(g^*\) is,

\begin{eqnarray}
  \tau^*  =  \frac{\tau}{1-\omega f} \\
  \omega^*  =  \frac{(1-f)\omega}{1-\omega f}   \\
  g^*  =  \frac{g-f}{1-f}  
\end{eqnarray}

From now on, as a phase function of the normalized scattering,

\begin{eqnarray}
  \hat{P}^\pm    =  \omega^* {W^-}^2 \left( 1 \pm 3g^* \mu \right)/2 \\
  \hat{S}_s^\pm  =  \omega^* W^-     \left( 1 \pm 3g^* \mu \mu_0 \right)/2
\end{eqnarray}

However, \(\mu\) is a two-stream directional cosine, and

\begin{eqnarray}
  \mu \equiv \left\{ \begin{array}{ll}
                   1/\sqrt{3} \; \; \;  可視・近赤外域 \\
                   1/1.66     \; \; \;  赤外域
                    \end{array}
             \right.
\end{eqnarray}

\begin{eqnarray}
  W^- \equiv \mu^{-1/2}
\end{eqnarray}

Furthermore,

\begin{eqnarray}
  X  =  \mu^{-1} - (\hat{P}^+ - \hat{P}^- ) \\
  Y  =  \mu^{-1} - (\hat{P}^+ + \hat{P}^- ) \\
  \hat{\sigma}_s^{\pm}  =  \hat{S}_s^+ \pm \hat{S}_s^- \\
  \lambda  =  \sqrt{XY}
\end{eqnarray}

the reflectance \(R\) and transmission \(T\) become

\begin{eqnarray}
 \frac{A^+{\tau^*}}{A^-{\tau^*}}
   =  \frac{X (1+e^{-\lambda\tau^*}) - \lambda (1-e^{-\lambda\tau^*})}
             {X (1+e^{-\lambda\tau^*}) + \lambda (1-e^{-\lambda\tau^*})} \\
 \frac{B^+{\tau^*}}{B^-{\tau^*}}
   =  \frac{X (1-e^{-\lambda\tau^*}) - \lambda (1+e^{-\lambda\tau^*})}
             {X (1-e^{-\lambda\tau^*}) + \lambda (1+e^{-\lambda\tau^*})}
\end{eqnarray}

\begin{eqnarray}
  R  =   \frac{1}{2} \left(  \frac{A^+{\tau^*}}{A^-{\tau^*}} 
                             + \frac{B^+{\tau^*}}{B^-{\tau^*}} \right) \\
  T  =   \frac{1}{2} \left(  \frac{A^+{\tau^*}}{A^-{\tau^*}} 
                             - \frac{B^+{\tau^*}}{B^-{\tau^*}} \right)
\end{eqnarray}

Next, we first find the source function from which the Planck function
is derived.

\begin{eqnarray}
  \hat{b}_n = 2 \pi (1-\omega^*) W^- b_n \; \; \; n=0,1,2 
\end{eqnarray}

The expansion coefficients of the radiant function can be found from

\begin{eqnarray}
  D_2^\pm  =  \frac{\hat{b}_2}{Y} \\
  D_1^\pm  =  \frac{\hat{b}_1}{Y} \mp  \frac{2 \hat{b}_2}{XY} \\
  D_0^\pm  =  \frac{\hat{b}_0}{Y} + \frac{2 \hat{b}_2}{XY^2} 
                \mp  \frac{\hat{b}_1}{XY} \\
\end{eqnarray}

\begin{eqnarray}
  D^\pm(0)       =  D_0^pm \\
  D^\pm(\tau^*)  =  D_0^pm + D_1^pm \tau^* + D_2^pm {\tau^*}^2
\end{eqnarray}

The source function \(\hat{\epsilon}_A^\pm\), which is derived from the
Planck function, is

\begin{eqnarray}
  \hat{\epsilon}_A^-  =  D^-(0) - R D^+(0) - T D^-(\tau^*) \\
  \hat{\epsilon}_A^+  =  D^+(0) - T D^+(0) - R D^-(\tau^*)
\end{eqnarray}

On the other hand, the source function of the solar-induced radiation is

\begin{eqnarray}
  Q\gamma = \frac{X\hat{\sigma}_s^+ + \mu_0^{-1} \hat{\sigma}_s^-}
                 {\lambda^2 - \mu_0^{-2} }
\end{eqnarray}

than ,

\begin{eqnarray}
  V_s^\pm = \frac{1}{2} \left[
             Q\gamma \pm \left( \frac{Q\gamma}{\mu X} 
                                + \frac{\hat{\sigma}_s^-}{X} \right)
                        \right]
\end{eqnarray}

we obtain the following by using

\begin{eqnarray}
  \hat{\epsilon}_S^-  =  V_s^- - R V_s^+ - T V_s^- e^{-\tau^*/\mu_0} \\
  \hat{\epsilon}_S^+  =  V_s^+ - T V_s^+ - R V_s^- e^{-\tau^*/\mu_0}
\end{eqnarray}

This calculation is done for each sub-channel and each layer.

\hypertarget{combinations-of-source-functions-for-each-layer.}{%
\subsubsection{Combinations of source functions for each
layer.}\label{combinations-of-source-functions-for-each-layer.}}

The Planck function origin and solar-induced origin The combined source
function is

\begin{eqnarray}
  \epsilon^\pm  = 
  \epsilon_A^\pm + \hat{\epsilon}_S^\pm e^{-<\tau^*>/\mu_0} F_0 \\
\end{eqnarray}

However, the \(<\tau^*>\) is not a good match for the upper atmosphere.
However, \(<\tau^*>\) has a value of to the top of the layer we're
considering now. It is the total optical thickness of the \(\tau^*\), It
is the incident flux in the wavelength range under consideration in
\(F_0\). In other words, \(e^{-<\tau^*>/\mu_0} F_0\) is It is the
incident flux at the top of the layer under consideration. This
calculation is actually ,

\begin{eqnarray}
  e^{-<\tau^*>/\mu_0} = \Pi' e^{-\tau^*/\mu_0}
\end{eqnarray}

The procedure is as follows. \(\Pi'\) will be taken from the uppermost
layer of the atmosphere by Represents the product up to one layer above
the layer we're considering now.

This calculation is done for each sub-channel and each layer.

\hypertarget{radiation-flux-at-each-layer-boundary-moduleadding}{%
\subsubsection{\texorpdfstring{Radiation flux at each layer boundary
\texttt{MODULE:{[}ADDING{]}}}{Radiation flux at each layer boundary MODULE:{[}ADDING{]}}}\label{radiation-flux-at-each-layer-boundary-moduleadding}}

Transmission coefficient of each layer \(R_l\), Reflection coefficient
\(T_l\), Radiation source function \(\epsilon^\pm_l\) is required in all
layers of \(l\), The radiation fluxes at each layer boundary can be
obtained by using the adding method. This means that the two layers of
\(R,T,\epsilon\) are known, The \(R,T,\epsilon\) of the whole combined
layer of the two layers can be easily calculated by It is an
exploitation of what is required . In a homogeneous layer, the
reflectance of the incident light from above, the transmission
coefficient and the It is the same as the reflectance and transmittance
of the incident light from below, Because it is different in
heterogeneous layers composed of multiple layers, The reflectance,
transmittance and transmittance of the incident light from above
(\(R^+, T^+\), \(R^+, T^+\)) Distinguish between the reflectance and the
transmittance of the incident light from below (\(R^-, T^-\)) and the
reflectance of the incident light from below (\(R^-, T^-\)). Now, in
layer 1 above and layer 2 below, these If
\(R^\pm_1, T\pm_1, \epsilon^\pm_1,  R^\pm_2, T\pm_2, \epsilon^\pm_2\)
are known, Value in the composite layer
\(R^\pm_{1,2}, T\pm_{1,2}, \epsilon^\pm_{1,2}\) is It looks like the
following.

\begin{eqnarray}
  R^+_{1,2}  =  R^+_1 + T^-_1 ( 1- R^+_2 R^-_1 )^{-1} R^+_2 T^+_1 \\
  R^-_{1,2}  =  R^-_2 + T^+_2 ( 1- R^+_1 R^-_2 )^{-1} R^-_1 T^-_2 \\
  T^+_{1,2}  =  T^+_2 ( 1- R^+_1 R^-_2 )^{-1} T^+_1 \\
  T^-_{1,2}  =  T^-_1 ( 1- R^+_1 R^-_2 )^{-1} T^-_2 \\
  \epsilon^+_{1,2}  =  \epsilon^+_2 
    + T^+_2 ( 1- R^+_2 R^-_1 )^{-1} ( R^-_1 \epsilon^-_2 + \epsilon^+_1 ) \\
  \epsilon^-_{1,2}  =  \epsilon^-_1 
    + T^-_1 ( 1- R^+_2 R^-_1 )^{-1} ( R^+_2 \epsilon^+_1 + \epsilon^-_2 ) 
\end{eqnarray}

Let's say there are layers 1, 2, \ldots{}\(N\) from the top. However,
the surface is considered to be a single layer and is the \(N\) layer.
Reflectance and source function of the layers from the \(n\) to the
\(N\) layer as a single layer Given the \(R^+_{n,N}, \epsilon^-_{n,N}\),
\(R^+_{n,N}, \epsilon^-_{n,N}\) ,

\begin{eqnarray}
  R^+_{n,N}  =  R^+_n 
      + T^-_n ( 1- R^+_{n+1,N} R^-_n )^{-1} R^+_{n+1,N} T^+_n \\
  \epsilon^-_{n,N}  =  \epsilon^-_n
    + T^-_n ( 1- R^+_{n,N} R^-_n )^{-1} 
      ( R^+_{n,N} \epsilon^+_n + \epsilon^-_{n,N} ) 
\end{eqnarray}

This is the value at the surface

\begin{eqnarray}
  R^+_{N,N}  =   R^+_N = 2 {W^+}^2 \alpha_s \\
  \epsilon^-_{N,N}  =   \epsilon^-_N 
    = W^+ \left( 2 \alpha_s \mu_0 e^{-<\tau^*>/\mu_0} F_0 
                 + 2 \pi (1-\alpha_s) B_N 
          \right)
\end{eqnarray}

It can be solved by \(n=N-1, \ldots 1\) in sequence, starting from
However,

\begin{eqnarray}
  W^+ \equiv \mu^{1/2}
\end{eqnarray}

In the next section, we consider the reflectance and source function of
the layers from the first to the \(n\) as a single layer Given the
\(R^-_{1,n}, \epsilon^+_{1,n}\), \(R^-_{1,n}, \epsilon^+_{1,n}\) ,

\begin{eqnarray}
  R^-_{1,n}  =  R^-_n 
      + T^+_n ( 1- R^+_{1,n-1} R^-_n )^{-1} R^-_{1,n-1} T^-_n \\
  \epsilon^+_{1,n}  =  \epsilon^+_n
    + T^+_n ( 1- R^+_{1,n-1} R^-_n )^{-1} 
      ( R^-_{1,n-1} \epsilon^-_n + \epsilon^+_{1,n-1} ) 
\end{eqnarray}

and this is also \(R^-_{1,1} = R^-_1, \epsilon^+_{1,1} = \epsilon^+_1\)
It can be solved by \(n=2, \ldots N\), starting from

With these , Downward flux at the boundary between layers \(n\) and
\(n+1\) \(u^+_{n,n+1}\) and upward flux \(u^-_{n,n+1}\) is , \(1\sim n\)
The combination of layers and \(n+1\sim N\) Reduced to a matter between
two layers of combined layers,

\begin{eqnarray}
 u^+_{n+1/2} = (1-R^-_{1,n} R^+_{n+1,N})^{-1}
    (\epsilon^+_{1,n} + R^-_{1,n} \epsilon^-_{n+1,N} ) \\
 u^-_{n+1/2} = R^+_{n+1,N}  u^+_{n,n+1} + \epsilon^-_{n+1,N}
\end{eqnarray}

It can be written as. However, the flux at the top of the atmosphere is
not

\begin{eqnarray}
 u^+_{1/2}  =  0 \\
 u^-_{1/2}  =  \epsilon^-_{1,N}
\end{eqnarray}

Finally, since this flux is scaled , We rescaled and added direct solar
incidence to the Find the radiation flux.

\begin{eqnarray}
  F^+_{n+1/2}  =  \frac{W^+}{\bar{W}} u^+_{n+1/2} 
                + \mu_0 e^{-<\tau^*>_{1,n}/\mu_0} F_0 \\\\
  F^-_{n+1/2}  =  \frac{W^+}{\bar{W}} u^-_{n+1/2} \\
\end{eqnarray}

This calculation is done for each sub-channel.

\hypertarget{add-in-the-flux.}{%
\subsubsection{Add in the flux.}\label{add-in-the-flux.}}

If the radiation flux \(F^\pm_c\) is found for each subchannel in each
layer, the It corresponds to a wavelength representative of the
subchannel By applying a weight (\(w_c\)) and adding them together, The
wavelength-integrated flux is found.

\begin{eqnarray}
  \bar{F}^\pm = \sum_c w_c F^\pm
\end{eqnarray}

In practice, the short wavelength range (solar region), Divided into
long wavelengths (earth's radiation region) and added together. In
addition, a part of the short wavelength region (shorter than the
wavelength of \(0.7\mu\)) The downward flux at the surface is obtained
as PAR (photosynthetically active radiation).

\hypertarget{the-temperature-derivative-of-the-flux}{%
\subsubsection{The temperature derivative of the
flux}\label{the-temperature-derivative-of-the-flux}}

To solve for surface temperature by implicit, Differential term of
upward flux with respect to surface temperature Calculating
\(dF^-/dT_g\). Therefore, the value for temperatures 1K higher than
\(T_g\) We also obtained \(\overline{B}^w(T_g+1)\) and used it to Redo
the flux calculation using the addition method, The difference from the
original value is set to \(dF^-/dT_g\). This is a meaningful value only
in the long-wavelength region (Earth's radiation region).

\hypertarget{handling-of-cloud-cover}{%
\subsubsection{Handling of cloud cover}\label{handling-of-cloud-cover}}

In the CCSR/NIES AGCM , Considering the horizontal coverage of clouds in
a single grid. There are two types of clouds

\begin{enumerate}
\def\labelenumi{\arabic{enumi}.}
\item
  stratus cloud. Diagnosed by the large scale condensation scheme
  \texttt{MODULE:{[}LSCOND{]}}. For each layer (\(n\)), the
  lattice-averaged cloud water content of \(l^l_n\) and The horizontal
  coverage factor (cloud cover) \(C^l_n\) is defined.
\item
  cumulus clouds. Diagnosed by the cumulus convection scheme
  \texttt{MODULE:{[}CUMLUS{]}}. For each layer (\(n\)) the
  lattice-averaged cloud water content \(l^c_n\) is defined, but
  Horizontal coverage (cloud cover) \(C^c\) shall be constant in the
  vertical direction.
\end{enumerate}

In these treatments, we assume that the stratocumulus clouds overlap
randomly in a vertical direction, Assuming that the cumulus cloud always
occupies the same area in the upper and lower layers (Assume that the
cloud cover is 0 or 1 if it is confined to that region). To do so, we
perform the following calculations.

\begin{enumerate}
\def\labelenumi{\arabic{enumi}.}
\item
  optical thickness of Rayleigh and particle scattering/absorption, etc.
  \(\tau^s, \tau_s^s, g, f\),

  \begin{enumerate}
  \def\labelenumii{\arabic{enumii}.}
  \item
    when cloud water of the \(l^l_n/C^l_n\) exists (stratocumulus)
  \item
    when there are no clouds at all
  \item
    when cloud water in the cloud cover of \(l^c_n/C^c\) is present
    (cumulus clouds)
  \end{enumerate}
\end{enumerate}

Calculate for.

\begin{enumerate}
\def\labelenumi{\arabic{enumi}.}
\setcounter{enumi}{1}
\item
  reflection and transmission coefficients for each layer, The radiant
  source function (Planck function origin, insolation origin) is
  Calculate for each of the three cases above. The values for no clouds.
  \(R^\circ\), in the case of stratus clouds \(R^l\), in the case of
  cumulus clouds \(R^c\) and so on.
\item
  reflection and transmission coefficients for each layer, The source
  function is averaged with the weight of the cloud cover of the
  stratocumulus, \(C^l\). The averages are represented by \(\bar{}\),
\end{enumerate}

\begin{eqnarray}
        \bar{R}  =  ( 1 - C^l ) R^\circ + C^l R^l \\
        \bar{T}  =  ( 1 - C^l ) T^\circ + C^l T^l \\
        \bar{\epsilon}  =  
            ( 1 - C^l ) \epsilon_A^\circ + C^l \epsilon_A^l \\        
           +  
            \left[ ( 1 - C^l ) \epsilon_S^\circ + C^l \epsilon_S^l \right] 
            e^{-\overline{<\tau^*>}/\mu_0} F_0 
\end{eqnarray}

However, the However ,

\begin{eqnarray}
        e^{-\overline{<\tau^*>}/\mu_0} 
        = \Pi' \left[ ( 1 - C^l ) e^{-\tau^{*\circ}/\mu_0} 
                       + C_l e^{-\tau^{*l}/\mu_0} \right]
\end{eqnarray}

It is. Also,

\begin{eqnarray}
        \epsilon^\circ  =  \epsilon_A^\circ +
                             \epsilon_S^\circ 
                              e^{-<\tau^{*\circ}>/\mu_0} F_0 \\
        \epsilon^c      =  \epsilon_A^c +
                             \epsilon_S^c 
                              e^{-<\tau^{*c}>/\mu_0} F_0        
\end{eqnarray}

Seek also.

\begin{enumerate}
\def\labelenumi{\arabic{enumi}.}
\setcounter{enumi}{3}
\item
  when the characteristic values of the average (e.g., \(\bar{R}\)) are
  used, When using a characteristic value without clouds (e.g.,
  \(R^\circ\)), When the characteristic values of cumulus clouds (e.g.,
  \(R^c\)) are used, fluxes by adding, respectively. Find
  \(\bar{F}, F^\circ, F^c\).
\item
  the final flux we seek is
\end{enumerate}

\begin{eqnarray}
        F = ( 1 - C^c ) \bar{F} + C^c F^c
\end{eqnarray}

\begin{verbatim}
 ($F^\circ$ is used to estimate cloud radiative forcing
\end{verbatim}

I'm doing the math.)

\hypertarget{incidence-flux-and-angle-of-incidence-moduleshtins}{%
\subsubsection{\texorpdfstring{Incidence flux and angle of incidence
\texttt{MODULE:{[}SHTINS{]}}}{Incidence flux and angle of incidence MODULE:{[}SHTINS{]}}}\label{incidence-flux-and-angle-of-incidence-moduleshtins}}

Incident Flux \(F_0\) is , Solar constant, \(F_{00}\), The distance
between the sun and the earth, The ratio of the ratio to the time
average is \(r_s\).

\begin{eqnarray}
F_0 = F_00 r_s^-2 
\end{eqnarray}

Here, \(r_s\) asks for the following.

\begin{eqnarray}
  M \equiv 2 \pi ( d - d_0 ) 
\end{eqnarray}

As ,

\begin{eqnarray}
  r_s = a_0 - a_1 \cos M - a_2 \cos 2M - a_3 \cos 3M
\end{eqnarray}

Note that \(d\) is the time in days since the beginning of the year.

The angle of incidence is obtained as follows. Solar angle position
\(\omega_s\)

\begin{eqnarray}
  \omega_s = M + b_1 \sin M + b_2 \sin 2M + b_3 \sin 3M
\end{eqnarray}

As the solar declination \(\delta_s\) is

\begin{eqnarray}
  \sin \delta_s = \sin \epsilon \sin ( \omega_s - \omega_0 ) 
\end{eqnarray}

Then the angle of incidence factor \(\mu = \cos \zeta\) (where \(\zeta\)
is the zenith angle) is

\begin{eqnarray}
\mu = \cos \zeta = \cos \varphi \cos \delta_s \cos h
                 + \sin \varphi \sin \delta_s
\end{eqnarray}

\(\varphi\) is a latitude, \(h\) is the time angle (local time minus
\(\pi\)).

Assuming that the eccentricity of the Earth's orbit is \(e\) (Katayama,
1974),

\begin{eqnarray}
   a_0  =   1 + e^2 \\
   a_1  =   e - 3/8 e^3 - 5/32 e^5 \\
   a_2  =   1/2 e^2 - 1/3e^4 \\
   a_3  =   3/8 e^3 - 135/64^5 \\
   b_1  =  2e - 1/4 e^3 + 5/96 e^5 \\
   b_2  =  5/4 e^2 - 11/24 e^4 \\
   b_3  =  13/12 e^3 - 645/940 e^5 \\
\end{eqnarray}

It is also possible to give average annual insolation. In this case, the
annual mean incidence and the annual mean angle of incidence are It
approximates to be as follows.

\begin{eqnarray}
\overline{F} = F_{00}/\pi
\end{eqnarray}

\begin{eqnarray}
\overline{\mu} \simeq 0.410 + 0.590 \cos^2 \varphi .
\end{eqnarray}

\hypertarget{other-notes.}{%
\subsubsection{Other Notes.}\label{other-notes.}}

The calculation of the radiation is usually not done at every step. To
do so, we have to save the radiation flux, If the time is not used for
radiation calculation, it is used. As for the shortwave radiation,
Percentage of time (time that is \(\mu_0>0\)) between next calculation
time (\(f\)) and Using the solar incidence angle factor
(\(\bar{\mu_0}\)) averaged over the daylight hours Seeking Flux
\(\bar{F}\),

\begin{eqnarray}
        F =  f \frac{\mu_0}{\bar{\mu_0}} \bar{F}
\end{eqnarray}

.

\begin{enumerate}
\def\labelenumi{\arabic{enumi}.}
\setcounter{enumi}{1}
\tightlist
\item
  cloud water depends on the temperature, Treated as water and ice cloud
  particles. Percentage treated as ice clouds \(f_I\) is ,
\end{enumerate}

\begin{eqnarray}
        f_I = \frac{ T_0 - T }{ T_0 - T_1 }
\end{eqnarray}

\begin{verbatim}
 (but with a maximum value of 1 and a minimum value of 0). Also,
 $T_0 = 273.15{K}, T_1 = 258.15{K}$.
\end{verbatim}
