\hypertarget{radiant-flux.}{%
\subsection{Radiant Flux.}\label{radiant-flux.}}
\hypertarget{summary-of-radiation-flux-calculations}{%
\subsubsection{Summary of Radiation Flux
Calculations}\label{summary-of-radiation-flux-calculations}}

The CCSR/NIES AGCM radiation scheme was created based on the Discrete 
Ordinate Method and the k-Distribution Method. The scheme calculate 
the value of the radiation flux at each level by taking into acount 
solar radiatin by gases and clouds/aerosols and the absorption, emission, 
and scattering processes of terrestrial radiation. The main input data are 
temperature \(T\), specific humidity \(q\), cloud cover \(l\), and cloud 
cover \(C\). The output data are upward and downward radiation fluxes \(F^-, F^+\) 
and the differential coefficient of upward radiation flux with respect to 
surface temperature \(dF^-/dT_g\).

The calculation is separated for several wavelengths.It is further divided 
into several sub-channels, based on the k-distribution method. As for gaseous 
absorption, line absorption bands in H\(_2\)O, CO\(_2\), O\(_3\), N\(_2\)O, CH\(_4\) 
and continuous absorption bans of H\(_2\)O, O\(_2\), O\(_3\) and CFC absorption is 
incorporated. As for scattering, Rayleigh scattering of gases and scattering by 
cloud and aerosol particles is incorporated.

The outline of the calculation procedure is as follows (subroutine names
in parentheses).

\begin{enumerate}
\def\labelenumi{\arabic{enumi}.}
\item
  Calculate the Planck function from atmospheric temperature
  \texttt{{[}PLANKS{]}}.
\item
  Calculates the optical thickness due to gas in each subchannel
  absorption \texttt{{[}PTFIT{]}}.
\item
  Calculate the optical thickness by continuous absorption and CFC absorption 
\texttt{{[}CNTCFC{]}}.
\item
   Calculate the optical thickness of Rayleigh scattering and particle scattering 
and scattering moment \texttt{{[}SCATMM{]}}.
\item
  Seeking sea-level albedo from the optical thickness and solar zenith angle of 
the scattering \texttt{{[}SSRFC{]}}.
\item
  Expand the Planck function by optical thickness for each sub-channel
  \texttt{{[}PLKEXP{]}}.
\item
  Calculate the transmission coefficient, reflection coefficient and source 
function for each layer for each sub-channel
  \texttt{{[}TWST{]}}
\item
  Calculate the radiation flux by adding method
  \texttt{{[}ADDING{]}}
\end{enumerate}

To account for the partial coverage of clouds, the transmission coefficients, 
reflection coefficients and source functions for each layer are calculated 
separately for cloud cover and cloud-free conditions at weighted average of 
the cloud cover. The cloud cover of the cumulus is also consider the cloud 
cover of the cumulus clouds. In addition, it also performs several additing 
and calculates the clear-sky radiation flux.

\hypertarget{wavelengths-and-subchannels}{%
\subsubsection{Wavelengths and
Subchannels}\label{wavelengths-and-subchannels}}

The basics of radiative flux calculations are represented by Beer-Lambert's Law.

\begin{eqnarray}
  F^\lambda(z) = F^\lambda(0) exp (-k^\lambda z)
\end{eqnarray}

\(F^\lambda\) is the radiant flux density at the wavelength of \(\lambda\) and \(k^\lambda\) 
is the absorption coefficient. In order to calculate the radiative fluxes related to the 
heating rate,theintegration operation with respect to the wavelength is required.

\begin{eqnarray}
  F(z) = \int F^\lambda(z) d \lambda 
 = \int F^\lambda(0) exp (-k^\lambda z) d \lambda \label{p-rad:beer}\\
\end{eqnarray}

However, it is not easy to evaluate this integration precisely because the 
absorption and emission of radiation by gas molecules have the complicated 
wavelength dependence of the absorption line attributed to the structure of 
the molecule. The k-distribution method is a method designed to make the 
relatively precise calculation easier.Within a certain wavelength range, 
considering the density function \(F(k)\) for \(\lambda\), and of the 
absorption coefficient of \(k\), \eqref{p-rad:beer} is approximated as follows, 

\begin{eqnarray}
 \int F^\lambda(0) exp (-k^\lambda z) d \lambda 
 \simeq \int \bar{F}^k(0) exp (-k z) F(k) dk
\end{eqnarray}

, by where \(\bar{F}^k(0)\) is  the absorption coefficient in this wavelength \(k\) 
the flux averaged over a wavelength having the absorption coefficient in this 
wavelength \(k\)in the \(z=0\). 

if \(\bar{F}_k, F(k)\) are a relatively smooth functions of the \(k\), this expression

\begin{eqnarray}
 \int F^\lambda(0) exp (-k^\lambda z) d \lambda 
 \simeq \sum \bar{F}^i(0) exp (-k^i z) F^i
\end{eqnarray}

\begin{quote}
\protect\hypertarget{p-rad:beer-kd}{}{ }
\end{quote}

, as such above, can be relatively precisely calculated by the addition of 
a finite number (subchannels) of exponential terms. This method has furthermore 
the advantage  easy to consider the absorption and scattering at the same time.

In the CCSR/NIES AGCM, By changing the radiation parameter data, the 
calculations can be performed at various wavelengths. In the version currently used
as a standard, the wavelength range is divided into 18 parts. In
addition, each wavelength range is divided into 1 to 6 sub-channels
(corresponding to the \(i\) in the above formula), There are 40
channels in total. The wavelength range is divided by a wavenumber (cm\(^{-1}\)),
10, 250, 530, 610, 670, 980, 1175, 1225, 2000, 2500, 3300, 6000, 10000, 23000, 
30000, 33500, 36000, 43500, and 50000. A global warming version with 29 bands and
111 chanels has been developed. Additionally, a chemical version is also with 37 
bands and 126 chanels for chemical transport model. The boundary of the SW region 
is also changed to 54000 cm\(^{-1}\).

\hypertarget{calculating-the-planck-function-moduleplanks}{%
\subsubsection{\texorpdfstring{Calculating the Planck function
\texttt{{[}PLANKS{]}}}{Calculating the Planck function {[}PLANKS{]}}}
\label{calculating-the-planck-function-moduleplanks}}

The Planck function \(\overline{B}^w(T)\) integrated in each wavelength
range is, evaluate by the following formula.

\begin{eqnarray}
  \overline{B}^w(T) 
   = \lambda^{-2} T \exp \left\{ \sum_{n=0}{4} B^w_n (\bar{\lambda}^w T)^{-n}
                         \right\}
\end{eqnarray}

\(\bar{\lambda}^w\) is the average wavelength of the wavelength range,
\(B^w_n\) is the parameter determined by function fitting. This is caluculated to the
atmospheric temperature of each layer, \(T_l\), and the boundary
atmospheric temperature of each layer, \(T_{l+1/2}\) and surface
temperature \(T_g\).

In the following, the index \(w\) is basically abbreviated for the
wavelength range.

\hypertarget{calculating-the-optical-thickness-by-gas-absorption-moduleptfit}{%
\subsubsection{\texorpdfstring{Calculating the optical thickness by gas
absorption
\texttt{{[}PTFIT{]}}}{Calculating the optical thickness by gas absorption {[}PTFIT{]}}}
\label{calculating-the-optical-thickness-by-gas-absorption-moduleptfit}}

The optical thickness of the gas absorption is expressed as follows
by using the index \(m\) as the type of molecule.

\begin{eqnarray}
  \tau^g = \sum_{m=1}{N_m} k^{(m)} C^{(m)}
\end{eqnarray}

where \(k^{(m)}\) is the absorption coefficient of the molecule \(m\),
which is different for each subchannel and determined as a function of 
temperature \(T\)(K) and atmospheric pressure \(p\)(hPa).

\begin{eqnarray}
 k^{(m)} = \exp\left\{ \sum_{i=0}{N_i} \sum_{j=0}{N_j} A^{(m)}_{ij}
                   (\ln p)^{i} (T-T_{STD})^{j}
               \right\}
\end{eqnarray}

\(C^{(m)}\) is the amount of gas in the layer represented by
mol cm\(^{-2}\). Based on the Volume Mixing Ratio \(r\) (in ppmv),

\begin{eqnarray}
  C = 1\times 10^{-5} \frac{p}{R_u T} \Delta z \cdot r
\end{eqnarray}

it can be calculated. Note that \(R_u\) is the gas constant
per mole (8.31 J mol\(_{-1}\) K\(^{-1}\)), The unit of air layer
thickness \(\Delta z\) is in km. The volume mixing ratio \(r\) at ppmv, 
from Mass Mixture Ratio \(q\),

\begin{eqnarray}
  r = 10^6 R^{(m)}/R^{(air)} q = 10^6 M^{(air)}/M^{(m)}
\end{eqnarray}

can be converted. \(R^{(m)},R^{(air)}\) are gas constant per
targetted molecule and atmospheric mass, and respectively,
\(M^{(m)},M^{(air)}\) is the average molecular weight of the
target molecule and the atmosphere.

This calculation is done for each sub-channel and each layer.

\hypertarget{optical-thickness-by-continuous-absorption-and-cfc-absorption-modulecntcfc}{%
\subsubsection{\texorpdfstring{Optical Thickness by Continuous
Absorption and CFC Absorption
\texttt{{[}CNTCFC{]}}}{Optical Thickness by Continuous Absorption and CFC Absorption {[}CNTCFC{]}}}
\label{optical-thickness-by-continuous-absorption-and-cfc-absorption-modulecntcfc}}

The optical thickness of the H\(_2\)O continuous absorption
\(\tau^{H_2O}\) is , in consideration of a dimer, Basically, evaluated
in proportion to the square of the volume mixing ratio of water vapor.

\begin{eqnarray}
\tau^{H_2O} = ( A^{H_2O} + f(T) \hat{A}^{H_2O} ) (r^{H_2O})^2 \rho \Delta z
\end{eqnarray}

The \(f(T)\) for the \(\hat{A}\) section is , the temperature dependence
of the absorption of the dimer. Furthermore, in the wavelength range
where normal gas absorption is ignored, a contribution proportional to 
the square of the volume mixing ratio of water vapor is incorporated.

The continuous absorption of O\(_2\) is assumed to be constant in the
mixing ratio.

\begin{eqnarray}
\tau^{O_2} = A^{O_2} \rho \Delta z
\end{eqnarray}

The continuous absorption of O\(_3\) is based on the mixing ratio
\(r^{O_3}\) and incorporates a temperature dependence.

\begin{eqnarray}
\tau^{O_3} = \sum_{n=0}{2} A^{O_3}_n r^{O_3} \frac{T}{T_{STD}}^n \rho \Delta z
\end{eqnarray}

Absorption of CFCs is considered for \(N_m\) types of CFCs.

\begin{eqnarray}
\tau^{CFC} = \sum_{m=1}{N_m} A^{CFC}_m r^{(m)} \rho \Delta z
\end{eqnarray}

The sum of these optical thicknesses is \(\tau^{CON}\).

\begin{eqnarray}
 \tau^{CON} =  \tau^{H_2O} + \tau^{O_2} + \tau^{O_3} + \tau^{CFC} 
\end{eqnarray}

This calculation is performed for each wavelength range and each layer.

\hypertarget{scattering-optical-thickness-and-scattering-moments-scatmm}{%
\subsubsection{\texorpdfstring{Scattering optical thickness and
scattering moments
\texttt{{[}SCATMM{]}}}{Scattering optical thickness and scattering moments {[}SCATMM{]}}}
\label{scattering-optical-thickness-and-scattering-moments-modulescatmm}}

The optical thicknesses of Rayleigh scattering and particle dissipation
(including scattering and absorption) are

\begin{eqnarray}
\tau^{s} 
 = \left( e^R + \sum_{p=1}{N_p} e^{(p)}_m r^{(p)}\right) \rho \Delta z
\end{eqnarray}

where \(e^R\) is the dissipation coefficient of Rayleigh scattering, the
\(e^{(p)}\) is the dissipation factor of the particle \(p\), the \(r^{(p)}\) is
the volume mixing ratio of the
particle \(p\) converted to standard conditions.

Here, The conversion from the mass mixing ratio of cloud water \(l\) to cloud 
grains of standard state-conversion volume mixing ratios (ppmv) is as follows.

\begin{eqnarray}
  r = 10^6 \frac{p_{STD}}{R T_{STD}}/\rho_w
\end{eqnarray}

\(\rho_w\) is the density of cloud particles.

On the other hand, the scattering-induced part of the optical thickness,
\(\tau_s^s\) is

\begin{eqnarray}
\tau_s^{s} 
 = \left( s^R + \sum_{p=1}{N_p} s^{(p)}_m r^{(p)}\right) \rho \Delta z
\end{eqnarray}

where \(s^R\) is the scattering coefficient of Rayleigh scattering,
\(s^{(p)}\) is the scattering coefficient for the particle \(p\).

Also, the standardized scattering moments \(g\) (asymmetry factor) and
\(f\) (forward scattering factor) are

\begin{eqnarray}
g = \frac{1}{\tau_s} \left[
    \left( g^R + \sum_{p=1}{N_p} g^{(p)}_m r^{(p)}\right) \rho \Delta z
    \right]
\end{eqnarray}

\begin{eqnarray}
f = \frac{1}{\tau_s} \left[ 
    \left( f^R + \sum_{p=1}{N_p} f^{(p)}_m r^{(p)}\right) \rho \Delta z
    \right]
\end{eqnarray}

Here, \(g^R, f^R\) are the scattering moments of Rayleigh scattering,
\(g^{(p)}, f^{(p)}\) is the scattering moment of the particle \(p\).

This calculation is performed for each wavelength range and each layer.

\hypertarget{albedo-at-sea-level-ssrfc}{%
\subsubsection{\texorpdfstring{Albedo at Sea Level
\texttt{{[}SSRFC{]}}}{Albedo at Sea Level {[}SSRFC{]}}}
\label{albedo-at-sea-level-modulessrfc}}

Albedo \(\alpha_s\) at sea level is expressed as follows by using the 
vertical addition of the optical thickness of the scattering \(<\tau^{s}>\) 
and the solar incidence angle factor \(\mu_0\).

\begin{eqnarray}
  \alpha_s = \exp\left\{ \sum_{i,j} C_{ij} {\mathcal T}^j {\mu_0}^j \right\}
\end{eqnarray}

However,

\begin{eqnarray}
 {\mathcal T} = ( 4 <\tau^{s}>/\mu )^{-1}
\end{eqnarray}

This calculation is done for each wavelength.

\hypertarget{total-optical-thickness}{%
\subsubsection{Total Optical
Thickness.}\label{total-optical-thickness.}}

All optical thickness including gaseous band absorption, continuous 
absorption, Rayleigh scattering, particle scattering and absorption is,

\begin{eqnarray}
  \tau = \tau^g + \tau^{CON} + \tau^{s}
\end{eqnarray}

where because \(\tau^g\) is different for each subchanne, the calculation 
is done for each sub-channel and each layer.

\hypertarget{planck-function-expansion-plkexp}{%
\subsubsection{\texorpdfstring{Planck function expansion
\texttt{{[}PLKEXP{]}}}{Planck function expansion {[}PLKEXP{]}}}
\label{planck-function-expansion-moduleplkexp}}

In each layer, the Planck function \(B\) is expanded as follows and the 
expansion coefficients \(b_0, b_1, b,2\) is obtained. 

\begin{eqnarray}
  B(\tau') = b_0 + b_1 \tau' + b_2 \left(\tau'\right)^2
\end{eqnarray}

 Here, as \(B(0)\) \(B\) at the top of each layer (bordering the top layer) 
is used, and as \(B(\tau)\), \(B\) at the bottom edge of each layer 
(the boundary with the layer below), and as \(B(\tau/2)\), the \(B\) at 
the representative level of each layer.

\begin{eqnarray}
  b_0 & = & B(0) \nonumber \\
  b_1 & = & ( 4B(\tau/2) - B(\tau) - 3B(0) )/\tau  \\
  b_2 & = & 2 ( B(\tau) + B(0) - 2B(\tau/2) )/\tau^2  \nonumber
\end{eqnarray}

This calculation is done for each sub-channel and each layer.

\hypertarget{transmission-and-reflection-coefficients-of-each-layer-the-source-function-twst}{%
\subsubsection{\texorpdfstring{Transmission and reflection coefficients
of each layer, the source function
\texttt{{[}TWST{]}}}{Transmission and reflection coefficients of each layer, 
the source function {[}TWST{]}}}
\label{transmission-and-reflection-coefficients-of-each-layer-the-source-function-moduletwst}}

Using the So far obtained, optical thickness \(\tau\), optical thickness of
scattering \(\tau^s\), scattering Moments \(g, f\), expansion
Coefficient for Planck Function \(b_0, b_1, b_2\), and solar
incidence angle factor \(\mu_0\), assuming a uniform layer, and using
the two-stream approximation, transmission Coefficient \(R\), reflection
Coefficient \(T\), downward Radiation Source Function \(\epsilon^+\), and 
the upward radiation source function \(\epsilon^-\) are founded.

The single-scattering albedo \(\omega\) is,

\begin{eqnarray}
  \omega = \tau_s^s/\tau
\end{eqnarray}

The optical thickness \(\tau^*\) corrected by the contribution from the forward scattering factor \(f\) 
and the single-scattering albedo \(\omega^*\), and asymmetric factor \(g^*\) are,

\begin{eqnarray}
  \tau^* & = & \frac{\tau}{1-\omega f} \\
  \omega^* & = & \frac{(1-f)\omega}{1-\omega f}   \\
  g^* & = & \frac{g-f}{1-f}  
\end{eqnarray}

As a phase function of the normalized scattering,

\begin{eqnarray}
  \hat{P}^\pm   & = & \omega^* {W^-}^2 \left( 1 \pm 3g^* \mu \right)/2 \\
  \hat{S}_s^\pm & = & \omega^* W^-     \left( 1 \pm 3g^* \mu \mu_0 \right)/2
\end{eqnarray}

However, \(\mu\) is a two-stream directional cosine, and

\begin{eqnarray}
  \mu \equiv \left\{ \begin{array}{ll}
                   1/\sqrt{3} \; \; \;  visible/near-infrared region \\
                   1/1.66     \; \; \;  infrared region
                    \end{array}
             \right.
\end{eqnarray}

\begin{eqnarray}
  W^- \equiv \mu^{-1/2}
\end{eqnarray}

Furthermore,

\begin{eqnarray}
  X & = & \mu^{-1} - (\hat{P}^+ - \hat{P}^- ) \\
  Y & = & \mu^{-1} - (\hat{P}^+ + \hat{P}^- ) \\
  \hat{\sigma}_s^{\pm} & = & \hat{S}_s^+ \pm \hat{S}_s^- \\
  \lambda & = & \sqrt{XY}
\end{eqnarray}

using the above formula, the reflectance \(R\) and transmission \(T\) become

\begin{eqnarray}
 \frac{A^+{\tau^*}}{A^-{\tau^*}}
   =  \frac{X (1+e^{-\lambda\tau^*}) - \lambda (1-e^{-\lambda\tau^*})}
             {X (1+e^{-\lambda\tau^*}) + \lambda (1-e^{-\lambda\tau^*})} \\
 \frac{B^+{\tau^*}}{B^-{\tau^*}}
   =  \frac{X (1-e^{-\lambda\tau^*}) - \lambda (1+e^{-\lambda\tau^*})}
             {X (1-e^{-\lambda\tau^*}) + \lambda (1+e^{-\lambda\tau^*})}
\end{eqnarray}

\begin{eqnarray}
  R  =   \frac{1}{2} \left(  \frac{A^+{\tau^*}}{A^-{\tau^*}} 
                             + \frac{B^+{\tau^*}}{B^-{\tau^*}} \right) \\
  T  =   \frac{1}{2} \left(  \frac{A^+{\tau^*}}{A^-{\tau^*}} 
                             - \frac{B^+{\tau^*}}{B^-{\tau^*}} \right)
\end{eqnarray}

Next, we find the source function from which the Planck function
is derived.

\begin{eqnarray}
  \hat{b}_n = 2 \pi (1-\omega^*) W^- b_n \; \; \; n=0,1,2 
\end{eqnarray}

The expansion coefficients of the radiant function can be found from the above formulathe.

\begin{eqnarray}
  D_2^\pm & = & \frac{\hat{b}_2}{Y} \\
  D_1^\pm & = & \frac{\hat{b}_1}{Y} \mp  \frac{2 \hat{b}_2}{XY} \\
  D_0^\pm & = & \frac{\hat{b}_0}{Y} + \frac{2 \hat{b}_2}{XY^2} 
                \mp  \frac{\hat{b}_1}{XY} \\
\end{eqnarray}

\begin{eqnarray}
  D^\pm(0)       =  D_0^pm \\
  D^\pm(\tau^*)  =  D_0^pm + D_1^pm \tau^* + D_2^pm {\tau^*}^2
\end{eqnarray}

The source function \(\hat{\epsilon}_A^\pm\), which is derived from the
Planck function, is

\begin{eqnarray}
  \hat{\epsilon}_A^- & = & D^-(0) - R D^+(0) - T D^-(\tau^*) \\
  \hat{\epsilon}_A^+ & = & D^+(0) - T D^+(0) - R D^-(\tau^*)
\end{eqnarray}

On the other hand, the source function of the solar-induced radiation is

\begin{eqnarray}
  Q\gamma = \frac{X\hat{\sigma}_s^+ + \mu_0^{-1} \hat{\sigma}_s^-}
                 {\lambda^2 - \mu_0^{-2} }
\end{eqnarray}

than, by using

\begin{eqnarray}
  V_s^\pm = \frac{1}{2} \left[
             Q\gamma \pm \left( \frac{Q\gamma}{\mu X} 
                                + \frac{\hat{\sigma}_s^-}{X} \right)
                        \right]
\end{eqnarray}

we obtain the following.

\begin{eqnarray}
  \hat{\epsilon}_S^-  =  V_s^- - R V_s^+ - T V_s^- e^{-\tau^*/\mu_0} \\
  \hat{\epsilon}_S^+  =  V_s^+ - T V_s^+ - R V_s^- e^{-\tau^*/\mu_0}
\end{eqnarray}

This calculation is done for each sub-channel and each layer.

\hypertarget{combinations-of-source-functions-for-each-layer}{%
\subsubsection{Combinations of source functions for each
layer}\label{combinations-of-source-functions-for-each-layer}}

The source function combined between the Planck function origin and solar-induced origin is

\begin{eqnarray}
  \epsilon^\pm  = 
  \epsilon_A^\pm + \hat{\epsilon}_S^\pm e^{-<\tau^*>/\mu_0} F_0 \\
\end{eqnarray}

,where the \(<\tau^*>\) is the total optical thickness adding the \(\tau^*\) to the top of 
the considered layer and \(F_0\) is the incident flux in the wavelength range under consideration. 
Thus, \(e^{-<\tau^*>/\mu_0} F_0\) is the incident flux at the top of the layer under consideration. 
This calculation is actually done as follows.

\begin{eqnarray}
  e^{-<\tau^*>/\mu_0} = \Pi' e^{-\tau^*/\mu_0}
\end{eqnarray}

\(\Pi'\) will be taken from the uppermost
layer of the atmosphere by Represents the product up to one layer above
the layer considering now.

\(\Pi'\) represents the product from the uppermost layer of the atmosphere 
to one layer above the layer considering now.

\hypertarget{radiation-flux-at-each-layer-boundary-adding}{%
\subsubsection{\texorpdfstring{Radiation flux at each layer boundary
\texttt{{[}ADDING{]}}}{Radiation flux at each layer boundary {[}ADDING{]}}}
\label{radiation-flux-at-each-layer-boundary-moduleadding}}

By using transmission coefficient \(R_l\), reflection coefficient
\(T_l\), and radiation source function \(\epsilon^\pm_l\) in all
layers \(l\), the radiation fluxes at each layer boundary can be
obtained by using the adding method. 
This means that the when two layers of
\(R,T,\epsilon\) are known, the \(R,T,\epsilon\) of the whole combined
layer of the two layers can be easily calculated. 
In a homogeneous layer, the
reflectance of the incident light and transmission
coefficient  from above are the same as those from below.
On the other hands, because it is different in
heterogeneous layers composed of multiple layers, the reflectance and 
transmittance of the incident light from above (\(R^+, T^+\), \(R^+, T^+\)) 
is distinguished with those from below (\(R^-, T^-\)). 
If \(R^\pm_1, T\pm_1, \epsilon^\pm_1,  R^\pm_2, T\pm_2, \epsilon^\pm_2\)
are known in above layer 1and below layer 2, the Value in the composite layer
\(R^\pm_{1,2}, T\pm_{1,2}, \epsilon^\pm_{1,2}\) are as folllows.  

\begin{eqnarray}
  R^+_{1,2} & = & R^+_1 + T^-_1 ( 1- R^+_2 R^-_1 )^{-1} R^+_2 T^+_1 \\
  R^-_{1,2} & = & R^-_2 + T^+_2 ( 1- R^+_1 R^-_2 )^{-1} R^-_1 T^-_2 \\
  T^+_{1,2} & = & T^+_2 ( 1- R^+_1 R^-_2 )^{-1} T^+_1 \\
  T^-_{1,2} & = & T^-_1 ( 1- R^+_1 R^-_2 )^{-1} T^-_2 \\
  \epsilon^+_{1,2} & = & \epsilon^+_2 
    + T^+_2 ( 1- R^+_2 R^-_1 )^{-1} ( R^-_1 \epsilon^-_2 + \epsilon^+_1 ) \\
  \epsilon^-_{1,2} & = & \epsilon^-_1 
    + T^-_1 ( 1- R^+_2 R^-_1 )^{-1} ( R^+_2 \epsilon^+_1 + \epsilon^-_2 ) 
\end{eqnarray}

There are layers 1, 2, \ldots{}\(N\) from the top and the surface is considered to 
be a single layer and the \(N\) layer.Given the reflectance and source function of 
the layers from the \(n\) to the \(N\) layer as a single layer \(R^+_{n,N}, 
\epsilon^-_{n,N}\),

\begin{eqnarray}
  R^+_{n,N} & = & R^+_n 
      + T^-_n ( 1- R^+_{n+1,N} R^-_n )^{-1} R^+_{n+1,N} T^+_n \\
  \epsilon^-_{n,N} & = & \epsilon^-_n
    + T^-_n ( 1- R^+_{n,N} R^-_n )^{-1} 
      ( R^+_{n,N} \epsilon^+_n + \epsilon^-_{n,N} ) 
\end{eqnarray}

This can be solved by \(n=N-1, \ldots 1\) in sequence, starting from the value 
at the surface

\begin{eqnarray}
  R^+_{N,N} & = &  R^+_N = 2 {W^+}^2 \alpha_s \\
  \epsilon^-_{N,N} & = &  \epsilon^-_N 
    = W^+ \left( 2 \alpha_s \mu_0 e^{-<\tau^*>/\mu_0} F_0 
                 + 2 \pi (1-\alpha_s) B_N 
          \right)
\end{eqnarray}

However,

\begin{eqnarray}
  W^+ \equiv \mu^{1/2}
\end{eqnarray}

Given the reflectance and source function regarded from the first to the 
\(n\) layers as a single layer \(R^-_{1,n}, \epsilon^+_{1,n}\), \(R^-_{1,n}, 
\epsilon^+_{1,n}\),

\begin{eqnarray}
  R^-_{1,n} & = & R^-_n 
      + T^+_n ( 1- R^+_{1,n-1} R^-_n )^{-1} R^-_{1,n-1} T^-_n \\
  \epsilon^+_{1,n} & = & \epsilon^+_n
    + T^+_n ( 1- R^+_{1,n-1} R^-_n )^{-1} 
      ( R^-_{1,n-1} \epsilon^-_n + \epsilon^+_{1,n-1} ) 
\end{eqnarray}

and this is also \(R^-_{1,1} = R^-_1, \epsilon^+_{1,1} = \epsilon^+_1\)
It can be solved by \(n=2, \ldots N\), starting from \(R^-_{1,1} = R^-_1, 
\epsilon^+_{1,1} = \epsilon^+_1\).

With these, downward flux at the boundary between layers \(n\) and
\(n+1\) \(u^+_{n,n+1}\) and upward flux \(u^-_{n,n+1}\) are came back to a problem between
two layers of combined layer, the combinations of layers \(1\sim n\) and \(n+1\sim N\).  
 
\begin{eqnarray}
 u^+_{n+1/2} = (1-R^-_{1,n} R^+_{n+1,N})^{-1}
    (\epsilon^+_{1,n} + R^-_{1,n} \epsilon^-_{n+1,N} ) \\
 u^-_{n+1/2} = R^+_{n+1,N}  u^+_{n,n+1} + \epsilon^-_{n+1,N}
\end{eqnarray}

can be written as above. However, the flux at the top of the atmosphere is

\begin{eqnarray}
 u^+_{1/2} & = & 0 \\
 u^-_{1/2} & = & \epsilon^-_{1,N}
\end{eqnarray}

Finally, since this flux is scaled , We rescaled and added direct solar
incidence to the find the radiation flux.

\begin{eqnarray}
  F^+_{n+1/2} & = & \frac{W^+}{\bar{W}} u^+_{n+1/2} 
                + \mu_0 e^{-<\tau^*>_{1,n}/\mu_0} F_0 \\\\
  F^-_{n+1/2} & = & \frac{W^+}{\bar{W}} u^-_{n+1/2} \\
\end{eqnarray}


This calculation is done for each sub-channel.

\hypertarget{add-in-the-flux}{%
\subsubsection{Add in the flux}\label{add-in-the-flux}}

If the radiation flux \(F^\pm_c\) is found for each subchannel in each
layer,  the wavelength-integrated flux is found by applying it a weight 
(\(w_c\)) correspondingly to a wavelength representative of the subchannel 
and adding them together.

\begin{eqnarray}
  \bar{F}^\pm = \sum_c w_c F^\pm
\end{eqnarray}

In practice, it is divided into the short wavelength range (solar region) and 
long wavelengths (earth's radiation region) and added together. 
In addition, the downward flux of a part of the short wavelength region 
(shorter than the wavelength of \(0.7\mu\)) at the surface is obtained
as PAR (photosynthetically active radiation).

\hypertarget{the-temperature-derivative-of-the-flux}{%
\subsubsection{The temperature derivative of the
flux}\label{the-temperature-derivative-of-the-flux}}

To implicitly solve for surface temperature, calculate differential term of
upward flux with respect to surface temperature \(dF^-/dT_g\). 

Therefore, we obtained the value for temperatures 1K higher than
\(T_g\) \(\overline{B}^w(T_g+1)\) and used it to redo
the flux calculation using the addition method, and the difference from the
original value is set to \(dF^-/dT_g\). 

This is a meaningful value only
in the long-wavelength region (Earth's radiation region).

\hypertarget{treatment-of-cloud-cover}{%
\subsubsection{Treatment of cloud cover}\label{treatment-of-cloud-cover}}

In the CCSR/NIES AGCM, the horizontal coverage of clouds in
a single grid is considered. There are two types of clouds as follows.

\begin{enumerate}
\def\labelenumi{\arabic{enumi}.}
\item
  Stratus cloud. Diagnosed by the large scale condensation scheme
  \texttt{{[}LSCOND{]}}. For each layer (\(n\)), the
  lattice-averaged cloud water content \(l^l_n\) and the horizontal
  coverage factor (cloud cover) \(C^l_n\) are defined.
\item
  Cumulus cloud. Diagnosed by the cumulus convection scheme
  \texttt{{[}CUMLUS{]}}. For each layer (\(n\)) the
  lattice-averaged cloud water content \(l^c_n\) is defined, but
  horizontal coverage (cloud cover) \(C^c\) is constant in the
  vertical direction.
\end{enumerate}

In these treatments, we assume that the stratocumulus clouds overlap
randomly in a vertical direction and the cumulus cloud always
occupies the same area in the upper and lower layers (the
cloud cover is 0 or 1 if it is confined to that region). In order to do that, we
perform the following calculations.

\begin{enumerate}
\def\labelenumi{\arabic{enumi}.}
\item
  Calculate for optical thickness of Rayleigh and particle scattering/absorption, etc.
  \(\tau^s, \tau_s^s, g, f\),

  \begin{enumerate}
  \def\labelenumii{\arabic{enumii}.}
  \item
    when cloud water of the \(l^l_n/C^l_n\) exists (stratocumulus)
  \item
    when there are no clouds at all
  \item
    when cloud water in the cloud cover of \(l^c_n/C^c\) is present
    (cumulus clouds)
  \end{enumerate}
\end{enumerate}


\begin{enumerate}
\def\labelenumi{\arabic{enumi}.}
\setcounter{enumi}{1}
\item
  Reflection, transmission coefficients for each layer, and radiant
  source function (Planck function origin, insolation origin) for each layer are
  calculated for each of the three cases above. The values for no clouds
  \(R^\circ\), in the case of stratus clouds \(R^l\), in the case of
  cumulus clouds \(R^c\) and so on.
\item
  Reflection, transmission coefficients for each layer, and source
  function for each layer are averaged with the weight of the cloud cover of the
  stratocumulus, \(C^l\). The averages are represented by \(\bar{}\),
\end{enumerate}

      \begin{eqnarray}
        \bar{R} & = & ( 1 - C^l ) R^\circ + C^l R^l \\
        \bar{T} & = & ( 1 - C^l ) T^\circ + C^l T^l \\
        \bar{\epsilon} & = & 
            ( 1 - C^l ) \epsilon_A^\circ + C^l \epsilon_A^l \\        
          & + & 
            \left[ ( 1 - C^l ) \epsilon_S^\circ + C^l \epsilon_S^l \right] 
            e^{-\overline{<\tau^*>}/\mu_0} F_0 
      \end{eqnarray}

However,

\begin{eqnarray}
        e^{-\overline{<\tau^*>}/\mu_0} 
        = \Pi' \left[ ( 1 - C^l ) e^{-\tau^{*\circ}/\mu_0} 
                       + C_l e^{-\tau^{*l}/\mu_0} \right]
\end{eqnarray}

Also, we seek

\begin{eqnarray}
        \epsilon^\circ  =  \epsilon_A^\circ +
                             \epsilon_S^\circ 
                              e^{-<\tau^{*\circ}>/\mu_0} F_0 \\
        \epsilon^c      =  \epsilon_A^c +
                             \epsilon_S^c 
                              e^{-<\tau^{*c}>/\mu_0} F_0        
\end{eqnarray}

\begin{enumerate}
\def\labelenumi{\arabic{enumi}.}
\setcounter{enumi}{3}
\item
  When the characteristic values of the average (e.g., \(\bar{R}\)) are
  used, when using a characteristic value without clouds (e.g.,
  \(R^\circ\)) and when the characteristic values of cumulus clouds (e.g.,
  \(R^c\)),  fluxes $\bar{F}, F^\circ, F^c$ are found, fluxes by adding, respectively.
\item
  The final flux we seek is
\end{enumerate}

\begin{eqnarray}
        F = ( 1 - C^c ) \bar{F} + C^c F^c
\end{eqnarray}

\item
(F^\circ is calculated to estimate cloud radiative forcing) 
\end{enumerate}

\hypertarget{incidence-flux-and-angle-of-incidence-shtins}{%
\subsubsection{\texorpdfstring{Incidence flux and angle of incidence
\texttt{{[}SHTINS{]}}}{Incidence flux and angle of incidence {[}SHTINS{]}}}
\label{incidence-flux-and-angle-of-incidence-moduleshtins}}

Incident Flux \(F_0\) is represented as follows

\begin{eqnarray}
F_0 = F_00 r_s^-2 
\end{eqnarray}

where \(F_{00}\) is the solar constant and \(r_s\) is the ratio of the ratio 
to the time of the distance between the sun and the earth. Also, \(r_s\) asks 
for the following.   

\begin{eqnarray}
  M \equiv 2 \pi ( d - d_0 ) 
\end{eqnarray}

\begin{eqnarray}
  r_s = a_0 - a_1 \cos M - a_2 \cos 2M - a_3 \cos 3M
\end{eqnarray}

Note that \(d\) is the time in days since the beginning of the year.

Also, the angle of incidence is obtained as follows. Solar angle position 
\(\omega_s\) is

\begin{eqnarray}
  \omega_s = M + b_1 \sin M + b_2 \sin 2M + b_3 \sin 3M
\end{eqnarray}

If the solar declination \(\delta_s\) is

\begin{eqnarray}
  \sin \delta_s = \sin \epsilon \sin ( \omega_s - \omega_0 ) 
\end{eqnarray}

the angle of incidence factor \(\mu = \cos \zeta\) (where \(\zeta\)
is the zenith angle) is

\begin{eqnarray}
\mu = \cos \zeta = \cos \varphi \cos \delta_s \cos h
                 + \sin \varphi \sin \delta_s
\end{eqnarray}

where \(\varphi\) is a latitude and \(h\) is the time angle (local time minus
\(\pi\)).

Assuming that the eccentricity of the Earth's orbit is \(e\) (Katayama,
1974),

\begin{eqnarray}
   a_0 & = &  1 + e^2 \\
   a_1 & = &  e - 3/8 e^3 - 5/32 e^5 \\
   a_2 & = &  1/2 e^2 - 1/3e^4 \\
   a_3 & = &  3/8 e^3 - 135/64^5 \\
   b_1 & = & 2e - 1/4 e^3 + 5/96 e^5 \\
   b_2 & = & 5/4 e^2 - 11/24 e^4 \\
   b_3 & = & 13/12 e^3 - 645/940 e^5 \\
\end{eqnarray}

It is also possible to give average annual insolation. In this case, the
annual mean incidence and the annual mean angle of incidence are approximated 
as follows.

\begin{eqnarray}
\overline{F} = F_{00}/\pi
\end{eqnarray}

\begin{eqnarray}
\overline{\mu} \simeq 0.410 + 0.590 \cos^2 \varphi .
\end{eqnarray}

\hypertarget{other-notes.}{%
\subsubsection{Other Notes.}\label{other-notes.}}

The calculation of the radiation is usually not done at every step. 
Thus, the radiation flux is saved, and it is used if the time is not used for
radiation calculation. 

As for the shortwave radiation,
using the percentage of time (time is that \(\mu_0>0\)) between next calculation
time (\(f\)) and the solar incidence angle factor averaged over the daylight hours (\(\bar{\mu_0}\)),
seek the Flux \(\bar{F}\),

\begin{eqnarray}
        F =  f \frac{\mu_0}{\bar{\mu_0}} \bar{F}
\end{eqnarray}


\begin{enumerate}
\def\labelenumi{\arabic{enumi}.}
\setcounter{enumi}{1}
\tightlist
\item
  Cloud water is dependent on the temperature, and treated as water and ice cloud
  particles. Percentage treated as ice clouds \(f_I\) is,
\end{enumerate}

\begin{eqnarray}
        f_I = \frac{ T_0 - T }{ T_0 - T_1 }
\end{eqnarray}
