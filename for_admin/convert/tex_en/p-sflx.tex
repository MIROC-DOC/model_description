\hypertarget{surface-flux.}{%
\subsection{Surface Flux.}\label{surface-flux.}}

\hypertarget{overview-of-the-surface-flux-scheme}{%
\subsubsection{Overview of the Surface Flux
Scheme}\label{overview-of-the-surface-flux-scheme}}

The surface flux scheme is , due to turbulent transport in the ground
boundary layer. Assessing the flux of physical quantities between
atmospheric surfaces. The main input data are wind speed \(u, v\),
\(u, v\), temperature \(T\), and specific humidity \(q\), The output
data are the vertical fluxes of momentum, heat, water vapor and It is
the differential value for obtaining an implicit solution.

Bulk coefficients are obtained according to Louis(1979), Louis {\emph{et
al.}}(1982). However, we take into account the difference between the
momentum and heat roughness in the correction.

The outline of the calculation procedure is as follows.

\begin{enumerate}
\def\labelenumi{\arabic{enumi}.}
\item
  as the stability of the atmosphere. Richardson numbers.
\item
  calculate the bulk coefficient from Richardson number
  \texttt{MODULE:{[}PSFCL{]}}.
\item
  calculate the flux and its derivative from the bulk coefficient.
\item
  if necessary, using the required flux After taking into account the
  roughness effect of sea level, the effect of free convection, and wind
  speed correction, Do the math again.
\end{enumerate}

\hypertarget{basic-formula-for-flux-calculations}{%
\subsubsection{Basic Formula for Flux
Calculations}\label{basic-formula-for-flux-calculations}}

Surface Flux \$ Fu, Fv, F\theta, Fq\$ are Using the bulk coefficients
\(C_M, C_H, C_E\) It is expressed as follows.

\begin{eqnarray}
Fu  =  - \rho C_M |\mathbf{v}| u
\end{eqnarray}

\begin{eqnarray}
Fv  =  - \rho C_M |\mathbf{v}| v
\end{eqnarray}

\begin{eqnarray}
F\theta  = \rho c_p C_H |\mathbf{v}| ( \theta_g - \theta )
\end{eqnarray}

\begin{eqnarray}
Fq^P =  \rho C_E |\mathbf{v}| ( q_g - q )
\end{eqnarray}

However, the \(Fq^P\) is the amount of possible evaporation. The
calculation of actual evaporation is a combination of ``surface
processes'' and We will discuss this in the section ``Solving
Diffusion-Based Budget Equations for Atmospheric Surface Systems''.

\hypertarget{richardson-number.}{%
\subsubsection{Richardson Number.}\label{richardson-number.}}

The standard of stability between atmospheric surfaces, Bulk Richardson
number \(R_{iB}\) is

\begin{eqnarray}
R_{iB} = \frac{ \frac{g}{\theta_s} (\theta_1 - \theta(z_0))/z_1 }
              { (u_1/z_1)^2                                  }
       = \frac{g}{\theta_s} 
         \frac{T_1 (p_s/p_1)^\kappa - T_0 }{u_1^2/z_1} f_T .
\end{eqnarray}

Here,

\begin{eqnarray}
f_T = (\theta_1 - \theta(z_0))/(\theta_1 - \theta_0)
\end{eqnarray}

is a correction factor and is approximated from the uncorrected bulk
Richardson number, while Here, the calculation method is abbreviated.

\hypertarget{bulk-factor.}{%
\subsubsection{Bulk factor.}\label{bulk-factor.}}

The bulk coefficients \(C_M,C_H,C_E\) are Louis(1979), Louis {\emph{et
al.}}(1982). However, we take into account the difference between the
momentum and heat roughness in the correction. i.e., the roughness to
momentum, heat, and water vapor. \(z_{0M}, z_{0H}, z_{0E}\) respectively
In general, it is \(z_{0M} > z_{0H}, z_{0E}\), but heat and water vapor
are also Bulk factor for flux from the height of the \(z_{0M}\) First
find the \(\widetilde{C_H}\) and \(\widetilde{C_E}\) and then correct
them.

\begin{eqnarray}
C_M = \left\{ 
      \begin{array}{lr}
      C_{0M} [ 1 + (b_M/e_M) R_{iB} ]^{-e_M} 
          R_{iB} \geq 0 \\
      C_{0M} \left[ 1 - b_M R_{iB} \left( 1+ d_M b_M C_{0M}
                                  \sqrt{\frac{z_1}{z_{0M}}| R_{iB}|} \,
                                  \right)^{-1} \right]      
          R_{iB} < 0 \\
      \end{array} \right.
\end{eqnarray}

\begin{eqnarray}
\widetilde{C_H} = \left\{ 
      \begin{array}{lr}
      \widetilde{C_{0H}} [ 1 + (b_H/e_H) R_{iB} ]^{-e_H} 
          R_{iB} \geq 0 \\
      \widetilde{C_{0H}} \left[ 1 - b_H R_{iB} 
                                  \left( 1+ d_H b_H \widetilde{C_{0H}}
                                  \sqrt{\frac{z_1}{z_{0M}}| R_{iB}|} \,
                                  \right)^{-1} \right]      
          R_{iB} < 0 \\
      \end{array} \right.
\end{eqnarray}

\begin{eqnarray}
C_H = \widetilde{C_H} f_T 
\end{eqnarray}

\begin{eqnarray}
\widetilde{C_E} = \left\{ 
      \begin{array}{lr}
      \widetilde{C_{0E}} [ 1 + (b_E/e_E) R_{iB} ]^{-e_E} 
          R_{iB} \geq 0 \\
      \widetilde{C_{0E}} \left[ 1 - b_E R_{iB} 
                                  \left( 1+ d_E b_E \widetilde{C_{0E}}
                                  \sqrt{\frac{z_1}{z_{0M}}| R_{iB}|} \,
                                  \right)^{-1} \right]      
          R_{iB} < 0 \\
      \end{array} \right.
\end{eqnarray}

\begin{eqnarray}
C_E = \widetilde{C_E} f_q 
\end{eqnarray}

The \(C_{0M}, \widetilde{C_{0H}}, \widetilde{C_{0E}}\) are The bulk
coefficient (for fluxes from \(z_{0M}\)) at neutral,

\begin{eqnarray}
C_{0M}  =  \widetilde{C_{0H}}  =  \widetilde{C_{0E}}  = 
       \frac{k^2}{\left[\ln \left(\frac{z_1}{z_{0M}}\right)\right]^2 } .
\end{eqnarray}

Correction Factor \(f_q\) is ,

\begin{eqnarray}
  f_q = (q_1 - q(z_0))/(q_1 - q^{\ast}(\theta_0))
\end{eqnarray}

but the method of calculation is abbreviated. The coefficient is
\(( b_M, d_M, e_M ) = ( 9.4, 7.4, 2.0 ), \; ( b_H, d_H, e_H ) = ( b_E, d_E, e_E ) = ( 9.4, 5.3, 2.0 )\).

The dependence of the bulk coefficients on \(Ri_B\) is illustrated in
Fig, Figure {[}p-sflx:cm{]} (\#p-sflx:cm), Figure {[}p-sflx:ch{]}
(\#p-sflx:ch).

\hypertarget{calculating-flux.}{%
\subsubsection{Calculating Flux.}\label{calculating-flux.}}

This will calculate the flux.

\begin{eqnarray}
\hat{F}u_{1/2}  =  - \rho_{1/2} C_M |\mathbf{v}_1| u_1
\end{eqnarray}

\begin{eqnarray}
\hat{F}v_{1/2}  =  - \rho_{1/2} C_M |\mathbf{v}_1| v_1
\end{eqnarray}

\begin{eqnarray}
\hat{F}\theta_{1/2}  = \rho_{1/2} c_p C_H |\mathbf{v}_1| 
                    \left( T_0 - \sigma_1^{-\kappa} T_1 \right)
\end{eqnarray}

\begin{eqnarray}
\hat{F}q^P_{1/2}  =  \rho_{1/2} C_E |\mathbf{v}_1| 
                    \left( q^*(T_0) - q_1 \right)
\end{eqnarray}

The differential term is as follows

\begin{eqnarray}
\frac{\partial Fu_{1/2}}{\partial u_1} = \frac{\partial Fv_{1/2}}{\partial v_1} 
= - \rho_{1/2} C_M |\mathbf{v}_1|
\end{eqnarray}

\begin{eqnarray}
\frac{\partial F\theta_{1/2}}{\partial T_1} 
= - \rho_{1/2} c_p C_H |\mathbf{v}_1| \sigma_1^{-\kappa}
\end{eqnarray}

\begin{eqnarray}
\frac{\partial F\theta_{1/2}}{\partial T_0} 
= \rho_{1/2} c_p C_H |\mathbf{v}_1|
\end{eqnarray}

\begin{eqnarray}
\frac{\partial Fq_{1/2}}{\partial q_1} 
 =  - \beta \rho_{1/2} C_E |\mathbf{v}_1| 
\end{eqnarray}

\begin{eqnarray}
\frac{\partial Fq^P_{1/2}}{\partial T_0} 
 =  \beta \rho_{1/2} C_E |\mathbf{v}_1| \left( \frac{dq^*}{dT} \right)_{1/2}
\end{eqnarray}

Here, it's important to note, \(T_0\) is a quantity that is not required
at this time. Epidermal temperature is , Conditions for surface heat
balance

\begin{eqnarray}
   F\theta(T_0,T_1) + L \beta Fq^P(T_0,q_1) + FR(T_0) - Fg(T_0,G_1) = 0
\end{eqnarray}

Determined to meet. At this point, for \(T_0\), we use the one from the
previous time step for evaluation. The true flux value that meets the
surface balance is , It is determined by solving this equation in
conjunction with surface processes. In that sense, I have added
\(\hat{{}}\) to the flux above.

\hypertarget{handling-at-sea-level.}{%
\subsubsection{handling at sea level.}\label{handling-at-sea-level.}}

At sea level, we follow Miller et al.~(1992) and consider the following
two effects.

\begin{itemize}
\item
  Free convection is preeminent when the wind speed is low
\item
  The roughness of the sea surface varies with the wind speed.
\end{itemize}

The effect of free convection is calculated using the buoyancy flux
\(F_B\),

\begin{eqnarray}
  F_B = F\theta/c_p + \epsilon T_0 F_q^P
\end{eqnarray}

When I was in \(F_B >0\),

\begin{eqnarray}
  w^* = ( H_{B} F_B )^{1/3}
\end{eqnarray}

\begin{eqnarray}
  |\mathbf{v}_1| = \left( u_1^2 + v_1^2 + (w^*)^2 \right)^{1/2}
\end{eqnarray}

to be considered by making \(H_B\) corresponds to the thickness scale of
the mixing layer. The current standard value is \(H_B=2000\) m.

The roughness variation of the sea surface is represented by the
friction velocity \(u^*\)

\begin{eqnarray}
  u^* = \left( \sqrt{Fu^2 + Fv^2}/\rho \right)^{1/2}
\end{eqnarray}

with ,

\begin{eqnarray}
  Z_{0M}  =  A_M + B_M (u^*)^2/g + C_M \nu/u^* \\
  Z_{0H}  =  A_H + B_H (u^*)^2/g + C_H \nu/u^* \\
  Z_{0E}  =  A_E + B_E (u^*)^2/g + C_E \nu/u^* 
\end{eqnarray}

Evaluate as follows. \(\nu=1.5\times10^{-5}\) m\(^2\) s\(^{-1}\) is It
is the kinematic viscosity of the atmosphere, The standard values for
the other coefficients are \$(A\_M, B\_M, C\_M) = (0, 0.018, 0.11) \$,
\$(A\_H, B\_H, C\_H) = (1.4\times10\^{}\{-5\}, 0, 0.4) \$, \$(A\_E,
B\_E, C\_E) = (1.3\times10\^{}\{-4\}, 0, 0.62) \$.

For the above calculations, \(Fu, Fv, F\theta, Fq\) are required,
Perform successive approximation calculations.

\hypertarget{wind-speed-correction}{%
\subsubsection{Wind Speed Correction}\label{wind-speed-correction}}

In general, the roughness of the ground surface is greater on large
surfaces than on small surfaces. The downward transport of momentum is
so efficient that the wind just above it is weak, The difference in wind
speed cancels out the difference in roughness in \(C_D\).

In the model, the wind speed passed to the surface flux calculation is
It is a value calculated by time integration of the mechanical
processes, The values are smoothed by spectral expansion. This is the
reason why the surface of the ground with widely different roughnesses,
such as sea level and land level, is In an area that is mixed on a small
scale , I can't describe this compensation effect well. Therefore, once
the momentum flux is calculated and The wind speed in the lowest layer
of the atmosphere is corrected for by it, and then Recalculate the
momentum, heat, and water fluxes again.

\hypertarget{minimum-wind-speed.}{%
\subsubsection{Minimum wind speed.}\label{minimum-wind-speed.}}

Consider the effects of small-scale exercise, Surface wind speed in the
calculation of surface flux Set the minimum value of \(|\mathbf{v}_1|\).
The current standard values are the same for all fluxes and It is 3 m/s.
