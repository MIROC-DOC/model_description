%\documentstyle[a4j,dennou]{jarticle}

\subsection{モデルの機能と構造}
\subsubsection{モデルの基本的な機能}

CCSR/NIES AGCMは, 全球3次元大気を物理的法則に基づいて記述し, 
初期値問題として系の時間発展を計算する数値モデルである.

入力するデータとしては, 以下のようなものがある.
\begin{itemize}
\item 各予報変数(水平風速,温度,地表気圧,比湿,雲水量,各地表量) の初期値データ
\item 境界条件データ(地表標高,地表状態,海面水温など)
\item モデルの各種パラメータ(大気成分, 物理過程パラメータなど)
\end{itemize}
%
一方, 出力されるものは, 以下のようなものである.
\begin{itemize}
\item 各予報変数および診断変数の, 各時刻または時間平均のデータ
\item 継続して実行を行なう場合の初期値となるデータ(再出発データ)
\item 経過と各種診断メッセージ
\end{itemize}
%
ここで, 予報変数とは, 時間発展の微分方程式を時間積分することによって
時系列として得られるデータであり,
診断変数とは, 予報変数と境界条件とパラメータから
時間積分を含まない何らかの方法により計算される量である.

より具体的に言えば,
モデルは, 基本的に以下のような方程式(予報方程式)の解を求めている.

\begin{eqnarray}
  \DP{u}{t} & = & \left( {\cal F}_x \right)_D + \left( {\cal F}_x \right)_P 
  \label{struct:u-eq-1} \\
  \DP{v}{t} & = & \left( {\cal F}_y \right)_D + \left( {\cal F}_y \right)_P \\
  \DP{T}{t} & = & \left( Q \right)_D + \left( Q \right)_P \\
  \DP{p_S}{t} & = & \left( M \right)_D + \left( M \right)_P \\
  \DP{q}{t} & = & \left( S \right)_D + \left( S \right)_P \\
  \DP{T_g}{t} & = & \left( Q_g \right)_D + \left( Q_g \right)_P 
\end{eqnarray}

ここで, $u,v,T,p_S,q,T_g$ は, 
それぞれ東西風, 南北風, 気温, 地表気圧,比湿など,地表面状態量
といった2次元または3次元分布を持つ予報変数であり,
右辺はその各予報変数の時間変化をもたらす項である.
この時間変化をもたらす項${\cal F}_x,{\cal F}_y,Q,S,Q_g$は,  
予報変数$u,v,T,p_S,q,T_g$ を元に計算されるが,
$u$,$v$で表される大気の運動による移流などの項(上式で添字$D$の項)と,
雲・放射などの各プロセスによる項(添字$P$の項)とに大別される.
前者を力学過程, 後者を物理過程と呼んで区別する.

力学過程の時間変化項の主要部分は移流項であり,
その計算においては空間微分の正確な見積りが重要である.
CCSR/NIES AGCM においては, 水平微分項の計算に
球面調和関数展開を利用している.
一方, 物理過程は水の相変化や放射の吸収射出などによるエネルギー変換, 
それらと結びついた小規模な大気運動の効果, 
地表面のさまざまなプロセスの効果などを, 
簡単なモデルでパラメータを用いて表現すること
(パラメタリゼーション)が重要となる.

予報方程式の時間積分は,
(\ref{struct:u-eq-1})などの
左辺を差分で近似することによって行なう. 例えば,
%
\begin{equation}
  \DP{q}{t} \rightarrow \frac{q^{t+\Delta t} - q^{t}}{\Delta t}
\end{equation}
%
とすることにより, 
\begin{equation}
  q^{t+\Delta t} = q^{t} 
       + \Delta t \left[ \left( S \right)_D + \left( S \right)_P  \right]
  \label{struct:sabun}
\end{equation}
となる. 
ここで, $S$ は予報変数$u,v,T,p_S,q$などの関数であるが,
その計算においてどの時刻の予報変数を用いて$S$を評価するかによって,
いろいろな時間差分スキームが考えられる.
CCSR/NIES AGCM では, 
$t$ での値をそのまま用いる Euler 法,
$t+\Delta t/2$ での値を用いる leap frog 法, 
$t+\Delta t$ における(近似的な)値を用いる implicit 法を併用している.

CCSR/NIES AGCM では
予報変数の時間積分は力学過程と物理過程とで別々に行なっている.
すなわち, まず力学の項は基本的に leap frog を用い,
\begin{equation}
  \tilde{q}^{t+\Delta t} = q^{t-\Delta t} + 2 \Delta t \left( S \right)_D^{t}
\end{equation}
を解く. ただし一部の項は implicit 扱いをしている.
物理過程では,
力学の項を積分した結果に基づき, 
Euler 法と implicit 法を併用して,
\begin{equation}
  q^{t+\Delta t} = \tilde{q}^{t+\Delta t} + 2 \Delta t \left( S \right)_P
\end{equation}
を求めている. 
ここで, (\ref{struct:sabun})の$\Delta t$を
$2 \Delta t$におきかえていることに注意.

\subsubsection{モデル実行の流れ}

モデル実行の流れを簡単に示すと, 以下のようになる.
[\ ]の中は該当するサブルーチン名である.

\begin{enumerate}
\item 実験のパラメータ, 座標などを設定する \Module{SETPAR,PCONST,SETCOR,SETZS}
\item 予報変数の初期値を読み込む \Module{RDSTRT}
\item 時間ステップを開始する \Module{TIMSTP}
\item 力学過程による時間積分を行なう \Module{DYNMCS}
\item 物理過程による時間積分を行なう \Module{PHYSCS}
\item 時刻を進める \Module{STPTIM, TFILT}
\item 必要ならデータを出力する \Module{HISTOU}
\item 必要なら再出発データを出力する \Module{WRRSTR}
\item 3. に戻る
\end{enumerate}

\subsubsection{予報変数}

予報変数は, 以下の通りである.
括弧内は座標系であり, $\lambda,\varphi,\sigma, z$ はそれぞれ,
経度, 緯度, 無次元気圧$\sigma$, 鉛直深を示す.
[\ ] 内は単位である.

\begin{tabular}{lll}
東西風速 & $u$ ($\lambda,\varphi,\sigma$) & [m/s] \\
南北風速 & $v$ ($\lambda,\varphi,\sigma$) & [m/s] \\
気温     & $T$ ($\lambda,\varphi,\sigma$) & [K] \\
地表気圧 & $p_S$ ($\lambda,\varphi$) & [hPa] \\
比湿     & $q$ ($\lambda,\varphi,\sigma$) & [kg/kg] \\
雲水混合比 &  $l$ ($\lambda,\varphi,\sigma$) & [kg/kg] \\
地中温度 & $T_g$ ($\lambda,\varphi,z$) & [K] \\
地中水分 & $W_g$ ($\lambda,\varphi,z$) & [m/m] \\
積雪量   & $W_y$ ($\lambda,\varphi$) & [kg/m$^2$] \\
海氷厚   & $h_I$ ($\lambda,\varphi$) & [m]
\end{tabular}

ただし, 海氷厚さは通常混合層結合モデルでのみ予報変数となる.
また, 地中温度も, 海氷に覆われていない海洋上にあっては
通常, 予報変数ではない.
また, CCSR/NIES AGCM では, $q$ と $l$ は独立な変数ではなく,
実際には $q+l$ が予報変数である.

このうち,
地表・地中に関する量 $T_g, W_g, W_y, h_I$ は
同時には1ステップの量のみを記憶するが,
大気に関する量 $u, v, T, p_S, q, l$ は, 
同時に2ステップ分の量を記憶する必要がある.
これは, 大気に関する量の力学過程の時間積分において
leap forg 法を用いているからである.

大気に関する量 $u, v, T, p_S, q, l$ は,
メインルーチン [AGCM5] の管理する変数である.
一方, 地表・地中に関する量 $T_g, W_g, W_y, h_I$ は
メインルーチンには現れず, 
物理過程のサブルーチン \Module{PHYSCS} が管理している.

\subsubsection{変数の時間発展の流れ}

モデルの流れを, 予報変数の時間発展を中心に簡単にまとめる.

\begin{enumerate}
\item 初期値の読み込み \Module{RDSTRT,PRSTRT}

初期値として, 大気に関する量 $u, v, T, p_S, q, l$ は, 本来,
$t$ および $t-\Delta t$ における2組の量を用意しなくてはならない.
これは, 以前のモデルの出力結果から出発する場合には用意できるが,
通常の観測値や気候値などから出発する際には用意することはできない.
その際には, 2つの時間ステップの値として同じ値から出発し,
細かい$\Delta t$を用いて計算を立ち上げる(詳しくは後述).

大気に関する量 $u, v, T, p_S, q, l$ の初期値読み込みは,
メインルーチンから呼ばれる  \Module{RDSTRT} で行なわれる.
一方, 地表・地中に関する量 $T_g, W_g, W_y, h_I$ の初期値の読み込みは
\Module{PHYSCS} から呼ばれる \Module{PRSTRT} で行なわれる.


\item 時間ステップの開始 \Module{TIMSTP}

時刻 $t$ (および一部は$t-\Delta t$)における予報変数  \\
$u^{t}, u^{t-\Delta t}, v^{t}, v^{t-\Delta t},
T^{t}, T^{t-\Delta t}, p_S^{t}, p_S^{t-\Delta t},
q^{t}, q^{t-\Delta t}, l^{t}, l^{t-\Delta t},
T_g^{t}, W_g^{t}, W_y^{t}$ \\
が揃っているものとする.

$\Delta t$ は基本的には外部から与えられるパラメータであるが,
一定時間ごとに計算の安定性の評価が行なわれ,
計算が不安定になるおそれのある場合には
$\Delta t$ を小さくする \Module{TIMSTP}.

\item 予報変数の出力の設定 \Module{AHSTIN}

大気の予報変数で, 通常出力されるのは,
この段階での, 時刻 $t$ の値
$u^{t}, v^{t}, T^{t}, p_S^{t}, q^{t}, l^{t}$
である.
実際に出力が行なわれるのは後の \Module{HISTOU} の
タイミングであるが, ここでバッファに送り込まれる.

\item 力学過程 \Module{DYNMCS} 

力学過程による予報変数の時間変化を解く. \\
$u^{t}, u^{t-\Delta t}, v^{t}, v^{t-\Delta t},
T^{t}, T^{t-\Delta t}, p_S^{t}, p_S^{t-\Delta t},
q^{t}, q^{t-\Delta t}, l^{t}, l^{t-\Delta t}$ \\
をもとに, 
力学過程のみを考慮した, $t+\Delta t$での予報変数の値 \\
$\hat{u}^{t+\Delta t}, \hat{v}^{t+\Delta t}, 
\hat{T}^{t+\Delta t}, \hat{p_S}^{t+\Delta t}, 
\hat{q}^{t+\Delta t}, \hat{l}^{t+\Delta t}$ \\
を求める.

\begin{enumerate}
\item 渦度・発散等への変換 \Module{UV2VDG, VIRTMD, HGRAD}

大気に関する予報変数 $u, v, T, p_S, q, l$ の
力学過程による変化項を見積もるために, まず
$u^{t}, v^{t}$ を渦度と発散の格子点値$\zeta^{t},D^{t}$に変換する.
これは, 力学の方程式が渦度と発散で書かれるためである.
この変換は空間微分を含むが,
球面調和関数展開を用いて行なうことにより正確に行なうことができる 
\Module{UV2VDG}.

さらに, 仮温度$T_v^{t}$ を計算し \Module{VIRTMD},
やはり球面調和関数展開を用いて
地表気圧の水平微分$\nabla \ln p_S$ を計算する \Module{HGRAD}.


\item 移流による時間変化項の計算 \Module{GRDDYN}

$u, v, T, p_S, q, l$ の $t$ における値を用いて,
水平および鉛直の移流による,
各大気変数の時間変化項の一部を計算する.
まず連続の方程式から鉛直速度 $\dot{\sigma}$ および
$p_S$ の時間変化項を診断的に求め,
それを用いて $u, v, T, q, l$ の 鉛直移流項を計算する.
さらに, $u, v, T, q, l$ の水平移流フラックスを計算する.

\item スペクトルへの変換 \Module{GD2WD, TENG2W}

大気に関する予報変数の$t-\Delta t$での格子点の値
$u^{t}, v^{t}, T^{t}, p_S^{t}, q^{t}, l^{t}$  から, 
球面調和関数展開におけるスペクトル空間での値
(ただし, 渦度発散になおしたもの) \\
$\tilde{\zeta}^{t}, \tilde{D}^{t}, \tilde{T}^{t}, 
\tilde{\pi}^{t}, \tilde{q}^{t}, \tilde{l}^{t}$ \\
を求める (ただし, $\pi \equiv \ln p_S$) \Module{GD2WD}.

さらに, $u, v, T, p_S, q, l$ の鉛直移流等による
時間変化項をスペクトルに展開する.
また, スペクトル空間での微分を利用して,
水平移流フラックスの収束を計算し, 
時間変化項のスペクトル表現に加える \Module{TENG2W}.

これによって, $\zeta, D, T, \pi, q, l$ の
時間変化項のほとんどの項がスペクトルの値として求められる.
ただし, $\zeta, D, T, \pi$ の時間変化項のうち,
水平発散$D$に線形に依存する項は
semi implicit 法によって時間積分を行なうために,
この時点では時間変化項に含まれていない.

\item 時間積分 \Module{TINTGR}

$\zeta, D, T, \pi$ の時間変化項のうち,
水平発散$D$に線形に依存する項(重力波項)を
semi implicit 法で扱い,
さらに $\zeta, D, T, q, l$ の水平拡散を
implicit で取り入れることにより
力学過程部分の時間積分を行なう. 
これによって, 力学過程のみを考慮した$t+\Delta t$の
予報値のスペクトル表現 \\
$\tilde{\zeta}^{t+\Delta t}, \tilde{D}^{t+\Delta t}, 
\tilde{T}^{t+\Delta t}, \tilde{\pi}^{t+\Delta t}, 
\tilde{q}^{t+\Delta t}, \tilde{l}^{t+\Delta t}$  \\
が求められる.

\item 格子点値への変換 \Module{GENGD}

スペクトル表現の予報変数から,
$u, v, T, p_S, q, l$の,
力学過程のみを考慮した$t+\Delta t$の
予報値の格子点値 \\
$\hat{u}^{t+\Delta t}, \hat{v}^{t+\Delta t}, 
\hat{T}^{t+\Delta t}, \hat{p_S}^{t+\Delta t}, 
\hat{q}^{t+\Delta t}, \hat{l}^{t+\Delta t}$  \\
を生成する.

\item 拡散の補正 \Module{CORDIF, CORFRC}

水平拡散は 等 $\sigma$ 面上で適用されるが,
山岳の傾斜の大きな領域では, 山を上る方向に水蒸気が輸送され,
山頂部での偽の降水をもたらすなどの問題を起こす.
それを緩和するために, 等 $p$ 面の拡散に近くなるような
補正を$T,q,l$について入れる \Module{CORDIF}. 

また, 摩擦による熱を$\hat{T}$に加える \Module{CORFRC}

\item 質量保存の補正 \Module{MASFIX}

$q$および$l$の全球積分値の保存が満たされ,
かつ$q$の負の値が無くなるように補正を行なう.
さらに, 乾燥空気の質量が一定となるような補正を行なう.

\end{enumerate}

DYNMCS を出た時点では,
$t-\Delta t$ での予報変数の値は捨てられ,
$t$ での予報変数の値で上書きされる.
$t$ の予報変数の入っていた領域には,
力学過程のみを考慮した 
$t+\Delta t$での予報変数の値が入る.

\item 物理過程 \Module{PHYSCS}

力学過程のみを考慮した, $t+\Delta t$での予報変数の値 \\
$\hat{u}^{t+\Delta t}, \hat{v}^{t+\Delta t}, 
\hat{T}^{t+\Delta t}, \hat{p_S}^{t+\Delta t}, 
\hat{q}^{t+\Delta t}, \hat{l}^{t+\Delta t}$  \\
を用いて, それに物理過程による時間変化項を加えることにより
$t+\Delta t$での予報変数の値 \\
$u^{t+\Delta t}, v^{t+\Delta t}, 
T^{t+\Delta t}, p_S^{t+\Delta t}, 
q^{t+\Delta t}, l^{t+\Delta t}$  \\
を求める.

\begin{enumerate}
\item 基本的な診断変数の計算 \Module{PSETUP}

仮温度, 各レベルでの気圧, 高度などの基本的な
診断変数を求める.

\item 積雲対流, 大規模凝結 \Module{CUMLUS, LSCOND}

積雲対流による $T, q, l$ の時間変化項を求め \Module{CUMLUS}
その項だけで \Module{GDINTG} で時間積分を行なう.
また, 大規模凝結による $T, q, l$ の時間変化項を求め \Module{LSCOND},
その項だけで \Module{GDINTG} で時間積分を行なう.
積雲対流および大規模凝結による降水量 $P_c, P_l$, 
雲量 $C_c, C_l$ などが求められる.
これによって, $T, q, l$ は,
対流凝結過程に対し調節された値
$\hat{T}^{t+\Delta t,a}, 
\hat{q}^{t+\Delta t,a}, \hat{l}^{t+\Delta t,a}$ 
となる.

\item 地表境界条件の設定 \Module{GNDSFC, GNDALB}

地表状態を与えられたデータにより設定する.
地表状態インデックス, 海面水温などが設定される \Module{GNDSFC}.
また, 海面以外の地表アルベドが設定される \Module{GNDALB}.
(海面のアルベドの計算は放射フラックスの計算ルーチンに組み込まれている.)

\item 放射フラックスの計算 \Module{RADCON, RADFLX}

放射フラックス計算に用いる大気成分データを設定する \Module{RADCON}.
通常, オゾンはファイルから読み込む.
雲水量および雲量は, 積雲対流および大規模凝結で求められたものを用いるが,
ここで外部から与えることもできる.
これらと $\hat{T}^{t+\Delta t,a}, \hat{q}^{t+\Delta t,a}$ を用いて
短波および長波の放射フラックス $F_R$, および
implicit計算に使用する地表温度に対する微分係数を求める \Module{RADFLX}.

\item 鉛直拡散フラックスの計算 \Module{VDFFLX, VFTND1}

$\hat{u}^{t+\Delta t}, \hat{v}^{t+\Delta t}, 
\hat{T}^{t+\Delta t,a}, \hat{q}^{t+\Delta t,a}, \hat{l}^{t+\Delta t,a}$ 
を用いて,
鉛直拡散過程による $u, v, T, q, l$ のフラックスと
implicit計算に使用する微分係数を求める \Module{VDFFLX}.
さらに, LU分解による implicit 解法計算を途中まで行なう \Module{VFTND1}.

\item 地表過程の計算・地中変数の時間積分 \Module{SURFCE}

地表大気間の  $u, v, T, q $ のフラックスを計算し,
地中の熱の伝導を考慮して地表でのエネルギーバランスを
implicit 解法を用いて解く.
これによって, 地表面温度 $T_s$ が診断的に求められ
地中温度の $t+\Delta t$ での値
$T_g^{t+\Delta t}$
が求められる.
さらに,第1層の大気の予報変数の時間変化率
$F_{x,1}, F_{y,1}, Q_1, S_1$ を求める.

積雪・融雪過程が考慮され, 
積雪量の$t+\Delta t$ での値 $W_y^{t+\Delta t}$ が求められ,
地中の水の移動を考慮して
地中水分 $W_g^{t+\Delta t}$ が求められる.

また, 海洋混合層モデルを用いた場合には,
海洋の温度ならびに海氷厚の 
$t+\Delta t$ での値が時間積分によって求められる.

\item 放射・鉛直拡散による時間変化の評価 \Module{VFTND2, RADTND, FLXCOR}

放射フラックスおよび鉛直拡散を総合した
大気の各予報変数の時間変化率 \\
${\cal F}_x, {\cal F}_y, Q, S$ を求める \Module{VFTND2}.
さらに, その中から放射による寄与を分離する \Module{RADTND}.
これはモデルでは直接利用しないが,
データ出力の便宜のために行なう.

これらの計算においては, implicit 法を用いているため,
地表温度および大気予報変数の変化による
フラックスの変化を考慮に入れている.
そのことを勘定に入れて収支が合うフラックスを
 \Module{FLXCOR} で計算する.
これもデータ出力の便宜のためである.

\item 重力波抵抗の評価 \Module{GRAVTY}

地形起源の重力波による大気の運動量の変化を計算し,
鉛直拡散による $u, v$ の時間変化率
${\cal F}_x, {\cal F}_y$ に加える.

\item 気圧変化項の評価 

降水と蒸発による気圧の変化を考慮し,
気圧の変化項 $M$ を求める.

\item 物理過程の時間積分 \Module{GDINTG}

以上で求められた放射, 鉛直拡散, 地表過程, 重力波抵抗等による
大気の各予報変数の時間変化率
${\cal F}_x, {\cal F}_y, Q, M, S$ を用いて,
$t+\Delta t$ での値を時間積分によって求める.

\item 乾燥対流調節 \Module{DADJST}

求められた$T, q, l$ が乾燥対流に対して不安定の場合
乾燥対流調節を施す.

\end{enumerate}

以上の手続きにより,
$t+\Delta t$での予報変数の値 \\
$u^{t+\Delta t}, v^{t+\Delta t}, 
T^{t+\Delta t}, p_S^{t+\Delta t}, 
q^{t+\Delta t}, l^{t+\Delta t}$  \\
が求められる.

\item 時間フィルタ \Module{TFILT}

leap frog による計算モードの発生を抑えるため,
時間フィルタを施す.
$u^{t-\Delta t}, u^{t}, u^{t+\Delta t}$ 
の3つの時刻のデータを用いて平滑操作を行なった結果を
$u^{t}$ におきかえる操作を各変数について行なう.
(実際には, \Module{TFILT} の段階では
$u^{t-\Delta t}$ の情報は消去されているため,
この操作は2段階で行なう.
第1段階の操作は力学過程の中で行なっている.)

\end{enumerate}
