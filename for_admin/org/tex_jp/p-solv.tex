
\subsection{大気・地表系の拡散型収支式の解法}

\subsubsection{基本的な解法}

放射, 鉛直拡散, 接地境界層・地表過程は
一括して一部の項を implicit 扱いで
時間変化項を計算し, 最後に時間積分を行なう.
ベクトル量 {\boldmath q} の時間変化項として,
Euler 法で扱う項 ${\cal A}$ と implicit 法で扱う項 ${\cal B}$ とに分けて考える.
%
\begin{equation}
  \mbox{\boldmath q}^+ 
      = \mbox{\boldmath q} + 2 \Delta t {\cal A}( \mbox{\boldmath q}   )
                           + 2 \Delta t {\cal B}( \mbox{\boldmath q}^+ ) \; . 
\end{equation}
%
一般の場合にこれを解くことは困難であるが,
近似的に $B$ を線形化することにより解くことが可能となる.
\begin{equation}
  {\cal B}( \mbox{\boldmath q}^+ ) 
                           \simeq {\cal B}( \mbox{\boldmath q} ) 
                            + B( \mbox{\boldmath q}^+ - \mbox{\boldmath q} )
\end{equation}
のように行列$B$を用いて線形化する.
ここで,
\begin{equation}
  B_{ij} = \DP{{\cal B}_i}{q_j}
\end{equation}
である. 
すると, 
$\Delta \mbox{\boldmath q} \equiv \mbox{\boldmath q}^+ - \mbox{\boldmath q}$
と書けば,
\begin{equation}
  ( I - 2 \Delta t B ) \Delta \mbox{\boldmath q} 
      = 2 \Delta t \left(  {\cal A}( \mbox{\boldmath q} )
                         + {\cal B}( \mbox{\boldmath q} ) \right) \; . 
\end{equation}
%
これは, 行列演算で原理的には簡単に解くことができる.

\subsubsection{基本方程式}

放射, 鉛直拡散, 接地境界層・地表過程の
方程式は, 基本的に以下のように表わされる.
%
\begin{eqnarray}
     \DP{u}{t}  & = &  - g \DP{}{p} F_u \; , \\
     \DP{v}{t}  & = &  - g \DP{}{p} F_v \; , \\
 c_p \DP{T}{t}  & = &  - g \DP{}{p} ( F_T + F_R ) \; , \\
     \DP{q}{t}  & = &  - g \DP{}{p} F_q \; , \\
 C_g \DP{G}{t}  & = &  -   \DP{}{z} F_g \; .
\end{eqnarray}
%
ここで, $F_u, F_v, F_T, F_q$は
鉛直拡散による, それぞれ$u, v, c_p T, q$
の鉛直上向きフラックス密度である.
また, $F_R$ は放射による
鉛直上向きエネルギーフラックス密度である.

大気は, $\sigma=p/p_S$ を座標系で離散化される.
風速, 気温等は層 $\sigma_k$ で定義される.
フラックスは, 層の境界 $\sigma_{k-1/2}$ で定義される.
下層から上層に$k$が増大する.
また, $\sigma_{1/2} = 1$, 
$\sigma_{k} \simeq (\sigma_{k-1/2} + \sigma_{k+1/2})/2$ である.
$\sigma$ 座標は, 鉛直1次元過程を考えている限りは, 
定数($p_S$)倍の違いを除いては $p$ 座標と同じであると考えてよい.
ここで,
\begin{equation}
  \Delta \sigma_{k} = \sigma_{k-1/2} - \sigma_{k+1/2} \; , 
\end{equation}
\begin{equation}
  \Delta m_{k} = \frac{1}{g} \Delta p_k  
   = \frac{p_S}{g} ( \sigma_{k-1/2} - \sigma_{k+1/2} )
\end{equation}
と書く.

\subsubsection{implicit時間差分}

鉛直拡散項などの線形化できる項に関しては, implicit法を用いる.
拡散係数なども予報変数に依存するが,
その係数は最初に求めるだけで, 反復的に解くことはしない.
ただし, 安定性の向上のために時間ステップの扱いを工夫している(後述). 

例えば, 離散化した$u$の方程式(\ref{u-eq.orig})は,
%
\begin{equation}
  (u_k^{m+1} - u_k^{m})/\Delta t 
    = (Fu^{m+1}_{k-1/2}-Fu^{m+1}_{k+1/2})/\Delta m_k
\end{equation}
ここで, $m$ は時間ステップである.
$Fu_{k-1/2}$等は, $u_k$ の関数であるから, その依存性を線形化して,
%
\begin{equation}
   Fu^{m+1}_{k-1/2} 
  =  Fu^{m}_{k-1/2} 
  +  \sum_{k'=1}^{K} 
     \DP{Fu^{m}_{k-1/2}}{u_{k'}} (u^{m+1}_{k'}-u^{m}_{k'})
  \label{u-flux.next}
\end{equation}

従って, $\delta u_k \equiv (u^{m+1}_{k}-u^{m}_{k})/\Delta t$ と置くと,
%
\begin{equation}
  \Delta m_k \delta u_k
  =   \left( Fu^{m}_{k-1/2}
         +  \sum_{k'=1}^{K} 
            \DP{Fu^{m}_{k-1/2}}{u_{k'}} \Delta t \delta u_{k'}
         -   Fu^{m}_{k+1/2}
         -  \sum_{k'=1}^{K}
            \DP{Fu^{m}_{k+1/2}}{u_{k'}} \Delta t \delta u_{k'}
      \right) 
\end{equation}
%
すなわち,
\begin{equation}
  \Delta m_k \delta u_k
  -  \sum_{k'=1}^{K} \left(  \DP{Fu^{m}_{k-1/2}}{u_{k'}} 
                       - \DP{Fu^{m}_{k+1/2}}{u_{k'}} \right)
                 \Delta t\delta u_{k'}
  = Fu^{m}_{k-1/2} - Fu^{m}_{k+1/2}
\end{equation}

これは, 以下のような行列形式で書くことができる.
%
\begin{equation}
  \sum_{k'=1}^{K} M^u_{k,k'} \delta u_k' = Fu^{m}_{k-1/2} - Fu^{m}_{k+1/2}
\end{equation}
%
\begin{equation}
M^u_{k,k'} \equiv \Delta m_k \delta_{k,k'}
          -  \left(  \DP{Fu^{m}_{k-1/2}}{u_{k'}} 
                   - \DP{Fu^{m}_{k+1/2}}{u_{k'}} \right) \Delta t
   \label{u-matrix}
\end{equation}
%
これを LU 分解などの方法で解けば良い.
通常は, $M^u_{k,k'}$ は3重対角となるので, 簡単に解ける.
解いた後は, (\ref{u-flux.next})を用いて
この方法にコンシステントなフラックスを計算しておく.
$v$ についても全く同様である.

\subsubsection{implicit時間差分の結合}

気温, 比湿, 地中温度については, 前節のように簡単にはいかない.
%
\begin{eqnarray}
  c_p \Delta m_k \delta T_k
  & - & \sum_{k'=0}^{K}
                 \left(  \DP{F\theta^{m}_{k-1/2}}{T_{k'}} 
                       - \DP{F\theta^{m}_{k+1/2}}{T_{k'}} \right)
                 \Delta t\delta T_{k'}
  - \sum_{k'=0}^{K}
                 \left(  \DP{FR^{m}_{k-1/2}}{T_{k'}} 
                       - \DP{FR^{m}_{k+1/2}}{T_{k'}} \right)
                 \Delta t\delta T_{k'} \nonumber \\
 & = &  ( F\theta^{m}_{k-1/2} - F\theta^{m}_{k+1/2} )
  + ( FR^{m}_{k-1/2} - FR^{m}_{k+1/2} )
\label{deq-theta}
\end{eqnarray}
%
\begin{equation}
  \Delta m_k \delta q_k
  -  \sum_{k'=0}^{K} \left(  \DP{Fq^{m}_{k-1/2}}{q_{k'}} 
                            - \DP{Fq^{m}_{k+1/2}}{q_{k'}} \right)
                 \Delta t\delta q_{k'}
  = ( Fq^{m}_{k-1/2} - Fq^{m}_{k+1/2} )
\label{deq-q}
\end{equation}
%
\begin{equation}
  Cg_l \Delta z_l \delta G_l
  +  \sum_{l'=0}^{L} \left(  \DP{Fg^{m}_{l-1/2}}{G_{l'}} 
                            - \DP{Fg^{m}_{l+1/2}}{G_{l'}} \right)
                 \Delta t\delta T_{k'}
  = - ( Fg^{m}_{l-1/2} - Fg^{m}_{l+1/2} )
\label{deq-g}
\end{equation}
%
ここで, 上記の式における $\sum_{k'}$, $\sum_{l'}$ は
$k'=0$, $l'=0$ から取っていることに注意. なぜならば,
地表でのフラックスは, 以下のようになるからである.
\begin{equation}
  F\theta_{1/2} =  c_p C_H |\Dvect{v}_{1/2}| (\theta_0 - \theta_1)
\end{equation}
\begin{equation}
  Fq_{1/2} =  \beta C_E |\Dvect{v}_{1/2}| (q_0 - q_1)
\end{equation}
\begin{equation}
  Fg_{1/2} =  K_g (G_1 - G_0)/z_1
\end{equation}
ここで, 地表面表皮温度を $T_0$ とすると,
$\theta_0 = T_0$, $q_0 = q^*(T_0)$ (飽和比湿), $G_0 = T_0$ である.
これらは, 全て, $T_0$ に依存する.
また, $FR_{k}$ は, 全ての$k$での値が $T_0$ に依存する. 

(\ref{u-matrix}) と同様に, 行列 $M^{\theta}, M^q, M^g$ を用いて
(\ref{deq-theta}), (\ref{deq-q}), (\ref{deq-g}) を書き直すと, 
%
 $k \ge 2$ ($\theta, q$について) または $l \ge 1$ ($G$について) のとき, 
%
  \begin{eqnarray}
    \sum_{k'=1}^{K}  M^\theta_{k,k'} \delta T_{k'}
       & = & (F\theta^{m}_{k-1/2} - F\theta^{m}_{k+1/2}) 
        + (FR^{m}_{k-1/2} - FR^{m}_{k+1/2})  \nonumber \\
& + & \left(\DP{FR^{m}_{k-1/2}}{T_0} - \DP{FR^{m}_{k+1/2}}{T_0} \right)
     \Delta t\delta T_0 \; ,
  \label{comb-theta2}
  \end{eqnarray}

\begin{equation}
 \sum_{k'=1}^{K}  M^q_{k,k'} \delta q_{k'}
         = (Fq^{m}_{k-1/2} - Fq^{m}_{k+1/2}) \; ,
  \label{comb-q2}
\end{equation}

\begin{equation}
  \sum_{l'=0}^{L} M^g_{l,l'} \delta G_{l'}
         = - (Fg^{m}_{l-1/2} - Fg^{m}_{l+1/2}) \; .
  \label{comb-g2}
\end{equation}
%
ただし, 
\begin{equation}
M^{\theta}_{k,k'} \equiv c_p \Delta m_k \delta_{k,k'}
          -  \left(  \DP{F\theta^{m}_{k-1/2}}{T_{k'}} 
                   - \DP{F\theta^{m}_{k+1/2}}{T_{k'}} \right) \Delta t
          -  \left(  \DP{FR^{m}_{k-1/2}}{T_{k'}} 
                   - \DP{FR^{m}_{k+1/2}}{T_{k'}} \right) \Delta t \; , 
\end{equation}
\begin{equation}
M^{q}_{k,k'} \equiv \Delta m_k \delta_{k,k'}
          -  \left(  \DP{Fq^{m}_{k-1/2}}{q_{k'}} 
                   - \DP{Fq^{m}_{k+1/2}}{q_{k'}} \right) \Delta t \; ,
\end{equation}
\begin{equation}
M^{g}_{l,l'} \equiv Cg_l \Delta z_l \delta_{l,l'}
          -  \left(  \DP{Fg^{m}_{l-1/2}}{G_{l'}} 
                   - \DP{Fg^{m}_{l+1/2}}{G_{l'}} \right) \Delta t \; .
\end{equation}

 $k=1$ ($\theta, q$について) または $l=0$ ($G$について) のとき, 
%
  \begin{eqnarray}
    \sum_{k'=1}^{K}  M^\theta_{1,k'} \delta T_{k'}
  +  \DP{F\theta^{m}_{1/2}}{T_1} \Delta t\delta T_1
       & = & (F\theta^{m}_{1/2} - F\theta^{m}_{3/2}) 
        + (FR^{m}_{1/2} - FR^{m}_{3/2})  \nonumber \\
& + &  \DP{F\theta^{m}_{1/2}}{T_0} \Delta t\delta T_0 
     \nonumber \\
& + & \left(\DP{FR^{m}_{1/2}}{T_0} - \DP{FR^{m}_{3/2}}{T_0} \right)
     \Delta t\delta T_0 \; ,
  \label{comb-theta}
  \end{eqnarray}
%
\begin{equation}
 \sum_{k'=1}^{K}  M^q_{1,k'} \delta q_{k'}
         - \DP{Fq^{m}_{1/2}}{q_1} \Delta t\delta q_1
         = (Fq^{m}_{1/2} - Fq^{m}_{3/2}) 
         + \DP{Fq^{m}_{1/2}}{T_0} \Delta t\delta T_0 \; ,
  \label{comb-q}
\end{equation}
%
\begin{eqnarray}
  \lefteqn{\sum_{l'=1}^{L} M^g_{0,l'} \delta G_{l'}
           +  \left(    \DP{F\theta^{m}_{1/2}}{T_0}
           +  L \DP{Fq^{m}_{1/2}}{T_0} 
           +    \DP{FR^{m}_{1/2}}{T_0}
           -  \DP{Fg^{m}_{1/2}}{T_0} \right) \Delta t\delta T_0  }
        \nonumber \\
       & & =  - F\theta^{m} - L Fq^{m} - FR^{m} +  Fg^{m}_{1/2} \nonumber \\
       & & -    \DP{F\theta^{m}_{1/2}}{T_1} \Delta t\delta T_1
           -  L \DP{Fq^{m}_{1/2}}{q_1} \Delta t\delta q_1
           -    \DP{FR^{m}_{1/2}}{T_1} \Delta t\delta T_1
           +    \DP{Fg^{m}_{1/2}}{G_1} \Delta t\delta G_1  
  \label{comb-g}
\end{eqnarray}
%
ただし, 
\begin{equation}
M^{\theta}_{1,k'} \equiv c_p \Delta m_1 \delta_{1,k'}
          -  \left(
                   - \DP{F\theta^{m}_{3/2}}{T_{k'}} \right) \Delta t
          -  \left( \DP{FR^{m}_{1/2}}{T_{k'}} 
                   - \DP{FR^{m}_{3/2}}{T_{k'}} \right) \Delta t \; , 
\end{equation}
\begin{equation}
M^{q}_{1,k'} \equiv \Delta m_1 \delta_{1,k'}
          -  \left(
                   - \DP{Fq^{m}_{3/2}}{q_{k'}} \right) \Delta t \; ,
\end{equation}
\begin{equation}
M^{g}_{0,l'} \equiv
             \left(
                   - \DP{Fg^{m}_{1/2}}{G_{l'}} \right) \Delta t \; .
\end{equation}
%
ただし, (\ref{comb-g})は, 地表面のバランスの条件
\begin{equation}
   F\theta^{m+1} + L Fq^{m+1} + FR^{m+1} - Fg^{m+1} = 0
\end{equation}
を, 土壌温度の式の $l=0$ の場合として扱ったもので, 
(\ref{deq-g})の表式には含まれていないことに注意. 

これら,
(\ref{comb-theta2}), (\ref{comb-q2}), (\ref{comb-g2}), 
(\ref{comb-theta}), (\ref{comb-q}), (\ref{comb-g})
を連立すると, $2K+L+1$ 個の未知変数に関して, 
同数の方程式があるので, 解くことができる.
実際の解法としては, LU分解を用いて行なうことができる.

解いた後は, 
(\ref{u-flux.next}) と同様に,
コンシステントなフラックスを求めておく.

\subsubsection{時間差分の結合式の解法}

(\ref{comb-theta}) などは, 以下のように書ける.
%
\begin{equation}
  \sum_{k'=1}^{K} ( M_{k,k'} + \delta_{1,k} \delta_{1,k'} \alpha)
    = F_k + \delta_{1,k} ( F_s + \gamma T_0 )
\end{equation}
ここで, $F_s, \alpha, \gamma$
の項は地表フラックスに伴う項,
その他は鉛直拡散に伴う項である.
%
ここで, 上下を逆にして行列で表現すると, 以下のようになる.
%
\begin{equation}
  \left( \begin{array}{lll} M_{KK} & \cdots & M_{K1} \\ \vdots & &
  \vdots \\ M_{1K} & \cdots & M_{11} + \alpha \\
\end{array}  \right)
\left( \begin{array}{l} T_K \\ \vdots \\ T_1 \\
\end{array}  \right)
= \left( \begin{array}{l} F_K \\ \vdots \\ F_1 + F_s + \gamma T_{0} \\
\end{array} \right)
\end{equation}

いま, 表記の簡単のために, $K=3$とする.  以後の議論は, これによって一般性
を失うことはない.
%
\begin{equation}
  \left( \begin{array}{lll} M_{33} & M_{32} & M_{31} \\ M_{23} &
  M_{22} & M_{21} \\ M_{13} & M_{12} & M_{11} + \alpha \\
\end{array} \right)
\left( \begin{array}{l} T_3 \\ T_2 \\ T_1 \\
\end{array} \right)
= \left( \begin{array}{l} F_3 \\ F_2 \\ F_1 + F_s + \gamma T_{0} \\
\end{array} \right)
\end{equation}
%
ここで,
$F_s = 0, \alpha=0, \gamma=0$ としたときの式,
(地表でのフラックス交換を考えない場合にあたる)
を LU 分解で解くことを考える.
%
\begin{equation}
  \left( \begin{array}{lll} M_{33} & M_{32} & M_{31} \\ M_{23} &
  M_{22} & M_{21} \\ M_{13} & M_{12} & M_{11} \\
         \end{array} \right)
  \left( \begin{array}{l}
         T'_3 \\ T'_2 \\ T'_1 \\
         \end{array} \right)
  = 
  \left(  \begin{array}{l}
          F_3 \\ F_2 \\ F_1 \\
          \end{array} \right)
  \label{solve-0}
\end{equation}

LU 分解すると,
%
\begin{equation}
  \left( \begin{array}{lll}
         1      & 0      & 0      \\
         L_{23} & 1      & 0      \\
         L_{13} & L_{12} & 1      \\
         \end{array} \right)
  \left( \begin{array}{lll}
         U_{33} & U_{32} & U_{31} \\
         0      & U_{22} & U_{21} \\
         0      & 0      & U_{11} \\
         \end{array} \right)
  \left( \begin{array}{l}
         T'_3 \\ T'_2 \\ T'_1 \\
         \end{array} \right)
  = 
  \left(  \begin{array}{l}
          F_3 \\ F_2 \\ F_1 \\
          \end{array} \right)
\end{equation}
%
これから, 
%
\begin{equation}
  \left( \begin{array}{lll}
         1      & 0      & 0      \\
         L_{23} & 1      & 0      \\
         L_{13} & L_{12} & 1      \\
         \end{array} \right)
  \left( \begin{array}{l}
         f'_3 \\ f'_2 \\ f'_1 \\
         \end{array} \right)
  = 
  \left(  \begin{array}{l}
          F_3 \\ F_2 \\ F_1 \\
          \end{array} \right)
 \label{solve-z}
\end{equation}
%
を$f'$について解き($f'_3=F_3$から出発すれば, 簡単に解ける), 
それから,
%
\begin{equation}
  \left( \begin{array}{lll}
         U_{33} & U_{32} & U_{31} \\
         0      & U_{22} & U_{21} \\
         0      & 0      & U_{11} \\
         \end{array} \right)
  \left( \begin{array}{l}
         T'_3 \\ T'_2 \\ T'_1 \\
         \end{array} \right)
  = 
  \left(  \begin{array}{l}
          f'_3 \\ f'_2 \\ f'_1 \\
          \end{array} \right)
\end{equation}
%
を, $f'$について, $x'_1=z'_1/U_{11}$ から出発して順に解くことができる.

$\alpha \neq 0, \gamma \neq 0$ だと, LU 分解は, 
%
\begin{equation}
  \left( \begin{array}{lll}
         1      & 0      & 0      \\
         L_{23} & 1      & 0      \\
         L_{13} & L_{12} & 1      \\
         \end{array} \right)
  \left( \begin{array}{lll}
         U_{33} & U_{32} & U_{31} \\
         0      & U_{22} & U_{21} \\
         0      & 0      & U_{11}+\alpha \\
         \end{array} \right)
  \left( \begin{array}{l}
         T_3 \\ T_2 \\ T_1 \\
         \end{array} \right)
  = 
  \left(  \begin{array}{l}
          F_3 \\ F_2 \\ F_1 + F_s + \gamma T_0 \\
          \end{array} \right)
\end{equation}
%
これから, 
%
\begin{equation}
  \left( \begin{array}{lll}
         1      & 0      & 0      \\
         L_{23} & 1      & 0      \\
         L_{13} & L_{12} & 1      \\
         \end{array} \right)
  \left( \begin{array}{l}
         f_3 \\ f_2 \\ f_1 \\
         \end{array} \right)
  = 
  \left(  \begin{array}{l}
          F_3 \\ F_2 \\ F_1 + F_s + \gamma T_0 \\
          \end{array} \right)
\end{equation}
%
だが, これと, (\ref{solve-z}) を見比べると, 以下の関係があることがわかる.
%
\begin{equation}
  \left( \begin{array}{l}
         f_3 \\ f_2 \\ f_1 \\
         \end{array} \right)
 = 
  \left( \begin{array}{l}
         f'_3 \\ f'_2 \\ f'_1 + F_s + \gamma T_0 \\
         \end{array} \right)
\end{equation}
%
これを用いると,
%
\begin{equation}
  \left( \begin{array}{lll}
         U_{33} & U_{32} & U_{31} \\
         0      & U_{22} & U_{21} \\
         0      & 0      & U_{11} + \alpha \\
         \end{array} \right)
  \left( \begin{array}{l}
         T_3 \\ T_2 \\ T_1 \\
         \end{array} \right)
  = 
  \left(  \begin{array}{l}
          f'_3 \\ f'_2 \\ f'_1 + F_s + \gamma x_0 \\
          \end{array} \right)
 \label{solve-x}
\end{equation}
%
となる. すなわち,
%
\begin{equation}
  ( U_{11} +  \alpha  ) T_1 = f'_1 + F_s + \gamma T_0 
  \label{solve-1}
\end{equation}
%
ここで, $U_{k,k'}$ および, $f'_k$ は,
$\alpha=\gamma=0$ とおいた式(\ref{solve-0}),
すなわち, 地表フラックスの項を考慮しない式で
LU分解を行なうことによって得られることに注意したい.
これらの項の物理的な意味としては,
地表面とのフラックス交換過程において,
大気全体が, 熱容量$U_{11}$を持ち, 
上からフラックス $f'_1$ が
供給される1つの層とみなすことができることを示す.

(\ref{comb-theta2})および(\ref{comb-theta}), 
(\ref{comb-q2})および(\ref{comb-q}), 
(\ref{comb-g2})および(\ref{comb-g})  のそれぞれにおいて
(\ref{solve-1}) に対応する式が得られ, 以下のようになる.

\begin{equation}
  ( U^{T}_{11} +  \alpha^{T}  ) \delta T_1 - \gamma^{T} \delta T_0 
      = f^{T}_1 + F\theta_{1/2}
\end{equation}
\begin{equation}
  ( U^{q}_{11} +  \alpha^{q}  ) \delta q_1 - \gamma^{q} \delta T_0
      = f^{q}_1 + Fq^P_{1/2}
\end{equation}
\begin{equation}
  ( U^{g}_{00} +  \alpha^{g}  ) \delta T_0 - \gamma^{g1} \delta T_1
                                           - \gamma^{g2} \delta q_1
      = f^{g}_0 - F\theta_{1/2} - L Fq_{1/2}
\end{equation}

従って, 上の3式を連立させれば,
未知変数 $\delta T_1, \delta q_1, \delta T_0$ を解くことができる.
これらが解ければ, 後は
(\ref{solve-x}) を順次$x_2,x_3$と解くことができる.
%
その後, 得られた温度にコンシステントなフラックスを
\begin{eqnarray}
  F\theta_c & = & F\theta_{1/2} 
                   + \gamma^{T} \delta T_0 -\alpha^{T} \delta T_1 \\
  Fq^P_c & = & Fq^P_{1/2} 
                   + \gamma^{q} \delta T_0 -\alpha^{q} \delta q_1
\end{eqnarray}
として計算する.
%
ここでは, $M$ が一般行列の場合を示したが,
実際には3重対角行列となるので, さらに簡単である.

プログラム中においては,
\Module{VFTND1(pimtx.F)} で大気部分について,
\Module{GNDHT1(pggnd.F)} で地中部分について, LU分解解法の前半
($f'_k$を求めるところ)を行ない, 
\Module{SLVSFC(pgslv.F)} において, $3\times 3$の方程式を解き,
$\delta q_1, \delta G_1, \delta T_0$ を求めている.
その後, \Module{GNDHT2(pggnd.F)} において
LU分解解法の後半を行ない, 地中について温度変化率を解き, 
収支が合うようにフラックスを補正する.
また, \Module{VFTND2(pimtx.F)} において大気中について
温度変化率を解き, 
\Module{FLXCOR(pimtx.F)} でフラックスを補正している.

\subsubsection{時間差分の結合式}

$\delta T_1, \delta q_1, \delta T_0$ を求める結合式は, 
以下の様に条件を変えながら 3回解く. 
\begin{enumerate}
\item 地表湿潤度 $\beta$ を 1 として解く. 地表温度は変数. 
\item \Module{GNDBET} で得られた地表湿潤度で解く. 
      地表温度は変数.
\item \Module{GNDBET} で得られた地表湿潤度で解く. 
      融雪等の場合, 地表温度は氷点に固定. 
\end{enumerate}

1 回目の計算は, 可能蒸発量 $Fq^P$ を見積もるために行なわれる.
(地表湿潤度が小さいときに, モデルのエネルギーバランスから
得られた $Fq$ を用いて, 可能蒸発量を
$\widetilde{Fq^P} = Fq / \beta$
として診断すると, 非現実的な大きな値になってしまう.)
可能蒸発量は,
\begin{eqnarray}
  Fq^P_c = Fq^P_{1/2}
       + \DP{Fq^P_{1/2}}{q_1} \delta q_1 2 \Delta t 
       + \DP{Fq^P_{1/2}}{T_0} \delta T_0 2 \Delta t 
\end{eqnarray}
となる.
添字 $c$ は, 補正後の意味で, 
これが得られた温度等にコンシステントなフラックスである.
    
2 回目以降の計算では, 
\begin{enumerate}
\item 1回めの計算で求めた可能蒸発量に
      地表湿潤度(蒸発効率) $\beta$ を
      かけたものを蒸発量 $Fq_1$ とする.
      \begin{equation}
        Fq = \beta Fq^P
      \end{equation}

\item 蒸発量$Fq_1$ は
      \begin{equation}
        \beta \rho C_E |\Dvect{v}| ( q_*(T_0) - q )
      \end{equation}
      で求められるものとして, 
      改めてエネルギーバランスを解き直す.
\end{enumerate}
の2通りの蒸発量の計算方法を用いることができる
(標準では1.の方法を用いる).
3 回目の計算は, 融雪, 融氷のときや, 混合層海洋で海氷が生成するときに
地表温度を氷点などに固定してエネルギーバランスを解くために行なう. 
このとき, 融雪などの水の相変化に使われるエネルギー量が診断的に求まり, 
後で融雪量などを計算する際に用いられる. 

結合式の具体的な形は以下の様である. 
%
\begin{eqnarray}
  \renewcommand{\arraystretch}{1.5}
  \left( \begin{array}{ccc}
      U_{11}^T - \DP{F\theta_{1/2}}{T_1} 2 \Delta t &
      0 &
      - \left( \DP{F\theta_{1/2}}{T_0} 
                         + \DP{FR_{1/2}}{T_0}\right) 2 \Delta t \\
      0 &
      U_{11}^q - \beta \DP{Fq^P_{1/2}}{q_1} \Delta t &
      - \beta \DP{Fq^P_{1/2}}{T_0} \Delta t \\
        \DP{F\theta_{1/2}}{T_1} \Delta t &
      L \beta \DP{Fq_{1/2}}{q_1} \Delta t &
      U_{00}^g + \left(\DP{F\theta_{1/2}}{T_0}
                + L \beta \DP{Fq_{1/2}}{T_0}
                + \DP{FR_{1/2}}{T_0} \right) 2 \Delta t \\
  \end{array} \right)
  \left( \begin{array}{l}
      \delta T_1 \\ \delta q_1 \\ \delta T_0 \\
  \end{array} \right)  \nonumber \\
=  \left( \begin{array}{l}
      f_1^T + F\theta_{1/2} \\  
      f_1^q + \beta Fq^P_{1/2} \\  
      f_0^g - F\theta_{1/2} - L \beta Fq^P_{1/2} \\  
  \end{array} \right) \; .
  \label{combin-eq}
\end{eqnarray}
%
ここで, $U_{11}^T, U_{11}^q, U_{00}^g$ および $f_1^T, f_1^q, f_0^g$ は, 
LU分解解法の前半を行なって得られる, 行列およびベクトルの成分である. 
地表面が雪または氷に覆われているときには, 潜熱 $L$ の代わりに
昇華の潜熱 $Ls = L + L_M$ を用いる. $L_M$ は水の融解の潜熱である. 
%
ただし, 第2回めの計算で, 
蒸発の見積もりとして第一の方法を用いた場合は, 以下のようになる.
\begin{eqnarray}
 \renewcommand{\arraystretch}{1.5}
  \left( \begin{array}{ccc}
      U_{11}^T - \DP{F\theta_{1/2}}{T_1} 2 \Delta t &
      0 &
      - \left( \DP{F\theta_{1/2}}{T_0} 
                         + \DP{FR_{1/2}}{T_0}\right) 2 \Delta t \\
      0 &
      U_{11}^q - \beta \DP{Fq^P_{1/2}}{q_1} \Delta t & \\
        \DP{F\theta_{1/2}}{T_1} \Delta t & 
                                         &
      U_{00}^g + \left(\DP{F\theta_{1/2}}{T_0}
                + \DP{FR_{1/2}}{T_0} \right) 2 \Delta t \\
  \end{array} \right)
  \left( \begin{array}{l}
      \delta T_1 \\ \delta q_1 \\ \delta T_0 \\
  \end{array} \right)  \nonumber \\
=
  \left( \begin{array}{l}
      f_1^T + F\theta_{1/2} \\  
      f_1^q + \beta Fq^P_c \\  
      f_0^g - F\theta_{1/2} - L \beta Fq^P0_c \\  
  \end{array} \right) \; .
\end{eqnarray}

3 回目の計算で, 地表温度を固定した場合の結合式は, 
\begin{eqnarray}
 \renewcommand{\arraystretch}{1.5}
  \left( \begin{array}{cc}
      U_{11}^T - \DP{F\theta_{1/2}}{T_1} 2 \Delta t &
      0 \\
      0 &
      U_{11}^q - \DP{Fq_{1/2}}{q_1} 2 \Delta t \\
  \end{array} \right)
  \left( \begin{array}{l}
      \delta T_1 \\ \delta q_1 \\
  \end{array} \right) \nonumber \\
=
  \left( \begin{array}{l}
      f_1^T + + F\theta_{1/2}
      + \left( \DP{F\theta_{1/2}}{T_0} 
                    + \DP{FR_{1/2}}{T_0} \right) \delta_0 T_0 2 \Delta t \\
      f_1^q +  \beta Fq^P_{1/2} 
      + \DP{Fq_{1/2}}{T_0} \delta_0 T_0 2 \Delta t \\
  \end{array} \right) \; .
  \label{combin-eq3}
\end{eqnarray}
ここで, $\delta_0 T_0$ は固定する温度までの変化率で, 
\begin{equation}
   \delta_0 T_0 = ( T_0^{fix} - T_0 ) / \Delta t \; .
\end{equation}
$T_0^{fix}$ は, 融雪, 融氷の場合は 273.15K, 
海氷の生成の場合は 271.15K である. 
また, 蒸発計算の第2の方法を用いる場合には,
同様に$Fq^P_{1/2}$ のかわりに $Fq^P_c$ を用い,
$F_q$ の微分項を 0 として計算する.
このとき, 
\begin{eqnarray}
\Delta s & = & f_0^g - F\theta_{1/2} - L \beta Fq^P_{1/2} - U_{00}^g
          -  \left(\DP{F\theta_{1/2}}{T_0}
                + L \beta \DP{Fq_{1/2}}{T_0}
                + \DP{FR_{1/2}}{T_0} \right) \delta_0 T_0 \Delta t 
                \nonumber \\
         & - & \DP{F\theta_{1/2}}{T_1} \delta T_1 \Delta t
                - L \beta \DP{Fq^P_{1/2}}{q_1} \delta q_1 \Delta t \; .
\end{eqnarray}
で計算される $\Delta s$ は地表エネルギーバランスで, 
水の相変化に使われる分のエネルギーである. 

\subsubsection{implicit 時間差分における時間ステップの扱い}

鉛直拡散項の時間差分には implicit 法を用いているが, 
一般に拡散係数が非線形であり, この係数を explicit に評価している
ことにより, 数値不安定の問題が生じ得る. 
安定性の向上のために, Kalnay and Kanamitsu (19??) にならって
時間ステップの扱いを工夫している. 

簡単化のために以下の常微分方程式を例に取って説明する. 
\begin{equation}
  \DP{X}{t} = - K(X) X
\end{equation}
係数 $K(X)$ が非線形性を表す. 
係数のみ explicit に評価して素直に implicit 差分化すると次式のようになる. 
\begin{equation}
  \frac{X^{m+1} - X^m}{\Delta t} = - K( X^m ) X^{m+1}
  \label{normal-fd}
\end{equation}
ところが, ここでは 2ステップ先の $X$ の値 $X^{\ast}$ を考えて, 
\begin{equation}
  \frac{X^{\ast} - X^m}{2\Delta t} = - K ( X^m ) X^{\ast}
  \label{modify-fd1}
\end{equation}
\begin{equation}
  X^{m+1} = \frac{X^{\ast} + X^m}2
  \label{modify-fd2}
\end{equation}
とする. 
一般に, (\ref{modify-fd1}), (\ref{modify-fd2}) のようにする方が
(\ref{normal-fd})よりも安定性が良いことが知られている. 

(\ref{modify-fd1}), (\ref{modify-fd2}) を, 時間変化率を求める
形に書き直すと以下を得る. 
\begin{equation}
  \left(\frac{\Delta X}{\Delta t}\right)^{\ast} = 
     - K( X^m ) \left( X^m + 
        \left(\frac{\Delta X}{\Delta t}\right)^{\ast} 
        2 \Delta t \right)
\end{equation}
\begin{equation}
  X^{m+1} = X^m + \left(\frac{\Delta X}{\Delta t}\right)^{\ast} \Delta t
\end{equation}
すなわち, 時間変化率を求める際の時間ステップには, 
時間積分のステップの 2 倍を用いる. 

