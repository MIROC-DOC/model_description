
\subsection{力学部分のまとめ}

ここでは, これまでの記述と重複するが,
力学過程部で行なわれる計算を列挙する.

\subsubsection{力学部分の計算の概要}

力学過程は, 以下のような順序で計算が行なわれる.

\begin{enumerate}
\item 水平風の渦度・発散への変換   \Module{UV2VDG(dvect)}
\item 仮温度の計算              \Module{VIRTMD(dvtmp)}
\item 気圧傾度項の計算           \Module{HGRAD(dvect)}
\item 鉛直流の診断的計算         \Module{GRDDYN/PSDOT(dgdyn)}
\item 移流による時間変化項 \Module{GRDDYN(dgdyn)}
\item 予報変数のスペクトルへの変換 \Module{GD2WD(dg2wd)}
\item 時間変化項のスペクトルへの変換 \Module{TENG2W(dg2wd)}
\item スペクトル値時間積分 \Module{TINTGR(dintg)}
\item 予報変数の格子点値への変換 \Module{GENGD(dgeng)}
\item 疑似等$p$面拡散補正   \Module{CORDIF(ddifc)}
\item 拡散による摩擦熱の考慮    \Module{CORFRC(ddifc)}
\item 質量の保存の補正          \Module{MASFIX(dmfix)}
\item (物理過程)             \Module{PHYSCS(padmn)}
\item (時間フィルター)        \Module{TFILT(aadvn)}
\end{enumerate}

\subsubsection{水平風の渦度・発散への変換}

水平風の格子点値$u_{ij}, v_{ij}$ 
から渦度・発散の格子点値$\zeta_{ij}, D_{ij}$ を求める.
まず, 渦度・発散のスペクトル
$\zeta_n^m, D_n^m$ を求める,
\begin{eqnarray}
\zeta_n^m & = & \frac{1}{I} \sum_{i=1}^{I} \sum_{j=1}^{J}  
                  im v_{ij} \cos\varphi_j {Y_n^{m*}}_{ij}
                \frac{w_j}{a(1-\mu_j^{2})} 
                \nonumber \\
          & + &   \frac{1}{I} \sum_{i=1}^{I} \sum_{j=1}^{J}  
                     u_{ij} \cos\varphi_j (1-\mu_j^2) 
                  \frac{\partial }{\partial \mu} {Y_n^{m*}}_{ij}
                 \frac{w_j}{a(1-\mu_j^{2})} \; ,
\label{d-summ:uv-zeta}                 
\end{eqnarray}
\begin{eqnarray}
    D_n^m & = & \frac{1}{I} \sum_{i=1}^{I} \sum_{j=1}^{J}  
                  im u_{ij} \cos\varphi_j {Y_n^{m*}}_{ij}
                \frac{w_j}{a(1-\mu_j^{2})} 
                \nonumber \\
          & - &   \frac{1}{I} \sum_{i=1}^{I} \sum_{j=1}^{J}  
                  v_{ij} \cos\varphi_j  (1-\mu_j^2) 
                  \frac{\partial }{\partial \mu} {Y_n^{m*}}_{ij}
                 \frac{w_j}{a(1-\mu_j^{2})} ; .
\label{d-summ:uv-D}                 
\end{eqnarray}
それをさらに, 
\begin{equation}
  \zeta_{ij} 
   =  {\cal R}\Dvect{e} \sum_{m=-N}^{N} \sum_{n=|m|}^{N} 
      \zeta_n^m  {Y_n^m}_{ij} \; ,
\end{equation}
等を用いて格子点値に変換する.

\subsubsection{仮温度の計算}

仮温度$T_v$ は, 
\begin{eqnarray}
  T_v = T ( 1 + \epsilon_v q - l ) \; ,
\end{eqnarray}
ただし, $\epsilon_v = R_v/R - 1$ であり, 
$R_v$ は水蒸気の気体定数
(461Jkg$^{-1}$K$^{-1}$)
$R$ は空気の気体定数
(287.04Jkg$^{-1}$K$^{-1}$)
である.

\subsubsection{気圧傾度項の計算}

気圧傾度項 $\nabla \pi = \frac{1}{p_S} \nabla p_S$ は,
まず, $\pi_n^m$ を
\begin{equation}
  \pi_n^m  =  \frac{1}{I} \sum_{i=1}^{I} \sum_{j=1}^{J}  
               (\ln {p_S})_{ij} {Y_n^{m *}}_{ij}  w_j \; ,
\end{equation}
でスペクトル表現に直してから,
\begin{equation}
   \frac{1}{a \cos \varphi} 
   \left( \frac{\partial \pi}{\partial \lambda} \right)_{ij}
     = 
   \frac{1}{a \cos \varphi} 
        {\cal R}\Dvect{e} \sum_{m=-N}^{N} \sum_{n=|m|}^{N} 
       im \tilde{X}_n^m {Y_n^m}_{ij}  \; ,
\end{equation}
\begin{equation}
   \frac{1}{a}
   \left( \frac{\partial \pi}{\partial \varphi} \right)_{ij}
     =  
   \frac{1}{a \cos \varphi} 
       {\cal R}\Dvect{e} \sum_{m=-N}^{N} \sum_{n=|m|}^{N} 
       \pi_n^m 
       ( 1-\mu^{2} ) \frac{\partial }{\partial \mu} {Y_n^m}_{ij}  \; .
\end{equation}

\subsubsection{鉛直流の診断的計算}

気圧変化項, および鉛直流,
\begin{equation}
  \frac{\partial \pi}{\partial t}
 = - \sum_{k=1}^{K} ( D_k + \Dvect{v}_k \cdot \nabla \pi ) 
       \Delta  \sigma_k
\end{equation}
%
\begin{equation}
  \dot{\sigma}_{k-1/2}
 = - \sigma_{k-1/2} \frac{\partial \pi}{\partial t}
   - \sum_{l=k}^{K} ( D_l + \Dvect{v}_l \cdot \nabla \pi )          
       \Delta  \sigma_l
\end{equation}
%
ならびにその非重力波成分を計算する.
%
\begin{equation}
  \left( \frac{\partial \pi}{\partial t} \right)^{NG}
   =   - \sum_{k=1}^{K} \Dvect{v}_{k} \cdot \nabla \pi  
       \Delta  \sigma_{k} \nonumber \\
\end{equation}
%
\begin{equation}
  \dot{\sigma}^{NG}_{k-1/2}
 = - \sigma_{k-1/2} \left( \frac{\partial \pi}{\partial t} \right)^{NG}
   - \sum_{l=k}^{K} \Dvect{v}_{l} \cdot \nabla \pi
       \Delta  \sigma_{l}
\end{equation}

\subsubsection{移流による時間変化項}

運動量移流項:
\begin{eqnarray}
  (A_u)_k 
   & = & ( \zeta_k + f ) v_k 
             - \frac{1}{2 \Delta \sigma_k} 
             [   \dot{\sigma}_{k-1/2} ( u_{k-1} - u_k   )
               + \dot{\sigma}_{k+1/2} ( u_k   - u_{k+1} ) ]
           \nonumber \\
   &   &     - \frac{C_{p} \hat{\kappa}_k T_{v,k}'}{ a \cos \varphi} 
                  \frac{\partial \pi}{\partial \lambda} 
\end{eqnarray}
%
\begin{eqnarray}
  (A_v)_k 
   & = & - ( \zeta_k + f ) u_k 
             - \frac{1}{2 \Delta \sigma_k} 
             [   \dot{\sigma}_{k-1/2} ( v_{k-1} - v_k   )
               + \dot{\sigma}_{k+1/2} ( v_k   - v_{k+1} ) ]
           \nonumber \\
   &   &     - \frac{C_{p} \hat{\kappa}_k T_{v,k}'}{a} 
             \frac{\partial \pi}{\partial \varphi} 
\end{eqnarray}
\begin{equation}
 \hat{E}_k    
  = \frac{1}{2} ( u^2 + v^2 ) 
    +  \sum_{k'=1}^{k} \left[  C_p \alpha_k ( T_v - T )_{k'}
                              + C_p \beta_k ( T_v - T )_{k'-1} \right]
\end{equation}

温度移流項:
\begin{equation}
 (u T')_k  = u_k (T_k - \bar{T} )
\end{equation}
\begin{equation}
 (v T')_k  = v_k (T_k - \bar{T} )
\end{equation}
%
\begin{eqnarray}
 \hat{H}_k & = & T_{k}^{\prime} D_{k} \nonumber \\
     &   & - \frac{1}{\Delta \sigma_{k}} 
             [   \dot{\sigma}_{k-1/2} ( \hat{T^{\prime}}_{k-1/2} 
                                         - T^{\prime}_{k}   )
               + \dot{\sigma}_{k+1/2} ( T^{\prime}_{k}  
                                         - \hat{T^{\prime}}_{k+1/2} ) ]
               \nonumber \\
     &   & - \frac{1}{\Delta \sigma_{k}} 
             [   \dot{\sigma}^{NG}_{k-1/2} ( \hat{\bar{T}}_{k-1/2} 
                                         - \bar{T}_{k}   )
               + \dot{\sigma}^{NG}_{k+1/2} ( \bar{T}_{k}  
                                         - \hat{\bar{T}}_{k+1/2} ) ]
               \nonumber \\
     &   & + \hat{\kappa}_{k} T_{v,k} \Dvect{v}_{k} \cdot \nabla \pi
               \nonumber \\
     &   & - \frac{\alpha_{k}}{\Delta \sigma_{k} } T_{v,k}
             \sum_{l=k}^{K} \Dvect{v}_{l} \cdot \nabla \pi 
               \Delta \sigma_{l}
           - \frac{\beta_{k}}{\Delta \sigma_{k} } T_{v,k}
             \sum_{l=k+1}^{K} \Dvect{v}_{l} \cdot \nabla \pi 
               \Delta \sigma_{l}
               \nonumber \\
     &   & - \frac{\alpha_{k}}{\Delta \sigma_{k} } T'_{v,k}
             \sum_{l=k}^{K} D_l  \Delta \sigma_{l}
           - \frac{\beta_{k}}{\Delta \sigma_{k} } T'_{v,k}
             \sum_{l=k+1}^{K} D_l  \Delta \sigma_{l}
\end{eqnarray}

水蒸気移流項:
\begin{equation}
 (u q)_k  = u_k q_k
\end{equation}
\begin{equation}
 (v q)_k  = v_k q_k
\end{equation}
%
\begin{equation}
R_k  =  q_k D_k 
       - \frac{1}{2 \Delta \sigma_k} 
             [   \dot{\sigma}_{k-1/2} ( q_{k-1} - q_k   )
               + \dot{\sigma}_{k+1/2} ( q_k   - q_{k+1} ) ]
\end{equation}

\subsubsection{予報変数のスペクトルへの変換}

(\ref{d-summ:uv-zeta}) および
(\ref{d-summ:uv-D}) を用いて

$u_{ij}^{t-\Delta t}, v_{ij}^{t-\Delta t}$ を
渦度・発散のスペクトル表現
$\zeta_n^m, D_n^m$ に変換する.
さらに,
温度$T^{t-\Delta t}$, 比湿$q^{t-\Delta t}$, 
$\pi = \ln p_S^{t-\Delta t}$ を
\begin{equation}
  X_n^m  =  \frac{1}{I} \sum_{i=1}^{I} \sum_{j=1}^{J}  
               X_{ij} {Y_n^{m *}}_{ij}  w_j \; ,
\end{equation}
でスペクトル表現に変換する.

\subsubsection{時間変化項のスペクトルへの変換}

渦度の時間変化項
\begin{eqnarray}
  \DP{\zeta_n^m}{t} 
  & = & \frac{1}{I} \sum_{i=1}^{I} \sum_{j=1}^{J}  
          im (A_v)_{ij} \cos \varphi_j
          {Y_n^{m *}}_{ij}
         \frac{w_j}{a(1-\mu_j^{2})} 
         \nonumber \\
  & + &   \frac{1}{I} \sum_{i=1}^{I} \sum_{j=1}^{J}  
          (A_u)_{ij} \cos \varphi_j
          (1-\mu_j^2) 
          \frac{\partial }{\partial \mu} {Y_n^{m *}}_{ij}
          \frac{w_j}{a(1-\mu_j^{2})} 
         \nonumber \\ 
\end{eqnarray}
%
発散の時間変化項の非重力波成分
\begin{eqnarray}
  \left( \DP{D_n^m}{t} \right)^{NG}
  & = & \frac{1}{I} \sum_{i=1}^{I} \sum_{j=1}^{J}  
          im (A_u)_{ij} \cos \varphi_j
          {Y_n^{m *}}_{ij}
         \frac{w_j}{a(1-\mu_j^{2})} 
         \nonumber \\
  & - &   \frac{1}{I} \sum_{i=1}^{I} \sum_{j=1}^{J}  
          (A_v)_{ij} \cos \varphi_j
          (1-\mu_j^2) 
          \frac{\partial }{\partial \mu} {Y_n^{m *}}_{ij}
          \frac{w_j}{a(1-\mu_j^{2})} 
         \nonumber \\
  & - &  \frac{n(n+1)}{a^{2}} 
         \frac{1}{I} \sum_{i=1}^{I} \sum_{j=1}^{J}  
          \hat{E}_{ij}  {Y_n^{m *}}_{ij} w_j
         \nonumber \\ 
\end{eqnarray}
%
温度の時間変化項の非重力波成分
\begin{eqnarray}
  \left( \DP{T_n^m}{t} \right)^{NG}
  & = & - \frac{1}{I} \sum_{i=1}^{I} \sum_{j=1}^{J}  
          im (u T')_{ij} \cos \varphi_j
          {Y_n^{m *}}_{ij}
         \frac{w_j}{a(1-\mu_j^{2})} 
         \nonumber \\
  &  & + \frac{1}{I} \sum_{i=1}^{I} \sum_{j=1}^{J}  
          (v T')_{ij} \cos \varphi_j
          (1-\mu_j^2) 
          \frac{\partial }{\partial \mu} {Y_n^{m *}}_{ij}
          \frac{w_j}{a(1-\mu_j^{2})} 
         \nonumber \\
  &  & + \frac{1}{I} \sum_{i=1}^{I} \sum_{j=1}^{J}  
          \hat{H}_{ij} 
          {Y_n^{m *}}_{ij} w_j
\end{eqnarray}
%
水蒸気の時間変化項
\begin{eqnarray}
  \DP{q_n^m}{t}
  & = & - \frac{1}{I} \sum_{i=1}^{I} \sum_{j=1}^{J}  
          im (uq)_{ij} \cos \varphi_j
          {Y_n^{m *}}_{ij}
         \frac{w_j}{a(1-\mu_j^{2})} 
         \nonumber \\
  &  & + \frac{1}{I} \sum_{i=1}^{I} \sum_{j=1}^{J}  
          (vq)_{ij} \cos \varphi_j
          (1-\mu_j^2) 
          \frac{\partial }{\partial \mu} {Y_n^{m *}}_{ij}
          \frac{w_j}{a(1-\mu_j^{2})} 
         \nonumber \\
  &  & + \frac{1}{I} \sum_{i=1}^{I} \sum_{j=1}^{J}  
          R_{ij} 
          {Y_n^{m *}}_{ij} w_j
\end{eqnarray}

\subsubsection{スペクトル値時間積分}

行列形式の方程式
\begin{eqnarray}
&   &   \left\{ ( 1+2\Delta t {\cal D}_H )( 1+2\Delta t {\cal D}_M )
           \underline{I}  
      - ( \Delta t )^{2}  ( \underline{W} \ \underline{h} 
           + (1+2\Delta t {\cal D}_M)
             \Dvect{G} \Dvect{C}^{T} ) \nabla^{2}_{\sigma}
  \right\}
      \overline{ \Dvect{D} }^{t} 
      \nonumber \\
& & = ( 1+2\Delta t {\cal D}_H )( 1-\Delta t {\cal D}_M ) 
       \Dvect{D}^{t-\Delta t}
  + \Delta t 
         \left( \frac{\partial \Dvect{D}}{\partial t} \right)_{NG}  
 \nonumber \\
& & -  \Delta t \nabla^{2}_{\sigma}     
                   \left\{  ( 1+2\Delta t {\cal D}_H ) \Dvect{\Phi}_{S} 
                          + \underline{W} 
                            \left[ ( 1-2\Delta t {\cal D}_H ) 
                                    \Dvect{T}^{t-\Delta t}
                                  + \Delta t 
                                      \left( \frac{\partial \Dvect{T}}
                                                  {\partial t}     
                                      \right)_{NG} \right]
                   \right.
 \nonumber \\
  &  &             \left.  \hspace*{20mm} 
                          + ( 1+2\Delta t {\cal D}_H ) \Dvect{G} 
                            \left[ \pi^{t-\Delta t} 
                                  + \Delta t
                                     \left( \frac{\partial \pi}
                                                 {\partial t} 
                                     \right)_{NG}  \right]
                   \right\} . 
\end{eqnarray}
%
を LU 分解を用いて解くことによって 
$\bar{D}$ を求め,
%
\begin{equation}
  \frac{\partial \Dvect{T}}{\partial t} 
      =   \left( \frac{\partial \Dvect{T}}
                        {\partial t}       \right)_{NG}  
         - \underline{h} \Dvect{D}
\end{equation}
%
\begin{equation}
  \frac{\partial \pi}{\partial t} 
      =   \left( \frac{\partial \pi}
                        {\partial t}       \right)_{NG}  
         - \Dvect{C} \cdot \Dvect{D}
\end{equation}

%
によって
$\partial \Dvect{T}/\partial t$,
$\partial \pi/\partial t$ 
を求めて, $t+\Delta t$ におけるスペクトルの値を計算する.
\begin{eqnarray}
  \zeta^{t+\Delta t} & = & \left( \zeta^{t-\Delta t}
                                +   2 \Delta t \DP{\zeta}{t} \right)
                          ( 1 + 2 \Delta t {\cal D}_M )^{-1} \\
  D^{t+\Delta t} & = & 2 \bar{D} - D^{t-\Delta t}\\
  T^{t+\Delta t} & = & \left( T^{t-\Delta t}
                                +  2 \Delta t  \DP{T}{t} \right)
                          ( 1 + 2 \Delta t {\cal D}_H )^{-1} \\
  q^{t+\Delta t} & = & \left( q^{t-\Delta t}
                                +  2 \Delta t \DP{q}{t} \right)
                          ( 1 + 2 \Delta t {\cal D}_E )^{-1} \\
\pi^{t+\Delta t} & = & \pi^{t-\Delta t}
                                +  2 \Delta t \DP{\pi}{t}
\end{eqnarray}

\subsubsection{予報変数の格子点値への変換}


渦度・発散のスペクトル値 $\zeta_n^m, D_n^m$ から
水平風速の格子点値 $u_{ij}, v_{ij}$ を求める.
\begin{equation}
  u_{ij}
  =  \frac{1}{\cos \varphi_j}
     {\cal R}\Dvect{e} \sum_{m=-N}^{N} 
                       \sum_{\stackrel{n=|m|}{n \neq 0}}^{N} 
    \left\{
             \frac{a}{n(n+1)} \zeta_n^m 
            (1-\mu^{2}) \DP{}{\mu} {Y_n^m}_{ij}
          -  \frac{im a}{n(n+1)} D_n^m {Y_n^m}_{ij}
    \right\}
\end{equation}
%
\begin{equation}
  v_{ij}
  =  \frac{1}{\cos \varphi_j}
     {\cal R}\Dvect{e} \sum_{m=-N}^{N}
                       \sum_{\stackrel{n=|m|}{n \neq 0}}^{N}
    \left\{
          -  \frac{im a}{n(n+1)} \zeta_n^m  {Y_n^m}_{ij}
          -  \frac{a}{n(n+1)} \tilde{D}_n^m 
            (1-\mu^{2}) \DP{}{\mu} {Y_n^m}_{ij}
    \right\}
\end{equation}

さらに,
\begin{equation}
  T_{ij} 
   =  {\cal R}\Dvect{e} \sum_{m=-N}^{N} \sum_{n=|m|}^{N} 
      T_n^m  {Y_n^m}_{ij} \; ,
\end{equation}
などによって, $T_{ij}, \pi_{ij}, q_{ij}$ を求め,
\begin{eqnarray}
  {p_S}_{ij} = \exp \pi_{ij} 
\end{eqnarray}
を計算する.

\subsubsection{疑似等$p$面拡散補正}

水平拡散は 等 $\sigma$ 面上で適用されるが,
山岳の傾斜の大きな領域では, 山を上る方向に水蒸気が輸送され,
山頂部での偽の降水をもたらすなどの問題を起こす.
それを緩和するために, 等 $p$ 面の拡散に近くなるような
補正を$T,q,l$について入れる.

\begin{eqnarray}
  {\cal D}_p (T) = (-1)^{N_D/2} K \nabla^{N_D}_p T  
               & \simeq & (-1)^{N_D/2} K \nabla^{N_D}_{\sigma} T  
                      - \DP{\sigma}{p} 
                      (-1)^{N_D/2} K \nabla^{N_D}_{\sigma} p
                      \cdot \DP{T}{\sigma}                  \nonumber \\
               & =  &    (-1)^{N_D/2} K \nabla^{N_D}_{\sigma} T  
                    -  (-1)^{N_D/2} K \nabla^{N_D}_{\sigma} \pi
                          \cdot \sigma \DP{T}{\sigma} \nonumber \\
               & = &   {\cal D} (T) 
                    -  {\cal D} (\pi) 
                       \sigma \DP{T}{\sigma}
\end{eqnarray}
%
であるから,
\begin{equation}
  T_k \leftarrow  T_k 
       -  2 \Delta t
        \sigma_{k} \frac{T_{k+1}-T_{k-1}}{\sigma_{k+1} - \sigma_{k-1}}
        {\cal D}(\pi)
\end{equation}
などととする.
${\cal D}(\pi)$ は, $pi$ のスペクトル値 $\pi_n^m$ に
拡散係数のスペクトル表現をかけたものを
格子の値に変換して用いる.

\subsubsection{拡散による摩擦熱の考慮}

拡散による摩擦熱は,
\begin{equation}
  Q_{DIF} = - \left( u_{ij} {\cal D}(u)_{ij} 
                   + v_{ij} {\cal D}(v)_{ij} \right)
\end{equation}
と見積もられる.
したがって,
\begin{equation}
  T_k \leftarrow  T_k 
       -  \frac{2 \Delta t}{C_p}
           \left( u_{ij} {\cal D}(u)_{ij} 
                 + v_{ij} {\cal D}(v)_{ij} \right)
\end{equation}

\subsubsection{質量の保存の補正}

スペクトル法による取扱いは,
$\pi = \ln p_S$ の全球積分は丸め誤差を除いて保存するが,
質量, すなわち $p_S$ の全球積分の保存は保証されない.
また, スペクトルの波数打ちきりにともない,
水蒸気の格子点値に負の値が出ることがある.
これらの事情から, 
乾燥大気の質量と水蒸気, 雲水の質量を保存させ,
さらに負の水蒸気量となる領域を除去するための補正を行なう.

まず, 力学の計算の最初に \Module{FIXMAS},
水蒸気, 雲水の各成分の全球積分値 $M_q, M_l$ を計算しておく.
\begin{eqnarray}
  M_q^0 & = & \sum_{ijk} q p_S  \Delta\lambda_i w_j \Delta\sigma_k  \\
  M_l^0 & = & \sum_{ijk} l p_S  \Delta\lambda_i w_j \Delta\sigma_k 
\end{eqnarray}
また, 計算の最初のステップで
乾燥質量 $M_d$ を計算し, 記憶する.
\begin{eqnarray}
  M_d^0 = \sum_{ijk} (1-q-l) p_S \Delta\lambda_i w_j \Delta\sigma_k 
\end{eqnarray}

力学計算の終りには \Module{MASFIX},
以下のような手順で補正を行なう.
\begin{enumerate}
\item まず, 負の水蒸気量となる格子点について,
      直下の格子点から水蒸気を分配して,
      負の水蒸気を除去する.
      $q_k < 0 $であるとすると,
      \begin{eqnarray}
        q_k'     & = & 0          \\
        q_{k-1}' & = & q_{k-1} + \frac{\Delta p_k}{\Delta p_{k-1}} q_k
      \end{eqnarray}
      ただし, これは $q_{k-1}' \ge 0 $ となる場合にのみ行なう.

\item 次に上の手続きで除去されなかった格子点について値を 0 とする.

\item 全球積分値 $M_q$ を計算し,
      これが $M_q^0$ と一致するように,
      全球の水蒸気量に一定割合をかける.

      \begin{equation}
        q'' = \frac{M_q^0}{M_q} q' 
      \end{equation}
      
\item 乾燥空気質量の補正を行なう.
      同様に$M_d$を計算し,

      \begin{equation}
        p_S'' = \frac{M_d^0}{M_d} p_S
      \end{equation}

\end{enumerate}

\subsubsection{水平拡散とレーリー摩擦}

水平拡散の係数をスペクトル表現すると,

\begin{equation}
 {{\cal D}_M}_n^m = K_M 
                      \left[ \left( \frac{n(n+1)}{a^2} \right)^{N_D/2} 
                                - \left( \frac{2}{a^2} \right)^{N_D/2} 
                      \right]
                  + K_R
\end{equation}
%
\begin{equation}
  {{\cal D}_H}_n^m = K_M \left( \frac{n(n+1)}{a^2} \right)^{N_D/2} 
\end{equation}
%
\begin{equation}
  {{\cal D}_E}_n^m = K_E \left( \frac{n(n+1)}{a^2} \right)^{N_D/2} 
\end{equation}

$K_R$ は レーリー摩擦係数である.
レーリー摩擦係数は
\begin{equation}
  K_R = K_R^0 \left[ 1+\tanh \left( \frac{z-z_R}{H_R} \right) \right]
\end{equation}
のようなプロファイルで与える.
ただし,
\begin{equation}
  z = - H \ln \sigma 
\end{equation}
と近似する.
標準値は, $K_R^0 = \mbox{(30day)}^{-1}$,
$z_R = -H \ln \sigma_{top}$ ($\sigma_{top}$ : モデルの最上レベル),
$H = 8000$m,
$H_R = 7000$m である.

\subsubsection{時間フィルター}


leap frog における計算モードの除去のために 
Asselin(1972) の時間フィルターを毎ステップ適用する.
%
\begin{equation}
  \bar{T}^{t}
    = ( 1-2 \epsilon_f ) T^{t}
    +  \epsilon_f 
        \left( \bar{T}^{t-\Delta t} + T^{t+\Delta t} \right)
\end{equation}
%
と$\bar{T}$を求める.
次のステップの力学過程で用いる $T^{t-\Delta t}$ としては,
この$\bar{T}^t$ を用いる.
$\epsilon_f$ としては標準的に 0.05 を使用する. 

実際には
まず, 予報変数の格子点値への変換 \Module{GENGD} の箇所で,
\begin{equation}
  \bar{T}^{t*}
    = ( 1 -\epsilon_f )^{-1} 
     \left[ ( 1-2 \epsilon_f ) T^{t} + \epsilon_f \bar{T}^{t-\Delta t}
     \right]
\end{equation}
を求めておき, 物理過程の処理が終わり
$T^{t+\Delta t}$ の値が確定した後で \Module{TFILT} で,
\begin{equation}
 \bar{T}^{t}
    = ( 1 -\epsilon_f ) \bar{T}^{t*}  
       +  \epsilon_f \bar{T}^{t+\Delta t}
\end{equation}
とする.
