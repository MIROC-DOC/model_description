
\subsection{重力波抵抗}

\subsubsection{重力波抵抗スキームの概要}

重力波抵抗スキームは,
サブグリッドスケールの地形によって励起される
重力波の上方への運動量フラックスを表現し,
その収束に伴う水平風の減速を計算する.
主な入力データは, 東西風 $u$, 南北風 $v$, 気温 $T$, であり,
出力データは東西風と南北風の時間変化率,
$\partial u/\partial t, \partial v/\partial t$, である.

計算手順の概略は以下の通りである.
%
\begin{enumerate}
\item 地表面での運動量フラックスを
      地表高度の分散, 
      最下層での水平風速, 成層安定度などから求める.
\item 運動量フラックスを持つ重力波の上方への伝播を考える.
      運動量フラックスが臨界フルード数から決まる
      臨界フラックスを越える場合には,
      砕波が起こってフラックスはその臨界値となるとする.
\item 運動量フラックスの各層での収束に応じた
      水平風の時間変化を計算する.
\end{enumerate}

\subsubsection{局所フルード数と運動量フラックスの関係}

地表起源の重力波による
水平運動量の鉛直フラックスを考えると,
ある高度でのフラックス $\tau$ と
局所フルード数 $F_L = NH/U$ との間には,
\begin{equation}
   F_L = \left(
            \frac{\tau N}{E_f \rho U^3}
           \right)^{1/2} \; ,
  \label{p-grav:tau-fl}
\end{equation}
の関係が成り立つ.
ここで, $N = g/\theta \partial \theta/\partial z$ は
ブラントバイサラ振動数, 
$\rho$ は大気の密度, 
$U$ は風速, $E_f$ は地表高度の波打ちの水平スケールに対応する
比例定数である.
これから,
\begin{equation}
  \tau = \frac{E_f F_L^2 \rho U^3}{N}
\label{p-grav:fl-tau}
\end{equation}

局所フルード数 $F_L$ は,
ある値, 臨界フルード数 $F_{c}$ を越えることができないとする.
(\ref{p-grav:tau-fl}) から計算される
局所フルード数が臨界フルード数 $F_{c}$ を越える場合には
重力波は過飽和となり,
臨界フルード数に対応する運動量フラックスまで
フラックスは減少する.

\subsubsection{地表での運動量フラックス}

地表面で励起される重力波による
水平運動量の鉛直フラックスの大きさ $\tau_{1/2}$ は,
ただし, 地表での局所フルード数 
$(F_L)_{1/2} = N_1 h/U_1$ を
(\ref{p-grav:fl-tau}) に代入することにより,
%
\begin{equation}
  \tau_{1/2} = E_f h^2 \rho_1 N_1 U_1 \; ,
\end{equation}
%
と見積られる.
ここで, 
$U_1 = |\mbox{\boldmath v}_1| = (u_1^2 + v_1^2)^{1/2}$ は地表風速,
$N_1, \rho_1$ はそれぞれ地表付近の大気の
安定度と密度である.
$h$ はサブグリッドの地表高度変化の指標であり,
地表高度の標準偏差 $Z_{SD}$ に等しいとする.

ここで, 地表での局所フルード数 
$(F_L)_{1/2} = N_1 Z_{SD}/U_1$ が 臨界フルード数
$F_c$ を越えるときは, 
運動量フラックスは $F_c$ を(\ref{p-grav:fl-tau}) に代入した値に
抑えられるとする.
すなわち,
\begin{equation}
  \tau_{1/2} = \min \left(
                   E_f Z_{SD}^{2} \rho_1 N_1 U_1, \; 
                  \frac{E_f F_c^{2} \rho_1 U_1^3}{N_1}
               \right)
\end{equation}

\subsubsection{上層での運動量フラックス}

レベル $k-1/2$ での運動量フラックス $\tau_{k-1/2}$ が
求められているとする.
$\tau_{k+1/2}$ は, 飽和が起こらないときには
$\tau_{k-1/2}$ に等しい.
この運動量フラックス $\tau_{k-1/2}$ が,
$k+1/2$ レベルでの臨界フルード数から計算される運動量フラックス
を上回るときには, $k$ 層内で砕波が起こり,
運動量フラックスは臨界に対応するフラックスまで減少する.

\begin{equation}
  \tau_{k+1/2} = \min \left( 
               \tau_{k-1/2}, \;
               \frac{E_f F_c^2 \rho_{k+1/2} U_{k+1/2}^3}{N_{k+1/2}}
                      \right),
\end{equation}

ただし $U_{k+1/2}$ は,
各層での風速ベクトルの,
最下層の水平風の方向に対する射影成分の大きさであり,
\begin{equation}
  U_{k+1/2} = \frac{\mbox{\boldmath v}_{k+1/2} 
                      \cdot \mbox{\boldmath v}_{1}}
                   {|\mbox{\boldmath v}_{1}|       }
\end{equation}

\subsubsection{運動量収束による水平風の時間変化の大きさ}

水平風の射影成分 $U_{k}$ の時間変化率は,
\begin{equation}
  \frac{\partial U}{\partial t} 
        = - \frac{1}{\rho} \frac{\partial \tau}{\partial z}
        = g  \frac{\partial \tau}{\partial p}
\end{equation}
%
によって求められる.すなわち,
%
\begin{equation}
  \frac{\partial U_{k}}{\partial t} 
        =  g  \frac{\tau_{k+1/2} - \tau{k-1/2}}{\Delta p}.
\end{equation}
%
これを用いて,
東西風と南北風の時間変化率は以下のように計算される.
\begin{eqnarray}
  \frac{\partial u_{k}}{\partial t} & = &
           \frac{\partial U_{k}}{\partial t} \frac{u_{1}}{U_{1}} \\
  \frac{\partial v_{k}}{\partial t} & = &
           \frac{\partial U_{k}}{\partial t} \frac{v_{1}}{U_{1}}
\end{eqnarray}

\subsubsection{その他の留意点}

\begin{enumerate}
\item 最下層の風速が小さく $U_{1} \le v_{min}$ のとき,また,
      地表の起伏が小さく $Z_{SD} \le (Z_{SD})_{min}$ のときは, 
      地表で重力波が励起されないと仮定する.
\end{enumerate}

