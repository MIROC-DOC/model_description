
\section{力学過程}

\subsection{基礎方程式}

\subsubsection{基礎方程式}

基礎方程式は,
球面($\lambda,\varphi$), $\sigma$座標におけるプリミティブ方程式系であり,
以下のように与えられる( Haltiner and Williams , 1980 ).


\begin{enumerate}
\item 連続の式

\begin{equation}
  \label{質量}
  \frac{\partial \pi}{\partial t} 
    + \Dvect{v}_{H} \cdot \nabla_{\sigma} \pi
     =  - \nabla_{\sigma} \cdot \Dvect{v}_{H} 
          - \frac{\partial \dot{\sigma}}{\partial \sigma}
\end{equation}

\item 静水圧の式

\begin{equation}
\label{静水圧}
  \frac{\partial \Phi}{\partial \sigma} = - \frac{RT_v}{\sigma} 
\end{equation}


\item 運動方程式

\begin{equation}
  \label{渦度}
  \frac{\partial \zeta}{\partial t} 
     =   \frac{1}{a\cos\varphi}
            \frac{\partial A_v}{\partial \lambda}
          - \frac{1}{a\cos \varphi}
            \frac{\partial}{\partial \varphi} ( A_u \cos\varphi )
          - {\cal D}(\zeta) 
\end{equation}
\begin{equation}
  \label{発散}
  \frac{\partial D}{\partial t} 
     =    \frac{1}{a\cos\varphi}
            \frac{\partial A_u}{\partial \lambda}
          + \frac{1}{a\cos\varphi}
            \frac{\partial }{\partial \varphi} ( A_v \cos\varphi )
          - \nabla^{2}_{\sigma}
           ( \Phi + R \bar{T} \pi + E ) 
          - {\cal D}(D) 
\end{equation}


\item 熱力学の式

\begin{eqnarray}
\label{熱力}
  \frac{\partial T}{\partial t}
    & = & - \frac{1}{a\cos\varphi}
               \frac{\partial uT'}{\partial \lambda}
          - \frac{1}{a}
               \frac{\partial }{\partial \varphi} ( vT' \cos\varphi )
          + T' D \nonumber \\
    &   & - \dot{\sigma} 
              \frac{\partial T }{\partial \sigma}
          + \kappa T \left( \frac{\partial \pi}{\partial t}
                            + \Dvect{v}_{H} \cdot \nabla_{\sigma} \pi 
                            + \frac{ \dot{\sigma} }{ \sigma } 
                     \right)
          + \frac{Q}{C_{p}}
          + \frac{Q_{diff}}{C_{p}}
          - {\cal D}(T) 
\end{eqnarray}


\item 水蒸気の式

\begin{eqnarray}
\label{水蒸気}
  \frac{\partial q}{\partial t}
  & = & - \frac{1}{a\cos\varphi}
               \frac{\partial uq}{\partial \lambda}
          - \frac{1}{a\cos\varphi}
               \frac{\partial }{\partial \varphi} (vq \cos\varphi)
          + q D \nonumber \\
    &   & - \dot{\sigma} \frac{\partial q }{\partial \sigma}
          + S_{q}
          - {\cal D}(q) 
\end{eqnarray}

\end{enumerate}

ここで,
%
\begin{eqnarray}
\theta & \equiv & T \left( p/p_{0} \right)^{-\kappa} \\
\kappa & \equiv & R/C_{p} \\
  \Phi & \equiv & gz \\
   \pi & \equiv & \ln p_{S} \\
%
 \dot{\sigma} & \equiv &  \frac{d \sigma}{d t} \\
%
     T_v & \equiv & T ( 1+\epsilon_v q ) \\
     T & \equiv &  \bar{T}(\sigma) + T^{\prime} \\
%
\label{渦度定義}
 \zeta & \equiv & \frac{1}{a \cos\varphi }
                    \frac{\partial v}{\partial \lambda} 
             -    \frac{1}{a \cos\varphi }
                    \frac{\partial }{\partial \varphi}
                    ( u \cos\varphi ) \\
%
\label{発散定義}
     D & \equiv & \frac{1}{a \cos\varphi }
                    \frac{\partial u}{\partial \lambda} 
             +    \frac{1}{a \cos\varphi }
                    \frac{\partial }{\partial \varphi}
                    ( v \cos\varphi ) \\
%
\label{B項}
    A_u & \equiv &  ( \zeta + f ) v
             - \dot{\sigma} \frac{\partial u}{\partial \sigma} 
             - \frac{RT^{\prime}}{a\cos\varphi} 
                  \frac{\partial \pi}{\partial \lambda} 
             + {\cal F}_x \\
%
\label{A項}
    A_v & \equiv & - ( \zeta + f ) u
             - \dot{\sigma} \frac{\partial v}{\partial \sigma} 
             - \frac{RT^{\prime}}{a}
                  \frac{\partial \pi}{\partial \varphi} 
             + {\cal F}_y \\
%
\label{E項}
     E & \equiv &  \frac{u^{2}+v^{2}}{2} \\
%
 \Dvect{v}_{H} \cdot \nabla
       & \equiv & \frac{u}{a \cos \varphi} 
         \left( \frac{\partial }{\partial \lambda} \right)_{\sigma}
     + \frac{v}{a}
         \left( \frac{\partial }{\partial \varphi} \right)_{\sigma} 
           \nonumber \\
  \nabla^{2}_{\sigma}  
       & \equiv & 
               \frac{1}{a^{2}\cos^2\varphi} 
                 \frac{\partial^{2} }{\partial \lambda^{2}} 
             + \frac{1}{a^{2}\cos\varphi} 
                 \frac{\partial }{\partial \varphi}
                 \left[ \cos\varphi
                       \frac{\partial }{\partial \varphi} \right]  .
\end{eqnarray}

${\cal D}(\zeta), {\cal D}(D), {\cal D}(T), {\cal D}(q)$
は水平拡散項,
${\cal F}_\lambda, {\cal F}_\varphi$
は小規模運動過程(`物理過程'として扱う)による力,
$Q$ は放射, 凝結, 小規模運動過程等の`物理過程'による
加熱・温度変化,
$S_q$は凝結, 小規模運動過程等の`物理過程'による
水蒸気ソース項である.
また, $Q_{diff}$ は摩擦熱であり,
%
\begin{equation}
  Q_{diff}
 = - \Dvect{v} \cdot  ( \frac{\partial \Dvect{v}}{\partial t} )_{diff} .
\end{equation}
%
$( \frac{\partial \Dvect{v}}{\partial t} )_{diff} $ は,
水平および鉛直の拡散による $u,v$ の時間変化項である.

\subsubsection{境界条件}

鉛直流に関する境界条件は
%
\begin{equation}
  \dot{\sigma} = 0  \ \ \ at \ \ \sigma = 0 , \ 1 .
\end{equation}
%
である. よって(\ref{質量}) から,
地表気圧の時間変化式と
$\sigma$系での鉛直速度$\dot{\sigma}$を求める診断式
%
\begin{equation}
   \label{気圧傾向}
   \frac{\partial \pi}{\partial t}
   = - \int_{0}^{1} \Dvect{v}_{H} \cdot \nabla_{\sigma} \pi d \sigma
     - \int_{0}^{1} D  d \sigma ,
\end{equation}
%
\begin{equation}
   \label{鉛直速度}
   \dot{\sigma} 
   = - \sigma 
     \frac{\partial \pi}{\partial t}
     - \int_{0}^{\sigma} D d \sigma
     - \int_{0}^{\sigma} 
         \Dvect{v}_{H} \cdot \nabla_{\sigma} \pi d \sigma ,
\end{equation}
%
が導かれる.



