
\section{力学過程}

\subsection{基礎方程式}

\subsubsection{基礎方程式}

基礎方程式は,
球面(TERM00191,TERM00191), TERM00192 座標におけるプリミティブ方程式系であり,
以下のように与えられる( Haltiner and Williams , 1980 ).


\begin{enumerate}
\item 連続の式

\begin{verbatim}
EQ=00008.
\end{verbatim}

\item 静水圧の式

\begin{verbatim}
EQ=00009.
\end{verbatim}


\item 運動方程式

\begin{verbatim}
EQ=00010.
\end{verbatim}
\begin{verbatim}
EQ=00011.
\end{verbatim}


\item 熱力学の式

\begin{verbatim}
EQ=00016.
EQ=00016.
\end{verbatim}


\item 水蒸気の式

\begin{verbatim}
EQ=00017.
EQ=00017.
\end{verbatim}

\end{enumerate}

ここで,
%
\begin{verbatim}
EQ=00018.
EQ=00018.
EQ=00018.
EQ=00018.
EQ=00018.
EQ=00018.
EQ=00018.
EQ=00018.
EQ=00018.
EQ=00018.
EQ=00018.
EQ=00018.
EQ=00018.
EQ=00018.
\end{verbatim}

TERM00197,TERM00197
は水平拡散項,
TERM00198,TERM00198
は小規模運動過程(`物理過程'として扱う)による力,
TERM00199 は放射, 凝結, 小規模運動過程等の`物理過程'による
加熱・温度変化,
TERM00200 は凝結, 小規模運動過程等の`物理過程'による
水蒸気ソース項である.
また, TERM00201 は摩擦熱であり,
%
\begin{verbatim}
EQ=00012.
\end{verbatim}
%
TERM00204 は,
水平および鉛直の拡散による TERM00205,TERM00205 の時間変化項である.

\subsubsection{境界条件}

鉛直流に関する境界条件は
%
\begin{verbatim}
EQ=00013.
\end{verbatim}
%
である. よって(14) から,
地表気圧の時間変化式と
TERM00206 系での鉛直速度 TERM00207 を求める診断式
%
\begin{verbatim}
EQ=00014.
\end{verbatim}
%
\begin{verbatim}
EQ=00015.
\end{verbatim}
%
が導かれる.



