\hypertarget{solving-the-diffuse-balance-equation-for-atmospheric-and-surface-systems}{%
\subsection{Solving the diffuse balance equation for atmospheric and
surface
systems}\label{solving-the-diffuse-balance-equation-for-atmospheric-and-surface-systems}}

\hypertarget{basic-solutions.}{%
\subsubsection{Basic Solutions.}\label{basic-solutions.}}

Radiative, vertical diffusion, ground boundary layer and surface
processes are Some terms are treated as implicit in the Compute the
time-varying term and do time integration at the end. As a time-varying
term for the vector quantity {q}, The term \({\mathcal A}\) for the
Euler method and the term \({\mathcal B}\) for the implicit method are
considered separately.

\begin{eqnarray}
  {\mathbf q}^+ 
      = {\mathbf q} + 2 \Delta t {\mathcal A}( {\mathbf q}   )
                           + 2 \Delta t {\mathcal B}( {\mathbf q}^+ ) \; . 
\end{eqnarray}

It is difficult to solve this in the general case, but, It can be solved
by linearizing \(B\) in an approximate manner.

\begin{eqnarray}
  {\mathcal B}( {\mathbf q}^+ ) 
                           \simeq {\mathcal B}( {\mathbf q} ) 
                            + B( {\mathbf q}^+ - {\mathbf q} )
\end{eqnarray}

using the matrix \(B\) as Here,

\begin{eqnarray}
  B_{ij} = \frac{\partial {\mathcal B}_i}{\partial q_j}
\end{eqnarray}

It is. So.., \(\Delta {\mathbf q} \equiv {\mathbf q}^+ - {\mathbf q}\)
And then you can write,

\begin{eqnarray}
  ( I - 2 \Delta t B ) \Delta {\mathbf q} 
      = 2 \Delta t \left(  {\mathcal A}( {\mathbf q} )
                         + {\mathcal B}( {\mathbf q} ) \right) \; . 
\end{eqnarray}

It is easy to solve in principle by matrix operations.

\hypertarget{fundamental-equations.}{%
\subsubsection{Fundamental Equations.}\label{fundamental-equations.}}

Radiation, vertical diffusion, ground boundary layer and surface
processes The equations are basically expressed as follows.

\begin{eqnarray}
     \frac{\partial u}{\partial t}   =   - g \frac{\partial }{\partial p} F_u \; , \\
     \frac{\partial v}{\partial t}   =   - g \frac{\partial }{\partial p} F_v \; , \\
 c_p \frac{\partial T}{\partial t}   =   - g \frac{\partial }{\partial p} ( F_T + F_R ) \; , \\
     \frac{\partial q}{\partial t}   =   - g \frac{\partial }{\partial p} F_q \; , \\
 C_g \frac{\partial G}{\partial t}   =   -   \frac{\partial }{\partial z} F_g \; .
\end{eqnarray}

where \(F_u, F_v, F_T, F_q\) are By vertical diffusion,
\(u, v, c_p T, q\), respectively is the vertical upward flux density of
Also, the \(F_R\) is a radiant It is a vertical upward energy flux
density.

The atmosphere is discretized in the \(\sigma=p/p_S\) coordinate system.
Wind speed, temperature, etc. are defined in layer \(\sigma_k\). The
flux is defined by the layer boundary \(\sigma_{k-1/2}\). \(k\)
increases from lower to higher levels. Also, \(\sigma_{1/2} = 1\), This
is the \(\sigma_{k} \simeq (\sigma_{k-1/2} + \sigma_{k+1/2})/2\).
\(\sigma\) The coordinates are only available when we consider a
one-dimensional vertical process, It can be considered to be the same as
\(p\) coordinates except for the difference in the constant (\(p_S\))
times. Here,

\begin{eqnarray}
  \Delta \sigma_{k} = \sigma_{k-1/2} - \sigma_{k+1/2} \; , 
\end{eqnarray}

\begin{eqnarray}
  \Delta m_{k} = \frac{1}{g} \Delta p_k  
   = \frac{p_S}{g} ( \sigma_{k-1/2} - \sigma_{k+1/2} )
\end{eqnarray}

And write .

\hypertarget{implicit-time-difference}{%
\subsubsection{implicit time
difference}\label{implicit-time-difference}}

For terms that can be linearized, such as the vertical diffusion term,
we use the implicit method. Diffusion coefficients and other factors
also depend on the forecast variables, The coefficients are only
calculated first, not iteratively. However, the treatment of the time
step is devised to improve the stability (see below).

For example, the discretized equation of \(u\)
({[}{[}u-eq.orig{]}{]}(\#u-eq.orig)) is

\begin{eqnarray}
  (u_k^{m+1} - u_k^{m})/\Delta t 
    = (Fu^{m+1}_{k-1/2}-Fu^{m+1}_{k+1/2})/\Delta m_k
\end{eqnarray}

Here, \(m\) is a time step. Since \(Fu_{k-1/2}\) etc. is a function of
\(u_k\), its dependency is linearized and the

\begin{eqnarray}
   Fu^{m+1}_{k-1/2} 
  =  Fu^{m}_{k-1/2} 
  +  \sum_{k'=1}^{K} 
     \frac{\partial Fu^{m}_{k-1/2}}{\partial u_{k'}} (u^{m+1}_{k'}-u^{m}_{k'})
\end{eqnarray}

\begin{quote}
\protect\hypertarget{u-flux.next}{}{\textbackslash{[}u-flux.next{]}}
\end{quote}

Thus, if you put \(\delta u_k \equiv (u^{m+1}_{k}-u^{m}_{k})/\Delta t\),

\begin{eqnarray}
  \Delta m_k \delta u_k
  =   \left( Fu^{m}_{k-1/2}
         +  \sum_{k'=1}^{K} 
            \frac{\partial Fu^{m}_{k-1/2}}{\partial u_{k'}} \Delta t \delta u_{k'}
         -   Fu^{m}_{k+1/2}
         -  \sum_{k'=1}^{K}
            \frac{\partial Fu^{m}_{k+1/2}}{\partial u_{k'}} \Delta t \delta u_{k'}
      \right) 
\end{eqnarray}

Namely,

\begin{eqnarray}
  \Delta m_k \delta u_k
  -  \sum_{k'=1}^{K} \left(  \frac{\partial Fu^{m}_{k-1/2}}{\partial u_{k'}} 
                       - \frac{\partial Fu^{m}_{k+1/2}}{\partial u_{k'}} \right)
                 \Delta t\delta u_{k'}
  = Fu^{m}_{k-1/2} - Fu^{m}_{k+1/2}
\end{eqnarray}

It can be written in the following matrix form

\begin{eqnarray}
  \sum_{k'=1}^{K} M^u_{k,k'} \delta u_k' = Fu^{m}_{k-1/2} - Fu^{m}_{k+1/2}
\end{eqnarray}

\begin{eqnarray}
M^u_{k,k'} \equiv \Delta m_k \delta_{k,k'}
          -  \left(  \frac{\partial Fu^{m}_{k-1/2}}{\partial u_{k'}} 
                   - \frac{\partial Fu^{m}_{k+1/2}}{\partial u_{k'}} \right) \Delta t
\end{eqnarray}

\begin{quote}
\protect\hypertarget{u-matrix}{}{\blazer[u-matrix]}
\end{quote}

This can be solved by LU decomposition or some other method. Normally,
\(M^u_{k,k'}\) is easy to solve since it is a triple diagonal. After
solving it, ({[}\protect\hyperlink{u-flux.next}{u-flux.next{]}}), you
can use We calculate the consistent flux to this method. The same is
true for \(v\).

\hypertarget{implicit-time-difference-coupling}{%
\subsubsection{implicit time difference
coupling}\label{implicit-time-difference-coupling}}

Temperature, specific humidity, and ground temperature are not as simple
as those in the previous section.

\begin{eqnarray}
  c_p \Delta m_k \delta T_k
   -  \sum_{k'=0}^{K}
                 \left(  \frac{\partial F\theta^{m}_{k-1/2}}{\partial T_{k'}} 
                       - \frac{\partial F\theta^{m}_{k+1/2}}{\partial T_{k'}} \right)
                 \Delta t\delta T_{k'}
  - \sum_{k'=0}^{K}
                 \left(  \frac{\partial FR^{m}_{k-1/2}}{\partial T_{k'}} 
                       - \frac{\partial FR^{m}_{k+1/2}}{\partial T_{k'}} \right)
                 \Delta t\delta T_{k'}  \\
  =   ( F\theta^{m}_{k-1/2} - F\theta^{m}_{k+1/2} )
  + ( FR^{m}_{k-1/2} - FR^{m}_{k+1/2} )
\end{eqnarray}

\begin{quote}
\protect\hypertarget{deq-theta}{}{{[}deq-theta{]}}
\end{quote}

\begin{eqnarray}
  \Delta m_k \delta q_k
  -  \sum_{k'=0}^{K} \left(  \frac{\partial Fq^{m}_{k-1/2}}{\partial q_{k'}} 
                            - \frac{\partial Fq^{m}_{k+1/2}}{\partial q_{k'}} \right)
                 \Delta t\delta q_{k'}
  = ( Fq^{m}_{k-1/2} - Fq^{m}_{k+1/2} )
\end{eqnarray}

\begin{quote}
\protect\hypertarget{deq-q}{}{\blazer[deq-q]}
\end{quote}

\begin{eqnarray}
  Cg_l \Delta z_l \delta G_l
  +  \sum_{l'=0}^{L} \left(  \frac{\partial Fg^{m}_{l-1/2}}{\partial G_{l'}} 
                            - \frac{\partial Fg^{m}_{l+1/2}}{\partial G_{l'}} \right)
                 \Delta t\delta T_{k'}
  = - ( Fg^{m}_{l-1/2} - Fg^{m}_{l+1/2} )
\end{eqnarray}

\begin{quote}
\protect\hypertarget{deq-g}{}{\blazer[deq-g]}
\end{quote}

Here, \(\sum_{k'}\) and \(\sum_{l'}\) in the above equations are Note
that I took this from \(k'=0\), \(l'=0\). because, This is because the
flux at the surface is as follows

\begin{eqnarray}
  F\theta_{1/2} =  c_p C_H |\mathbf{v}_{1/2}| (\theta_0 - \theta_1)
\end{eqnarray}

\begin{eqnarray}
  Fq_{1/2} =  \beta C_E |\mathbf{v}_{1/2}| (q_0 - q_1)
\end{eqnarray}

\begin{eqnarray}
  Fg_{1/2} =  K_g (G_1 - G_0)/z_1
\end{eqnarray}

Where the surface skin temperature is set to \(T_0\),
\(\theta_0 = T_0\), \(q_0 = q^*(T_0)\) (saturated specific humidity),
\(G_0 = T_0\). They all depend on the \(T_0\). Also, the value of the
\(FR_{k}\) depends on the \(T_0\) for all \(k\) values.

(as with {[}{[}u-matrix{]}{]}(\#u-matrix)), using the matrices
\(M^{\theta}, M^q, M^g\) ({[}deq-theta{]} (\#deq-theta)), ({[}deq-q{]}
(\#deq-q)), ({[}deq-g\textbackslash{]} (\#deq-g)), and
({[}deq-g\textbackslash{]} (\#deq-g)) are rewritten, For \(k \ge 2\)
(for \(\theta, q\)) or \(l \ge 1\) (for \(G\)),

\begin{eqnarray}
    \sum_{k'=1}^{K}  M^\theta_{k,k'} \delta T_{k'}
        =  (F\theta^{m}_{k-1/2} - F\theta^{m}_{k+1/2}) 
        + (FR^{m}_{k-1/2} - FR^{m}_{k+1/2})   \\
 +  \left(\frac{\partial FR^{m}_{k-1/2}}{\partial T_0} - \frac{\partial FR^{m}_{k+1/2}}{\partial T_0} \right)
     \Delta t\delta T_0 \; ,
\end{eqnarray}

\begin{quote}
\protect\hypertarget{combo-theta2}{}{\textbackslash{[}combo-theta2\textbackslash blazer{]}}
\end{quote}

\begin{eqnarray}
 \sum_{k'=1}^{K}  M^q_{k,k'} \delta q_{k'}
         = (Fq^{m}_{k-1/2} - Fq^{m}_{k+1/2}) \; ,
\end{eqnarray}

\begin{quote}
\protect\hypertarget{combo-q2}{}{\blazer[combo-q2]}
\end{quote}

\begin{eqnarray}
  \sum_{l'=0}^{L} M^g_{l,l'} \delta G_{l'}
         = - (Fg^{m}_{l-1/2} - Fg^{m}_{l+1/2}) \; .
\end{eqnarray}

\begin{quote}
\protect\hypertarget{combo-g2}{}{\blazer[combo-g2]}
\end{quote}

However,

\begin{eqnarray}
M^{\theta}_{k,k'} \equiv c_p \Delta m_k \delta_{k,k'}
          -  \left(  \frac{\partial F\theta^{m}_{k-1/2}}{\partial T_{k'}} 
                   - \frac{\partial F\theta^{m}_{k+1/2}}{\partial T_{k'}} \right) \Delta t
          -  \left(  \frac{\partial FR^{m}_{k-1/2}}{\partial T_{k'}} 
                   - \frac{\partial FR^{m}_{k+1/2}}{\partial T_{k'}} \right) \Delta t \; , 
\end{eqnarray}

\begin{eqnarray}
M^{q}_{k,k'} \equiv \Delta m_k \delta_{k,k'}
          -  \left(  \frac{\partial Fq^{m}_{k-1/2}}{\partial q_{k'}} 
                   - \frac{\partial Fq^{m}_{k+1/2}}{\partial q_{k'}} \right) \Delta t \; ,
\end{eqnarray}

\begin{eqnarray}
M^{g}_{l,l'} \equiv Cg_l \Delta z_l \delta_{l,l'}
          -  \left(  \frac{\partial Fg^{m}_{l-1/2}}{\partial G_{l'}} 
                   - \frac{\partial Fg^{m}_{l+1/2}}{\partial G_{l'}} \right) \Delta t \; .
\end{eqnarray}

In case of \(k=1\) (for \(\theta, q\)) or \(l=0\) (for \(G\)),

\begin{eqnarray}
    \sum_{k'=1}^{K}  M^\theta_{1,k'} \delta T_{k'}
  +  \frac{\partial F\theta^{m}_{1/2}}{\partial T_1} \Delta t\delta T_1
        =  (F\theta^{m}_{1/2} - F\theta^{m}_{3/2}) 
        + (FR^{m}_{1/2} - FR^{m}_{3/2})   \\
 +   \frac{\partial F\theta^{m}_{1/2}}{\partial T_0} \Delta t\delta T_0 
      \\
 +  \left(\frac{\partial FR^{m}_{1/2}}{\partial T_0} - \frac{\partial FR^{m}_{3/2}}{\partial T_0} \right)
     \Delta t\delta T_0 \; ,
\end{eqnarray}

\begin{quote}
\protect\hypertarget{comb-theta}{}{\centric[comb-theta]}
\end{quote}

\begin{eqnarray}
 \sum_{k'=1}^{K}  M^q_{1,k'} \delta q_{k'}
         - \frac{\partial Fq^{m}_{1/2}}{\partial q_1} \Delta t\delta q_1
         = (Fq^{m}_{1/2} - Fq^{m}_{3/2}) 
         + \frac{\partial Fq^{m}_{1/2}}{\partial T_0} \Delta t\delta T_0 \; ,
\end{eqnarray}

\begin{quote}
\textbackslash centric\centric\centric\centric\centric\centric\centric\centric\centric\centric\centric\centric\centric\centric\centric\centric\centric\centric\centric\centric\centric\centric\centric\centric\centric\centric\centric\centric\centric\centric\fadabraz.
\end{quote}

\begin{eqnarray}
  {\sum_{l'=1}^{L} M^g_{0,l'} \delta G_{l'}
           +  \left(    \frac{\partial F\theta^{m}_{1/2}}{\partial T_0}
           +  L \frac{\partial Fq^{m}_{1/2}}{\partial T_0} 
           +    \frac{\partial FR^{m}_{1/2}}{\partial T_0}
           -  \frac{\partial Fg^{m}_{1/2}}{\partial T_0} \right) \Delta t\delta T_0  }
         \\
         =  - F\theta^{m} - L Fq^{m} - FR^{m} +  Fg^{m}_{1/2}  \\
         -    \frac{\partial F\theta^{m}_{1/2}}{\partial T_1} \Delta t\delta T_1
           -  L \frac{\partial Fq^{m}_{1/2}}{\partial q_1} \Delta t\delta q_1
           -    \frac{\partial FR^{m}_{1/2}}{\partial T_1} \Delta t\delta T_1
           +    \frac{\partial Fg^{m}_{1/2}}{\partial G_1} \Delta t\delta G_1  
\end{eqnarray}

\begin{quote}
\protect\hypertarget{combo-g}{}{\cleaner[combo-g]}
\end{quote}

However,

\begin{eqnarray}
M^{\theta}_{1,k'} \equiv c_p \Delta m_1 \delta_{1,k'}
          -  \left(
                   - \frac{\partial F\theta^{m}_{3/2}}{\partial T_{k'}} \right) \Delta t
          -  \left( \frac{\partial FR^{m}_{1/2}}{\partial T_{k'}} 
                   - \frac{\partial FR^{m}_{3/2}}{\partial T_{k'}} \right) \Delta t \; , 
\end{eqnarray}

\begin{eqnarray}
M^{q}_{1,k'} \equiv \Delta m_1 \delta_{1,k'}
          -  \left(
                   - \frac{\partial Fq^{m}_{3/2}}{\partial q_{k'}} \right) \Delta t \; ,
\end{eqnarray}

\begin{eqnarray}
M^{g}_{0,l'} \equiv
             \left(
                   - \frac{\partial Fg^{m}_{1/2}}{\partial G_{l'}} \right) \Delta t \; .
\end{eqnarray}

However, ({[}comb-g{]}{]}(\#comb-g)), the balance condition of the
ground surface

\begin{eqnarray}
   F\theta^{m+1} + L Fq^{m+1} + FR^{m+1} - Fg^{m+1} = 0
\end{eqnarray}

as the case of \(l=0\) in the soil temperature equation, (Note that it
is not included in the formula of \protect\hyperlink{deq-g}{deq-g}{]}.

These, ({[}comb-theta2{]} (\#comb-theta2)), ({[}comb-q2{]} (\#comb-q2)),
({[}comb-g2{]} (\#comb-g2)), ({[}comb-theta{]}(\#comb-theta)),
(\protect\hyperlink{comb-q}{comb-q}),
({[}comb-g\textbackslash lopen\protect\hyperlink{comb-g}{comb-g{]}}) for
the \(2K+L+1\) unknowns, There are equations of equality that can be
solved. In practice, the LU decomposition can be used to solve the
problem.

Once you're untied, ({[}u-flux.next{]}{]}(\#u-flux.next)) as well as ,
Consistent flux should be sought.

\hypertarget{solving-the-coupling-formula-for-time-difference}{%
\subsubsection{Solving the Coupling Formula for Time
Difference}\label{solving-the-coupling-formula-for-time-difference}}

({[}comb-theta{]}{]}(\#comb-theta)), etc., can be written as follows.

\begin{eqnarray}
  \sum_{k'=1}^{K} ( M_{k,k'} + \delta_{1,k} \delta_{1,k'} \alpha)
    = F_k + \delta_{1,k} ( F_s + \gamma T_0 )
\end{eqnarray}

Where, \(F_s, \alpha, \gamma\) The term in Section 3.1 is a term
associated with surface flux, The others are terms associated with
vertical diffusion. Here, if we reverse the top and bottom and represent
it as a matrix, we get the following.

\begin{eqnarray}
  \left( \begin{array}{lll} M_{KK}  \cdots  M_{K1} \\ \vdots  
  \vdots \\ M_{1K}  \cdots  M_{11} + \alpha \\
\end{array}  \right)
\left( \begin{array}{l} T_K \\ \vdots \\ T_1 \\
\end{array}  \right)
= \left( \begin{array}{l} F_K \\ \vdots \\ F_1 + F_s + \gamma T_{0} \\
\end{array} \right)
\end{eqnarray}

For the sake of brevity, we shall now refer to this document as \(K=3\).
For the sake of notation simplicity, we will now refer to it as \(K=3\).
You can't lose the.

\begin{eqnarray}
  \left( \begin{array}{lll} M_{33}  M_{32}  M_{31} \\ M_{23} 
  M_{22}  M_{21} \\ M_{13}  M_{12}  M_{11} + \alpha \\
\end{array} \right)
\left( \begin{array}{l} T_3 \\ T_2 \\ T_1 \\
\end{array} \right)
= \left( \begin{array}{l} F_3 \\ F_2 \\ F_1 + F_s + \gamma T_{0} \\
\end{array} \right)
\end{eqnarray}

Here, The expression for \(F_s = 0, \alpha=0, \gamma=0\), (This
corresponds to the case where flux replacement at the surface is not
considered.) by LU decomposition.

\begin{eqnarray}
  \left( \begin{array}{lll} M_{33}  M_{32}  M_{31} \\ M_{23} 
  M_{22}  M_{21} \\ M_{13}  M_{12}  M_{11} \\
         \end{array} \right)
  \left( \begin{array}{l}
         T'_3 \\ T'_2 \\ T'_1 \\
         \end{array} \right)
  = 
  \left(  \begin{array}{l}
          F_3 \\ F_2 \\ F_1 \\
          \end{array} \right)
\end{eqnarray}

\begin{quote}
\protect\hypertarget{summe-0}{}{\blazer[summe-0]}
\end{quote}

LU. Take it apart,

\begin{eqnarray}
  \left( \begin{array}{lll}
         1       0       0      \\
         L_{23}  1       0      \\
         L_{13}  L_{12}  1      \\
         \end{array} \right)
  \left( \begin{array}{lll}
         U_{33}  U_{32}  U_{31} \\
         0       U_{22}  U_{21} \\
         0       0       U_{11} \\
         \end{array} \right)
  \left( \begin{array}{l}
         T'_3 \\ T'_2 \\ T'_1 \\
         \end{array} \right)
  = 
  \left(  \begin{array}{l}
          F_3 \\ F_2 \\ F_1 \\
          \end{array} \right)
\end{eqnarray}

Now..,

\begin{eqnarray}
  \left( \begin{array}{lll}
         1       0       0      \\
         L_{23}  1       0      \\
         L_{13}  L_{12}  1      \\
         \end{array} \right)
  \left( \begin{array}{l}
         f'_3 \\ f'_2 \\ f'_1 \\
         \end{array} \right)
  = 
  \left(  \begin{array}{l}
          F_3 \\ F_2 \\ F_1 \\
          \end{array} \right)
\end{eqnarray}

\begin{quote}
\protect\hypertarget{solve-z}{}{\blazer[solve-z]}
\end{quote}

for \(f'\) (which can be easily solved by starting from \(f'_3=F_3\)),
And then..,

\begin{eqnarray}
  \left( \begin{array}{lll}
         U_{33}  U_{32}  U_{31} \\
         0       U_{22}  U_{21} \\
         0       0       U_{11} \\
         \end{array} \right)
  \left( \begin{array}{l}
         T'_3 \\ T'_2 \\ T'_1 \\
         \end{array} \right)
  = 
  \left(  \begin{array}{l}
          f'_3 \\ f'_2 \\ f'_1 \\
          \end{array} \right)
\end{eqnarray}

for \(f'\), starting from \(x'_1=z'_1/U_{11}\), and solving in sequence.

For \(\alpha \neq 0, \gamma \neq 0\), the LU decomposition is

\begin{eqnarray}
  \left( \begin{array}{lll}
         1       0       0      \\
         L_{23}  1       0      \\
         L_{13}  L_{12}  1      \\
         \end{array} \right)
  \left( \begin{array}{lll}
         U_{33}  U_{32}  U_{31} \\
         0       U_{22}  U_{21} \\
         0       0       U_{11}+\alpha \\
         \end{array} \right)
  \left( \begin{array}{l}
         T_3 \\ T_2 \\ T_1 \\
         \end{array} \right)
  = 
  \left(  \begin{array}{l}
          F_3 \\ F_2 \\ F_1 + F_s + \gamma T_0 \\
          \end{array} \right)
\end{eqnarray}

Now..,

\begin{eqnarray}
  \left( \begin{array}{lll}
         1       0       0      \\
         L_{23}  1       0      \\
         L_{13}  L_{12}  1      \\
         \end{array} \right)
  \left( \begin{array}{l}
         f_3 \\ f_2 \\ f_1 \\
         \end{array} \right)
  = 
  \left(  \begin{array}{l}
          F_3 \\ F_2 \\ F_1 + F_s + \gamma T_0 \\
          \end{array} \right)
\end{eqnarray}

However, comparing this with
(\protect\hyperlink{solve-z}{\ltra[solve-z]}), we see the following
relationship.

\begin{eqnarray}
  \left( \begin{array}{l}
         f_3 \\ f_2 \\ f_1 \\
         \end{array} \right)
 = 
  \left( \begin{array}{l}
         f'_3 \\ f'_2 \\ f'_1 + F_s + \gamma T_0 \\
         \end{array} \right)
\end{eqnarray}

With this,

\begin{eqnarray}
  \left( \begin{array}{lll}
         U_{33}  U_{32}  U_{31} \\
         0       U_{22}  U_{21} \\
         0       0       U_{11} + \alpha \\
         \end{array} \right)
  \left( \begin{array}{l}
         T_3 \\ T_2 \\ T_1 \\
         \end{array} \right)
  = 
  \left(  \begin{array}{l}
          f'_3 \\ f'_2 \\ f'_1 + F_s + \gamma x_0 \\
          \end{array} \right)
\end{eqnarray}

\begin{quote}
\protect\hypertarget{solve-x}{}{\blazer[solve-x]}
\end{quote}

The result is That is, ,

\begin{eqnarray}
  ( U_{11} +  \alpha  ) T_1 = f'_1 + F_s + \gamma T_0 
\end{eqnarray}

\begin{quote}
\protect\hypertarget{solve-1}{}{\blazer[solve-1\blazer]}
\end{quote}

where, \(U_{k,k'}\) and, \(f'_k\) are,

In other words, without considering the surface flux term Note that this
can be obtained by performing LU decomposition. The physical meaning of
these terms is , During the flux exchange process with the ground
surface, The entire atmosphere has a heat capacity of \(U_{11}\), Flux
(\(f'_1\)) from the top Indicates that it can be regarded as one layer
to be supplied.

({[}comb-theta2{]}{]}(\#comb-theta2)) and
(\protect\hyperlink{comb-theta}{comb-theta{]}}),
({[}combo-q2{]}(\#comb-q2)) and
(\protect\hyperlink{comb-q}{combo-q\textbackslash\textbackslash clean\}}),
({[}comb-g2{]}(\#comb-g2)) and (\protect\hyperlink{comb-g}{comb-g}) (the
formula corresponding to {[}\ltra[solve-1\end{eqnarray}](\#solve-1)) is obtained and
is as follows

\begin{eqnarray}
  ( U^{T}_{11} +  \alpha^{T}  ) \delta T_1 - \gamma^{T} \delta T_0 
      = f^{T}_1 + F\theta_{1/2}
\end{eqnarray}

\begin{eqnarray}
  ( U^{q}_{11} +  \alpha^{q}  ) \delta q_1 - \gamma^{q} \delta T_0
      = f^{q}_1 + Fq^P_{1/2}
\end{eqnarray}

\begin{eqnarray}
  ( U^{g}_{00} +  \alpha^{g}  ) \delta T_0 - \gamma^{g1} \delta T_1
                                           - \gamma^{g2} \delta q_1
      = f^{g}_0 - F\theta_{1/2} - L Fq_{1/2}
\end{eqnarray}

Therefore, if we concatenate the three equations above, we get We can
solve for the unknown variables \(\delta T_1, \delta q_1, \delta T_0\).
If we can solve these problems, we can then
({[}{[}solve-x{]}{]}(\#solve-x)) can be solved sequentially as
\(x_2,x_3\). Afterwards, the consistuous flux is applied to the obtained
temperature

\begin{eqnarray}
  F\theta_c  =  F\theta_{1/2} 
                   + \gamma^{T} \delta T_0 -\alpha^{T} \delta T_1 \\
  Fq^P_c  =  Fq^P_{1/2} 
                   + \gamma^{q} \delta T_0 -\alpha^{q} \delta q_1
\end{eqnarray}

Calculate as. Here, we show the case where \(M\) is a general matrix, It
is even simpler since it is actually a triple diagonal matrix.

During the program, For atmospheric parts in
\texttt{MODULE:{[}VFTND1(pimtx.F){]}},
MODULE:{[}GNDHT1(pggnd.F){]}\texttt{for\ the\ underground\ part,\ the\ first\ half\ of\ the\ LU\ decomposition\ method.\ (where\ \$f\textquotesingle{}\_k\$\ is\ obtained),\ In}MODULE:{[}SLVSFC(pgslv.F){]}\texttt{,\ solve\ the\ equation\ of\ \$3\textbackslash{}times\ 3\$,\ Seeking\ \$\textbackslash{}delta\ q\_1,\ \textbackslash{}delta\ G\_1,\ \textbackslash{}delta\ T\_0\$.\ Then,\ in}MODULE:{[}GNDHT2(pggnd.F){]}\texttt{The\ second\ half\ of\ the\ LU\ decomposition\ method\ is\ performed\ and\ the\ rate\ of\ change\ of\ temperature\ in\ the\ ground\ is\ solved,\ Correct\ the\ fluxes\ so\ that\ the\ balance\ is\ matched.\ Also,\ in}MODULE:{[}VFTND2(pimtx.F){]}\texttt{,\ for\ the\ atmosphere\ Solving\ the\ rate\ of\ temperature\ change,\ Fluxes\ are\ corrected\ with}MODULE:{[}FLXCOR(pimtx.F){]}`.

\hypertarget{combined-expression-for-time-difference}{%
\subsubsection{Combined expression for time
difference}\label{combined-expression-for-time-difference}}

The coupling formula for finding \(\delta T_1, \delta q_1, \delta T_0\)
is , Solve three times under different conditions as follows.

Solve for surface wetness \(\beta\) as 1. Surface temperature is a
variable.

\begin{enumerate}
\def\labelenumi{\arabic{enumi}.}
\setcounter{enumi}{1}
\item
  the surface wetness obtained by \texttt{MODULE:{[}GNDBET{]}}. Surface
  temperature is a variable .
\item
  the surface wetness obtained by \texttt{MODULE:{[}GNDBET{]}}. In case
  of snowmelt, the surface temperature is fixed at the freezing point.
\end{enumerate}

The first calculation is performed to estimate the possible evaporation
rate, \(Fq^P\). (When the surface wetness is small, the energy balance
of the model indicates that Using the obtained \(Fq\), the possible
evaporation rate can be calculated by \(\widetilde{Fq^P} = Fq / \beta\)
diagnosed as, would result in unrealistically large values). Possible
evaporation rate is ,

\begin{eqnarray}
  Fq^P_c = Fq^P_{1/2}
       + \frac{\partial Fq^P_{1/2}}{\partial q_1} \delta q_1 2 \Delta t 
       + \frac{\partial Fq^P_{1/2}}{\partial T_0} \delta T_0 2 \Delta t 
\end{eqnarray}

That would be. The subscript \(c\) means after correction, This is a
consistent flux to the obtained temperature etc.

In the second and subsequent calculations ,

\begin{enumerate}
\def\labelenumi{\arabic{enumi}.}
\tightlist
\item
  to the amount of possible evaporation found in the first calculation
  Surface wetness (evaporation efficiency) \(\beta\) multiplied by the
  amount of evaporation \(Fq_1\).
\end{enumerate}

\begin{eqnarray}
        Fq = \beta Fq^P
\end{eqnarray}

\begin{enumerate}
\def\labelenumi{\arabic{enumi}.}
\setcounter{enumi}{1}
\tightlist
\item
  evaporation quantity \(Fq_1\) is
\end{enumerate}

\begin{eqnarray}
        \beta \rho C_E |\mathbf{v}| ( q_*(T_0) - q )
\end{eqnarray}

As required by , Once again, rebalancing the energy.

Two methods of calculating the amount of evaporation can be used (The
standard uses the method in 1.). The third calculation is performed
during snow and ice melt and sea ice formation in the mixed layer ocean.
This is done in order to fix the surface temperature at the freezing
point, for example, to unbalance the energy. In this case, the amount of
energy used for the phase change of water, such as snowmelt, is
diagnostically determined, It will be used later to calculate the amount
of snow melt, etc.

The concrete form of the coupling formula is as follows.

\begin{eqnarray}
  \renewcommand{\arraystretch}{1.5}
  \left( \begin{array}{ccc}
      U_{11}^T - \frac{\partial F\theta_{1/2}}{\partial T_1} 2 \Delta t 
      0 
      - \left( \frac{\partial F\theta_{1/2}}{\partial T_0} 
                         + \frac{\partial FR_{1/2}}{\partial T_0}\right) 2 \Delta t \\
      0 
      U_{11}^q - \beta \frac{\partial Fq^P_{1/2}}{\partial q_1} \Delta t 
      - \beta \frac{\partial Fq^P_{1/2}}{\partial T_0} \Delta t \\
        \frac{\partial F\theta_{1/2}}{\partial T_1} \Delta t 
      L \beta \frac{\partial Fq_{1/2}}{\partial q_1} \Delta t 
      U_{00}^g + \left(\frac{\partial F\theta_{1/2}}{\partial T_0}
                + L \beta \frac{\partial Fq_{1/2}}{\partial T_0}
                + \frac{\partial FR_{1/2}}{\partial T_0} \right) 2 \Delta t \\
  \end{array} \right)
  \left( \begin{array}{l}
      \delta T_1 \\ \delta q_1 \\ \delta T_0 \\
  \end{array} \right)   \\
=  \left( \begin{array}{l}
      f_1^T + F\theta_{1/2} \\  
      f_1^q + \beta Fq^P_{1/2} \\  
      f_0^g - F\theta_{1/2} - L \beta Fq^P_{1/2} \\  
  \end{array} \right) \; .
\end{eqnarray}

\begin{quote}
\protect\hypertarget{combin-eq}{}{\textbackslash brain{[}combin-eq{]}}
\end{quote}

Here, \(U_{11}^T, U_{11}^q, U_{00}^g\), \(U_{11}^T, U_{11}^q, U_{00}^g\)
and \(f_1^T, f_1^q, f_0^g\), \(f_1^T, f_1^q, f_0^g\) are, The components
of the matrices and vectors obtained by doing the first half of the LU
decomposition method. When the ground is covered with snow or ice,
instead of the latent heat \(L\) Using the Latent Heat of Sublimation
\(Ls = L + L_M\) \(L_M\) is the latent heat of melting of water.
However, in the second calculation, If the first method is used as an
estimate of evaporation, we get the following.

\begin{eqnarray}
 \renewcommand{\arraystretch}{1.5}
  \left( \begin{array}{ccc}
      U_{11}^T - \frac{\partial F\theta_{1/2}}{\partial T_1} 2 \Delta t 
      0 
      - \left( \frac{\partial F\theta_{1/2}}{\partial T_0} 
                         + \frac{\partial FR_{1/2}}{\partial T_0}\right) 2 \Delta t \\
      0 
      U_{11}^q - \beta \frac{\partial Fq^P_{1/2}}{\partial q_1} \Delta t  \\
        \frac{\partial F\theta_{1/2}}{\partial T_1} \Delta t  
      U_{00}^g + \left(\frac{\partial F\theta_{1/2}}{\partial T_0}
                + \frac{\partial FR_{1/2}}{\partial T_0} \right) 2 \Delta t \\
  \end{array} \right)
  \left( \begin{array}{l}
      \delta T_1 \\ \delta q_1 \\ \delta T_0 \\
  \end{array} \right)   \\
=
  \left( \begin{array}{l}
      f_1^T + F\theta_{1/2} \\  
      f_1^q + \beta Fq^P_c \\  
      f_0^g - F\theta_{1/2} - L \beta Fq^P0_c \\  
  \end{array} \right) \; .
\end{eqnarray}

In the third calculation, the concatenation equation for a fixed surface
temperature is

\begin{eqnarray}
 \renewcommand{\arraystretch}{1.5}
  \left( \begin{array}{cc}
      U_{11}^T - \frac{\partial F\theta_{1/2}}{\partial T_1} 2 \Delta t 
      0 \\
      0 
      U_{11}^q - \frac{\partial Fq_{1/2}}{\partial q_1} 2 \Delta t \\
  \end{array} \right)
  \left( \begin{array}{l}
      \delta T_1 \\ \delta q_1 \\
  \end{array} \right)  \\
=
  \left( \begin{array}{l}
      f_1^T + + F\theta_{1/2}
      + \left( \frac{\partial F\theta_{1/2}}{\partial T_0} 
                    + \frac{\partial FR_{1/2}}{\partial T_0} \right) \delta_0 T_0 2 \Delta t \\
      f_1^q +  \beta Fq^P_{1/2} 
      + \frac{\partial Fq_{1/2}}{\partial T_0} \delta_0 T_0 2 \Delta t \\
  \end{array} \right) \; .
\end{eqnarray}

\begin{quote}
\protect\hypertarget{combin-eq3}{}{\brain[combin-eq3]}
\end{quote}

Here, \(\delta_0 T_0\) is the rate of change to the temperature to be
fixed,

\begin{eqnarray}
   \delta_0 T_0 = ( T_0^{fix} - T_0 ) / \Delta t \; .
\end{eqnarray}

\(T_0^{fix}\) is 273.15K for snow and ice melting, In the case of sea
ice production it is 271.15K. If the second method of evaporation
calculation is used, then Similarly, use \(Fq^P_c\) instead of
\(Fq^P_{1/2}\), Calculate the differential term of \(F_q\) as 0. In this
case,

\begin{eqnarray}
\Delta s  =  f_0^g - F\theta_{1/2} - L \beta Fq^P_{1/2} - U_{00}^g
          -  \left(\frac{\partial F\theta_{1/2}}{\partial T_0}
                + L \beta \frac{\partial Fq_{1/2}}{\partial T_0}
                + \frac{\partial FR_{1/2}}{\partial T_0} \right) \delta_0 T_0 \Delta t 
                 \\
          -  \frac{\partial F\theta_{1/2}}{\partial T_1} \delta T_1 \Delta t
                - L \beta \frac{\partial Fq^P_{1/2}}{\partial q_1} \delta q_1 \Delta t \; .
\end{eqnarray}

\(\Delta s\), calculated by the Surface Energy Balance, It is the amount
of energy used for the phase change of water.

\hypertarget{implicit-treatment-of-time-steps-in-time-differences}{%
\subsubsection{implicit Treatment of Time Steps in Time
Differences}\label{implicit-treatment-of-time-steps-in-time-differences}}

Although the implicit method is used for the time difference of the
vertical diffusion term, In general, the diffusion coefficients are
nonlinear, and we explicitly evaluate these coefficients This can cause
problems of numerical instability. To improve stability, Kalnay and
Kanamitsu (19?) following the I'm working on how to handle the time
steps.

For simplicity, we will take the following ordinary differential
equations as an example.

\begin{eqnarray}
  \frac{\partial X}{\partial t} = - K(X) X
\end{eqnarray}

The coefficient \(K(X)\) represents the nonlinearity. If we evaluate
only the coefficients explicitly and make them implicitly different, we
get the following equation.

\begin{eqnarray}
  \frac{X^{m+1} - X^m}{\Delta t} = - K( X^m ) X^{m+1}
\end{eqnarray}

\begin{quote}
\protect\hypertarget{normal-fd}{}{\centric[normal-fd]}
\end{quote}

However, consider the value of \(X\) two steps ahead, \(X^{\ast}\),

\begin{eqnarray}
  \frac{X^{\ast} - X^m}{2\Delta t} = - K ( X^m ) X^{\ast}
\end{eqnarray}

\begin{quote}
\protect\hypertarget{modify-fd1}{}{{[}modify-fd1\textbackslash blade{]}}
\end{quote}

\begin{eqnarray}
  X^{m+1} = \frac{X^{\ast} + X^m}2
\end{eqnarray}

\begin{quote}
\protect\hypertarget{modify-fd2}{}{{[}modify-fd2\Backlash{]}}
\end{quote}

. Generally, (\protect\hyperlink{modify-fd1}{modify-fd1{]}}),
(\protect\hyperlink{normal-fd}{modify-fd2{]}(\#modify-fd2)) is better
({[}normal-fd{]}}) is known to be more stable than {[}normal-fd{]}).

({[}modifie-fd1{]}(\#modifie-fd1)),
(\protect\hyperlink{modifie-fd2}{modifie-fd2\lenses}) to find the rate
of change in time Rewriting it into a form yields the following.

\begin{eqnarray}
  \left(\frac{\Delta X}{\Delta t}\right)^{\ast} = 
     - K( X^m ) \left( X^m + 
        \left(\frac{\Delta X}{\Delta t}\right)^{\ast} 
        2 \Delta t \right)
\end{eqnarray}

\begin{eqnarray}
  X^{m+1} = X^m + \left(\frac{\Delta X}{\Delta t}\right)^{\ast} \Delta t
\end{eqnarray}

That is, the time step in determining the rate of change of the rate of
change of time includes the following, Using twice the time integration
step.
