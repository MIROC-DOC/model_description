\hypertarget{cumulus-convection}{%
\subsection{cumulus convection}\label{cumulus-convection}}

\hypertarget{overview-of-the-cumulus-convection-scheme}{%
\subsubsection{Overview of the Cumulus Convection
Scheme}\label{overview-of-the-cumulus-convection-scheme}}

The cumulus convection scheme is , This figure represents the
condensation, precipitation and convection processes involved in cumulus
convection, Due to the latent heat release and associated convective
motion Calculate precipitation with temperature and with changes in
water vapor content. We also calculate the cloud water content and cloud
coverage involved in the radiation. The main input data are temperature
\(T\) and specific humidity \(q\), The output data is the time rate of
change of temperature and specific humidity,
\(\partial T/\partial t, \partial q/\partial t, \partial l/\partial t\),
The cloud water content of the cumulus clouds used for radiation is
\(l^{cR}\) Cloud volume \(C^c\).

The framework of the cumulus convection scheme is Basically based on
Arakawa and Schubert (1974). Vertical air columns in one horizontal
grid. Considered as the basic unit of parameterization. Clouds are
determined by the temperature, specific humidity, cloud water content
and Characterized by a vertical upward mass flux, Considering multiple
clouds with different cloud tops within a single vertical air column.
Clouds occupy part of the horizontal lattice, and the rest of the
surrounding region is There is a downward flow equal to the cloud mass
flux (compensating downward flow). This compensatory downward flow and
outflow of air into the surrounding region in the clouds (detraining)
The temperature and the specific humidity field in the surrounding
region are changed by The area of the upwelling of the cumulus
convection is assumed to be small, The lattice-averaged temperature and
specific humidity fields and Since we treat the temperature and specific
humidity fields in the surrounding area as the same, we are able to This
gives the changes in the lattice mean temperature and specific humidity.

It is the cloud model that determines the temperature, specific humidity
and cloud water content in clouds. Here, we use an entrained-purume
model, As with Moorthi and Suarez (1992) , We assume a linear mass flux
increase with respect to height. The cloud base is used as the lifted
condensation height of the surface atmosphere, of the percentage of air
uptake (entrainment) in the surrounding area. Consider multiple cloud
top altitudes depending on the difference. However, if a cloud with a
cloud base cannot exist, then Consider the possibility of clouds with
higher cloud bases.

The mass flux of each cloud is diagnostically determined using the cloud
work function. The cloud work function is defined as It is defined as
the vertical integral of the work done by buoyancy. This cloud-work
function is driven by the compensating downward motion of cumulus
clouds, etc. It gives a mass flux that approaches zero at a certain
relaxation time.

In addition, the evaporation of precipitation and The effect of the
downdrafting that goes with it. Consider in a very simple way .

The outline of the calculation procedure is as follows. Parentheses are
the names of the corresponding subroutines.

\begin{enumerate}
\def\labelenumi{\arabic{enumi}.}
\item
  cloud-bottom height as the lifted condensation height of the surface
  atmosphere Evaluate .
\item
  using a cloud model, Corresponding to each cloud top altitude of cloud
  temperature, specific humidity, cloud water content, and mass flux
  (relative value) Calculates the vertical distribution
  \texttt{MODULE:{[}UPDRF{]}}.
\item
  calculate the cloud work function \texttt{MODULE:{[}CWF{]}}.
\item
  due to a cloud of unit mass fluxes. Calculates the hypothetical change
  of temperature and specific humidity in the surrounding area
  \texttt{MODULE:{[}CLDTST{]}}.
\item
  for a hypothetical change in temperature and specific humidity
  Calculate the cloud work function \texttt{MODULE:{[}CWF{]}}.
\item
  using the cloud work function before and after the virtual change
  Calculates the cloud mass flux at the cloud base
  \texttt{MODULE:{[}CBFLX{]}}.
\item
  the cloud mass flux detrainment. Calculate the vertical distribution
  and precipitation \texttt{MODULE:{[}CMFLX{]}}.
\item
  evaluate cloud water and cloud cover due to cumulus clouds
  \texttt{MODULE:{[}CMCLD{]}}.
\item
  by detainment. Calculate the change of temperature and specific
  humidity \texttt{MODULE:{[}CLDDET{]}}.
\item
  by compensatory downstream flow. Calculate the change of temperature
  and specific humidity \texttt{MODULE:{[}CLDSBH{]}}.
\item
  evaporation of precipitation and The downdraft. of cloud temperature,
  specific humidity and mass flux Calculates the vertical distribution
  \texttt{MODULE:{[}DWNEVP{]}}.
\item
  by downdraft detrainment. Calculate the change of temperature and
  specific humidity \texttt{MODULE:{[}CLDDDE{]}}.
\item
  by the compensatory upward flow of downdrafts. Calculate the change of
  temperature and specific humidity \texttt{MODULE:{[}CLDSBH{]}}.
\end{enumerate}

\hypertarget{the-basic-framework-of-the-arakawa-schubert-scheme}{%
\subsubsection{The Basic Framework of the Arakawa-Schubert
Scheme}\label{the-basic-framework-of-the-arakawa-schubert-scheme}}

Cloud Mass Flux \(M\), Detrainment \(D\) is,

\begin{eqnarray}
  M(z)     =  M_B \eta(z) \; , \\
  D(z)     =  M_B \eta(z_T) \delta (z-z_T) \; .
\end{eqnarray}

represented as . The mass flux at the cloud base (\(M_B\)) is the mass
flux at \(z_B\), \(\eta\) is a dimensionless mass flux in it.

From this, the time variation of the mean field is calculated as

\begin{eqnarray}
  \frac{\partial \bar{h}}{\partial t}  =  M \frac{\partial \bar{h}}{\partial z} 
                       + D( h^t - \bar{h} ) \; , \\
  \frac{\partial \bar{q}}{\partial t}  =  M\frac{\partial \bar{q}}{\partial z} 
                       + D( q^t + l^t - \bar{q} ) \; .
\end{eqnarray}

However, \(\bar{h}, \bar{q}\) are based on the wet static energy of the
mean field and the specific humidity, \(h^t, q^t, l^t\) are the air in
the detrainment It is the wet static energy, specific humidity, and
cloud water content.

\(\eta, h^t, q^t, l^t\) are required by the cloud model. \(M_B\) is
obtained by the closure assumption using the cloud work function.

\hypertarget{cloud-model.}{%
\subsubsection{Cloud Model.}\label{cloud-model.}}

The cloud model is essentially an entrained-purume model. Each type of
cloud is characterized by an entrainment rate, It will have various
cloud top heights accordingly. However, for the sake of later
calculations, Here, you can specify the cloud top altitude, By finding
the corresponding entrainment rate Find the vertical structure of
clouds. By assuming a linear mass flux increase with respect to height.
This calculation is simplified to a form that does not include a
sequential approximation.

Let's set the cloudbase altitude at \(z_T\), The lifted condensation
altitude of the surface atmosphere, i.e., the height of condensation,

\begin{eqnarray}
   \bar{q}(0) \geq
                \bar{q}^*(z)
                + \frac{\gamma}{L(1+\gamma)} 
                    \left(\bar{h}(0)-\bar{h}(z) \right) \; , 
\end{eqnarray}

Define it as the minimum \(z\) that meets the following criteria

The dimensionless mass flux \(\eta\) is, The entrainment rate is set to
\(\lambda\),

\begin{eqnarray}
  \frac{\partial \eta}{\partial z} = \lambda \; ,
\end{eqnarray}

Namely,

\begin{eqnarray}
  \eta (z)  =  1 + \lambda ( z - z_B ) \\
            \equiv  1 + \lambda \hat{\eta}(z)  \; .
\end{eqnarray}

The balance on wet static energy \(h^c\) and total water content \(w^c\)
in the clouds is,

\begin{eqnarray}
  \frac{\partial }{\partial z}( \eta h^c   )  =  \lambda \bar{h}      \; , \\
  \frac{\partial }{\partial z}( \eta w^c   )  =  \lambda \bar{q} - \pi  \; .
\end{eqnarray}

Here, \(\bar{h}, \bar{q}, \pi\) are respectively, \(h\) and \(q\), in
mean field, are precipitation generation.

Integrating,

\begin{eqnarray}
   \eta (z) h^c(z)  =  h^c(z_B)
                         + \lambda \int_{z_B}^{z} \bar{h}(\xi) d\xi \\
                    \equiv  h^c(z_B) + \lambda \hat{h}^c(z) \; ,
\end{eqnarray}

\begin{eqnarray}
   \eta (z) w^c(z)  =  w^c(z_B)
                         + \lambda \int_{z_B}^{z} \bar{q}(\xi) d\xi
                         - R(z) \\
                    \equiv  w^c(z_B) + \lambda \hat{w}^c(z) 
                              - R(z)             \\
                    \equiv  \eta(z) w^a(z) - R(z)  \; . 
\end{eqnarray}

The mass flux is assumed to be zero at the surface, It is assumed to
increase linearly below the cloud base,

\begin{eqnarray}
 \eta (z) =   \frac{z}{z_B} \; \; \; ( z<z_B ) \; .
\end{eqnarray}

By calculating the entrainment below this cloud base, \(h^c,w^c\) are
required at cloudbase. That is, ,

\begin{eqnarray}
  h^c(z_B)  =  \frac{1}{z_B} \int_0^{z_B} \bar{h}(z) dz \; , \\
  w^c(z_B)  =  \frac{1}{z_B} \int_0^{z_B} \bar{q}(z) dz \; .
\end{eqnarray}

The buoyancy per unit mass flux due to clouds is ,

\begin{eqnarray}
   B  =   \frac{g}{\bar{T}} ( T_v^c - \bar{T}_v ) \\
      =   \frac{g}{\bar{T}} 
            \left[ T^c ( 1+\epsilon q^c-l^c ) 
                      - \bar{T} ( 1+\epsilon \bar{q} ) \right] \\
      \simeq  \frac{g}{\bar{T}} 
               \left[ ( T^c - \bar{T} ) 
               - \bar{T} \left( \epsilon(q^c-\bar{q}) -l^c \right) 
                                                     \right] \\
      \simeq  \frac{g}{\bar{T}} 
                \left[ \frac{1}{C_p(1+\gamma)} (h^c-\bar{h}^*)
                       + \bar{T} \left( \epsilon \frac{\gamma}{L(1+\gamma)} 
                                                     (h^c-\bar{h}^*)
                               + \epsilon (\bar{q}^* - \bar{q} ) 
                               - l^c                      \right) \right] \; .
\end{eqnarray}

where \(T_v\) is the provisional temperature and \(q^*\) is the
saturation specific humidity, \(\epsilon = R_{{H}_2{O}}/R_{{air}} -1\),
It is \(\gamma = L/C_p \partial q^*/\partial T\),
\(\bar{q}^*, \bar{h}^*\) indicate the values at mean-field saturation,
respectively. \(q^c, l^c\) are the amounts of cloud water vapor and
cloud water,

\begin{eqnarray}
  q^c  =  q^*(T^c) \, \simeq \,
           \bar{q}^* + \frac{1}{L(1+\gamma)} ( h^c - \bar{h}^* ) \; , \\  
  l^c  =  w^c - q^c \; .
\end{eqnarray}

For the cloud top \(z_T\), the buoyancy \(B\) is assumed to be zero.
Thus, solving the \(B(z_T)=0\) corresponds to the given cloud top height
of \(z_T\) \(\lambda\) can be obtained. Here, for precipitation rate
\(R(z)\) integrated from the ground upward, we have a problem, Using the
known function \(r(z)\) Assume that it is represented.

\begin{eqnarray}
  R(z)   = \eta(z) r(z) \left[ w^a(z) - q^c(z) \right] \; .
\end{eqnarray}

So..,

\begin{eqnarray}
\frac{\bar{T}}{g} B \simeq 
 \frac{1}{1+\gamma} 
 \left[ \frac{1}{C_p} + \bar{T} (\epsilon+1-r) \frac{\gamma}{L} \right]
  (h^c-\bar{h}^*)
  + (\epsilon+1-r) \bar{T} \bar{q}^* 
  - \epsilon  \bar{T} \bar{q}
  - \bar{T} (1-r) w^a \; .
\end{eqnarray}

\$B(z\_T) =0 \$ is easy to solve and,

\begin{eqnarray}
  \lambda = \frac{ a\left[ h^c(z_B)-\bar{h}^*(z_T) \right]
                  +\bar{T}(z_T)\left[ b -(1-r(z_T))q^c(z_B) \right] }
                 { a\left[ \hat{\eta}(z_T) \bar{h}^*(z_T) 
                               - \hat{h}^c(z_T) \right]
                  -\bar{T}(z_T)\left[ b \hat{\eta}(z_T) 
                                     - (1-r(z_T))\hat{q}_t^c(z_T) \right] }
\end{eqnarray}

However,

\begin{eqnarray}
a  \equiv  \frac{1}{1+\gamma}
             \left[ \frac{1}{C_p} 
                + \bar{T}(z_T) 
                  \left( \epsilon+1-r(z_T) \right) 
               \frac{\gamma}{L}                \right] \; ,\\
b  \equiv  \left(\epsilon+1-r(z_T) \right) \bar{q}^*(z_T) 
                    - \epsilon \bar{q}(z_T) \; .
\end{eqnarray}

As mentioned above, you should specify \(\lambda\) to obtain \(z_T\), A
physically meaningful \(\lambda\) for a given \(z_T\) There is no
guarantee that we will seek it. That scrutiny is necessary, but here it
is, The smaller the \(\lambda\) is, the more the \(z_T\) is Take into
account that it should be lower.

\begin{eqnarray}
  \frac{\partial \lambda}{\partial z_T} < 0
\end{eqnarray}

We will examine whether or not the If the value is not satisfied, assume
that the cloud with cloud top \(z_T\) does not exist. Also, a minimum
value has been set for \(\lambda\), We assume that there are no smaller
\(\lambda\) clouds. This means that the entrainment rate can be reduced
by Given the inverse proportions , The equivalent of having a maximum in
the size of the plume.

Cloud water content \(l^c(z)\) is ,

\begin{eqnarray}
  l^c(z)  =  w^a(z)-q^c(z)-R(z)/\eta(z)   \\
          =  \left( 1-r(z) \right) \left[ w^a(z)-q^c(z) \right] \; .
\end{eqnarray}

However, in the case of \(w^a(z) < q^c(z)\), it is \(l^c(z)=0\).
Furthermore, it is unlikely that a precipitation event will be followed
by cloudy water, \(R(z)\) must be an increasing function of \(z\). This
will limit the \(r(z)\).

The characteristic value of the detrainment air is ,

\begin{eqnarray}
  h^t  =  h^c(z_T) \; , \\
  q^t  =  q^c(z_T) \; , \\
  l^t  =  l^c(z_T) \; .
\end{eqnarray}

In the case of \$ h\^{}c(z\_B) \textless{} \bar\{h\}\^{}* (z\_T) \$,
Suppose that clouds do not exist. In this case,

\begin{eqnarray}
  \bar{h}(z'_B) > \bar{h}^* (z_T) \; , \;\;\; z_B < z < z_T 
\end{eqnarray}

If there is a \(z'_B\) that satisfies , The area directly above it has
been renamed \(z_B\),

\begin{eqnarray}
  h^c(z_B)  =  \bar{h}(z'_B) \; , \\
  w^c(z_B)  =  \bar{q}(z'_B) \; 
\end{eqnarray}

Seek as .

\hypertarget{cloud-work-function-cwf}{%
\subsubsection{Cloud Work Function
(CWF)}\label{cloud-work-function-cwf}}

The cloud work function (CWF), \(A\) is,

\begin{eqnarray}
  A \equiv \int_{z_B}^{z_T} B \eta dz 
\end{eqnarray}

It is,

\begin{eqnarray}
A = \int_{z_B}^{z_T} \frac{g}{\bar{T}} \left[
        (T^c-\bar{T})
      + \bar{T} \left\{ \epsilon (q^c - \bar{q} ) 
                     - l^c                 \right\}
       \right] \eta dz \; .
\end{eqnarray}

\begin{quote}
\protect\hypertarget{p-cum:cwf}{}{\blaze[p-cum:cwf]}
\end{quote}

Essentially, the work associated with the downdraft, discussed below,
should be It should be accounted for, but we'll ignore it here for
simplicity's sake.

In this calculation, we start at the bottom and Once a positively
buoyant cloud is negatively buoyant, if , Since there should be cloud
tops where they are supposed to be negative, Assume that the cloud with
the cloud top we are considering does not exist.

\hypertarget{cloud-mass-flux-at-cloudbase}{%
\subsubsection{Cloud Mass Flux at
Cloudbase}\label{cloud-mass-flux-at-cloudbase}}

The cloud mass flux at the cloud base is , On some time scale
\(\tau_a\), Cloud action determines the cloud work function to be close
to zero I make the assumption that.

In order to estimate it, we firstly estimated the unit cloud-bottom mass
flux of \(M_0\) Find the time variation of the mean field.

\begin{eqnarray}
  \frac{\partial \bar{h}'}{\partial t}  =  M_0 \eta \frac{\partial \bar{h}}{\partial z} 
                       + \eta(z_T) \delta(z-z_T) ( h^t - \bar{h} ) \; , \\
  \frac{\partial \bar{q}'}{\partial t}  =  M_0 \eta \frac{\partial \bar{q}}{\partial z} 
                       + \eta(z_T) \delta(z-z_T) ( q^t + l^t - \bar{q} ) \; .
\end{eqnarray}

With this,

\begin{eqnarray}
  \bar{h}'  =  \bar{h} + \frac{\partial \bar{h}'}{\partial t} \delta t \; , \\
  \bar{q}'  =  \bar{q} + \frac{\partial \bar{q}'}{\partial t} \delta t 
\end{eqnarray}

and using \(\bar{h}', \bar{q}'\) Let \(A'\) be the cloud work function
calculated from (\protect\hyperlink{p-cum:cwf}{p-cum:cwf{]}}).

So..,

\begin{eqnarray}
  M_B = \frac{A}{A-A'} \frac{\delta t}{\tau_a} M_0 
\end{eqnarray}

That would be. Here, when obtaining \(A'\), the original
\(\bar{h}', \bar{q}'\) were used I should recalculate the vertical
structure of the clouds, Now we are using the same cloud structure.

\hypertarget{cloud-mass-flux-precipitation}{%
\subsubsection{Cloud Mass Flux,
Precipitation}\label{cloud-mass-flux-precipitation}}

The sum of the clouds at each cloud top altitude, Cloud Mass Flux \(M\)

\begin{eqnarray}
  M(z)   = \int^i M_B^i \eta^i(z) \; .
\end{eqnarray}

Also, precipitation flux \(P(z)\) is

\begin{eqnarray}
 P(z) = \int_i M_B^i \left[ R^i(z_T)-R^i(z) \right]  \; .
\end{eqnarray}

\hypertarget{time-variation-of-the-average-field}{%
\subsubsection{Time variation of the average
field}\label{time-variation-of-the-average-field}}

by compensated downstream flow and detraining. The time variation of the
mean field is calculated as follows

\begin{eqnarray}
  \frac{\partial \bar{h}}{\partial t}  =  M \frac{\partial \bar{h}}{\partial z} 
                    + \int_i D^i ( (h^t)^i - \bar{h} ) \; , \\
  \frac{\partial \bar{q}}{\partial t}  =  M\frac{\partial \bar{q}}{\partial z} 
                    + \int_i D^i ( (q^t)^i + (l^t)^i - \bar{q}(z_T^i) ) \; .
\end{eqnarray}

However, it is \(D^i = M_B^i \eta^i(z_T^i)\).

\hypertarget{evaporation-and-downdrafting-of-precipitation}{%
\subsubsection{Evaporation and downdrafting of
precipitation}\label{evaporation-and-downdrafting-of-precipitation}}

Precipitation falls through the unsaturated atmosphere, while some of it
evaporates. In addition, some of them form a downdraft.

Evaporation Rate \(E\) is ,

\begin{eqnarray}
 E = \rho a_e {\rho_p}^{b_e} \left( \bar{q}_{w} - \bar{q} \right) \; ,
\end{eqnarray}

Note that \(\bar{q}_{w}\) is the saturation specific humidity
corresponding to the wet bulb temperature,

\begin{eqnarray}
  \bar{q}_w = \bar{q} 
            + \frac{\bar{q}^* - \bar{q}}{1+ \frac{L}{C_P}\frac{\partial q^*}{\partial T}} \; .
\end{eqnarray}

\(a_e, b_e\) is a parameter of the microphysics. \(\rho_p\) is the
density of precipitation particles and \(V_T\) is the terminal velocity
of precipitation,

\begin{eqnarray}
  \rho_p = \frac{P}{V_T} \; .
\end{eqnarray}

The current standard values are \(a_e=0.25\), \(b_e=1\) and \(V_T=10\)
m/s.

For downdrafting, we make the following assumptions.

\begin{itemize}
\item
  \(\bar{h}\) decreases monotonically with altitude above cloudbase If
  the upper end of the region is set to \(z_d\), the downdraft is It
  occurs in the region of \(z < z_d\).
\item
  A certain percentage of the precipitation evaporation that occurs at
  each altitude It is used to form downdrafts. Evaporation of
  precipitation has just saturated it. The air in the surrounding area.
  Taken into the downdraft (Entrainment).
\item
  In \(z < z_B\), detraining occurs, The mass flux decreases linearly.
\end{itemize}

That is, in \(z_B < z < z_d\), the mass flux \(M^d(z)\), The downdraft
air masses \(h^d(z),q^d(z)\) follow the following equation. Upon
evaporation of precipitation, the wet static energy should be conserved,
and the specific humidity when saturated by evaporation. Note that this
is \(\bar{q}_{w}\).

\begin{eqnarray}
  \frac{\partial M^d}{\partial z} =  - f_d \frac{E}{\bar{q}_{w}-\bar{q}} \;  ,
\end{eqnarray}

\begin{eqnarray}
  \frac{\partial }{\partial z} ( M^d h^d )  =  \bar{h}     \frac{\partial M^d}{\partial z} \; ,\\
  \frac{\partial }{\partial z} ( M^d q^d )  =  \bar{q}_{w} \frac{\partial M^d}{\partial z} \; .
\end{eqnarray}

In the above equation, \(f_d\) is the portion of the evaporation that is
taken up by the downdraft, \((1-f_d)\) evaporates directly into the mean
field. However, the downdraft mass flux \(M^d\) is The total mass flux
of cloud base shall not exceed the \(f_m\) of \(M\). The current
standard value is \(f_d=0.5, f_m=1.0\).

\hypertarget{cloud-water-and-cloud-cover}{%
\subsubsection{cloud water and cloud
cover}\label{cloud-water-and-cloud-cover}}

The lattice-averaged cloud water content used for radiation, \(l^{cR}\),
is Strong upwelling areas of cumulus clouds, including cloud water
\(l^c\) If the ratio of the ratio to the \(\delta^c\) ,

\begin{eqnarray}
  l^{cR} = \delta^c l^c \; .
\end{eqnarray}

The mass flux \(M^c\) is the same as this \(\delta^c\) Using the
vertical velocity of the upstream stream, \(v^c\)

\begin{eqnarray}
  M = \delta^c \rho v^c 
\end{eqnarray}

So, in the end,

\begin{eqnarray}
  l^{cR} = \frac{M^c}{\rho v^c} l^c = \alpha M^c l^c \; .
\end{eqnarray}

The cloud cover used to estimate radiation, \(C^c\), is , that there is
actually a horizontal spread in the distribution of upwelling and cloud
water. It is reasonable to take a larger value than this \(\delta^c\).
Here, in brief,

\begin{eqnarray}
  C^c = \beta M_B
\end{eqnarray}

. The current standard values are \(\alpha=0.3\) and \(\beta=10\).
