
\subsection{基本的な設定}

ここでは, モデルの基本的な設定を示す.

\subsubsection{座標系}

座標系は, 基本的に,
経度 TERM00134, 緯度 TERM00135, 正規化気圧 TERM00136 
(TERM00137,TERM00137 は地表気圧)
を用い, それぞれは直交するとして扱う.
ただし, 地中の鉛直座標は TERM00138 を用いる.

経度は等間隔に離散化される \texttt{MODULE:[ASETL]}.
\begin{verbatim}
EQ=00005.
\end{verbatim}

緯度は力学の項で述べる Gauss 緯度 TERM00139 であり \texttt{MODULE:[ASETL]},
Gauss-Legendre 積分公式から導かれる.
これは, TERM00140 を引数とする
J 次の Legendre 多項式の 0 点である \texttt{MODULE:[GAUSS]}. 

J が大きい場合には, 近似的に,
\begin{verbatim}
EQ=00006.
\end{verbatim}

通常, 経度・緯度の格子間隔はほぼ等しく TERM00141 と取る. 
これは, スペクトル法の三角形切断に基づく.

正規化気圧 TERM00142 は, 大気の鉛直構造を良く表現するように,
不等間隔で適当に離散化される \texttt{MODULE:[ASETS]}.
後で力学の項で述べるように, 層の境界の値
TERM00143 を TERM00144 で定義してから,
%
\begin{verbatim}
EQ=00007.
\end{verbatim}
によって層を代表する TERM00145 を求める.
図\ref{a-setup:level}に, 標準的に用いられる 20層のレベルを示す.

\begin{figure}[hbtp]
  \begin{center}
    \epsfile{file=vert-cord.ps,width=80mm}
  \end{center}
  \caption{標準的に用いられるレベル}
  \label{a-setup:level}
\end{figure}

各予報変数は全て, TERM00146,TERM00146
または TERM00147,TERM00147 の格子上で定義される.
(地中のレベル TERM00148 については物理過程の項で述べる.)

時間方向には, 等間隔 TERM00149 で離散化され,
予報方程式の時間積分が行なわれる.
ただし, 時間積分の安定性が損なわれるおそれのあるときには
TERM00150 は変化しうる.

\subsubsection{物理定数}

基本的な物理定数を以下に示す \texttt{MODULE:[APCON]}.

\begin{description}
\item[TAB00002:0.0] 地球半径
\item[TAB00002:0.1] TERM00151
\item[TAB00002:0.2] m
\item[TAB00002:0.3] 6.37 TERM00152
\item[TAB00002:1.0] 重力加速度
\item[TAB00002:1.1] TERM00153
\item[TAB00002:1.2] TERM00154
\item[TAB00002:1.3] 9.8
\item[TAB00002:2.0] 大気定圧比熱
\item[TAB00002:2.1] TERM00155
\item[TAB00002:2.2] J TERM00156 TERM00157
\item[TAB00002:2.3] 1004.6
\item[TAB00002:3.0] 大気気体定数
\item[TAB00002:3.1] TERM00158
\item[TAB00002:3.2] J TERM00159 TERM00160
\item[TAB00002:3.3] 287.04
\item[TAB00002:4.0] 水の蒸発潜熱
\item[TAB00002:4.1] TERM00161
\item[TAB00002:4.2] J TERM00162
\item[TAB00002:4.3] 2.5 TERM00163
\item[TAB00002:5.0] 水蒸気定圧比熱
\item[TAB00002:5.1] TERM00164
\item[TAB00002:5.2] J TERM00165 TERM00166
\item[TAB00002:5.3] 1810.
\item[TAB00002:6.0] 水の気体定数
\item[TAB00002:6.1] TERM00167
\item[TAB00002:6.2] J TERM00168 TERM00169
\item[TAB00002:6.3] 461.
\item[TAB00002:7.0] 液体水の密度
\item[TAB00002:7.1] TERM00170
\item[TAB00002:7.2] J TERM00171 TERM00172
\item[TAB00002:7.3] 1000.
\item[TAB00002:8.0] 0 TERM00173 での
飽和蒸気
\item[TAB00002:8.1] TERM00174(273K)
\item[TAB00002:8.2] Pa
\item[TAB00002:8.3] 611
\item[TAB00002:9.0] Stefan Bolzman
定数
\item[TAB00002:9.1] TERM00175
\item[TAB00002:9.2] W TERM00176 TERM00177
\item[TAB00002:9.3] 5.67 
                                                          TERM00178
\item[TAB00002:10.0] K\'{a}rman 定数
\item[TAB00002:10.1] TERM00179
\item[TAB00002:10.2] 
\item[TAB00002:10.3] 0.4
\item[TAB00002:11.0] 氷の融解潜熱
\item[TAB00002:11.1] TERM00180
\item[TAB00002:11.2] J TERM00181
\item[TAB00002:11.3] 3.4 TERM00182
\item[TAB00002:12.0] 水の氷点
\item[TAB00002:12.1] TERM00183
\item[TAB00002:12.2] K
\item[TAB00002:12.3] 273.15
\item[TAB00002:13.0] 水の定圧比熱
\item[TAB00002:13.1] TERM00184
\item[TAB00002:13.2] J TERM00185
\item[TAB00002:13.3] 4200.
\item[TAB00002:14.0] 海水の氷点
\item[TAB00002:14.1] TERM00186
\item[TAB00002:14.2] K
\item[TAB00002:14.3] 271.35
\item[TAB00002:15.0] 氷の定圧比熱比
\item[TAB00002:15.1] TERM00187
\item[TAB00002:15.2] 
\item[TAB00002:15.3] 2397.
\item[TAB00002:16.0] 水蒸気分子量比
\item[TAB00002:16.1] TERM00188
\item[TAB00002:16.2] 
\item[TAB00002:16.3] 0.622
\item[TAB00002:17.0] 仮温度の係数
\item[TAB00002:17.1] TERM00189
\item[TAB00002:17.2] 
\item[TAB00002:17.3] 0.606
\item[TAB00002:18.0] 比熱と気体定数の比
\item[TAB00002:18.1] TERM00190
\item[TAB00002:18.2] 
\item[TAB00002:18.3] 0.286
\end{description}

