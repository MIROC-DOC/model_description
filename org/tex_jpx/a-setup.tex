
\subsection{基本的な設定}

ここでは, モデルの基本的な設定を示す.

\subsubsection{座標系}

座標系は, 基本的に,
経度 TERM00000, 緯度 TERM00001, 正規化気圧 TERM00002 
(TERM00003,TERM00003 は地表気圧)
を用い, それぞれは直交するとして扱う.
ただし, 地中の鉛直座標は TERM00004 を用いる.

経度は等間隔に離散化される \texttt{MODULE:[ASETL]}.
\begin{quote}
EQ=00000.
\end{quote}

緯度は力学の項で述べる Gauss 緯度 TERM00005 であり \texttt{MODULE:[ASETL]},
Gauss-Legendre 積分公式から導かれる.
これは, TERM00006 を引数とする
J 次の Legendre 多項式の 0 点である \texttt{MODULE:[GAUSS]}. 

J が大きい場合には, 近似的に,
\begin{quote}
EQ=00001.
\end{quote}

通常, 経度・緯度の格子間隔はほぼ等しく TERM00007 と取る. 
これは, スペクトル法の三角形切断に基づく.

正規化気圧 TERM00008 は, 大気の鉛直構造を良く表現するように,
不等間隔で適当に離散化される \texttt{MODULE:[ASETS]}.
後で力学の項で述べるように, 層の境界の値
TERM00009 を TERM00010 で定義してから,
%
\begin{quote}
EQ=00002.
\end{quote}
によって層を代表する TERM00011 を求める.
図\ref{a-setup:level}に, 標準的に用いられる 20層のレベルを示す.

\begin{figure}[hbtp]
  \begin{center}
    \epsfile{file=vert-cord.ps,width=80mm}
  \end{center}
  \caption{標準的に用いられるレベル}
  \label{a-setup:level}
\end{figure}

各予報変数は全て, TERM00012,TERM00012
または TERM00013,TERM00013 の格子上で定義される.
(地中のレベル TERM00014 については物理過程の項で述べる.)

時間方向には, 等間隔 TERM00015 で離散化され,
予報方程式の時間積分が行なわれる.
ただし, 時間積分の安定性が損なわれるおそれのあるときには
TERM00016 は変化しうる.

\subsubsection{物理定数}

基本的な物理定数を以下に示す \texttt{MODULE:[APCON]}.

\begin{description}
\item[TAB00000:0.0] 地球半径
\item[TAB00000:0.1] TERM00017
\item[TAB00000:0.2] m
\item[TAB00000:0.3] 6.37 TERM00018
\item[TAB00000:1.0] 重力加速度
\item[TAB00000:1.1] TERM00019
\item[TAB00000:1.2] TERM00020
\item[TAB00000:1.3] 9.8
\item[TAB00000:2.0] 大気定圧比熱
\item[TAB00000:2.1] TERM00021
\item[TAB00000:2.2] J TERM00022 TERM00023
\item[TAB00000:2.3] 1004.6
\item[TAB00000:3.0] 大気気体定数
\item[TAB00000:3.1] TERM00024
\item[TAB00000:3.2] J TERM00025 TERM00026
\item[TAB00000:3.3] 287.04
\item[TAB00000:4.0] 水の蒸発潜熱
\item[TAB00000:4.1] TERM00027
\item[TAB00000:4.2] J TERM00028
\item[TAB00000:4.3] 2.5 TERM00029
\item[TAB00000:5.0] 水蒸気定圧比熱
\item[TAB00000:5.1] TERM00030
\item[TAB00000:5.2] J TERM00031 TERM00032
\item[TAB00000:5.3] 1810.
\item[TAB00000:6.0] 水の気体定数
\item[TAB00000:6.1] TERM00033
\item[TAB00000:6.2] J TERM00034 TERM00035
\item[TAB00000:6.3] 461.
\item[TAB00000:7.0] 液体水の密度
\item[TAB00000:7.1] TERM00036
\item[TAB00000:7.2] J TERM00037 TERM00038
\item[TAB00000:7.3] 1000.
\item[TAB00000:8.0] 0 TERM00039 での
飽和蒸気
\item[TAB00000:8.1] TERM00040(273K)
\item[TAB00000:8.2] Pa
\item[TAB00000:8.3] 611
\item[TAB00000:9.0] Stefan Bolzman
定数
\item[TAB00000:9.1] TERM00041
\item[TAB00000:9.2] W TERM00042 TERM00043
\item[TAB00000:9.3] 5.67 
                                                          TERM00044
\item[TAB00000:10.0] K\'{a}rman 定数
\item[TAB00000:10.1] TERM00045
\item[TAB00000:10.2] 
\item[TAB00000:10.3] 0.4
\item[TAB00000:11.0] 氷の融解潜熱
\item[TAB00000:11.1] TERM00046
\item[TAB00000:11.2] J TERM00047
\item[TAB00000:11.3] 3.4 TERM00048
\item[TAB00000:12.0] 水の氷点
\item[TAB00000:12.1] TERM00049
\item[TAB00000:12.2] K
\item[TAB00000:12.3] 273.15
\item[TAB00000:13.0] 水の定圧比熱
\item[TAB00000:13.1] TERM00050
\item[TAB00000:13.2] J TERM00051
\item[TAB00000:13.3] 4200.
\item[TAB00000:14.0] 海水の氷点
\item[TAB00000:14.1] TERM00052
\item[TAB00000:14.2] K
\item[TAB00000:14.3] 271.35
\item[TAB00000:15.0] 氷の定圧比熱比
\item[TAB00000:15.1] TERM00053
\item[TAB00000:15.2] 
\item[TAB00000:15.3] 2397.
\item[TAB00000:16.0] 水蒸気分子量比
\item[TAB00000:16.1] TERM00054
\item[TAB00000:16.2] 
\item[TAB00000:16.3] 0.622
\item[TAB00000:17.0] 仮温度の係数
\item[TAB00000:17.1] TERM00055
\item[TAB00000:17.2] 
\item[TAB00000:17.3] 0.606
\item[TAB00000:18.0] 比熱と気体定数の比
\item[TAB00000:18.1] TERM00056
\item[TAB00000:18.2] 
\item[TAB00000:18.3] 0.286
\end{description}

