
\section{物理過程}

\subsection{物理過程の概要}

物理過程として, 以下のような過程を考える
\begin{itemize}
\item 積雲対流過程
\item 大規模凝結過程
\item 放射過程
\item 鉛直拡散過程
\item 地表フラックス
\item 地表面・地中過程
\item 重力波抵抗
\end{itemize}
これらの過程による予報変数の時間変化項
TERM00000,TERM00000 を計算し, 時間積分を行なう.
また, 大気・地表フラックスを評価するために
地表面サブモデルを利用する.
地表面サブモデルにおいては,
地中温度 TERM00001, 地中水分 TERM00002, 積雪量 TERM00003 などを
予報変数として用いている.


\subsubsection{基本方程式}

TERM00004 座標系の大気の運動方程式, 熱力学の式,
水蒸気などの物質の連続の式を考える.
運動量, 熱, 水蒸気等の鉛直方向のフラックスを考慮し,
その収束による時間変化を求める.
鉛直フラックスは全て上向きを正とする.

\begin{enumerate}
\item 運動方程式

\begin{verbatim}
EQ=00000.
\end{verbatim}
\begin{verbatim}
EQ=00001.
\end{verbatim}

TERM00005,TERM00005: 東西, 南北風; 
TERM00006,TERM00006: それらの鉛直フラックス.

\item 熱力学の式

\begin{verbatim}
EQ=00002.
\end{verbatim}

TERM00007: 温度; 
TERM00008: 定圧比熱; 
TERM00009: 温位;
TERM00010: 鉛直顕熱フラックス;
TERM00011: 鉛直放射フラックス.

ここで, TERM00012 とおくと, これは,
\begin{verbatim}
EQ=00003.
\end{verbatim}

鉛直1次元過程を考える限りにおいては,
TERM00013 の代わりに TERM00014 を考えればよい.
以下, 簡単のために, 混同のおそれがない限り,
TERM00015 を TERM00016 と書く.

\item 水蒸気の連続の式

\begin{verbatim}
EQ=00004.
\end{verbatim}

TERM00017: 比湿; 
TERM00018: 鉛直水蒸気フラックス.

\subsubsection{地中の基本方程式}

下向きを正とした TERM00019 座標で考える. 
やはり鉛直フラックスは全て上向きを正とする.

\item 熱の式

\begin{verbatim}
EQ=00005.
\end{verbatim}

TERM00020: 地中温度; TERM00021: 定圧比熱; 
TERM00022: 鉛直熱フラックス;
TERM00023; 加熱項(相変化などによる).

\item 地中水分の式

\begin{verbatim}
EQ=00006.
\end{verbatim}

TERM00024: 地中水分; 
TERM00025: 鉛直水フラックス;
TERM00026; 水のソース(流出など).

\item エネルギーの収支式

地表表面で, エネルギーのバランスが成立する.

\begin{verbatim}
EQ=00007.
\end{verbatim}

TERM00027: 蒸発の潜熱;
TERM00028: 地表エネルギーバランス(相変化などにともなう).

\item 地表の水の収支

\begin{verbatim}
EQ=00008.
\end{verbatim}

TERM00029: 降水;
TERM00030: 表面流出.

\item 雪の収支

\begin{verbatim}
EQ=00009.
\end{verbatim}

TERM00031: 積雪量(kg/TERM00032);
TERM00033: 降雪;
TERM00034: 昇華;
TERM00035: 融雪.

\end{enumerate}

\subsubsection{物理過程の時間積分法}

予報変数の時間積分の観点から物理過程を分類すると,
実行順に以下の3つに分けることができる.
\begin{enumerate}
\item 積雲対流および大規模凝結
\item 放射, 鉛直拡散, 接地境界層・地表過程       
\item 重力波抵抗, 質量調節, 乾燥対流調節
\end{enumerate}

積雲対流および大規模凝結は,
\begin{verbatim}
EQ=00010.
\end{verbatim}
\begin{verbatim}
EQ=00011.
\end{verbatim}
のように, 通常の Euler 差分によって値を順次更新する.
大規模凝結スキームには, 
積雲対流スキームによって更新された値が受け渡されることに注意.
実際には, 積雲対流や大規模凝結のルーチンでは加熱率等が出力され,
時間積分はその直後の \texttt{MODULE:[GDINTG]} によって行なわれる.

次のグループの放射, 鉛直拡散, 接地境界層・地表過程
の計算は, 基本的には全てこの更新された値
( TERM00036,TERM00036 等 )
を用いて行なわれる.
ただし, 一部の項を implicit 扱いで計算するために,
これらの項を全て一括して加熱率等を計算して, 
最後に時間積分を行なう.
すなわち, シンボリックに書けば,
\begin{verbatim}
EQ=00012.
\end{verbatim}
となる.

重力波抵抗, 質量調節, 乾燥対流調節に関しては,
積雲対流および大規模凝結と同様である.
\begin{verbatim}
EQ=00013.
\end{verbatim}



\subsubsection{各種の物理量}

予報変数から簡単な計算で求められる
各種の物理量の定義を示す.
このうちいくつかは, 
\texttt{MODULE:[PSETUP]} で計算される.

\begin{enumerate}
\item 仮温度

仮温度 TERM00037 は, 
\begin{verbatim}
EQ=00014.
\end{verbatim}

\item 大気密度

大気密度 TERM00038 は, 以下のように計算される.
\begin{verbatim}
EQ=00015.
\end{verbatim}

\item 高度

高度 TERM00039 は, 力学過程での
ジオポテンシャルの計算と同じ方式によって評価する.
\begin{verbatim}
EQ=00016.
\end{verbatim}
\begin{verbatim}
EQ=00017.
\end{verbatim}
%
\begin{verbatim}
EQ=00018.
\end{verbatim}


\item 層の境界の温度

層の境界の温度は, TERM00040 すなわち TERM00041 に対する
線形補間を行なって計算する.
\begin{verbatim}
EQ=00019.
\end{verbatim}

\item 飽和比湿

飽和比湿 TERM00042,TERM00042
は飽和蒸気圧 TERM00043 を用いて近似的に,
%
\begin{verbatim}
EQ=00020.
\end{verbatim}
%
ここで, TERM00044 であり,
%
\begin{verbatim}
EQ=00021.
\end{verbatim}
%
よって, 蒸発の潜熱 TERM00045, 水蒸気の気体定数 TERM00046 を一定とすれば,
%
\begin{verbatim}
EQ=00022.
\end{verbatim}
%
TERM00047\ [Pa] である.

(\ref{e-sat})より,
%
\begin{verbatim}
EQ=00023.
\end{verbatim}

ここで, 温度が氷点 273.15K よりも低い場合には,
潜熱 TERM00048 として昇華の潜熱 TERM00049 を用いる.

\item 乾燥静的エネルギー, 湿潤静的エネルギー

乾燥静的エネルギー TERM00050 は
\begin{verbatim}
EQ=00024.
\end{verbatim}
%
湿潤静的エネルギー TERM00051 は
\begin{verbatim}
EQ=00025.
\end{verbatim}
で定義される.

\end{enumerate}
