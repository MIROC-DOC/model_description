
\section{物理過程}

\subsection{物理過程の概要}

物理過程として, 以下のような過程を考える
\begin{itemize}
\item 積雲対流過程
\item 大規模凝結過程
\item 放射過程
\item 鉛直拡散過程
\item 地表フラックス
\item 地表面・地中過程
\item 重力波抵抗
\end{itemize}
これらの過程による予報変数の時間変化項
TERM00436,TERM00436 を計算し, 時間積分を行なう.
また, 大気・地表フラックスを評価するために
地表面サブモデルを利用する.
地表面サブモデルにおいては,
地中温度 TERM00437, 地中水分 TERM00438, 積雪量 TERM00439 などを
予報変数として用いている.


\subsubsection{基本方程式}

TERM00440 座標系の大気の運動方程式, 熱力学の式,
水蒸気などの物質の連続の式を考える.
運動量, 熱, 水蒸気等の鉛直方向のフラックスを考慮し,
その収束による時間変化を求める.
鉛直フラックスは全て上向きを正とする.

\begin{enumerate}
\item 運動方程式

\begin{verbatim}
EQ=00157.
\end{verbatim}
\begin{verbatim}
EQ=00158.
\end{verbatim}

TERM00441,TERM00441: 東西, 南北風; 
TERM00442,TERM00442: それらの鉛直フラックス.

\item 熱力学の式

\begin{verbatim}
EQ=00159.
\end{verbatim}

TERM00443: 温度; 
TERM00444: 定圧比熱; 
TERM00445: 温位;
TERM00446: 鉛直顕熱フラックス;
TERM00447: 鉛直放射フラックス.

ここで, TERM00448 とおくと, これは,
\begin{verbatim}
EQ=00160.
\end{verbatim}

鉛直1次元過程を考える限りにおいては,
TERM00449 の代わりに TERM00450 を考えればよい.
以下, 簡単のために, 混同のおそれがない限り,
TERM00451 を TERM00452 と書く.

\item 水蒸気の連続の式

\begin{verbatim}
EQ=00161.
\end{verbatim}

TERM00453: 比湿; 
TERM00454: 鉛直水蒸気フラックス.

\subsubsection{地中の基本方程式}

下向きを正とした TERM00455 座標で考える. 
やはり鉛直フラックスは全て上向きを正とする.

\item 熱の式

\begin{verbatim}
EQ=00162.
\end{verbatim}

TERM00456: 地中温度; TERM00457: 定圧比熱; 
TERM00458: 鉛直熱フラックス;
TERM00459; 加熱項(相変化などによる).

\item 地中水分の式

\begin{verbatim}
EQ=00163.
\end{verbatim}

TERM00460: 地中水分; 
TERM00461: 鉛直水フラックス;
TERM00462; 水のソース(流出など).

\item エネルギーの収支式

地表表面で, エネルギーのバランスが成立する.

\begin{verbatim}
EQ=00164.
\end{verbatim}

TERM00463: 蒸発の潜熱;
TERM00464: 地表エネルギーバランス(相変化などにともなう).

\item 地表の水の収支

\begin{verbatim}
EQ=00165.
\end{verbatim}

TERM00465: 降水;
TERM00466: 表面流出.

\item 雪の収支

\begin{verbatim}
EQ=00166.
\end{verbatim}

TERM00467: 積雪量(kg/TERM00468);
TERM00469: 降雪;
TERM00470: 昇華;
TERM00471: 融雪.

\end{enumerate}

\subsubsection{物理過程の時間積分法}

予報変数の時間積分の観点から物理過程を分類すると,
実行順に以下の3つに分けることができる.
\begin{enumerate}
\item 積雲対流および大規模凝結
\item 放射, 鉛直拡散, 接地境界層・地表過程       
\item 重力波抵抗, 質量調節, 乾燥対流調節
\end{enumerate}

積雲対流および大規模凝結は,
\begin{verbatim}
EQ=00167.
\end{verbatim}
\begin{verbatim}
EQ=00168.
\end{verbatim}
のように, 通常の Euler 差分によって値を順次更新する.
大規模凝結スキームには, 
積雲対流スキームによって更新された値が受け渡されることに注意.
実際には, 積雲対流や大規模凝結のルーチンでは加熱率等が出力され,
時間積分はその直後の \texttt{MODULE:[GDINTG]} によって行なわれる.

次のグループの放射, 鉛直拡散, 接地境界層・地表過程
の計算は, 基本的には全てこの更新された値
( TERM00472,TERM00472 等 )
を用いて行なわれる.
ただし, 一部の項を implicit 扱いで計算するために,
これらの項を全て一括して加熱率等を計算して, 
最後に時間積分を行なう.
すなわち, シンボリックに書けば,
\begin{verbatim}
EQ=00169.
\end{verbatim}
となる.

重力波抵抗, 質量調節, 乾燥対流調節に関しては,
積雲対流および大規模凝結と同様である.
\begin{verbatim}
EQ=00170.
\end{verbatim}



\subsubsection{各種の物理量}

予報変数から簡単な計算で求められる
各種の物理量の定義を示す.
このうちいくつかは, 
\texttt{MODULE:[PSETUP]} で計算される.

\begin{enumerate}
\item 仮温度

仮温度 TERM00473 は, 
\begin{verbatim}
EQ=00171.
\end{verbatim}

\item 大気密度

大気密度 TERM00474 は, 以下のように計算される.
\begin{verbatim}
EQ=00172.
\end{verbatim}

\item 高度

高度 TERM00475 は, 力学過程での
ジオポテンシャルの計算と同じ方式によって評価する.
\begin{verbatim}
EQ=00173.
\end{verbatim}
\begin{verbatim}
EQ=00174.
\end{verbatim}
%
\begin{verbatim}
EQ=00175.
\end{verbatim}


\item 層の境界の温度

層の境界の温度は, TERM00476 すなわち TERM00477 に対する
線形補間を行なって計算する.
\begin{verbatim}
EQ=00176.
\end{verbatim}

\item 飽和比湿

飽和比湿 TERM00478,TERM00478
は飽和蒸気圧 TERM00479 を用いて近似的に,
%
\begin{verbatim}
EQ=00177.
\end{verbatim}
%
ここで, TERM00480 であり,
%
\begin{verbatim}
EQ=00178.
\end{verbatim}
%
よって, 蒸発の潜熱 TERM00481, 水蒸気の気体定数 TERM00482 を一定とすれば,
%
\begin{verbatim}
EQ=00179.
\end{verbatim}
%
TERM00483\ [Pa] である.

(199)より,
%
\begin{verbatim}
EQ=00180.
\end{verbatim}

ここで, 温度が氷点 273.15K よりも低い場合には,
潜熱 TERM00484 として昇華の潜熱 TERM00485 を用いる.

\item 乾燥静的エネルギー, 湿潤静的エネルギー

乾燥静的エネルギー TERM00486 は
\begin{verbatim}
EQ=00181.
\end{verbatim}
%
湿潤静的エネルギー TERM00487 は
\begin{verbatim}
EQ=00182.
\end{verbatim}
で定義される.

\end{enumerate}
