\subsection{放射フラックス}

\subsubsection{放射フラックス計算の概要}

CCSR/NIES AGCM の放射計算スキームは, 
Discrete Ordinate Method および 
k-Distribution Method に基づいて作成されたものである.
気体および雲・エアロゾルによる
太陽放射および地球放射の吸収・射出・散乱過程を考慮し,
放射フラックスの各レベルでの値を計算する.
主な入力データは, 気温 TERM00000, 比湿 TERM00001, 雲水量 TERM00002, 雲量 TERM00003 であり,
出力データは, 上向きおよび下向きの放射フラックス, TERM00004,TERM00004,
および上向き放射フラックスの地表温度に対する微分係数
TERM00005 である.

計算は複数の波長域に分けて行なわれる.
各波長域は k-distribution 法に基づき,
さらに複数のサブチャネルに分かれる.
気体吸収としては, 
TERM00006, TERM00007, TERM00008, TERM00009, TERM00010 のバンド吸収と,
TERM00011, TERM00012, TERM00013 の連続吸収
およびCFCの吸収を取り入れている.
また, 散乱としては, 気体のレーリー散乱と
雲・エアロゾル粒子による散乱を取り入れている.

計算手順の概略は以下の通りである(括弧内はサブルーチン名).
%
\begin{enumerate}
\item 大気温度からプランク関数を計算する \texttt{MODULE:[PLANKS]}.
\item 各サブチャネルにおける,
      気体吸収による光学的厚さを計算する \texttt{MODULE:[PTFIT]}.
\item 連続吸収およびCFCの吸収による
      光学的厚さを計算する \texttt{MODULE:[CNTCFC]}.
\item レーリー散乱および粒子散乱の
      光学的厚さと散乱モーメントを計算する \texttt{MODULE:[SCATMM]}.
\item 散乱の光学的厚さと太陽天頂角から, 
      海面のアルベドを求める \texttt{MODULE:[SSRFC]}.
\item 各サブチャネルごとに,
      プランク関数を光学的厚さで展開する \texttt{MODULE:[PLKEXP]}.
\item 各サブチャネルごとに,
      各層の透過係数, 反射係数, 放射源関数を計算する \texttt{MODULE:[TWST]}
\item adding 法によって, 各層の境界での
      放射フラックスを計算する \texttt{MODULE:[ADDING]}
\end{enumerate}

雲の partial の被覆率を考慮するために,
各層の透過係数, 反射係数, 放射源関数は
雲に覆われた場合と雲がない場合とを別々に計算し,
雲量の重みをかけて平均をとる.
また, 積雲の雲量の考慮も行なっている.
さらに, adding も複数回行ない, 晴天放射フラックスを計算する.

\subsubsection{波長域とサブチャネル}

放射フラックス計算の基本は,
Beer-Lambert の法則
\begin{quote}
EQ=00000.
\end{quote}
に表される. TERM00014 は波長 TERM00015 の放射フラックス密度であり.
TERM00016 は吸収係数である.
加熱率にかかわる放射フラックスを計算するためには,
波長に対する積分操作が必要である.
%
\begin{quote}
EQ=00001.
\label{p-rad:beer}
\end{quote}
%
しかし, 気体分子による放射の吸収・射出は,
分子の吸収線構造により, 非常に複雑な波長依存性を持つため,
この積分を精密に評価することは容易ではない.
その比較的精密な計算を容易に行なうために考案された方法が
k-distribution 法である.
ある波長域の中で, 吸収係数 TERM00017 の,
TERM00018 に関する密度関数 TERM00019 を考え,
(\ref{p-rad:beer}) を
\begin{quote}
EQ=00002.
\end{quote}
で近似する. ここで, TERM00020 は
TERM00021 における, この波長域で吸収係数 TERM00022 をもつ
波長で平均したフラックスである.
この式は, TERM00023,TERM00023 が TERM00024 の
比較的滑らかな関数であれば, 
\begin{quote}
EQ=00003.
\label{p-rad:beer-kd}
\end{quote}
のように, 指数関数項の有限個(サブチャネル)の足しあわせで
比較的精密に計算可能である.
この方法はさらに,
吸収と散乱を同時に考慮することが容易であるという利点を持つ.

CCSR/NIES AGCM では,
放射パラメータデータを変えることにより
いろいろな波長分割数で計算を行なうことができる.
現在標準で用いられるものでは,
波長域は18に分割されている.
さらに各波長域は1から6個のサブチャネル(上式の TERM00025 に対応)に分割され,
全体で37チャネルとなる.
波長域は, 波数(TERM00026)で
50, 250, 400, 550, 770, 990, 1100, 1400, 2000,
2500, 4000, 14500, 31500, 33000, 34500, 36000, 43000, 46000, 50000
で分割されている.

\subsubsection{プランク関数の計算 \texttt{MODULE:[PLANKS]}}

各波長域で積分したプランク関数 TERM00027 は,
以下の式で評価する.

\begin{quote}
EQ=00004.
\end{quote}

TERM00028 は波長域の平均波長,
TERM00029 は function fitting によって定められたパラメータである.
これは, 各層の大気温度 TERM00030, 各層の境界の大気温度 TERM00031
と地表面温度 TERM00032 に対し計算する.

以下, 波長域に関する添字 TERM00033 は基本的に省略する.

\subsubsection{気体吸収による光学的厚さの計算 \texttt{MODULE:[PTFIT]}}

気体吸収による光学的厚さは, 添字 TERM00034 を分子の種類として,
以下のようになる. 

\begin{quote}
EQ=00005.
\end{quote}

ここで, TERM00035 は分子 TERM00036 の吸収係数であり, サブチャネルごとに異なる.

\begin{quote}
EQ=00006.
\end{quote}

という形で, 温度 TERM00037(K), 気圧 TERM00038(hPa) の関数として与えられる.
TERM00039 は, mol TERM00040 で表した層の中の気体の量であり,
体積混合比 TERM00041(単位ppmv)から,
\begin{quote}
EQ=00007.
\end{quote}
と計算できる. 
ただし, TERM00042 はモルあたりの気体定数(8.31 J TERM00043 TERM00044)であり,
気層の厚さ TERM00045 の単位は km である.
また, ppmv での体積混合比 TERM00046 は, 
質量混合比 TERM00047 から, 
\begin{quote}
EQ=00008.
\end{quote}
によって換算できる.
TERM00048,TERM00048 は
それぞれ対象分子と大気の質量あたりの気体定数,
TERM00049,TERM00049 は
それぞれ対象分子と大気の平均分子量である.

この計算は, サブチャネルごと, 各層ごとに行なう.

\subsubsection{連続吸収およびCFCの吸収による光学的厚さ \texttt{MODULE:[CNTCFC]}}

TERM00050 の連続吸収による光学的厚さ TERM00051 は,
ダイマーによるものを考え, 
基本的に水蒸気の体積混合比の二乗に比例した形で評価する.
\begin{quote}
EQ=00009.
\end{quote}

TERM00052 の項にかかる TERM00053 は, 
ダイマーの吸収の温度依存性を表す.
さらに, 通常の気体吸収を無視する波長帯においては,
水蒸気の体積混合比の一乗に比例する寄与を取り入れる.

TERM00054 の連続吸収は, 混合比一定と仮定して,
\begin{quote}
EQ=00010.
\end{quote}
としている.

TERM00055 の連続吸収は, 混合比 TERM00056 を用い, 温度依存性を取り入れて,
\begin{quote}
EQ=00011.
\end{quote}

CFCの吸収は, TERM00057 種類の CFC を考えて,
\begin{quote}
EQ=00012.
\end{quote}

これらの光学的厚さの総和を TERM00058 とする.
\begin{quote}
EQ=00013.
\end{quote}

この計算は各波長域ごと, 各層ごとに行なう.

\subsubsection{散乱の光学的厚さと散乱モーメント \texttt{MODULE:[SCATMM]}}

レーリー散乱および粒子消散の(散乱と吸収を含めた)光学的厚さは
\begin{quote}
EQ=00014.
\end{quote}
ここで, TERM00059 はレーリー散乱の消散係数,
TERM00060 は粒子 TERM00061 の消散係数,
TERM00062 は標準状態に換算した
粒子 TERM00063 の体積混合比である.

ここで, 雲水の質量混合比 TERM00064 から
雲粒の標準状態換体積混合比(ppmv)への換算は以下のようになる.
\begin{quote}
EQ=00015.
\end{quote}
ただし, TERM00065 は雲粒の密度である.

一方, 光学的厚さのうち散乱に起因する部分 TERM00066 は,
\begin{quote}
EQ=00016.
\end{quote}
ここで, TERM00067 はレーリー散乱の散乱係数,
TERM00068 は粒子 TERM00069 の散乱係数である.

また, 規格化された散乱のモーメント 
TERM00070 (非対称因子) および TERM00071 (前方散乱因子)は,
\begin{quote}
EQ=00017.
\end{quote}
\begin{quote}
EQ=00018.
\end{quote}
ここで, TERM00072,TERM00072 はレーリー散乱の散乱モーメント,
TERM00073,TERM00073 は粒子 TERM00074 の散乱モーメントである.

この計算は各波長域ごと, 各層ごとに行なう.

\subsubsection{海面のアルベド \texttt{MODULE:[SSRFC]}}

海面のアルベド TERM00075 は散乱の光学的厚さを鉛直に足し合わせたもの
TERM00076 および太陽入射角ファクタ TERM00077 を用いて,
\begin{quote}
EQ=00019.
\end{quote}
のように表される.
ただし,
\begin{quote}
EQ=00020.
\end{quote}
である.

この計算は各波長域ごとに行なう.

\subsubsection{光学的厚さの総計}

気体バンド吸収, 連続吸収, レーリー散乱, 粒子散乱・吸収を
全て考慮した光学的厚さは, 
%
\begin{quote}
EQ=00021.
\end{quote}
%
となる. ここで, TERM00078 はサブチャネルごとに異なるため,
サブチャネルごと, 各層ごとに計算を行なう.

\subsubsection{プランク関数の展開 \texttt{MODULE:[PLKEXP]}}

各層の中で, プランク関数 TERM00079 を
\begin{quote}
EQ=00022.
\end{quote}
のように展開して表現し, 展開係数 TERM00080,TERM00080 を求める.
ここで, TERM00081 として
各層の上端(上の層との境界)での TERM00082 を,
TERM00083 として, 各層の下端(下の層との境界)での TERM00084 を,
TERM00085 として, 各層の代表レベルでの TERM00086 を用いる.
\begin{quote}
\nonumber
EQ=00045.\\
EQ=00045.\\
\nonumber
EQ=00045.
\end{quote}

この計算は, サブチャネルごと, 各層ごとに行なう.

\subsubsection{各層の透過・反射係数, 放射源関数 \texttt{MODULE:[TWST]}}

これまで求められた, 光学的厚さ TERM00087, 散乱の光学的厚さ TERM00088,
散乱モーメント TERM00089,TERM00089, プランク関数の展開係数 TERM00090,TERM00090,
太陽入射角ファクタ TERM00091 を用いて,
均一な層を仮定し, 2ストリーム近似で
透過係数 TERM00092, 反射係数 TERM00093, 下方向への放射源関数 TERM00094,
上方向への放射源関数 TERM00095 を求める.

単一散乱アルベド TERM00096 は,
\begin{quote}
EQ=00023.
\end{quote}
前方散乱因子 TERM00097 による寄与を
補正した光学的厚さ TERM00098,
単一散乱アルベド TERM00099, 非対称因子 TERM00100 は,
\begin{quote}
EQ=00046.\\
EQ=00046.\\
EQ=00046.
\end{quote}

これから, 規格化された散乱の位相関数として,
\begin{quote}
EQ=00047.\\
EQ=00047.
\end{quote}
ただし, TERM00101 は2ストリームの方向余弦であり,
\begin{quote}
EQ=00024.
\end{quote}
\begin{quote}
EQ=00025.
\end{quote}

さらに,
\begin{quote}
EQ=00048.\\
EQ=00048.\\
EQ=00048.\\
EQ=00048.
\end{quote}
を用いると, 反射率 TERM00102 および透過率 TERM00103 は以下のようになる.
\begin{quote}
EQ=00049.\\
EQ=00049.
\end{quote}
\begin{quote}
EQ=00050.\\
EQ=00050.
\end{quote}

次にまず, プランク関数起源の放射源関数を求める.
\begin{quote}
EQ=00026.
\end{quote}
から, 放射源関数の展開係数が求められ,
\begin{quote}
EQ=00051.\\
EQ=00051.\\
EQ=00051.\\
EQ=00051.
\end{quote}
\begin{quote}
EQ=00052.\\
EQ=00052.
\end{quote}
によりプランク関数起源の放射源関数 TERM00104 は,
\begin{quote}
EQ=00053.\\
EQ=00053.
\end{quote}

一方,  太陽入射起源の放射源関数は,
\begin{quote}
EQ=00027.
\end{quote}
より,
\begin{quote}
EQ=00028.
\end{quote}
を用いることにより, 以下の様になる.
\begin{quote}
EQ=00054.\\
EQ=00054.
\end{quote}

この計算は, サブチャネルごと, 各層ごとに行なう.

\subsubsection{各層の放射源関数の組み合わせ}

プランク関数起源と太陽入射起源の
両者を合わせた放射源関数は
\begin{quote}
EQ=00029.
\end{quote}
となる. ただし, TERM00105 は大気上端から
いま考慮している層の上端までの
TERM00106 を合計した光学的厚さであり, 
TERM00107 はいま考慮している波長域における入射フラックスである.
すなわち, TERM00108 は
いま考慮している層の上端での入射フラックスである.
%
この計算は実際には, 
\begin{quote}
EQ=00030.
\end{quote}
のように行なう. TERM00109 は大気最上層から
今考えている層の1つ上の層までの積を表す.

この計算は, サブチャネルごと, 各層ごとに行なう.

\subsubsection{各層境界での放射フラックス \texttt{MODULE:[ADDING]}}

各層の透過係数 TERM00110, 反射係数 TERM00111, 放射源関数 TERM00112
が全ての層 TERM00113 で求められると,
adding 法を用いて各層境界での放射フラックスを求めることができる.
これは, 2つの層の TERM00114,TERM00114 がわかっていると,
2つの層を合成した層全体の TERM00115,TERM00115 が簡単な計算により
求められることを利用したものである.
均質な層では, 上から入射した場合の反射率, 透過率と
下から入射した場合の反射率, 透過率とは同じであるが,
複数の層を合成した不均質な層では異なるため,
上から入射した場合の反射率, 透過率 TERM00116,TERM00116 と
下から入射した場合の反射率, 透過率 TERM00117,TERM00117 とを区別する.
今, 上の層1 と 下の層2 でこれら
TERM00118,TERM00118 が既知であると,
合成した層での値
TERM00119,TERM00119 は
以下のようになる.
\begin{quote}
EQ=00055.\\
EQ=00055.\\
EQ=00055.\\
EQ=00055.\\
EQ=00055.\\
EQ=00055.
\end{quote}

上から1, 2, \ldots TERM00120 層まであるとする. 
ただし, 地表を一つの層と考え, 第 TERM00121 層とする.
第 TERM00122 層から TERM00123 層までを1つの層と考えたときの反射率, 放射源関数
TERM00124,TERM00124 を考えると,
\begin{quote}
EQ=00056.\\
EQ=00056.
\end{quote}
これは, 地表での値
\begin{quote}
EQ=00057.\\
EQ=00057.
\end{quote}
から出発して, 順次 TERM00125,TERM00125 で解くことができる.
ただし,
\begin{quote}
EQ=00031.
\end{quote}


次に, 第1層から第 TERM00126 層までを1つの層と考えたときの反射率, 放射源関数
TERM00127,TERM00127 を考えると,
\begin{quote}
EQ=00058.\\
EQ=00058.
\end{quote}
となり, これも TERM00128,TERM00128
から出発して順次 TERM00129,TERM00129 で解くことができる.

これらを用いると,
層 TERM00130 と TERM00131 の境界における下向きのフラックス TERM00132
および上向きのフラックス TERM00133 は,
TERM00134 層を組み合わせた層と
TERM00135 層を組み合わせた層の2つの層の間の問題に還元され,
\begin{quote}
EQ=00059.\\
EQ=00059.
\end{quote}
と書き表すことができる.
ただし, 大気上端でのフラックスは,
\begin{quote}
EQ=00060.\\
EQ=00060.
\end{quote}

最後にこのフラックスはスケールされたものであるので,
再スケーリングを行ない, さらに直達太陽入射を加えて
放射フラックスを求める.

\begin{quote}
EQ=00061.\\
EQ=00061.\\
EQ=00061.\\
EQ=00061.
\end{quote}

この計算は, サブチャネルごとに行なう.

\subsubsection{フラックスの足し込み}

各層のサブチャネルごとの放射フラックス TERM00136 が求められると,
それをサブチャネルの代表する波長幅に対応する
重み TERM00137 をかけて足し合わせることにより,
波長積分のフラックスが求められる.

\begin{quote}
EQ=00032.
\end{quote}

実際には, 短波長域(太陽光領域), 
長波長域(地球放射領域)に分けて足し合わせる.
また, 短波長域の一部(波長 TERM00138 より短い領域)の
地表での下向きフラックスを PAR (光合成活性放射)として得る.

\subsubsection{フラックスの温度微分}

地表面温度を implicit で解くために,
上向きフラックスの地表面温度に対する微分項
TERM00139 を計算する.
そのために, TERM00140 より1K高い温度に対する値 
TERM00141 も求め, それを用いて
adding 法によるフラックスの計算をやりなおし,
元の値との差を TERM00142 とする.
これは長波長域(地球放射領域)のみ意味のある値となる.

\subsubsection{雲量の取扱い}

CCSR/NIES AGCM では,
1つの格子の中での雲の水平方向の被覆率を考慮している.
雲は以下の2種類である.
\begin{enumerate}
\item 層雲. 大規模凝結スキーム \texttt{MODULE:[LSCOND]} で診断される.
      各層(TERM00143)ごとに格子平均の雲水量 TERM00144 と
      水平被覆率(雲量) TERM00145 が定義される.      
\item 積雲. 積雲対流スキーム \texttt{MODULE:[CUMLUS]} で診断される.
      各層(TERM00146)ごとに格子平均の雲水量 TERM00147 が定義されるが,
      水平被覆率(雲量) TERM00148 は鉛直に一定とする.
\end{enumerate}
これらの取扱いにおいて, 層雲は鉛直にランダムに重なり合うと仮定し,
積雲は上下層で常に同じ領域を占めると仮定する
(その領域の中に限れば雲量は0もしくは1であるとする).
そのために, 以下のように計算を行なう.

\begin{enumerate}
\item レーリーおよび粒子散乱・吸収の光学的厚さ等
      TERM00149,TERM00149 を,
      \begin{enumerate}
      \item 雲水量 TERM00150 の雲が存在する場合(層雲)
      \item 雲の全くない場合
      \item 雲水量 TERM00151 の雲が存在する場合(積雲)
      \end{enumerate}
      について計算する.

\item 各層の反射係数, 透過係数, 
      放射源関数(プランク関数起源, 日射起源)を
      上の3つの場合についてそれぞれ計算する.
      雲なしの場合の値を
      TERM00152, 層雲のある場合を TERM00153, 積雲のある場合を
      TERM00154 などとする.

\item 各層の反射係数, 透過係数, 
      放射源関数を, 層雲の雲量の重み TERM00155 をつけて平均する.
      平均したものを TERM00156 をつけて表すと,
      \begin{quote}
EQ=00062.\\
EQ=00062.\\
EQ=00062.\\
EQ=00062.
\end{quote}
      とする. ただし,
      \begin{quote}
EQ=00033.
\end{quote}
      である. 
      また,
      \begin{quote}
EQ=00063.\\
EQ=00063.
\end{quote}
      も求める.

\item 平均の特性値(TERM00157 など)を用いた場合,
      雲なしの特性値(TERM00158 など)を用いた場合,
      積雲の特性値(TERM00159 など)を用いた場合について,
      それぞれadding によってフラックス
      TERM00160,TERM00160 を求める.
      
\item 最終的に求めるフラックスは
      \begin{quote}
EQ=00034.
\end{quote}
      (TERM00161 は cloud radiative forcing の見積りのために
       計算している)

\end{enumerate}

\subsubsection{入射フラックスと入射角 \texttt{MODULE:[SHTINS]}}

入射フラックス TERM00162 は,
太陽定数を TERM00163, 
太陽地球間の距離の, 
その時間平均値との比を TERM00164 とすると.
%
\begin{quote}
EQ=00035.
\end{quote}
ここで, TERM00165 は以下のように求める.
%
\begin{quote}
EQ=00036.
\end{quote}
として,
\begin{quote}
EQ=00037.
\end{quote}
ただし, TERM00166 は年初から日単位で表した時刻である.

また,入射角は以下のように求める.
太陽の角度位置 TERM00167 を
\begin{quote}
EQ=00038.
\end{quote}
として,  太陽の赤緯 TERM00168 は,
\begin{quote}
EQ=00039.
\end{quote}
%
すると, 入射角ファクタ TERM00169 (TERM00170 は天頂角)は,
\begin{quote}
EQ=00040.
\end{quote}
TERM00171 は緯度,
TERM00172 は時角(地方時から TERM00173 を引いたもの)である.

以上において, 地球軌道の離心率を TERM00174 とすると(Katayama, 1974),
\begin{quote}
EQ=00064.\\
EQ=00064.\\
EQ=00064.\\
EQ=00064.\\
EQ=00064.\\
EQ=00064.\\
EQ=00064.\\
EQ=00064.
\end{quote}

年平均日射を与えることも可能である.
この場合, 年平均入射量および年平均入射角は, 
近似的に次のようになる.
%
\begin{quote}
EQ=00041.
\end{quote}
%
\begin{quote}
EQ=00042.
\end{quote}

\subsubsection{その他の留意点}

\begin{enumerate}
\item 放射の計算は通常, 毎ステップ行なうわけではない.
      そのために, 放射フラックスをセーブしておき, 
      放射計算をしない時刻にはそれを用いる.
      その際, 短波放射に関しては,
      次回の計算時刻との間の可照時間(TERM00175 である時間)の割合 TERM00176 と
      可照時間内で平均した太陽入射角ファクタ TERM00177 を用いて
      フラックス TERM00178 を求め,
      \begin{quote}
EQ=00043.
\end{quote}
      とする.


\item 雲水は, 温度に依存して, 
      水雲粒および氷雲粒として扱われる.
      氷雲として扱われる割合 TERM00179 は,
      \begin{quote}
EQ=00044.
\end{quote}
      (ただし, 最大値1, 最小値0) である. また,
      TERM00180,TERM00180 とする.

\end{enumerate}




