
\subsection{地表フラックス}

\subsubsection{地表フラックススキームの概要}

地表フラックススキームは, 
接地境界層における乱流輸送による
大気地表間の物理量のフラックスを評価する.
主な入力データは, 風速 TERM00000,TERM00000, 気温 TERM00001, 比湿 TERM00002 であり,
出力データは, 運動量, 熱, 水蒸気の鉛直フラックスと
implicit 解を得るための微分値である.

バルク係数は Louis(1979), Louis {\em et al.}(1982) に従って求める. 
ただし, 運動量と熱に対する粗度の違いを考慮した補正を行なっている. 

計算手順の概略は以下の通りである.
\begin{enumerate}
\item 大気の安定度として
      Richardson 数を計算する.
\item Richardson 数からバルク係数を計算する \texttt{MODULE:[PSFCL]}.
\item バルク係数からフラックスとその微分を計算する.
\item 必要であれば, 求められたフラックスを用いて
      海面の粗度効果・自由対流の効果・風速補正を考慮した後に,
      もう一度計算を行なう.
\end{enumerate}

\subsubsection{フラックス計算の基本式}

地表フラックス TERM00003,TERM00003 は
バルク係数 TERM00004,TERM00004 を用いて
次のように表される.
%
\begin{verbatim}
EQ=00000.
\end{verbatim}
\begin{verbatim}
EQ=00001.
\end{verbatim}
\begin{verbatim}
EQ=00002.
\end{verbatim}
\begin{verbatim}
EQ=00003.
\end{verbatim}
%
ただし, TERM00009 は可能蒸発量である.
実蒸発量の計算は「地表過程」ならび
「大気地表系の拡散型収支式の解法」の節で述べる.

\subsubsection{Richardson 数}

大気地表間の安定度の基準となる,
バルクRichardson数 TERM00010 は
%
\begin{verbatim}
EQ=00004.
\end{verbatim}
ここで, 
\begin{verbatim}
EQ=00005.
\end{verbatim}
は補正ファクターで, 補正前のバルク Richardson数から近似的に求めるが, 
ここでは計算方法は略す. 

\subsubsection{バルク係数}

バルク係数 TERM00011,TERM00011 は
Louis(1979), Louis {\em et al.}(1982) に従って求める. 
ただし, 運動量と熱に対する粗度の違いを考慮した補正を行なっている. 
すなわち, 運動量, 熱, 水蒸気に対する粗度を
それぞれ TERM00012,TERM00012 とすると
一般に TERM00013,TERM00013 であるが, 熱, 水蒸気についても
TERM00014 の高さからのフラックスに対するバルク係数
TERM00015, TERM00016 をまず求め, その後に補正する. 
%
\begin{verbatim}
EQ=00006.
\end{verbatim}
%
\begin{verbatim}
EQ=00007.
\end{verbatim}
\begin{verbatim}
EQ=00008.
\end{verbatim}
%
\begin{verbatim}
EQ=00009.
\end{verbatim}
\begin{verbatim}
EQ=00010.
\end{verbatim}

TERM00017,TERM00017 は
中立時の(TERM00018 からのフラックスに対する)バルク係数で,
%
\begin{verbatim}
EQ=00011.
\end{verbatim}

補正ファクター TERM00019 は, 
\begin{verbatim}
EQ=00012.
\end{verbatim}
であるが, 計算方法は略す. 
係数は, TERM00020,TERM00020 である. 

バルク係数の TERM00021 依存性を図示すると,
図\ref{p-sflx:cm}, 図\ref{p-sflx:ch}のようになる.

\begin{figure}[htbp]
  \begin{center}
    \epsfile{file=sflx-cm.ps,width=70mm}
    \caption{運動量に対する粗度}
    \label{p-sflx:cm}
  \end{center}
\end{figure}
\begin{figure}[htbp]
  \begin{center}
    \epsfile{file=sflx-ch.ps,width=70mm}
    \caption{熱に対する粗度. TERM00022 の場合}
    \label{p-sflx:ch}
  \end{center}
\end{figure}

\subsubsection{フラックスの計算}

これにより, フラックスが計算される.
%
\begin{verbatim}
EQ=00013.
\end{verbatim}
\begin{verbatim}
EQ=00014.
\end{verbatim}
\begin{verbatim}
EQ=00015.
\end{verbatim}
\begin{verbatim}
EQ=00016.
\end{verbatim}

微分項は, 以下のようになる.
\begin{verbatim}
EQ=00017.
\end{verbatim}
\begin{verbatim}
EQ=00018.
\end{verbatim}
\begin{verbatim}
EQ=00019.
\end{verbatim}
\begin{verbatim}
EQ=00020.
\end{verbatim}
\begin{verbatim}
EQ=00021.
\end{verbatim}

ここで, 注意したいのは,
TERM00032 はこの時点では求められていない量であることである.
表皮温度は, 
地表熱バランスの条件
\begin{verbatim}
EQ=00022.
\end{verbatim}
を満たすように決まる.
この時点では, TERM00033 としては前の時間ステップにおけるものを使って評価する.
地表バランスを満たす本当のフラックスの値は,
地表過程と結合してこの式を解いてから定まる.
その意味で, 上のフラックスに TERM00034 をつけておいた.

\subsubsection{海面における取扱い}

海面では, Miller et al.(1992) に従い, 以下の2つの効果を考慮している.
\begin{itemize}
\item 風速が弱いときに自由対流運動が卓越すること
\item 海面の粗度が風速によって変化すること
\end{itemize}

自由対流運動の効果は, 浮力フラックス TERM00035 を計算し,
\begin{verbatim}
EQ=00023.
\end{verbatim}
TERM00036 のときに,
\begin{verbatim}
EQ=00024.
\end{verbatim}
\begin{verbatim}
EQ=00025.
\end{verbatim}
とすることで考慮する.  TERM00038 は混合層の厚さのスケールに対応する.
現在の標準値は TERM00039 m である.
% この自由対流運動の効果は, 海面以外でも考慮している.

海面の粗度変化は, 摩擦速度 TERM00040
\begin{verbatim}
EQ=00026.
\end{verbatim}
を用いて,
\begin{verbatim}
EQ=00027.
EQ=00027.
EQ=00027.
\end{verbatim}
のように評価する. TERM00041 TERM00042 TERM00043 は
大気の動粘性係数であり, 
他の係数の標準値は
TERM00044,TERM00044,
TERM00045,TERM00045,
TERM00046,TERM00046 である.

以上の計算では, TERM00047,TERM00047 が必要であるため,
逐次近似計算を行なう.

\subsubsection{風速の補正}

一般に粗度の大きな地表では, 粗度の小さな地表に比べて
運動量の下向き輸送が効率的であるためにその直上の風が弱く,
粗度による TERM00048 の違いを風速の違いによって打ち消す効果が働く.

モデルにおいて地表フラックス計算に渡される風速は
力学過程の時間積分によって計算された値であり,
スペクトル展開によって平滑化された値となっている.
そのために, 海面と陸面など, 粗度の大きく違う地表が
小さなスケールで混在している領域では, 
この補償効果がうまく表現できない.
そのため, 一度運動量フラックスを計算し,
大気最下層の風速をそれによって補正してから
もういちど運動量・熱・水のフラックスを計算しなおす.

\subsubsection{風速の最小値}

小規模運動の効果を考え,
地表フラックスの算出の際の地表風速
TERM00049 の最小値を設定する.
現在の標準値は, 各フラックスに共通で
3m/s である.

