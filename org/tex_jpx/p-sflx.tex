
\subsection{地表フラックス}

\subsubsection{地表フラックススキームの概要}

地表フラックススキームは, 
接地境界層における乱流輸送による
大気地表間の物理量のフラックスを評価する.
主な入力データは, 風速 TERM00850,TERM00850, 気温 TERM00851, 比湿 TERM00852 であり,
出力データは, 運動量, 熱, 水蒸気の鉛直フラックスと
implicit 解を得るための微分値である.

バルク係数は Louis(1979), Louis {\em et al.}(1982) に従って求める. 
ただし, 運動量と熱に対する粗度の違いを考慮した補正を行なっている. 

計算手順の概略は以下の通りである.
\begin{enumerate}
\item 大気の安定度として
      Richardson 数を計算する.
\item Richardson 数からバルク係数を計算する \texttt{MODULE:[PSFCL]}.
\item バルク係数からフラックスとその微分を計算する.
\item 必要であれば, 求められたフラックスを用いて
      海面の粗度効果・自由対流の効果・風速補正を考慮した後に,
      もう一度計算を行なう.
\end{enumerate}

\subsubsection{フラックス計算の基本式}

地表フラックス TERM00853,TERM00853 は
バルク係数 TERM00854,TERM00854 を用いて
次のように表される.
%
\begin{verbatim}
EQ=00328.
\end{verbatim}
\begin{verbatim}
EQ=00329.
\end{verbatim}
\begin{verbatim}
EQ=00330.
\end{verbatim}
\begin{verbatim}
EQ=00331.
\end{verbatim}
%
ただし, TERM00859 は可能蒸発量である.
実蒸発量の計算は「地表過程」ならび
「大気地表系の拡散型収支式の解法」の節で述べる.

\subsubsection{Richardson 数}

大気地表間の安定度の基準となる,
バルクRichardson数 TERM00860 は
%
\begin{verbatim}
EQ=00332.
\end{verbatim}
ここで, 
\begin{verbatim}
EQ=00333.
\end{verbatim}
は補正ファクターで, 補正前のバルク Richardson数から近似的に求めるが, 
ここでは計算方法は略す. 

\subsubsection{バルク係数}

バルク係数 TERM00861,TERM00861 は
Louis(1979), Louis {\em et al.}(1982) に従って求める. 
ただし, 運動量と熱に対する粗度の違いを考慮した補正を行なっている. 
すなわち, 運動量, 熱, 水蒸気に対する粗度を
それぞれ TERM00862,TERM00862 とすると
一般に TERM00863,TERM00863 であるが, 熱, 水蒸気についても
TERM00864 の高さからのフラックスに対するバルク係数
TERM00865, TERM00866 をまず求め, その後に補正する. 
%
\begin{verbatim}
EQ=00334.
\end{verbatim}
%
\begin{verbatim}
EQ=00335.
\end{verbatim}
\begin{verbatim}
EQ=00336.
\end{verbatim}
%
\begin{verbatim}
EQ=00337.
\end{verbatim}
\begin{verbatim}
EQ=00338.
\end{verbatim}

TERM00867,TERM00867 は
中立時の(TERM00868 からのフラックスに対する)バルク係数で,
%
\begin{verbatim}
EQ=00339.
\end{verbatim}

補正ファクター TERM00869 は, 
\begin{verbatim}
EQ=00340.
\end{verbatim}
であるが, 計算方法は略す. 
係数は, TERM00870,TERM00870 である. 

バルク係数の TERM00871 依存性を図示すると,
図\ref{p-sflx:cm}, 図\ref{p-sflx:ch}のようになる.

\begin{figure}[htbp]
  \begin{center}
    \epsfile{file=sflx-cm.ps,width=70mm}
    \caption{運動量に対する粗度}
    \label{p-sflx:cm}
  \end{center}
\end{figure}
\begin{figure}[htbp]
  \begin{center}
    \epsfile{file=sflx-ch.ps,width=70mm}
    \caption{熱に対する粗度. TERM00872 の場合}
    \label{p-sflx:ch}
  \end{center}
\end{figure}

\subsubsection{フラックスの計算}

これにより, フラックスが計算される.
%
\begin{verbatim}
EQ=00341.
\end{verbatim}
\begin{verbatim}
EQ=00342.
\end{verbatim}
\begin{verbatim}
EQ=00343.
\end{verbatim}
\begin{verbatim}
EQ=00344.
\end{verbatim}

微分項は, 以下のようになる.
\begin{verbatim}
EQ=00345.
\end{verbatim}
\begin{verbatim}
EQ=00346.
\end{verbatim}
\begin{verbatim}
EQ=00347.
\end{verbatim}
\begin{verbatim}
EQ=00348.
\end{verbatim}
\begin{verbatim}
EQ=00349.
\end{verbatim}

ここで, 注意したいのは,
TERM00882 はこの時点では求められていない量であることである.
表皮温度は, 
地表熱バランスの条件
\begin{verbatim}
EQ=00350.
\end{verbatim}
を満たすように決まる.
この時点では, TERM00883 としては前の時間ステップにおけるものを使って評価する.
地表バランスを満たす本当のフラックスの値は,
地表過程と結合してこの式を解いてから定まる.
その意味で, 上のフラックスに TERM00884 をつけておいた.

\subsubsection{海面における取扱い}

海面では, Miller et al.(1992) に従い, 以下の2つの効果を考慮している.
\begin{itemize}
\item 風速が弱いときに自由対流運動が卓越すること
\item 海面の粗度が風速によって変化すること
\end{itemize}

自由対流運動の効果は, 浮力フラックス TERM00885 を計算し,
\begin{verbatim}
EQ=00351.
\end{verbatim}
TERM00886 のときに,
\begin{verbatim}
EQ=00352.
\end{verbatim}
\begin{verbatim}
EQ=00353.
\end{verbatim}
とすることで考慮する.  TERM00888 は混合層の厚さのスケールに対応する.
現在の標準値は TERM00889 m である.
% この自由対流運動の効果は, 海面以外でも考慮している.

海面の粗度変化は, 摩擦速度 TERM00890
\begin{verbatim}
EQ=00354.
\end{verbatim}
を用いて,
\begin{verbatim}
EQ=00355.
EQ=00355.
EQ=00355.
\end{verbatim}
のように評価する. TERM00891 TERM00892 TERM00893 は
大気の動粘性係数であり, 
他の係数の標準値は
TERM00894,TERM00894,
TERM00895,TERM00895,
TERM00896,TERM00896 である.

以上の計算では, TERM00897,TERM00897 が必要であるため,
逐次近似計算を行なう.

\subsubsection{風速の補正}

一般に粗度の大きな地表では, 粗度の小さな地表に比べて
運動量の下向き輸送が効率的であるためにその直上の風が弱く,
粗度による TERM00898 の違いを風速の違いによって打ち消す効果が働く.

モデルにおいて地表フラックス計算に渡される風速は
力学過程の時間積分によって計算された値であり,
スペクトル展開によって平滑化された値となっている.
そのために, 海面と陸面など, 粗度の大きく違う地表が
小さなスケールで混在している領域では, 
この補償効果がうまく表現できない.
そのため, 一度運動量フラックスを計算し,
大気最下層の風速をそれによって補正してから
もういちど運動量・熱・水のフラックスを計算しなおす.

\subsubsection{風速の最小値}

小規模運動の効果を考え,
地表フラックスの算出の際の地表風速
TERM00899 の最小値を設定する.
現在の標準値は, 各フラックスに共通で
3m/s である.

