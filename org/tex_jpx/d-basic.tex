
\section{力学過程}

\subsection{基礎方程式}

\subsubsection{基礎方程式}

基礎方程式は,
球面(TERM00000,TERM00000), TERM00001 座標におけるプリミティブ方程式系であり,
以下のように与えられる( Haltiner and Williams , 1980 ).


\begin{enumerate}
\item 連続の式

\begin{quote}
EQ=00000.
\label{質量}
\end{quote}

\item 静水圧の式

\begin{quote}
EQ=00001.
\label{静水圧}
\end{quote}


\item 運動方程式

\begin{quote}
EQ=00002.
\label{渦度}
\end{quote}
\begin{quote}
EQ=00003.
\label{発散}
\end{quote}


\item 熱力学の式

\begin{quote}
\label{熱力}
\nonumber
EQ=00008.\\
EQ=00008.
\end{quote}


\item 水蒸気の式

\begin{quote}
\label{水蒸気}
\nonumber
EQ=00009.\\
EQ=00009.
\end{quote}

\end{enumerate}

ここで,
%
\begin{quote}
EQ=00010.\\
EQ=00010.\\
EQ=00010.\\
EQ=00010.\\
EQ=00010.\\
EQ=00010.\\
EQ=00010.\\
\label{渦度定義}
EQ=00010.\\
\label{発散定義}
EQ=00010.\\
\label{B項}
EQ=00010.\\
\label{A項}
EQ=00010.\\
\label{E項}
EQ=00010.\\
\nonumber
EQ=00010.\\
EQ=00010.
\end{quote}

TERM00002,TERM00002
は水平拡散項,
TERM00003,TERM00003
は小規模運動過程(`物理過程'として扱う)による力,
TERM00004 は放射, 凝結, 小規模運動過程等の`物理過程'による
加熱・温度変化,
TERM00005 は凝結, 小規模運動過程等の`物理過程'による
水蒸気ソース項である.
また, TERM00006 は摩擦熱であり,
%
\begin{quote}
EQ=00004.
\end{quote}
%
TERM00007 は,
水平および鉛直の拡散による TERM00008,TERM00008 の時間変化項である.

\subsubsection{境界条件}

鉛直流に関する境界条件は
%
\begin{quote}
EQ=00005.
\end{quote}
%
である. よって(\ref{質量}) から,
地表気圧の時間変化式と
TERM00009 系での鉛直速度 TERM00010 を求める診断式
%
\begin{quote}
EQ=00006.
\label{気圧傾向}
\end{quote}
%
\begin{quote}
EQ=00007.
\label{鉛直速度}
\end{quote}
%
が導かれる.



