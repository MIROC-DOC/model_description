
\section*{参考文献}

\begin{description}
\item Arakawa, A. and W.H. Schubert, 1974:
      Interactions of cumulus cloud ensemble with the large-scale
      environment. Part I. {\em J. Atmos. Sci.,\/} {\bf 31,} 671--701.

\item Arakawa A., Suarez M.J., 1983:
      Vertical differencing of the primitive equations
      in sigma coodinates.
      {\it Mon. Weather Rev.}, {\bf 111}, 34--45.

\item Bourke, W., 1988: 
      Spectral methods in global climate and weather prediction models.
      {\it  in Physically-Based Modelling and Simulation of Climate 
              and Climatic Change. Part I.}, 169--220., Kluwer.

\item Haltiner, G.J. and R.T. Williams,  1980: 
      Numerical Prediction and Dynamic Meteorology (2nd ed.),
      John Wiley \& Sons, 477pp.

\item Kondo J., 1993:
      A new bucket model for predicting water content 
      in the surface soil layer.
      {\it J. Japan Soc. Hydrol. Water Res.}, {\bf 6}, 344-349. (in Japanese)

\item Le Treut H. and Z.-X. Li, 1991:
      Sensitivity of an atmospheric general circulation model to
      prescribed SST changes: feedback effects associated with the
      simulation of cloud optical properties.
      {\it Climate Dynamics}, {\bf 5}, 175-187.

\item Louis, J., 1979: 
      A parametric model of vertical eddy fluxes in the 
      atmosphere. 
      {\it Bound. Layer Meteor.}, {\bf 17}, 187--202.

\item Louis, J., M. Tiedtle, J.-F. Geleyn, 1982:
      A short history of the PBL parameterization at ECMWF.
      {\it Workshop on Planetary Boundary layer Parameterization},
      59-80, ECMWF, Reading U.K.

\item Manabe, S., J. Smagorinsky and R.F. Strickler, 1965:
      Simulated climatology of a general circulation model
      with a hydrologic cycle. 
      {\it Mon. Weather Rev.} , {\bf 93}, 769--798.

\item Miller, M.J., A.C.M. Beljaars and T.N. Palmer, 1992:
      The sensitivity of the ECMWF model 
      to the parameterization of evaporation from the tropical oceans.
      {\it J. Climate}, {\bf 5}, 418-434.

\item Moorthi S. and M.J. Suarez, 1992:
      Relaxed Arakawa-Scubert: A parameterization of moist convection 
      for general circulation models.
      {\em Mon. Weather Rev.,\/} {\bf 120} 978--1002.

\item Mellor, G.L. and T. Yamada, 1974:
      A hierarchy of turbulence closure models
      for planetary boundary layers.
      {\it J. Atmos. Sci.}, {\bf 31}, 1791--1806.

\item Mellor, G.L. and T. Yamada, 1982:
      Development of a turbulence closure
      model for geophysical fluid problems.
      {\it Rev. Geophys Atmos. Phys.}, {\bf 20}, 851--875.  

\item Nakajima T. and M. Tanaka, 1986:
      Matrix formulation for the transfer of solar radiation
      in a plane-parallel scattering atmosphere.
      {\it J. Quant. Spectrosc. Radiat. Transfer}, {\bf 35}, 13-21.

\item Randall D.A., Pan D-.M., 1993:
      Imprementation of the Arakawa-Schubert cumulus parameterization
      with a prognostic closure.
      {Meteorological Monograph.\/}
\end{description}
