%\documentstyle[a4j,dennou]{jarticle}


\section{モデルの概要}

\subsection{CCSR/NIES AGCM の特徴}

        AGCM5.4 は, 東京大学気候システム研究センター(CCSR)と
        国立環境研究所(NIES)の共同研究によって作成された, 
        全球3次元大気大循環モデルである. 
        モデルの特徴を以下に示す.

        \begin{center}
        \begin{description}
\item[TAB00000:0.0] 方程式系
\item[TAB00000:0.1] 静水圧プリミティブ方程式系
\item[TAB00000:1.0] 領域
\item[TAB00000:1.1] 全球3次元
\item[TAB00000:2.0] 予報変数
\item[TAB00000:2.1] 水平風速, 温度, 地表気圧, 比湿, 雲水量,
                          陸地表面温度, 土壌水分
\item[TAB00000:3.0] 水平離散化
\item[TAB00000:3.1] スペクトル変換法
\item[TAB00000:4.0] 鉛直離散化
\item[TAB00000:4.1] σ系(Arakawa and Suarez, 1983)
\item[TAB00000:5.0] 放射
\item[TAB00000:5.1] 2ストリーム DOM/adding 法
\item[TAB00000:6.0] 
\item[TAB00000:6.1] (Nakajima and Tanaka, 1986 に基づく)
\item[TAB00000:7.0] 大規模雲過程
\item[TAB00000:7.1] 総水混合比を予報変数とするスキーム
\item[TAB00000:8.0] 
\item[TAB00000:8.1] (Le Treut and Li, 1991 に基づく)
\item[TAB00000:9.0] 積雲対流
\item[TAB00000:9.1] 簡易型 Arakawa-Schubert スキーム
\item[TAB00000:10.0] 鉛直拡散
\item[TAB00000:10.1] Mellor and Yamada(1974) level2
\item[TAB00000:11.0] 地表flux
\item[TAB00000:11.1] Louis(1979) バルク式
\item[TAB00000:12.0] 
\item[TAB00000:12.1] (気孔抵抗, Miller et al. 1992 の対流効果を考慮)
\item[TAB00000:13.0] 地表面熱過程
\item[TAB00000:13.1] 多層熱伝導
\item[TAB00000:14.0] 地表水文過程
\item[TAB00000:14.1] バケツモデル
\item[TAB00000:15.0] 
\item[TAB00000:15.1] (または, 新バケツモデル, 多層水輸送)
\item[TAB00000:16.0] 重力波抵抗
\item[TAB00000:16.1] McFarlane(1987)に基づくスキーム
\item[TAB00000:17.0] オプション
\item[TAB00000:17.1] 南北-鉛直および東西-鉛直2次元モデル. 
                          鉛直1次元モデル.
\item[TAB00000:18.0] 
\item[TAB00000:18.1] 海洋混合層結合モデル
\end{description}
        \end{center}

