
\subsection{力学部分のまとめ}

ここでは, これまでの記述と重複するが,
力学過程部で行なわれる計算を列挙する.

\subsubsection{力学部分の計算の概要}

力学過程は, 以下のような順序で計算が行なわれる.

\begin{enumerate}
\item 水平風の渦度・発散への変換   \texttt{MODULE:[UV2VDG(dvect)]}
\item 仮温度の計算              \texttt{MODULE:[VIRTMD(dvtmp)]}
\item 気圧傾度項の計算           \texttt{MODULE:[HGRAD(dvect)]}
\item 鉛直流の診断的計算         \texttt{MODULE:[GRDDYN/PSDOT(dgdyn)]}
\item 移流による時間変化項 \texttt{MODULE:[GRDDYN(dgdyn)]}
\item 予報変数のスペクトルへの変換 \texttt{MODULE:[GD2WD(dg2wd)]}
\item 時間変化項のスペクトルへの変換 \texttt{MODULE:[TENG2W(dg2wd)]}
\item スペクトル値時間積分 \texttt{MODULE:[TINTGR(dintg)]}
\item 予報変数の格子点値への変換 \texttt{MODULE:[GENGD(dgeng)]}
\item 疑似等 TERM00000 面拡散補正   \texttt{MODULE:[CORDIF(ddifc)]}
\item 拡散による摩擦熱の考慮    \texttt{MODULE:[CORFRC(ddifc)]}
\item 質量の保存の補正          \texttt{MODULE:[MASFIX(dmfix)]}
\item (物理過程)             \texttt{MODULE:[PHYSCS(padmn)]}
\item (時間フィルター)        \texttt{MODULE:[TFILT(aadvn)]}
\end{enumerate}

\subsubsection{水平風の渦度・発散への変換}

水平風の格子点値 TERM00001,TERM00001 
から渦度・発散の格子点値 TERM00002,TERM00002 を求める.
まず, 渦度・発散のスペクトル
TERM00003,TERM00003 を求める,
\begin{verbatim}
EQ=00033.
EQ=00033.
\end{verbatim}
\begin{verbatim}
EQ=00034.
EQ=00034.
\end{verbatim}
それをさらに, 
\begin{verbatim}
EQ=00000.
\end{verbatim}
等を用いて格子点値に変換する.

\subsubsection{仮温度の計算}

仮温度 TERM00005 は, 
\begin{verbatim}
EQ=00035.
\end{verbatim}
ただし, TERM00006 であり, 
TERM00007 は水蒸気の気体定数
(461 TERM00008TERM00009)
TERM00010 は空気の気体定数
(287.04 TERM00011TERM00012)
である.

\subsubsection{気圧傾度項の計算}

気圧傾度項 TERM00013 は,
まず, TERM00014 を
\begin{verbatim}
EQ=00001.
\end{verbatim}
でスペクトル表現に直してから,
\begin{verbatim}
EQ=00002.
\end{verbatim}
\begin{verbatim}
EQ=00003.
\end{verbatim}

\subsubsection{鉛直流の診断的計算}

気圧変化項, および鉛直流,
\begin{verbatim}
EQ=00004.
\end{verbatim}
%
\begin{verbatim}
EQ=00005.
\end{verbatim}
%
ならびにその非重力波成分を計算する.
%
\begin{verbatim}
EQ=00006.
\end{verbatim}
%
\begin{verbatim}
EQ=00007.
\end{verbatim}

\subsubsection{移流による時間変化項}

運動量移流項:
\begin{verbatim}
EQ=00036.
EQ=00036.
\end{verbatim}
%
\begin{verbatim}
EQ=00037.
EQ=00037.
\end{verbatim}
\begin{verbatim}
EQ=00008.
\end{verbatim}

温度移流項:
\begin{verbatim}
EQ=00009.
\end{verbatim}
\begin{verbatim}
EQ=00010.
\end{verbatim}
%
\begin{verbatim}
EQ=00038.
EQ=00038.
EQ=00038.
EQ=00038.
EQ=00038.
EQ=00038.
\end{verbatim}

水蒸気移流項:
\begin{verbatim}
EQ=00011.
\end{verbatim}
\begin{verbatim}
EQ=00012.
\end{verbatim}
%
\begin{verbatim}
EQ=00013.
\end{verbatim}

\subsubsection{予報変数のスペクトルへの変換}

(1) および
(2) を用いて

TERM00024,TERM00024 を
渦度・発散のスペクトル表現
TERM00025,TERM00025 に変換する.
さらに,
温度 TERM00026, 比湿 TERM00027, 
TERM00028 を
\begin{verbatim}
EQ=00014.
\end{verbatim}
でスペクトル表現に変換する.

\subsubsection{時間変化項のスペクトルへの変換}

渦度の時間変化項
\begin{verbatim}
EQ=00039.
EQ=00039.
EQ=00039.
\end{verbatim}
%
発散の時間変化項の非重力波成分
\begin{verbatim}
EQ=00040.
EQ=00040.
EQ=00040.
EQ=00040.
\end{verbatim}
%
温度の時間変化項の非重力波成分
\begin{verbatim}
EQ=00041.
EQ=00041.
EQ=00041.
\end{verbatim}
%
水蒸気の時間変化項
\begin{verbatim}
EQ=00042.
EQ=00042.
EQ=00042.
\end{verbatim}

\subsubsection{スペクトル値時間積分}

行列形式の方程式
\begin{verbatim}
EQ=00043.
EQ=00043.
EQ=00043.
EQ=00043.
\end{verbatim}
%
を LU 分解を用いて解くことによって 
TERM00038 を求め,
%
\begin{verbatim}
EQ=00015.
\end{verbatim}
%
\begin{verbatim}
EQ=00016.
\end{verbatim}

%
によって
TERM00044,
TERM00045 
を求めて, TERM00046 におけるスペクトルの値を計算する.
\begin{verbatim}
EQ=00044.
EQ=00044.
EQ=00044.
EQ=00044.
EQ=00044.
\end{verbatim}

\subsubsection{予報変数の格子点値への変換}


渦度・発散のスペクトル値 TERM00047,TERM00047 から
水平風速の格子点値 TERM00048,TERM00048 を求める.
\begin{verbatim}
EQ=00017.
\end{verbatim}
%
\begin{verbatim}
EQ=00018.
\end{verbatim}

さらに,
\begin{verbatim}
EQ=00019.
\end{verbatim}
などによって, TERM00052,TERM00052 を求め,
\begin{verbatim}
EQ=00045.
\end{verbatim}
を計算する.

\subsubsection{疑似等 TERM00053 面拡散補正}

水平拡散は 等 TERM00054 面上で適用されるが,
山岳の傾斜の大きな領域では, 山を上る方向に水蒸気が輸送され,
山頂部での偽の降水をもたらすなどの問題を起こす.
それを緩和するために, 等 TERM00055 面の拡散に近くなるような
補正を TERM00056,TERM00056 について入れる.

\begin{verbatim}
EQ=00046.
EQ=00046.
EQ=00046.
\end{verbatim}
%
であるから,
\begin{verbatim}
EQ=00020.
\end{verbatim}
などととする.
TERM00057 は, TERM00058 のスペクトル値 TERM00059 に
拡散係数のスペクトル表現をかけたものを
格子の値に変換して用いる.

\subsubsection{拡散による摩擦熱の考慮}

拡散による摩擦熱は,
\begin{verbatim}
EQ=00021.
\end{verbatim}
と見積もられる.
したがって,
\begin{verbatim}
EQ=00022.
\end{verbatim}

\subsubsection{質量の保存の補正}

スペクトル法による取扱いは,
TERM00060 の全球積分は丸め誤差を除いて保存するが,
質量, すなわち TERM00061 の全球積分の保存は保証されない.
また, スペクトルの波数打ちきりにともない,
水蒸気の格子点値に負の値が出ることがある.
これらの事情から, 
乾燥大気の質量と水蒸気, 雲水の質量を保存させ,
さらに負の水蒸気量となる領域を除去するための補正を行なう.

まず, 力学の計算の最初に \texttt{MODULE:[FIXMAS]},
水蒸気, 雲水の各成分の全球積分値 TERM00062,TERM00062 を計算しておく.
\begin{verbatim}
EQ=00047.
EQ=00047.
\end{verbatim}
また, 計算の最初のステップで
乾燥質量 TERM00063 を計算し, 記憶する.
\begin{verbatim}
EQ=00048.
\end{verbatim}

力学計算の終りには \texttt{MODULE:[MASFIX]},
以下のような手順で補正を行なう.
\begin{enumerate}
\item まず, 負の水蒸気量となる格子点について,
      直下の格子点から水蒸気を分配して,
      負の水蒸気を除去する.
      TERM00064 であるとすると,
      \begin{verbatim}
EQ=00049.
EQ=00049.
\end{verbatim}
      ただし, これは TERM00065 となる場合にのみ行なう.

\item 次に上の手続きで除去されなかった格子点について値を 0 とする.

\item 全球積分値 TERM00066 を計算し,
      これが TERM00067 と一致するように,
      全球の水蒸気量に一定割合をかける.

      \begin{verbatim}
EQ=00023.
\end{verbatim}
      
\item 乾燥空気質量の補正を行なう.
      同様に TERM00068 を計算し,

      \begin{verbatim}
EQ=00024.
\end{verbatim}

\end{enumerate}

\subsubsection{水平拡散とレーリー摩擦}

水平拡散の係数をスペクトル表現すると,

\begin{verbatim}
EQ=00025.
\end{verbatim}
%
\begin{verbatim}
EQ=00026.
\end{verbatim}
%
\begin{verbatim}
EQ=00027.
\end{verbatim}

TERM00069 は レーリー摩擦係数である.
レーリー摩擦係数は
\begin{verbatim}
EQ=00028.
\end{verbatim}
のようなプロファイルで与える.
ただし,
\begin{verbatim}
EQ=00029.
\end{verbatim}
と近似する.
標準値は, TERM00070,
TERM00071 (TERM00072 : モデルの最上レベル),
TERM00073 m,
TERM00074 m である.

\subsubsection{時間フィルター}


leap frog における計算モードの除去のために 
Asselin(1972) の時間フィルターを毎ステップ適用する.
%
\begin{verbatim}
EQ=00030.
\end{verbatim}
%
と TERM00075 を求める.
次のステップの力学過程で用いる TERM00076 としては,
この TERM00077 を用いる.
TERM00078 としては標準的に 0.05 を使用する. 

実際には
まず, 予報変数の格子点値への変換 \texttt{MODULE:[GENGD]} の箇所で,
\begin{verbatim}
EQ=00031.
\end{verbatim}
を求めておき, 物理過程の処理が終わり
TERM00079 の値が確定した後で \texttt{MODULE:[TFILT]} で,
\begin{verbatim}
EQ=00032.
\end{verbatim}
とする.
