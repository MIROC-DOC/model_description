\subsection{水平離散化}

水平方向の離散化は
スペクトル変換法を用いる(Bourke, 1988).
経度, 緯度に関する微分の項は直交関数展開によって評価し,
一方, 非線型項は格子点上で計算する.

\subsubsection{スペクトル展開}

展開関数系としては球面上の Laplacianの固有関数系である
球面調和関数 TERM00000,TERM00000 を用いる.
ただし, TERM00001 である.
TERM00002 は次のような方程式を満たし,
%
\begin{quote}
EQ=00000.
\end{quote}
%
Legendre 陪関数 TERM00003 を用いて次のように書き表される.
%
\begin{quote}
EQ=00001.
\end{quote}
%
ただし, TERM00004 である.

球面調和関数による展開は \footnotemark ,
\begin{quote}
EQ=00002.
\end{quote}
と書くと,
%
\footnotetext{ここでは三角形切断を用いている.
              平行四辺形切断の場合には, 
              TERM00005
              と置き換えること.}
%
\begin{quote}
EQ=00003.
\label{球面展開}
\end{quote}
%
その逆変換は,
\begin{quote}
EQ=00017.\\
\label{展開係数}
EQ=00017.
\end{quote}
%
のように表される.
%
積分を和に置き換えて評価する際には,
TERM00006 積分については Gauss の台形公式を,
TERM00007 積分については Gauss-Legendre 積分公式を用いる.
TERM00008 は Gauss 緯度, TERM00009 は Gauss 荷重である.
また, TERM00010 は等間隔の格子である.

スペクトル展開を用いれば,
微分を含む項の格子点値は次のように求められる.
%
\begin{quote}
EQ=00004.
\label{気圧x}
\end{quote}
%
\begin{quote}
EQ=00005.
\label{気圧y}
\end{quote}
%
さらに,
TERM00011, TERM00012 のスペクトル成分から, 
TERM00013,TERM00013 の格子点値が以下のように得られる.
%
\begin{quote}
EQ=00006.
\label{Uを求める}
\end{quote}
%
\begin{quote}
EQ=00007.
\label{Vを求める}
\end{quote}

方程式の移流項に現れる微分は,
次のように求められる.
%
\begin{quote}
\label{A積分}
\nonumber
EQ=00018.\\
\nonumber
EQ=00018.\\
EQ=00018.
\end{quote}
%
\begin{quote}
\label{B積分}
\nonumber
EQ=00019.\\
\nonumber
EQ=00019.\\
EQ=00019.
\end{quote}
%
さらに,
\begin{quote}
EQ=00008.
\end{quote}
を TERM00014 の項の評価のために用いる.

\subsubsection{水平拡散項}

水平拡散項は, 次のように TERM00015 の形で入れる.
%
\begin{quote}
EQ=00009.
\label{水平拡散}
\end{quote}
%
\begin{quote}
EQ=00010.
\end{quote}
%
\begin{quote}
EQ=00011.
\end{quote}
%
\begin{quote}
EQ=00012.
\end{quote}
%
この水平拡散項は計算の安定化のための意味合いが強い.
小さなスケールに選択的な水平拡散を表すため,
TERM00016 としては, 4 TERM00017 16を用いる.
ここで, 渦度および発散の拡散についている余分な項は,
TERM00018 の剛体回転の項が減衰しないことを表したものである.

\subsubsection{方程式のスペクトル表現}


\begin{enumerate}
\item 連続の式

\begin{quote}
\nonumber
EQ=00020.\\
EQ=00020.
\end{quote}
%
ここで,
\begin{quote}
EQ=00013.
\end{quote}

\item 運動方程式

\begin{quote}
\nonumber
EQ=00021.\\
\nonumber
EQ=00021.\\
EQ=00021.
\end{quote}
%
\begin{quote}
\nonumber
EQ=00022.\\
\nonumber
EQ=00022.\\
\nonumber
EQ=00022.\\
EQ=00022.
\end{quote}
%
ただし,
%
\begin{quote}
EQ=00014.
\end{quote}

\item 熱力学の式

\begin{quote}
\nonumber
EQ=00023.\\
\nonumber
EQ=00023.\\
\nonumber
EQ=00023.\\
EQ=00023.
\end{quote}
%
ただし,
%
\begin{quote}
EQ=00015.
\end{quote}



\item 水蒸気の式

\begin{quote}
\nonumber
EQ=00024.\\
\nonumber
EQ=00024.\\
\nonumber
EQ=00024.\\
EQ=00024.
\end{quote}

ただし,
%
\begin{quote}
EQ=00016.
\end{quote}



\end{enumerate}

