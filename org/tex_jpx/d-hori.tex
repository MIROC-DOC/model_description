\subsection{水平離散化}

水平方向の離散化は
スペクトル変換法を用いる(Bourke, 1988).
経度, 緯度に関する微分の項は直交関数展開によって評価し,
一方, 非線型項は格子点上で計算する.

\subsubsection{スペクトル展開}

展開関数系としては球面上の Laplacianの固有関数系である
球面調和関数 TERM00000,TERM00000 を用いる.
ただし, TERM00001 である.
TERM00002 は次のような方程式を満たし,
%
\begin{verbatim}
EQ=00000.
\end{verbatim}
%
Legendre 陪関数 TERM00003 を用いて次のように書き表される.
%
\begin{verbatim}
EQ=00001.
\end{verbatim}
%
ただし, TERM00004 である.

球面調和関数による展開は \footnotemark ,
\begin{verbatim}
EQ=00002.
\end{verbatim}
と書くと,
%
\footnotetext{ここでは三角形切断を用いている.
              平行四辺形切断の場合には, 
              TERM00005
              と置き換えること.}
%
\begin{verbatim}
EQ=00003.
\end{verbatim}
%
その逆変換は,
\begin{verbatim}
EQ=00017.
EQ=00017.
\end{verbatim}
%
のように表される.
%
積分を和に置き換えて評価する際には,
TERM00007 積分については Gauss の台形公式を,
TERM00008 積分については Gauss-Legendre 積分公式を用いる.
TERM00009 は Gauss 緯度, TERM00010 は Gauss 荷重である.
また, TERM00011 は等間隔の格子である.

スペクトル展開を用いれば,
微分を含む項の格子点値は次のように求められる.
%
\begin{verbatim}
EQ=00004.
\end{verbatim}
%
\begin{verbatim}
EQ=00005.
\end{verbatim}
%
さらに,
TERM00014, TERM00015 のスペクトル成分から, 
TERM00016,TERM00016 の格子点値が以下のように得られる.
%
\begin{verbatim}
EQ=00006.
\end{verbatim}
%
\begin{verbatim}
EQ=00007.
\end{verbatim}

方程式の移流項に現れる微分は,
次のように求められる.
%
\begin{verbatim}
EQ=00018.
EQ=00018.
EQ=00018.
\end{verbatim}
%
\begin{verbatim}
EQ=00019.
EQ=00019.
EQ=00019.
\end{verbatim}
%
さらに,
\begin{verbatim}
EQ=00008.
\end{verbatim}
を TERM00019 の項の評価のために用いる.

\subsubsection{水平拡散項}

水平拡散項は, 次のように TERM00020 の形で入れる.
%
\begin{verbatim}
EQ=00009.
\end{verbatim}
%
\begin{verbatim}
EQ=00010.
\end{verbatim}
%
\begin{verbatim}
EQ=00011.
\end{verbatim}
%
\begin{verbatim}
EQ=00012.
\end{verbatim}
%
この水平拡散項は計算の安定化のための意味合いが強い.
小さなスケールに選択的な水平拡散を表すため,
TERM00021 としては, 4 TERM00022 16を用いる.
ここで, 渦度および発散の拡散についている余分な項は,
TERM00023 の剛体回転の項が減衰しないことを表したものである.

\subsubsection{方程式のスペクトル表現}


\begin{enumerate}
\item 連続の式

\begin{verbatim}
EQ=00020.
EQ=00020.
\end{verbatim}
%
ここで,
\begin{verbatim}
EQ=00013.
\end{verbatim}

\item 運動方程式

\begin{verbatim}
EQ=00021.
EQ=00021.
EQ=00021.
\end{verbatim}
%
\begin{verbatim}
EQ=00022.
EQ=00022.
EQ=00022.
EQ=00022.
\end{verbatim}
%
ただし,
%
\begin{verbatim}
EQ=00014.
\end{verbatim}

\item 熱力学の式

\begin{verbatim}
EQ=00023.
EQ=00023.
EQ=00023.
EQ=00023.
\end{verbatim}
%
ただし,
%
\begin{verbatim}
EQ=00015.
\end{verbatim}



\item 水蒸気の式

\begin{verbatim}
EQ=00024.
EQ=00024.
EQ=00024.
EQ=00024.
\end{verbatim}

ただし,
%
\begin{verbatim}
EQ=00016.
\end{verbatim}



\end{enumerate}

