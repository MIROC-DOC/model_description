\subsection{水平離散化}

水平方向の離散化は
スペクトル変換法を用いる(Bourke, 1988).
経度, 緯度に関する微分の項は直交関数展開によって評価し,
一方, 非線型項は格子点上で計算する.

\subsubsection{スペクトル展開}

展開関数系としては球面上の Laplacianの固有関数系である
球面調和関数 TERM00229,TERM00229 を用いる.
ただし, TERM00230 である.
TERM00231 は次のような方程式を満たし,
%
\begin{verbatim}
EQ=00039.
\end{verbatim}
%
Legendre 陪関数 TERM00232 を用いて次のように書き表される.
%
\begin{verbatim}
EQ=00040.
\end{verbatim}
%
ただし, TERM00233 である.

球面調和関数による展開は \footnotemark ,
\begin{verbatim}
EQ=00041.
\end{verbatim}
と書くと,
%
\footnotetext{ここでは三角形切断を用いている.
              平行四辺形切断の場合には, 
              TERM00234
              と置き換えること.}
%
\begin{verbatim}
EQ=00042.
\end{verbatim}
%
その逆変換は,
\begin{verbatim}
EQ=00056.
EQ=00056.
\end{verbatim}
%
のように表される.
%
積分を和に置き換えて評価する際には,
TERM00236 積分については Gauss の台形公式を,
TERM00237 積分については Gauss-Legendre 積分公式を用いる.
TERM00238 は Gauss 緯度, TERM00239 は Gauss 荷重である.
また, TERM00240 は等間隔の格子である.

スペクトル展開を用いれば,
微分を含む項の格子点値は次のように求められる.
%
\begin{verbatim}
EQ=00043.
\end{verbatim}
%
\begin{verbatim}
EQ=00044.
\end{verbatim}
%
さらに,
TERM00243, TERM00244 のスペクトル成分から, 
TERM00245,TERM00245 の格子点値が以下のように得られる.
%
\begin{verbatim}
EQ=00045.
\end{verbatim}
%
\begin{verbatim}
EQ=00046.
\end{verbatim}

方程式の移流項に現れる微分は,
次のように求められる.
%
\begin{verbatim}
EQ=00057.
EQ=00057.
EQ=00057.
\end{verbatim}
%
\begin{verbatim}
EQ=00058.
EQ=00058.
EQ=00058.
\end{verbatim}
%
さらに,
\begin{verbatim}
EQ=00047.
\end{verbatim}
を TERM00248 の項の評価のために用いる.

\subsubsection{水平拡散項}

水平拡散項は, 次のように TERM00249 の形で入れる.
%
\begin{verbatim}
EQ=00048.
\end{verbatim}
%
\begin{verbatim}
EQ=00049.
\end{verbatim}
%
\begin{verbatim}
EQ=00050.
\end{verbatim}
%
\begin{verbatim}
EQ=00051.
\end{verbatim}
%
この水平拡散項は計算の安定化のための意味合いが強い.
小さなスケールに選択的な水平拡散を表すため,
TERM00250 としては, 4 TERM00251 16を用いる.
ここで, 渦度および発散の拡散についている余分な項は,
TERM00252 の剛体回転の項が減衰しないことを表したものである.

\subsubsection{方程式のスペクトル表現}


\begin{enumerate}
\item 連続の式

\begin{verbatim}
EQ=00059.
EQ=00059.
\end{verbatim}
%
ここで,
\begin{verbatim}
EQ=00052.
\end{verbatim}

\item 運動方程式

\begin{verbatim}
EQ=00060.
EQ=00060.
EQ=00060.
\end{verbatim}
%
\begin{verbatim}
EQ=00061.
EQ=00061.
EQ=00061.
EQ=00061.
\end{verbatim}
%
ただし,
%
\begin{verbatim}
EQ=00053.
\end{verbatim}

\item 熱力学の式

\begin{verbatim}
EQ=00062.
EQ=00062.
EQ=00062.
EQ=00062.
\end{verbatim}
%
ただし,
%
\begin{verbatim}
EQ=00054.
\end{verbatim}



\item 水蒸気の式

\begin{verbatim}
EQ=00063.
EQ=00063.
EQ=00063.
EQ=00063.
\end{verbatim}

ただし,
%
\begin{verbatim}
EQ=00055.
\end{verbatim}



\end{enumerate}

