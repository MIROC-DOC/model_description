
\subsection{鉛直拡散}

\subsubsection{鉛直拡散スキームの概要}

鉛直拡散スキームは,
サブグリッドスケールの乱流拡散による
物理量の鉛直フラックスを評価する.
主な入力データは, 風速, TERM00826,TERM00826, 気温 TERM00827, 比湿 TERM00828, 雲水量 TERM00829, であり,
出力データは, 運動量, 熱, 水蒸気, 雲水の鉛直フラックスと
implicit 解を得るための微分値である.

鉛直拡散係数の見積りには
Mellor and Yamada(1974, 1982)の
乱流クロージャーモデルの
level 2 のパラメタリゼーションを用いる.

計算手順の概略は以下の通りである.
\begin{enumerate}
\item 大気の安定度として
      Richardson 数を計算する.
\item Richardson 数から拡散係数を計算する \texttt{MODULE:[VDFCOF]}.
\item 拡散係数からフラックスとその微分を計算する.
\end{enumerate}

\subsubsection{フラックス計算の基本式}

大気中の鉛直拡散フラックスは, 
拡散係数 TERM00830 を用いて, 以下のように評価される.

\begin{verbatim}
EQ=00305.
\end{verbatim}
\begin{verbatim}
EQ=00306.
\end{verbatim}
\begin{verbatim}
EQ=00307.
\end{verbatim}
\begin{verbatim}
EQ=00308.
\end{verbatim}

\subsubsection{Richardson 数}

大気の成層安定度の基準となる,
バルクRichardson数 TERM00831 は

\begin{verbatim}
EQ=00309.
\end{verbatim}
で定義される.
ここで, TERM00832 は TERM00833 を表す.
また, TERM00834 は, 静水圧の式より,
\begin{verbatim}
EQ=00310.
\end{verbatim}

フラックスRicahrdson数 TERM00835 は,

\begin{verbatim}
EQ=00311.
\end{verbatim}
ただし, 

\begin{verbatim}
EQ=00325.
EQ=00325.
EQ=00325.
EQ=00325.
EQ=00325.
EQ=00325.
\end{verbatim}
\begin{verbatim}
EQ=00312.
\end{verbatim}

\begin{verbatim}
EQ=00313.
\end{verbatim}

TERM00836 と TERM00837 の関係を図示すると,
図\ref{p-dif:rib-rif} のようになる.

\begin{figure}[htbp]
  \begin{center}
    \epsfile{file=dif-rif.ps,width=70mm}
    \caption{バルクRichardson数とフラックスRichadson数の関係}
    \label{p-dif:rib-rif}
  \end{center}
\end{figure}

\subsubsection{拡散係数}

拡散係数は,
各層の境界(TERM00838 レベル)ごとに,
次のように与えられる.

\begin{verbatim}
EQ=00326.
EQ=00326.
\end{verbatim}

ここで, TERM00841,TERM00841 は,
\begin{verbatim}
EQ=00314.
\end{verbatim}
%
\begin{verbatim}
EQ=00315.
\end{verbatim}
を用いて,
\begin{verbatim}
EQ=00327.
EQ=00327.
\end{verbatim} 

TERM00842 は混合距離であり, Blakadar(1962)に従って,
\begin{verbatim}
EQ=00316.
\end{verbatim}
ととる. 
TERM00843 は K\'{a}rman 定数である. 
% 漸近混合距離$l_0$は, 
% $R_{iB} > R_{iBC}$ となるような最大の$\sigma$ を
% 境界層の上端$\sigma_B$と考え, それ以上と以下で異なる値を与える.
% \begin{equation}
% \sigma_B = max(\sigma; R_{iB} > R_{iBC})
% \end{equation}
%
% \begin{equation}
%  l_0 = \left\{
%  \begin{array}{l}
%    l_{0B} \; \; ( \sigma > \sigma_B ) \\
%    l_{0F} \; \; ( \sigma \le \sigma_B ) \\
%  \end{array} 
%  \right.
% \end{equation}
% 
% 現在の標準値は, 
% $R_{iBC}=2, l_{0B}=1000\mbox{m}, l_{0F}=10\mbox{m}$ である.
現在の標準値は, TERM00844 m である.

TERM00845,TERM00845 を TERM00846 の関数として示すと,
図\ref{p-dif:smsh-rif} のようになる.

\begin{figure}[htbp]
  \begin{center}
    \epsfile{file=dif-smsh.ps,width=70mm}
    \caption{フラックスRichardson数と TERM00847,TERM00847 の関係}
    \label{p-dif:smsh-rif}
  \end{center}
\end{figure}


\subsubsection{フラックスの計算}

以上を用い, フラックスおよびフラックス微分を計算する.

\begin{verbatim}
EQ=00317.
\end{verbatim}
\begin{verbatim}
EQ=00318.
\end{verbatim}
\begin{verbatim}
EQ=00319.
\end{verbatim}
\begin{verbatim}
EQ=00320.
\end{verbatim}

\begin{verbatim}
EQ=00321.
\end{verbatim}
\begin{verbatim}
EQ=00322.
\end{verbatim}
\begin{verbatim}
EQ=00323.
\end{verbatim}
\begin{verbatim}
EQ=00324.
\end{verbatim}

\subsubsection{拡散係数の最小値}

非常に安定な場合に, 以上の見積りは拡散係数として 0 を与える.
これをそのまま用いるとモデルの振舞いにさまざまな
悪影響をもたらすので, 適当な最小値を置く.
現在の標準値は, 全てのフラックスに共通で
TERM00848 0.15 TERM00849/s である.

\subsubsection{その他の留意点}

浅い積雲対流 \texttt{MODULE:[SHLCOF]} を呼んでいるが, 
標準ではこれはダミーである.
