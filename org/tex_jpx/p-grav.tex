
\subsection{重力波抵抗}

\subsubsection{重力波抵抗スキームの概要}

重力波抵抗スキームは,
サブグリッドスケールの地形によって励起される
重力波の上方への運動量フラックスを表現し,
その収束に伴う水平風の減速を計算する.
主な入力データは, 東西風 TERM01164, 南北風 TERM01165, 気温 TERM01166, であり,
出力データは東西風と南北風の時間変化率,
TERM01167,TERM01167, である.

計算手順の概略は以下の通りである.
%
\begin{enumerate}
\item 地表面での運動量フラックスを
      地表高度の分散, 
      最下層での水平風速, 成層安定度などから求める.
\item 運動量フラックスを持つ重力波の上方への伝播を考える.
      運動量フラックスが臨界フルード数から決まる
      臨界フラックスを越える場合には,
      砕波が起こってフラックスはその臨界値となるとする.
\item 運動量フラックスの各層での収束に応じた
      水平風の時間変化を計算する.
\end{enumerate}

\subsubsection{局所フルード数と運動量フラックスの関係}

地表起源の重力波による
水平運動量の鉛直フラックスを考えると,
ある高度でのフラックス TERM01168 と
局所フルード数 TERM01169 との間には,
\begin{verbatim}
EQ=00490.
\end{verbatim}
の関係が成り立つ.
ここで, TERM01170 は
ブラントバイサラ振動数, 
TERM01171 は大気の密度, 
TERM01172 は風速, TERM01173 は地表高度の波打ちの水平スケールに対応する
比例定数である.
これから,
\begin{verbatim}
EQ=00491.
\end{verbatim}

局所フルード数 TERM01174 は,
ある値, 臨界フルード数 TERM01175 を越えることができないとする.
(589) から計算される
局所フルード数が臨界フルード数 TERM01176 を越える場合には
重力波は過飽和となり,
臨界フルード数に対応する運動量フラックスまで
フラックスは減少する.

\subsubsection{地表での運動量フラックス}

地表面で励起される重力波による
水平運動量の鉛直フラックスの大きさ TERM01177 は,
ただし, 地表での局所フルード数 
TERM01178 を
(590) に代入することにより,
%
\begin{verbatim}
EQ=00492.
\end{verbatim}
%
と見積られる.
ここで, 
TERM01179 は地表風速,
TERM01180,TERM01180 はそれぞれ地表付近の大気の
安定度と密度である.
TERM01181 はサブグリッドの地表高度変化の指標であり,
地表高度の標準偏差 TERM01182 に等しいとする.

ここで, 地表での局所フルード数 
TERM01183 が 臨界フルード数
TERM01184 を越えるときは, 
運動量フラックスは TERM01185 を(590) に代入した値に
抑えられるとする.
すなわち,
\begin{verbatim}
EQ=00493.
\end{verbatim}

\subsubsection{上層での運動量フラックス}

レベル TERM01186 での運動量フラックス TERM01187 が
求められているとする.
TERM01188 は, 飽和が起こらないときには
TERM01189 に等しい.
この運動量フラックス TERM01190 が,
TERM01191 レベルでの臨界フルード数から計算される運動量フラックス
を上回るときには, TERM01192 層内で砕波が起こり,
運動量フラックスは臨界に対応するフラックスまで減少する.

\begin{verbatim}
EQ=00494.
\end{verbatim}

ただし TERM01193 は,
各層での風速ベクトルの,
最下層の水平風の方向に対する射影成分の大きさであり,
\begin{verbatim}
EQ=00495.
\end{verbatim}

\subsubsection{運動量収束による水平風の時間変化の大きさ}

水平風の射影成分 TERM01194 の時間変化率は,
\begin{verbatim}
EQ=00496.
\end{verbatim}
%
によって求められる.すなわち,
%
\begin{verbatim}
EQ=00497.
\end{verbatim}
%
これを用いて,
東西風と南北風の時間変化率は以下のように計算される.
\begin{verbatim}
EQ=00498.
EQ=00498.
\end{verbatim}

\subsubsection{その他の留意点}

\begin{enumerate}
\item 最下層の風速が小さく TERM01195 のとき,また,
      地表の起伏が小さく TERM01196 のときは, 
      地表で重力波が励起されないと仮定する.
\end{enumerate}

