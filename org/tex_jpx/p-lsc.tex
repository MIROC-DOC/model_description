
\subsection{大規模凝結}

\subsubsection{大規模凝結スキームの概要}

大規模凝結スキームは,
積雲対流以外の雲にかかわる凝結過程を表現し,
潜熱の放出と水蒸気の減少, 降水を計算する.
また, 放射に関与する雲水量と雲の被覆率を計算する.
主な入力データは, 気温 TERM00000, 比湿 TERM00001, 雲水量 TERM00002, であり,
出力データは気温・比湿・雲水量の時間変化率,
TERM00003,TERM00003,
雲量 TERM00004 である.

CCSR/NIES AGCM では, 水蒸気混合比 (比湿 TERM00005) に加えて
雲水量 (TERM00006) もモデルの予報変数となっている.
実際には, この大規模凝結のルーチンの中で
まずその和である総水量 (TERM00007) を計算し, 
それを再び雲水と水蒸気に分けることを行なっているので,
事実上は予報変数は総水量 (TERM00008) 一つである.
格子内の TERM00009 の変動の分布を仮定することにより,
各格子内での雲量(雲の水平被覆率)と雲水量を診断する.
また, 雲水の降水への変換と降水の落下途中の蒸発を考慮する.

計算手順の概略は以下の通りである.
%
\begin{enumerate}
\item 水蒸気量 TERM00010 雲水量 TERM00011 とを加え,
      総水量 TERM00012 とする.
      気温は雲水を蒸発させた, 
      liquid water temperature  TERM00013 とする.
\item TERM00014 の変動の分布を仮定し,
      雲量を求め, 雲水と水蒸気に再び分離する.
\item 凝結による温度の変化を考慮し,
      逐次近似によって
      雲量, 雲水量, 水蒸気の配分を決定する.
\item 雲水量の降水への変換を評価する.
\item 氷の落下を評価する.
\item 降水と落下氷の蒸発を評価する.
\end{enumerate}

\subsubsection{雲水量の診断}

格子平均の総水量 TERM00015 が与えられたとき,
総水量 TERM00016 の格子内での分布を,
TERM00017 から TERM00018 の間の
一様分布であると仮定する. すなわち確率密度関数は,
\begin{quote}
EQ=00000.
\end{quote}
この分布は, 水平方向の分布であると考える.
一方, 飽和比湿は格子平均の値 TERM00019 を用いる.

格子点の中で,
TERM00020 である領域に雲が存在すると考える(図\ref{lsc:fig-cloud}).

\begin{figure}[hbtp]
  \begin{center}
  \epsfile{file=p-lsc-cld.ps,width=50mm}    
  \end{center}
  \caption{総水量の分布, 飽和比湿の分布と雲量}
  \label{lsc:fig-cloud}
\end{figure}


すると, 図の陰影で示すような,
総水量が飽和を越える部分の水平方向の比率 TERM00021 は,
\begin{quote}
EQ=00001.
\end{quote}
となり, これが雲量(水平雲被覆率)である.

また, 雲水量 TERM00022 は, TERM00023 である領域で
TERM00024 を積分したもので,
\begin{quote}
EQ=00002.
\label{p-lsc:l}
\end{quote}

\subsubsection{逐次近似による決定}

まず, 水蒸気 TERM00025 と 雲水 TERM00026 温度 TERM00027 から,
総水量 TERM00028 と liquid water temperature TERM00029 を求める.
\begin{quote}
EQ=00017.\\
EQ=00017.
\end{quote}
TERM00030 は, 雲水を全て蒸発させたときの温度に対応する.
TERM00031, TERM00032 と置く.

温度 TERM00033 に対する飽和比湿によって, 
前述の方法で評価した雲水量を TERM00034 とすると,
それによって温度が変わり,
\begin{quote}
EQ=00003.
\label{p-lsc:itereate1}
\end{quote}
この温度に対する飽和比湿によって, 評価した雲水量を TERM00035,
それによって変化した温度を TERM00036 \ldots として逐次近似で解く.
この逐次収束を速めるために, Newton 法の取扱いを行なう.
すなわち, (\ref{p-lsc:itereate1})の代わりに
\begin{quote}
EQ=00004.
\end{quote}
とする.
TERM00037 は, (\ref{p-lsc:l}) を用いて解析的に求めることができる.

\subsubsection{降水過程}

降水は診断された雲水量に依存して起こる.
降水率(単位1/s)を TERM00038 とすると,
\begin{quote}
EQ=00005.
\end{quote}

TERM00039 は降水の時間スケールであり,
\begin{quote}
EQ=00006.
\end{quote}
ここで, TERM00040 は臨界雲水量であり,
Bergeron-Findeisen 効果を考慮して,
\begin{quote}
EQ=00007.
\end{quote}
TERM00041, TERM00042, TERM00043, 
TERM00044 K, TERM00045 K である.

降水は TERM00046 の減少をもたらす.
\begin{quote}
EQ=00018.\\
EQ=00018.
\end{quote}
これを TERM00047 の間積分すると,
\begin{quote}
EQ=00008.
\end{quote}

ある高さ TERM00048 での降水フラックス
(単位kg TERM00049 TERM00050)を TERM00051 とすると,
\begin{quote}
EQ=00009.
\end{quote}

\subsubsection{氷の落下過程}

雲水は, 温度に応じて氷雲, 水雲に分かれる.
氷雲の比率は
\begin{quote}
EQ=00010.
\end{quote}
(ただし, 最大値1, 最小値0) である. また,
TERM00052,TERM00052.
氷雲は, ゆっくりとした速度で降下するとして, 
その効果を考える. 降下速度 TERM00053 は,
\begin{quote}
EQ=00011.
\end{quote}
ただし, TERM00054 m/s, TERM00055.
すると, 
\begin{quote}
EQ=00012.
\end{quote}
として, 降水と同様に処理できる.

\subsubsection{降水の蒸発過程}

降水の蒸発 TERM00056 は, 次のように見積もる.

\begin{quote}
EQ=00013.
\end{quote}
ただし, TERM00057 のときは 0 とする.
TERM00058 は湿球温度に対応する飽和比湿で,
\begin{quote}
EQ=00014.
\end{quote}
%
これにより, 降水は
\begin{quote}
EQ=00015.
\end{quote}
となる. また, 蒸発による温度の下降を見積もる.
\begin{quote}
EQ=00016.
\end{quote}

\subsubsection{その他の留意点}

\begin{enumerate}
\item 計算は最上層から下に向かって行なう.
      便宜上, 計算はその上の層起源の降水の
      その層での蒸発を評価するところからはじめる.
\item 落下した氷はすぐ下の層で
      その層に既に存在する雲水と同じ扱いとなり,
      総水量に組み入れられる.
\end{enumerate}


