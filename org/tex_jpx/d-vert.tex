
\subsection{鉛直離散化}

Arakawa and Suarez(1983) に従って, 
基礎方程式を鉛直方向に差分によって離散化する.
このスキームは次のような特徴をもつ.
%
\begin{itemize}
\item 全領域積分した質量を保存
\item 全領域積分したエネルギーを保存
\item 全領域積分の角運動量を保存
\item 全質量積分した温位を保存
\item 静水圧の式がlocalにきまる(下層の高度が上層の温度に依存しない)
\item 水平方向に一定の, ある特定の温度分布について,
      静水圧の式が正確になり, 気圧傾度力が0になる.
\item 等温位大気はいつまでも等温位に留まる
\end{itemize}      

\subsubsection{レベルのとりかた}

下の層から上へと層の番号をつける.
TERM00000,TERM00000 の物理量は整数レベル(layer)で定義されるとする.
一方, TERM00001 は半整数レベル(level)で定義される.
%
まず, 半整数レベルでの TERM00002 の値
TERM00003,TERM00003
を定義する.
%
ただし, レベル TERM00004 は下端(TERM00005),
レベル TERM00006 は上端(TERM00007)とする.

整数レベルの TERM00008 の値
TERM00009,TERM00009
は次の式から求められる.
%
\begin{verbatim}
EQ=00000.
\end{verbatim}
%
さらに,
\begin{verbatim}
EQ=00001.
\end{verbatim}
を定義しておく.

\subsubsection{鉛直離散化表現}

各方程式の離散化表現は次のようになる.

\begin{enumerate}
\item 連続の式, 鉛直速度

\begin{verbatim}
EQ=00002.
\end{verbatim}
%
\begin{verbatim}
EQ=00003.
\end{verbatim}
%
\begin{verbatim}
EQ=00004.
\end{verbatim}
   
\item 静水圧の式

\begin{verbatim}
EQ=00010.
EQ=00010.
\end{verbatim}
%
\begin{verbatim}
EQ=00011.
EQ=00011.
\end{verbatim}
%
ここで,
%
\begin{verbatim}
EQ=00012.
EQ=00012.
\end{verbatim}

\item 運動方程式

\begin{verbatim}
EQ=00005.
\end{verbatim}
%
\begin{verbatim}
EQ=00006.
\end{verbatim}
%
ここで,
%
\begin{verbatim}
EQ=00013.
EQ=00013.
\end{verbatim}
%
\begin{verbatim}
EQ=00014.
EQ=00014.
\end{verbatim}
%
\begin{verbatim}
EQ=00015.
EQ=00015.
\end{verbatim}

\begin{verbatim}
EQ=00007.
\end{verbatim}

\item 熱力学の式

\begin{verbatim}
EQ=00016.
EQ=00016.
EQ=00016.
\end{verbatim}
%
ここで,
\begin{verbatim}
EQ=00017.
EQ=00017.
EQ=00017.
EQ=00017.
EQ=00017.
EQ=00017.
EQ=00017.
\end{verbatim}
%
\begin{verbatim}
EQ=00018.
EQ=00018.
\end{verbatim}
%
\begin{verbatim}
EQ=00019.
EQ=00019.
\end{verbatim}

\item 水蒸気の式

\begin{verbatim}
EQ=00008.
\end{verbatim}
%
\begin{verbatim}
EQ=00009.
\end{verbatim}

\end{enumerate}
