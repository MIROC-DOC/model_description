
\subsection{大規模凝結}

\subsubsection{大規模凝結スキームの概要}

大規模凝結スキームは,
積雲対流以外の雲にかかわる凝結過程を表現し,
潜熱の放出と水蒸気の減少, 降水を計算する.
また, 放射に関与する雲水量と雲の被覆率を計算する.
主な入力データは, 気温 $T$, 比湿 $q$, 雲水量 $l$, であり,
出力データは気温・比湿・雲水量の時間変化率,
$\partial T/\partial t, \partial q/\partial t, \partial l/\partial t$,
雲量 $C$ である.

CCSR/NIES AGCM では, 水蒸気混合比 (比湿 $q$) に加えて
雲水量 ($l$) もモデルの予報変数となっている.
実際には, この大規模凝結のルーチンの中で
まずその和である総水量 ($q^t = q+l$) を計算し, 
それを再び雲水と水蒸気に分けることを行なっているので,
事実上は予報変数は総水量 ($q^t$) 一つである.
格子内の $q^t$ の変動の分布を仮定することにより,
各格子内での雲量(雲の水平被覆率)と雲水量を診断する.
また, 雲水の降水への変換と降水の落下途中の蒸発を考慮する.

計算手順の概略は以下の通りである.
%
\begin{enumerate}
\item 水蒸気量 $q$ 雲水量 $l$ とを加え,
      総水量 $q^t$ とする.
      気温は雲水を蒸発させた, 
      liquid water temperature  $T_l$ とする.
\item $q^t$ の変動の分布を仮定し,
      雲量を求め, 雲水と水蒸気に再び分離する.
\item 凝結による温度の変化を考慮し,
      逐次近似によって
      雲量, 雲水量, 水蒸気の配分を決定する.
\item 雲水量の降水への変換を評価する.
\item 氷の落下を評価する.
\item 降水と落下氷の蒸発を評価する.
\end{enumerate}

\subsubsection{雲水量の診断}

格子平均の総水量 $\bar{q}^t = \bar{q} + \bar{l}$ が与えられたとき,
総水量 $q^t$ の格子内での分布を,
$(1-b)\bar{q}^t$ から $(1+b)\bar{q}^t$ の間の
一様分布であると仮定する. すなわち確率密度関数は,
\begin{equation}
  F(q^t) = \left\{ 
           \begin{array}{ll}
             (2b\bar{q}^t)^{-1} \; \; &
                 (1-b)\bar{q}^t < q^t <  (1+b)\bar{q}^t \\
             0                        & その他
           \end{array}
           \right. \; .
\end{equation}
この分布は, 水平方向の分布であると考える.
一方, 飽和比湿は格子平均の値 $\bar{q}^*$ を用いる.

格子点の中で,
$q^t > q^*$ である領域に雲が存在すると考える(図\ref{lsc:fig-cloud}).

\begin{figure}[hbtp]
  \begin{center}
  \epsfile{file=p-lsc-cld.ps,width=50mm}    
  \end{center}
  \caption{総水量の分布, 飽和比湿の分布と雲量}
  \label{lsc:fig-cloud}
\end{figure}


すると, 図の陰影で示すような,
総水量が飽和を越える部分の水平方向の比率 $C$ は,
\begin{equation}
  C = \left\{ 
           \begin{array}{ll}
             0 \; \; & (1+b)\bar{q}^t \leq \bar{q}^* \\
             \displaystyle
             \frac{(1+b)\bar{q}^t - \bar{q}^*}
                  {2b\bar{q}^t}                    
               \; \; & (1-b)\bar{q}^t < \bar{q}^* < (1+b)\bar{q}^t \\
             1 \; \; & (1-b)\bar{q}^t \leq \bar{q}^*
           \end{array}
        \right.
\end{equation}
となり, これが雲量(水平雲被覆率)である.

また, 雲水量$l$は, $q^t > q^*$ である領域で
$q^t - q^*$ を積分したもので,
\begin{equation}
  l = \left\{ 
           \begin{array}{ll} \displaystyle
             0 \; \; & (1+b)\bar{q}^t \leq \bar{q}^* \\
            \displaystyle
             \frac{\left[(1+b)\bar{q}^t - \bar{q}^*\right]^2}
                  {4b\bar{q}^t}
               \; \; & (1-b)\bar{q}^t \leq \bar{q}^* \leq (1+b)\bar{q}^t  \\
            \displaystyle
             \bar{q}^t - \bar{q}^*
                \; \; & (1-b)\bar{q}^t \geq \bar{q}^*
           \end{array}
        \right. 
  \label{p-lsc:l}
\end{equation}

\subsubsection{逐次近似による決定}

まず, 水蒸気 $q$ と 雲水 $l$ 温度 $T$ から,
総水量$q^t$ と liquid water temperature $T_l$ を求める.
\begin{eqnarray}
  q^t  & = & q + l \; , \\
  T_l  & = & T - \frac{L}{C_P} l \; .
\end{eqnarray}
$T_l$ は, 雲水を全て蒸発させたときの温度に対応する.
$T^{(0)} = T_l$, $l^{(0)} = 0$ と置く.

温度 $T_l$ に対する飽和比湿によって, 
前述の方法で評価した雲水量を $l^{(1)}$ とすると,
それによって温度が変わり,
\begin{equation}
  T^{(1)} = T_l +  \frac{L}{C_P} l^{(1)} \; .
  \label{p-lsc:itereate1}
\end{equation}
この温度に対する飽和比湿によって, 評価した雲水量を $l^{(2)}$,
それによって変化した温度を $T^{(2)}$ \ldots として逐次近似で解く.
この逐次収束を速めるために, Newton 法の取扱いを行なう.
すなわち, (\ref{p-lsc:itereate1})の代わりに
\begin{equation}
  T^{(1)} = T_l +  \frac{L}{C_P} l^{(1)} 
                   \left( 1 - \frac{L}{C_P} \frac{dl}{dT} \right)^{-1}
\end{equation}
とする.
$dl/dT$ は, (\ref{p-lsc:l}) を用いて解析的に求めることができる.

\subsubsection{降水過程}

降水は診断された雲水量に依存して起こる.
降水率(単位1/s)を $P$ とすると,
\begin{equation}
  P = l / \tau_P \; .
\end{equation}

$\tau_P$ は降水の時間スケールであり,
\begin{equation}
  \tau_P  = \tau_0 \left\{ 1 - \exp\left[ - \left(\frac{l}{l_C}\right)^2  
                                   \right]  \right\}^{-1} \; .
\end{equation}
ここで, $l_C$ は臨界雲水量であり,
Bergeron-Findeisen 効果を考慮して,
\begin{equation}
  l_C = \left\{ 
        \begin{array}{ll}
          l_C^0 \; \; & T \ge T_0 \\
          l_C^0 \left\{ 1+\alpha \exp\left[ - \beta(T-Tc)^2 \right] 
                \right\}^{-1}\; \; 
                      & T_0 > T >  T_c \\
          l_C^0 ( 1+\alpha )^{-1}
                      & T \le T_c
        \end{array}
        \right. \; .
\end{equation}
$l_C^0=10^{-4}$, $\alpha=50$, $\beta=0.03$, 
$T_0=273.15$K, $T_c=258.15$K である.

降水は $l$ の減少をもたらす.
\begin{eqnarray}
  P         & = & l / \tau_P \; , \\
  \DP{l}{t} & = & -P \; .
\end{eqnarray}
これを$\Delta t$の間積分すると,
\begin{equation}
  P \Delta t  =  \left\{ 1- \exp(- \Delta t/\tau_P) \right\} l \; .
\end{equation}

ある高さ $p$ での降水フラックス
(単位kg m$^{-2}$ s$^{-1}$)を$F_P$とすると,
\begin{equation}
  F_P(p) = \int_0^p P \frac{dp}{g} \; .
\end{equation}

\subsubsection{氷の落下過程}

雲水は, 温度に応じて氷雲, 水雲に分かれる.
氷雲の比率は
\begin{equation}
   f_I = \frac{ T_0 - T }{ T_0 - T_1 }
\end{equation}
(ただし, 最大値1, 最小値0) である. また,
$T_0 = 273.15\mbox{K}, T_1 = 258.15\mbox{K}$.
氷雲は, ゆっくりとした速度で降下するとして, 
その効果を考える. 降下速度$V_S$は,
\begin{equation}
  V_S = V_S^0 ( \rho_a f_I l )^\gamma \; .
\end{equation}
ただし, $V_S^0=3$m/s, $\gamma=0.17$.
すると, 
\begin{equation}
  \tau_S = \frac{\Delta p}{\rho g V_S} 
\end{equation}
として, 降水と同様に処理できる.

\subsubsection{降水の蒸発過程}

降水の蒸発 $E$ は, 次のように見積もる.

\begin{equation}
E = k_E (q^w - q) \frac{F_P}{V_T} \; .
\end{equation}
ただし, $q^w < q$ のときは 0 とする.
$q_w$ は湿球温度に対応する飽和比湿で,
\begin{equation}
  q^w = q + \frac{q^* - q}{1+ \frac{L}{C_P}\DP{q^*}{T}} \; .
\end{equation}
%
これにより, 降水は
\begin{equation}
  F_P(p) = \int_0^p (P - E) \frac{dp}{g}
\end{equation}
となる. また, 蒸発による温度の下降を見積もる.
\begin{equation}
  \DP{T}{t} = - \frac{L}{C_P} E \; .
\end{equation}

\subsubsection{その他の留意点}

\begin{enumerate}
\item 計算は最上層から下に向かって行なう.
      便宜上, 計算はその上の層起源の降水の
      その層での蒸発を評価するところからはじめる.
\item 落下した氷はすぐ下の層で
      その層に既に存在する雲水と同じ扱いとなり,
      総水量に組み入れられる.
\end{enumerate}


