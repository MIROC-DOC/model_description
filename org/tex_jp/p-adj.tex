
\subsection{乾燥対流調節}

\subsubsection{乾燥対流調節の概要}

乾燥対流調節は, 
連続した2つのレベルの間の層において成層が対流不安定, 
すなわち温度減率が乾燥断熱減率よりも大きい場合に
温度減率を乾燥断熱減率に調節する. この際, 水蒸気等を混合する.
主な入力データは, 気温 $T$, 比湿 $q$ であり,
出力データは調節された気温 $T$, 比湿 $q$ である.

本来は鉛直拡散が効率的であれば, それによって
鉛直の対流不安定は基本的に取り除かれるはずである.
ただし, 成層圏等ではそれが不足するおそれがあるので,
計算の安定のために対流調節を入れてある.

\subsubsection{乾燥対流調節の手続き}

層$(k-1,k)$が対流不安定である条件は,
%
\begin{equation}
\frac{T_{k-1} - T_{k}}{p_{k-1} - p_{k}} 
  > \frac{R}{C_p} \bar{T_{k-1/2}}
  = \frac{R}{C_p}
    \frac{\Delta p_{k-1} T_{k-1} + \Delta p_{k} T_{k}}
         {\Delta p_{k-1} + \Delta p_{k}} 
\end{equation}
%
すなわち,
\begin{equation}
 S = T_{k-1} - T_{k}
     - \frac{R}{C_p} 
        \frac{\Delta p_{k-1} T_{k-1} + \Delta p_{k} T_{k}}
         {\Delta p_{k-1} + \Delta p_{k}} 
       (p_{k-1} - p_{k})
   > 0 
  \label{p-adj:cond}
\end{equation}
が条件である.

これが満たされるときには,
\begin{eqnarray}
T_{k-1} & \leftarrow & \frac{\Delta p_{k}}{\Delta p_{k-1} + \Delta p_{k}} S \\
T_{k} & \leftarrow & \frac{\Delta p_{k-1}}{\Delta p_{k-1} + \Delta p_{k}} S 
\end{eqnarray}
によって, 温度を補正する.
さらに,
\begin{equation}
q_{k-1}, q_{k} \leftarrow
     \frac{\Delta p_{k-1} q_{k-1} + \Delta p_{k} q_{k}}
          {\Delta p_{k-1} + \Delta p_{k}} 
\end{equation}
によって, 二層の比湿等の値を平均化する.

このような操作を行なうと,
その上下の層が不安定化する可能性がある. そのため,
この操作を下層から上層に繰り返すことを
対流不安定な層が無くなるまで繰り返す.
ただし, 計算誤差等を考え, 
(\ref{p-adj:cond}) の条件として,
S が 0 でないある小さな有限値以下になれば収束したとみなす.

現在, 標準的には 下から2層目と3層めの間より上を調節している.

