\subsection{水平離散化}

水平方向の離散化は
スペクトル変換法を用いる(Bourke, 1988).
経度, 緯度に関する微分の項は直交関数展開によって評価し,
一方, 非線型項は格子点上で計算する.

\subsubsection{スペクトル展開}

展開関数系としては球面上の Laplacianの固有関数系である
球面調和関数 $Y_n^m(\lambda,\mu)$ を用いる.
ただし, $\mu \equiv \sin\varphi$ である.
$Y_n^m$は次のような方程式を満たし,
%
\begin{equation} 
\nabla^{2}_{\sigma} Y_n^m(\lambda,\mu) 
= - \frac{n(n+1)}{a^{2}} Y_n^m(\lambda,\mu) 
\end{equation}
%
Legendre 陪関数 $P_n^m$ を用いて次のように書き表される.
%
\begin{equation}
Y_n^m(\lambda,\mu) = P_n^m (\mu) e^{im \lambda}
\end{equation}
%
ただし, $ n \geq | m | $ である.

球面調和関数による展開は \footnotemark ,
\begin{equation}
   {Y_n^m}_{ij} \equiv Y_n^m ( \lambda_i, \mu_j )
\end{equation}
と書くと,
%
\footnotetext{ここでは三角形切断を用いている.
              平行四辺形切断の場合には, 
              $\sum_{m=-N}^{N} \sum_{n=|m|}^{N} 
               \rightarrow
               \sum_{m=-N}^{N} \sum_{n=|m|}^{N+|m|}$
              と置き換えること.}
%
\begin{equation}
  \label{球面展開}
  X_{ij} \equiv X ( \lambda_i, \mu_j )
  =  {\cal R}\Dvect{e} \sum_{m=-N}^{N} \sum_{n=|m|}^{N} 
        X_n^m {Y_n^m}_{ij} ,
\end{equation}
%
その逆変換は,
\begin{eqnarray}
  X_n^m 
        & = & \frac{1}{4 \pi} 
             \int_{-1}^{1} d \mu \int_{0}^{\pi} d \lambda 
               X( \lambda, \mu ) Y_n^{m *} ( \lambda, \mu ) \\
        & = & \frac{1}{I} \sum_{i=1}^{I} \sum_{j=1}^{J}  
               X_{ij} {Y_n^{m*}}_{ij} w_j 
\label{展開係数}
\end{eqnarray}
%
のように表される.
%
積分を和に置き換えて評価する際には,
$\lambda$ 積分については Gauss の台形公式を,
$\mu$ 積分については Gauss-Legendre 積分公式を用いる.
$\mu_j$ は Gauss 緯度, $w_j$は Gauss 荷重である.
また, $\lambda_i$ は等間隔の格子である.

スペクトル展開を用いれば,
微分を含む項の格子点値は次のように求められる.
%
\begin{equation}
   \label{気圧x}
        \left(  \frac{\partial X}{\partial \lambda} \right)_{ij}
     =  
        {\cal R}\Dvect{e} \sum_{m=-N}^{N} \sum_{n=|m|}^{N} 
       im X_n^m {Y_n^m}_{ij}
\end{equation}
%
\begin{equation}
   \label{気圧y}
   \left( \cos\varphi \frac{\partial X}{\partial \varphi} \right)_{ij}
     =  {\cal R}\Dvect{e} \sum_{m=-N}^{N} \sum_{n=|m|}^{N} 
       X_n^m 
       ( 1-\mu^{2} ) \frac{\partial }{\partial \mu} {Y_n^m}_{ij}
\end{equation}
%
さらに,
$\zeta$, $D$ のスペクトル成分から, 
$u,v$の格子点値が以下のように得られる.
%
\begin{equation}
  \label{Uを求める}
  u_{ij}
  = \frac{1}{\cos\varphi}
     {\cal R}\Dvect{e} \sum_{m=-N}^{N} 
                       \sum_{\stackrel{n=|m|}{n \neq 0}}^{N} 
    \left\{
             \frac{a}{n(n+1)} \zeta_n^m 
            (1-\mu^{2}) \DP{}{\mu} {Y_n^m}_{ij}
          -  \frac{im a}{n(n+1)} D_n^m {Y_n^m}_{ij}
    \right\}
\end{equation}
%
\begin{equation}
  \label{Vを求める}
  v_{ij}
  = \frac{1}{\cos\varphi}
   {\cal R}\Dvect{e} \sum_{m=-N}^{N}
                     \sum_{\stackrel{n=|m|}{n \neq 0}}^{N}
    \left\{
          -  \frac{im a}{n(n+1)} \zeta_n^m {Y_n^m}_{ij}
          -  \frac{a}{n(n+1)} D_n^m 
            (1-\mu^{2}) \DP{}{\mu} {Y_n^m}_{ij}
    \right\}
\end{equation}

方程式の移流項に現れる微分は,
次のように求められる.
%
\begin{eqnarray}
  \label{A積分}
  \left( \frac{1}{a\cos\varphi} \DP{A}{\lambda} \right)_n^m 
  & = & \frac{1}{4 \pi} 
        \int_{-1}^{1} d \mu \int_{0}^{\pi} d \lambda 
          \frac{1}{a\cos\varphi} \DP{A}{\lambda} Y_n^{m *} \nonumber\\
  & = & \frac{1}{4 \pi} 
        \int_{-1}^{1} d \mu \int_{0}^{\pi} d \lambda \,
          im A \cos\varphi \frac{1}{a(1-\mu^{2})} Y_n^{m *} \nonumber\\
  & = & \frac{1}{I} \sum_{i=1}^{I} \sum_{j=1}^{J}  
          im A_{ij} \cos\varphi_j
          {Y_n^{m *}}_{ij} \frac{w_j}{a(1-\mu_j^{2})} 
\end{eqnarray}
%
\begin{eqnarray}
  \label{B積分}
  \left( \frac{1}{a\cos\varphi} 
         \DP{}{\varphi} (A\cos\varphi) \right)_n^m 
   & = & \frac{1}{4 \pi a} 
         \int_{-1}^{1} d \mu \int_{0}^{\pi} d \lambda 
           \DP{}{\mu} (A\cos\varphi) Y_n^{m *} \nonumber \\
   & = & - \frac{1}{4 \pi a} 
         \int_{-1}^{1} d \mu \int_{0}^{\pi} d \lambda 
           A \cos\varphi \frac{\partial }{\partial \mu} Y_n^{m *}
           \nonumber \\
  & = & - \frac{1}{I} \sum_{i=1}^{I} \sum_{j=1}^{J}  
          A_{ij}  \cos\varphi_j
          (1-\mu_j^2)  \frac{\partial }{\partial \mu} 
          {Y_n^{m *}}_{ij} \frac{w_j}{a(1-\mu_j^{2})} 
\end{eqnarray}
%
さらに,
\begin{equation}
     \left( \nabla^{2}_{\sigma} X \right)_n^m
       =  - \frac{n(n+1)}{a^{2}} X_n^m
\end{equation}
を$\nabla^2$の項の評価のために用いる.

\subsubsection{水平拡散項}

水平拡散項は, 次のように$\nabla^{N_D}$の形で入れる.
%
\begin{equation}
  \label{水平拡散}
  {\cal D}(\zeta) = K_{MH} 
                      \left[ (-1)^{N_D/2} \nabla^{N_D}
                              - \left( \frac{2}{a^2} \right)^{N_D/2} 
                      \right]
                    \zeta ,
\end{equation}
%
\begin{equation}
     {\cal D}(D) = K_{MH} 
                      \left[ (-1)^{N_D/2} \nabla^{N_D}
                              - \left( \frac{2}{a^2} \right)^{N_D/2} 
                      \right]
                    D ,
\end{equation}
%
\begin{equation}
    {\cal D}(T) = (-1)^{N_D/2} K_{HH} \nabla^{N_D} T ,
\end{equation}
%
\begin{equation}
    {\cal D}(q) = (-1)^{N_D/2} K_{EH} \nabla^{N_D} q .
\end{equation}
%
この水平拡散項は計算の安定化のための意味合いが強い.
小さなスケールに選択的な水平拡散を表すため,
$N_D$としては, 4$\sim$16を用いる.
ここで, 渦度および発散の拡散についている余分な項は,
$n=1$の剛体回転の項が減衰しないことを表したものである.

\subsubsection{方程式のスペクトル表現}


\begin{enumerate}
\item 連続の式

\begin{eqnarray}
  \DP{\pi_m^m}{t}
 & = & - \sum_{k=1}^{K} (D_n^m)_k \Delta  \sigma_k \nonumber \\
 &   & + \frac{1}{I} \sum_{i=1}^{I} \sum_{j=1}^{J}  
               Z_{ij} {Y_n^{m *}}_{ij} w_j  ,
\end{eqnarray}
%
ここで,
\begin{equation}
Z \equiv - \sum_{k=1}^{K} \Dvect{v}_k \cdot \nabla \pi .
\end{equation}

\item 運動方程式

\begin{eqnarray}
  \DP{\zeta_n^m}{t} 
  & = & \frac{1}{I} \sum_{i=1}^{I} \sum_{j=1}^{J}  
          im (A_v)_{ij} \cos\varphi_j
          {Y_n^{m *}}_{ij}
         \frac{w_j}{a(1-\mu_j^{2})} 
         \nonumber \\
  & + &   \frac{1}{I} \sum_{i=1}^{I} \sum_{j=1}^{J}  
          (A_u)_{ij} \cos\varphi_j
          (1-\mu_j^2) 
          \frac{\partial }{\partial \mu} {Y_n^{m *}}_{ij}
          \frac{w_j}{a(1-\mu_j^{2})} 
         \nonumber \\ 
  & - &  ({\cal D}_M)_n^m \zeta_n^m  \; ,
\end{eqnarray}
%
\begin{eqnarray}
  \DP{\tilde{D}_n^m}{t} 
  & = & \frac{1}{I} \sum_{i=1}^{I} \sum_{j=1}^{J}  
          im (A_u)_{ij} \cos\varphi_j
          {Y_n^{m *}}_{ij}
         \frac{w_j}{a(1-\mu_j^{2})} 
         \nonumber \\
  & - &   \frac{1}{I} \sum_{i=1}^{I} \sum_{j=1}^{J}  
          (A_v)_{ij} \cos\varphi_j
          (1-\mu_j^2) 
          \frac{\partial }{\partial \mu} {Y_n^{m *}}_{ij}
          \frac{w_j}{a(1-\mu_j^{2})} 
         \nonumber \\
  & - &  \frac{n(n+1)}{a^{2}} 
         \frac{1}{I} \sum_{i=1}^{I} \sum_{j=1}^{J}  
          E_{ij} {Y_n^{m *}}_{ij} w_j
         \nonumber \\ 
  & + &  \frac{n(n+1)}{a^{2}} 
          ( \Phi_n^m + C_{p} \hat{\kappa}_k \bar{T}_k \pi_n^m ) 
          -  ({\cal D}_M)_n^m D_n^m  ,
\end{eqnarray}
%
ただし,
%
\begin{equation}
({\cal D}_M)_n^m = K_{MH} \left[ 
                            \left( \frac{n(n+1)}{a^{2}} \right)^{N_D/2}
                            - \left( \frac{2}{a^2} \right)^{N_D/2}
                            \right]  .
\end{equation}

\item 熱力学の式

\begin{eqnarray}
  \DP{T_n^m}{t}
  & = & - \frac{1}{I} \sum_{i=1}^{I} \sum_{j=1}^{J}  
          im u_{ij} T'_{ij} \cos\varphi_j
          {Y_n^{m *}}_{ij}
         \frac{w_j}{a(1-\mu_j^{2})} 
         \nonumber \\
  &  & + \frac{1}{I} \sum_{i=1}^{I} \sum_{j=1}^{J}  
          v_{ij} T'_{ij} \cos\varphi_j
          (1-\mu_j^2) 
          \frac{\partial }{\partial \mu} {Y_n^{m *}}_{ij}
          \frac{w_j}{a(1-\mu_j^{2})} 
         \nonumber \\
  &  & + \frac{1}{I} \sum_{i=1}^{I} \sum_{j=1}^{J}  
          \left( H_{ij} + \frac{Q_{ij}+Q_{diff}}{C_{p}} \right)
          {Y_n^{m *}}_{ij} w_j
         \nonumber \\ 
  &  & - (\tilde{\cal D}_H)_n^m T_n^m \; ,
\end{eqnarray}
%
ただし,
%
\begin{equation}
({\cal D}_H)_n^m 
   =  K_{HH} \left( \frac{n(n+1)}{a^{2}} \right)^{N_D/2} .
\end{equation}



\item 水蒸気の式

\begin{eqnarray}
  \DP{q_n^m}{t}
  & = & - \frac{1}{I} \sum_{i=1}^{I} \sum_{j=1}^{J}  
          im u_{ij} q_{ij} \cos\varphi_j
          {Y_n^{m *}}_{ij} \frac{w_j}{a(1-\mu_j^{2})} 
         \nonumber \\
  &  & + \frac{1}{I} \sum_{i=1}^{I} \sum_{j=1}^{J}  
          v_{ij} q_{ij} \cos\varphi_j
          (1-\mu_j^2) 
          \frac{\partial }{\partial \mu} {Y_n^{m *}}_{ij}
          \frac{w_j}{a(1-\mu_j^{2})} 
         \nonumber \\
  &  & + \frac{1}{I} \sum_{i=1}^{I} \sum_{j=1}^{J}  
          \left( \hat{R}_{ij} + S_{q,ij} \right)
          {Y_n^{m *}}_{ij} w_j
         \nonumber \\ 
  &  & + ({\cal D}_H)_n^m q_n^m
\end{eqnarray}

ただし,
%
\begin{equation}
({\cal D}_E)_n^m 
   =  K_{EH} \left( \frac{n(n+1)}{a^{2}} \right)^{N_D/2} .
\end{equation}



\end{enumerate}

