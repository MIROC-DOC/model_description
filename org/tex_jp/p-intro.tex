
\section{物理過程}

\subsection{物理過程の概要}

物理過程として, 以下のような過程を考える
\begin{itemize}
\item 積雲対流過程
\item 大規模凝結過程
\item 放射過程
\item 鉛直拡散過程
\item 地表フラックス
\item 地表面・地中過程
\item 重力波抵抗
\end{itemize}
これらの過程による予報変数の時間変化項
$F_x, F_y, Q, M, S$ を計算し, 時間積分を行なう.
また, 大気・地表フラックスを評価するために
地表面サブモデルを利用する.
地表面サブモデルにおいては,
地中温度 $T_g$, 地中水分 $W_g$, 積雪量 $W_y$ などを
予報変数として用いている.


\subsubsection{基本方程式}

$\sigma$座標系の大気の運動方程式, 熱力学の式,
水蒸気などの物質の連続の式を考える.
運動量, 熱, 水蒸気等の鉛直方向のフラックスを考慮し,
その収束による時間変化を求める.
鉛直フラックスは全て上向きを正とする.

\begin{enumerate}
\item 運動方程式

\begin{equation}
  \rho \DD{u}{t} = \DP{Fu}{\sigma}
\label{u-eq.orig}
\end{equation}
\begin{equation}
  \rho \DD{v}{t} = \DP{Fv}{\sigma}
\end{equation}

$u, v$: 東西, 南北風; 
$Fu, Fv$: それらの鉛直フラックス.

\item 熱力学の式

\begin{equation}
  \rho \DD{c_p T}{t} = \frac{T}{\theta} \DP{F{\theta}}{\sigma} 
                     + \DP{F{R}}{\sigma} 
\end{equation}

$T$: 温度; 
$c_p$: 定圧比熱; 
$\theta=T(p/p_0)^{-R/c_p}=T(p/p_0)^{-\kappa}$: 温位;
$F\theta$: 鉛直顕熱フラックス;
$FR$: 鉛直放射フラックス.

ここで, $\theta'=T(p/p_s)^{-\kappa}=T\sigma^{-\kappa}$ とおくと, これは,
\begin{equation}
  \rho \DD{c_p T}{t} = \sigma^\kappa \DP{F{\theta'}}{\sigma} 
                     + \DP{F{R}}{\sigma} 
\end{equation}

鉛直1次元過程を考える限りにおいては,
$\theta$の代わりに $\theta'$ を考えればよい.
以下, 簡単のために, 混同のおそれがない限り,
$\theta'$ を $\theta$ と書く.

\item 水蒸気の連続の式

\begin{equation}
  \rho \DD{q}{t} = \DP{Fq}{\sigma} 
\end{equation}

$q$: 比湿; 
$F{q}$: 鉛直水蒸気フラックス.

\subsubsection{地中の基本方程式}

下向きを正とした $z$ 座標で考える. 
やはり鉛直フラックスは全て上向きを正とする.

\item 熱の式

\begin{equation}
  \DP{C_g G}{t} = \DP{Fg}{z} + Sg
\end{equation}

$G$: 地中温度; $C_g$: 定圧比熱; 
$F{g}$: 鉛直熱フラックス;
$Sg$; 加熱項(相変化などによる).

\item 地中水分の式

\begin{equation}
  C_w \DP{w}{t} = \DP{Fw}{z} + Sw
\end{equation}

$w$: 地中水分; 
$F{w}$: 鉛直水フラックス;
$Sw$; 水のソース(流出など).

\item エネルギーの収支式

地表表面で, エネルギーのバランスが成立する.

\begin{equation}
    F{\theta} + L F{q} + F{R} - F{g} = \Delta s \; \; (\sigma=1, z=0)
\end{equation}

$L$: 蒸発の潜熱;
$\Delta s$: 地表エネルギーバランス(相変化などにともなう).

\item 地表の水の収支

\begin{equation}
  Pg + Fw - Rg = 0
\end{equation}

$Pg$: 降水;
$Rg$: 表面流出.

\item 雪の収支

\begin{equation}
  \DP{Wy}{t} = Py - Fy - My
\end{equation}

$Wy$: 積雪量(kg/m$^2$);
$Py$: 降雪;
$Fy$: 昇華;
$My$: 融雪.

\end{enumerate}

\subsubsection{物理過程の時間積分法}

予報変数の時間積分の観点から物理過程を分類すると,
実行順に以下の3つに分けることができる.
\begin{enumerate}
\item 積雲対流および大規模凝結
\item 放射, 鉛直拡散, 接地境界層・地表過程       
\item 重力波抵抗, 質量調節, 乾燥対流調節
\end{enumerate}

積雲対流および大規模凝結は,
\begin{equation}
  \hat{T}^{t+\Delta t,(1)} = \hat{T}^{t+\Delta t} 
                         +  2 \Delta t Q_{CUM}(\hat{T}^{t+\Delta t})
\end{equation}
\begin{equation}
  \hat{T}^{t+\Delta t,(2)} = \hat{T}^{t+\Delta t,(1)} 
                         +  2 \Delta t Q_{LSC}(\hat{T}^{t+\Delta t,(1)})
\end{equation}
のように, 通常の Euler 差分によって値を順次更新する.
大規模凝結スキームには, 
積雲対流スキームによって更新された値が受け渡されることに注意.
実際には, 積雲対流や大規模凝結のルーチンでは加熱率等が出力され,
時間積分はその直後の \Module{GDINTG} によって行なわれる.

次のグループの放射, 鉛直拡散, 接地境界層・地表過程
の計算は, 基本的には全てこの更新された値
( $\hat{T}^{t+\Delta t,(1)}, \hat{q}^{t+\Delta t,(2)}$ 等 )
を用いて行なわれる.
ただし, 一部の項を implicit 扱いで計算するために,
これらの項を全て一括して加熱率等を計算して, 
最後に時間積分を行なう.
すなわち, シンボリックに書けば,
\begin{equation}
  \hat{T}^{t+\Delta t,(3)} = \hat{T}^{t+\Delta t,(2)} 
              + 2 \Delta t Q_{RAD,DIF,SFC}
               (\hat{T}^{t+\Delta t,(2)},\hat{T}^{t+\Delta t,(3)})
\end{equation}
となる.

重力波抵抗, 質量調節, 乾燥対流調節に関しては,
積雲対流および大規模凝結と同様である.
\begin{equation}
  \hat{T}^{t+\Delta t,(4)} = \hat{T}^{t+\Delta t,(3)} 
              +  2 \Delta t Q_{ADJ}(\hat{T}^{t+\Delta t,(3)})
\end{equation}



\subsubsection{各種の物理量}

予報変数から簡単な計算で求められる
各種の物理量の定義を示す.
このうちいくつかは, 
\Module{PSETUP} で計算される.

\begin{enumerate}
\item 仮温度

仮温度 $T_v$ は, 
\begin{equation}
  T_v = T ( 1 + \epsilon_v q - l )
\end{equation}

\item 大気密度

大気密度$\rho$ は, 以下のように計算される.
\begin{equation}
  \rho = \frac{p}{RT_v}
\end{equation}

\item 高度

高度 $z$ は, 力学過程での
ジオポテンシャルの計算と同じ方式によって評価する.
\begin{equation}
  z = \frac{\Phi}{g} 
\end{equation}
\begin{equation}
 \Phi_{1}  =  \Phi_{s} + C_{p} ( \sigma_{1}^{-\kappa} - 1  ) T_{v,1}
\end{equation}
%
\begin{equation}
 \Phi_k - \Phi_{k-1} 
   =  C_{p}
   \left[ \left( \frac{ \sigma_{k-1/2} }{ \sigma_k } \right)^{\kappa}
          - 1 \right] T_{v,k} 
       + C_{p}
   \left[ 1- 
         \left( \frac{ \sigma_{k-1/2} }{ \sigma_{k-1} } \right)^{\kappa}
              \right] T_{v,k-1}
\end{equation}


\item 層の境界の温度

層の境界の温度は, $\ln p$ すなわち $\ln \sigma$ に対する
線形補間を行なって計算する.
\begin{equation}
  T_{k-1/2} = \frac{\ln \sigma_{k-1} - \ln \sigma_{k-1/2}}
                   {\ln \sigma_{k-1} - \ln \sigma_k      } T_k
            + \frac{\ln \sigma_{k-1/2} - \ln \sigma_k}
                   {\ln \sigma_{k-1} - \ln \sigma_k      } T_{k-1}
\end{equation}

\item 飽和比湿

飽和比湿 $q^*(T,p)$
は飽和蒸気圧$e^*(T)$ を用いて近似的に,
%
\begin{equation}
q^*(T,p) = \frac{\epsilon e^*(T)}{p} .
\end{equation}
%
ここで, $\epsilon=0.622$ であり,
%
\begin{equation}
\frac{1}{e^*_v} \DP{e^*_v}{T} = \frac{L}{R_v T^2}
\label{e-sat}
\end{equation}
%
よって, 蒸発の潜熱$L$, 水蒸気の気体定数 $R_v$ を一定とすれば,
%
\begin{equation}
  e^*(T) = e^*(T=273\mbox{K}) 
                      \exp \left[ \frac{L}{R_v} 
                            \left( \frac{1}{273} - \frac{1}{T} \right)
                       \right] ,
\end{equation}
%
$e^*(T=273\mbox{K}) = 611$\ [Pa] である.

(\ref{e-sat})より,
%
\begin{equation}
\DP{q^*}{T} = \frac{L}{R_v T^2} q^*(T,p) .
\end{equation}

ここで, 温度が氷点 273.15K よりも低い場合には,
潜熱 $L$ として昇華の潜熱 $L+L_M$ を用いる.

\item 乾燥静的エネルギー, 湿潤静的エネルギー

乾燥静的エネルギー $s$ は
\begin{equation}
  s = C_p T + g z \; ,
\end{equation}
%
湿潤静的エネルギー $h$ は
\begin{equation}
  h = C_p T + g z + L q \; ,
\end{equation}
で定義される.

\end{enumerate}
