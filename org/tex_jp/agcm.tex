\documentstyle[a4j,12pt,dennou]{jarticle}
%
\Dtitle{CCSR/NIES AGCM Document}
\Dpath{/}
\Ddate{1995/01/19}
% \Dnofoot
\def\Module#1{{[\tt #1]}}
% \Dparindent

% \includeonly{p-lsc}
% \includeonly{dynamics}
% \includeonly{physics}
%
%\makeindex
\begin{document}
%
\pagenumbering{roman}
\title{\Huge CCSR/NIES AGCM の解説}
\author{}
\date{1995年1月19日}
\maketitle 
\tableofcontents
\clearpage
\pagenumbering{arabic}
%
% ccsr/nies agcm の特徴の list
\Dinclude{summary}
% agcm の概念と構造
\Dinclude{a-intro}
% 基本設定
\Dinclude{a-setup}
% 基本方程式
\Dinclude{d-basic}
% 力学:鉛直離散化
\Dinclude{d-vert}
% 力学:水平離散化
\Dinclude{d-hori}
% 力学:時間離散化
\Dinclude{d-time}
% 力学:拡散項等
%\Dinclude{d-diff}
% 力学:まとめ
\Dinclude{d-summ}
% 物理過程:イントロ
\Dinclude{p-intro}
% 物理過程:積雲対流
\Dinclude{p-cum}
% 物理過程:大規模凝結
\Dinclude{p-lsc}
% 物理過程:放射
\Dinclude{p-rad}
% 物理過程:拡散フラックス
\Dinclude{p-dif}
% 物理過程:地表フラックス
\Dinclude{p-sflx}
% 物理過程:地表モデル
\Dinclude{p-sfc}
% 物理過程:implicit 解法
\Dinclude{p-solv}
% 物理過程:重力波抵抗
\Dinclude{p-grav}
% 物理過程:対流調節
\Dinclude{p-adj}
% コードの解説: プログラムの読み方
\Dinclude{c-intro}
% コードの解説: ルーチンのリスト
%\Dinclude{c-list}
% コードの解説: データ入出力
%\Dinclude{c-io}
% コードの解説: JCL
%\Dinclude{c-jcl}
% コードの解説: 開発のためのメモ
%\Dinclude{c-dvlp}
% 参考文献
\Dinclude{referenc}
%
%\Dinclude{index}

\end{document}
