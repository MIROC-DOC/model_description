\documentclass[11pt,a4paper,onecolumn]{jarticle}
%
\usepackage[top=30truemm,bottom=30truemm,left=25truemm,right=25truemm]{geometry}
\usepackage{amsmath,amssymb}
\usepackage{bm}
\usepackage[dvipdfmx]{graphicx}
\usepackage{ascmac}
\usepackage{longtable}
\usepackage{booktabs}
\usepackage{ltablex,booktabs}
\usepackage[dvipdfm]{hyperref}
%

\title{\Huge Description for MIROC6}
\author{}
\date{\today}
\pagenumbering{arabic}
%
%
%
\begin{document}
%
	\maketitle
	\tableofcontents
	\clearpage
	%
	\def\tightlist{\itemsep1pt\parskip0pt\parsep0pt}
	% please change the line below as your tex file (without ".tex")
	% agcm の概念と構造
	% %\documentstyle[a4j,dennou]{jarticle}

\subsection{モデルの機能と構造}
\subsubsection{モデルの基本的な機能}

CCSR/NIES AGCMは, 全球3次元大気を物理的法則に基づいて記述し, 
初期値問題として系の時間発展を計算する数値モデルである.

入力するデータとしては, 以下のようなものがある.
\begin{itemize}
\item 各予報変数(水平風速,温度,地表気圧,比湿,雲水量,各地表量) の初期値データ
\item 境界条件データ(地表標高,地表状態,海面水温など)
\item モデルの各種パラメータ(大気成分, 物理過程パラメータなど)
\end{itemize}
%
一方, 出力されるものは, 以下のようなものである.
\begin{itemize}
\item 各予報変数および診断変数の, 各時刻または時間平均のデータ
\item 継続して実行を行なう場合の初期値となるデータ(再出発データ)
\item 経過と各種診断メッセージ
\end{itemize}
%
ここで, 予報変数とは, 時間発展の微分方程式を時間積分することによって
時系列として得られるデータであり,
診断変数とは, 予報変数と境界条件とパラメータから
時間積分を含まない何らかの方法により計算される量である.

より具体的に言えば,
モデルは, 基本的に以下のような方程式(予報方程式)の解を求めている.

\begin{verbatim}
EQ=00004.
EQ=00004.
EQ=00004.
EQ=00004.
EQ=00004.
EQ=00004.
\end{verbatim}

ここで, TERM00000,TERM00000 は, 
それぞれ東西風, 南北風, 気温, 地表気圧,比湿など,地表面状態量
といった2次元または3次元分布を持つ予報変数であり,
右辺はその各予報変数の時間変化をもたらす項である.
この時間変化をもたらす項 TERM00001,TERM00001 は,  
予報変数 TERM00002,TERM00002 を元に計算されるが,
TERM00003,TERM00004 で表される大気の運動による移流などの項(上式で添字 TERM00005 の項)と,
雲・放射などの各プロセスによる項(添字 TERM00006 の項)とに大別される.
前者を力学過程, 後者を物理過程と呼んで区別する.

力学過程の時間変化項の主要部分は移流項であり,
その計算においては空間微分の正確な見積りが重要である.
CCSR/NIES AGCM においては, 水平微分項の計算に
球面調和関数展開を利用している.
一方, 物理過程は水の相変化や放射の吸収射出などによるエネルギー変換, 
それらと結びついた小規模な大気運動の効果, 
地表面のさまざまなプロセスの効果などを, 
簡単なモデルでパラメータを用いて表現すること
(パラメタリゼーション)が重要となる.

予報方程式の時間積分は,
(\ref{struct:u-eq-1})などの
左辺を差分で近似することによって行なう. 例えば,
%
\begin{verbatim}
EQ=00000.
\end{verbatim}
%
とすることにより, 
\begin{verbatim}
EQ=00001.
\end{verbatim}
となる. 
ここで, TERM00007 は予報変数 TERM00008,TERM00008 などの関数であるが,
その計算においてどの時刻の予報変数を用いて TERM00009 を評価するかによって,
いろいろな時間差分スキームが考えられる.
CCSR/NIES AGCM では, 
TERM00010 での値をそのまま用いる Euler 法,
TERM00011 での値を用いる leap frog 法, 
TERM00012 における(近似的な)値を用いる implicit 法を併用している.

CCSR/NIES AGCM では
予報変数の時間積分は力学過程と物理過程とで別々に行なっている.
すなわち, まず力学の項は基本的に leap frog を用い,
\begin{verbatim}
EQ=00002.
\end{verbatim}
を解く. ただし一部の項は implicit 扱いをしている.
物理過程では,
力学の項を積分した結果に基づき, 
Euler 法と implicit 法を併用して,
\begin{verbatim}
EQ=00003.
\end{verbatim}
を求めている. 
ここで, (\ref{struct:sabun})の TERM00013 を
TERM00014 におきかえていることに注意.

\subsubsection{モデル実行の流れ}

モデル実行の流れを簡単に示すと, 以下のようになる.
[\ ]の中は該当するサブルーチン名である.

\begin{enumerate}
\item 実験のパラメータ, 座標などを設定する \texttt{MODULE:[SETPAR,PCONST,SETCOR,SETZS]}
\item 予報変数の初期値を読み込む \texttt{MODULE:[RDSTRT]}
\item 時間ステップを開始する \texttt{MODULE:[TIMSTP]}
\item 力学過程による時間積分を行なう \texttt{MODULE:[DYNMCS]}
\item 物理過程による時間積分を行なう \texttt{MODULE:[PHYSCS]}
\item 時刻を進める \texttt{MODULE:[STPTIM, TFILT]}
\item 必要ならデータを出力する \texttt{MODULE:[HISTOU]}
\item 必要なら再出発データを出力する \texttt{MODULE:[WRRSTR]}
\item 3. に戻る
\end{enumerate}

\subsubsection{予報変数}

予報変数は, 以下の通りである.
括弧内は座標系であり, TERM00015,TERM00015 はそれぞれ,
経度, 緯度, 無次元気圧 TERM00016, 鉛直深を示す.
[\ ] 内は単位である.

\begin{description}
\item[TAB00000:0.0] 東西風速
\item[TAB00000:0.1] TERM00017 (TERM00018,TERM00018)
\item[TAB00000:0.2] [m/s]
\item[TAB00000:1.0] 南北風速
\item[TAB00000:1.1] TERM00019 (TERM00020,TERM00020)
\item[TAB00000:1.2] [m/s]
\item[TAB00000:2.0] 気温
\item[TAB00000:2.1] TERM00021 (TERM00022,TERM00022)
\item[TAB00000:2.2] [K]
\item[TAB00000:3.0] 地表気圧
\item[TAB00000:3.1] TERM00023 (TERM00024,TERM00024)
\item[TAB00000:3.2] [hPa]
\item[TAB00000:4.0] 比湿
\item[TAB00000:4.1] TERM00025 (TERM00026,TERM00026)
\item[TAB00000:4.2] [kg/kg]
\item[TAB00000:5.0] 雲水混合比
\item[TAB00000:5.1] TERM00027 (TERM00028,TERM00028)
\item[TAB00000:5.2] [kg/kg]
\item[TAB00000:6.0] 地中温度
\item[TAB00000:6.1] TERM00029 (TERM00030,TERM00030)
\item[TAB00000:6.2] [K]
\item[TAB00000:7.0] 地中水分
\item[TAB00000:7.1] TERM00031 (TERM00032,TERM00032)
\item[TAB00000:7.2] [m/m]
\item[TAB00000:8.0] 積雪量
\item[TAB00000:8.1] TERM00033 (TERM00034,TERM00034)
\item[TAB00000:8.2] [kg/TERM00035]
\item[TAB00000:9.0] 海氷厚
\item[TAB00000:9.1] TERM00036 (TERM00037,TERM00037)
\item[TAB00000:9.2] [m]
\end{description}

ただし, 海氷厚さは通常混合層結合モデルでのみ予報変数となる.
また, 地中温度も, 海氷に覆われていない海洋上にあっては
通常, 予報変数ではない.
また, CCSR/NIES AGCM では, TERM00038 と TERM00039 は独立な変数ではなく,
実際には TERM00040 が予報変数である.

このうち,
地表・地中に関する量 TERM00041,TERM00041 は
同時には1ステップの量のみを記憶するが,
大気に関する量 TERM00042,TERM00042 は, 
同時に2ステップ分の量を記憶する必要がある.
これは, 大気に関する量の力学過程の時間積分において
leap forg 法を用いているからである.

大気に関する量 TERM00043,TERM00043 は,
メインルーチン [AGCM5] の管理する変数である.
一方, 地表・地中に関する量 TERM00044,TERM00044 は
メインルーチンには現れず, 
物理過程のサブルーチン \texttt{MODULE:[PHYSCS]} が管理している.

\subsubsection{変数の時間発展の流れ}

モデルの流れを, 予報変数の時間発展を中心に簡単にまとめる.

\begin{enumerate}
\item 初期値の読み込み \texttt{MODULE:[RDSTRT,PRSTRT]}

初期値として, 大気に関する量 TERM00045,TERM00045 は, 本来,
TERM00046 および TERM00047 における2組の量を用意しなくてはならない.
これは, 以前のモデルの出力結果から出発する場合には用意できるが,
通常の観測値や気候値などから出発する際には用意することはできない.
その際には, 2つの時間ステップの値として同じ値から出発し,
細かい TERM00048 を用いて計算を立ち上げる(詳しくは後述).

大気に関する量 TERM00049,TERM00049 の初期値読み込みは,
メインルーチンから呼ばれる  \texttt{MODULE:[RDSTRT]} で行なわれる.
一方, 地表・地中に関する量 TERM00050,TERM00050 の初期値の読み込みは
\texttt{MODULE:[PHYSCS]} から呼ばれる \texttt{MODULE:[PRSTRT]} で行なわれる.


\item 時間ステップの開始 \texttt{MODULE:[TIMSTP]}

時刻 TERM00051 (および一部は TERM00052)における予報変数  \\
TERM00053,TERM00053 \\
が揃っているものとする.

TERM00054 は基本的には外部から与えられるパラメータであるが,
一定時間ごとに計算の安定性の評価が行なわれ,
計算が不安定になるおそれのある場合には
TERM00055 を小さくする \texttt{MODULE:[TIMSTP]}.

\item 予報変数の出力の設定 \texttt{MODULE:[AHSTIN]}

大気の予報変数で, 通常出力されるのは,
この段階での, 時刻 TERM00056 の値
TERM00057,TERM00057
である.
実際に出力が行なわれるのは後の \texttt{MODULE:[HISTOU]} の
タイミングであるが, ここでバッファに送り込まれる.

\item 力学過程 \texttt{MODULE:[DYNMCS]} 

力学過程による予報変数の時間変化を解く. \\
TERM00058,TERM00058 \\
をもとに, 
力学過程のみを考慮した, TERM00059 での予報変数の値 \\
TERM00060,TERM00060 \\
を求める.

\begin{enumerate}
\item 渦度・発散等への変換 \texttt{MODULE:[UV2VDG, VIRTMD, HGRAD]}

大気に関する予報変数 TERM00061,TERM00061 の
力学過程による変化項を見積もるために, まず
TERM00062,TERM00062 を渦度と発散の格子点値 TERM00063,TERM00063 に変換する.
これは, 力学の方程式が渦度と発散で書かれるためである.
この変換は空間微分を含むが,
球面調和関数展開を用いて行なうことにより正確に行なうことができる 
\texttt{MODULE:[UV2VDG]}.

さらに, 仮温度 TERM00064 を計算し \texttt{MODULE:[VIRTMD]},
やはり球面調和関数展開を用いて
地表気圧の水平微分 TERM00065 を計算する \texttt{MODULE:[HGRAD]}.


\item 移流による時間変化項の計算 \texttt{MODULE:[GRDDYN]}

TERM00066,TERM00066 の TERM00067 における値を用いて,
水平および鉛直の移流による,
各大気変数の時間変化項の一部を計算する.
まず連続の方程式から鉛直速度 TERM00068 および
TERM00069 の時間変化項を診断的に求め,
それを用いて TERM00070,TERM00070 の 鉛直移流項を計算する.
さらに, TERM00071,TERM00071 の水平移流フラックスを計算する.

\item スペクトルへの変換 \texttt{MODULE:[GD2WD, TENG2W]}

大気に関する予報変数の TERM00072 での格子点の値
TERM00073,TERM00073  から, 
球面調和関数展開におけるスペクトル空間での値
(ただし, 渦度発散になおしたもの) \\
TERM00074,TERM00074 \\
を求める (ただし, TERM00075) \texttt{MODULE:[GD2WD]}.

さらに, TERM00076,TERM00076 の鉛直移流等による
時間変化項をスペクトルに展開する.
また, スペクトル空間での微分を利用して,
水平移流フラックスの収束を計算し, 
時間変化項のスペクトル表現に加える \texttt{MODULE:[TENG2W]}.

これによって, TERM00077,TERM00077 の
時間変化項のほとんどの項がスペクトルの値として求められる.
ただし, TERM00078,TERM00078 の時間変化項のうち,
水平発散 TERM00079 に線形に依存する項は
semi implicit 法によって時間積分を行なうために,
この時点では時間変化項に含まれていない.

\item 時間積分 \texttt{MODULE:[TINTGR]}

TERM00080,TERM00080 の時間変化項のうち,
水平発散 TERM00081 に線形に依存する項(重力波項)を
semi implicit 法で扱い,
さらに TERM00082,TERM00082 の水平拡散を
implicit で取り入れることにより
力学過程部分の時間積分を行なう. 
これによって, 力学過程のみを考慮した TERM00083 の
予報値のスペクトル表現 \\
TERM00084,TERM00084  \\
が求められる.

\item 格子点値への変換 \texttt{MODULE:[GENGD]}

スペクトル表現の予報変数から,
TERM00085,TERM00085 の,
力学過程のみを考慮した TERM00086 の
予報値の格子点値 \\
TERM00087,TERM00087  \\
を生成する.

\item 拡散の補正 \texttt{MODULE:[CORDIF, CORFRC]}

水平拡散は 等 TERM00088 面上で適用されるが,
山岳の傾斜の大きな領域では, 山を上る方向に水蒸気が輸送され,
山頂部での偽の降水をもたらすなどの問題を起こす.
それを緩和するために, 等 TERM00089 面の拡散に近くなるような
補正を TERM00090,TERM00090 について入れる \texttt{MODULE:[CORDIF]}. 

また, 摩擦による熱を TERM00091 に加える \texttt{MODULE:[CORFRC]}

\item 質量保存の補正 \texttt{MODULE:[MASFIX]}

TERM00092 および TERM00093 の全球積分値の保存が満たされ,
かつ TERM00094 の負の値が無くなるように補正を行なう.
さらに, 乾燥空気の質量が一定となるような補正を行なう.

\end{enumerate}

DYNMCS を出た時点では,
TERM00095 での予報変数の値は捨てられ,
TERM00096 での予報変数の値で上書きされる.
TERM00097 の予報変数の入っていた領域には,
力学過程のみを考慮した 
TERM00098 での予報変数の値が入る.

\item 物理過程 \texttt{MODULE:[PHYSCS]}

力学過程のみを考慮した, TERM00099 での予報変数の値 \\
TERM00100,TERM00100  \\
を用いて, それに物理過程による時間変化項を加えることにより
TERM00101 での予報変数の値 \\
TERM00102,TERM00102  \\
を求める.

\begin{enumerate}
\item 基本的な診断変数の計算 \texttt{MODULE:[PSETUP]}

仮温度, 各レベルでの気圧, 高度などの基本的な
診断変数を求める.

\item 積雲対流, 大規模凝結 \texttt{MODULE:[CUMLUS, LSCOND]}

積雲対流による TERM00103,TERM00103 の時間変化項を求め \texttt{MODULE:[CUMLUS]}
その項だけで \texttt{MODULE:[GDINTG]} で時間積分を行なう.
また, 大規模凝結による TERM00104,TERM00104 の時間変化項を求め \texttt{MODULE:[LSCOND]},
その項だけで \texttt{MODULE:[GDINTG]} で時間積分を行なう.
積雲対流および大規模凝結による降水量 TERM00105,TERM00105, 
雲量 TERM00106,TERM00106 などが求められる.
これによって, TERM00107,TERM00107 は,
対流凝結過程に対し調節された値
TERM00108,TERM00108 
となる.

\item 地表境界条件の設定 \texttt{MODULE:[GNDSFC, GNDALB]}

地表状態を与えられたデータにより設定する.
地表状態インデックス, 海面水温などが設定される \texttt{MODULE:[GNDSFC]}.
また, 海面以外の地表アルベドが設定される \texttt{MODULE:[GNDALB]}.
(海面のアルベドの計算は放射フラックスの計算ルーチンに組み込まれている.)

\item 放射フラックスの計算 \texttt{MODULE:[RADCON, RADFLX]}

放射フラックス計算に用いる大気成分データを設定する \texttt{MODULE:[RADCON]}.
通常, オゾンはファイルから読み込む.
雲水量および雲量は, 積雲対流および大規模凝結で求められたものを用いるが,
ここで外部から与えることもできる.
これらと TERM00109,TERM00109 を用いて
短波および長波の放射フラックス TERM00110, および
implicit計算に使用する地表温度に対する微分係数を求める \texttt{MODULE:[RADFLX]}.

\item 鉛直拡散フラックスの計算 \texttt{MODULE:[VDFFLX, VFTND1]}

TERM00111,TERM00111 
を用いて,
鉛直拡散過程による TERM00112,TERM00112 のフラックスと
implicit計算に使用する微分係数を求める \texttt{MODULE:[VDFFLX]}.
さらに, LU分解による implicit 解法計算を途中まで行なう \texttt{MODULE:[VFTND1]}.

\item 地表過程の計算・地中変数の時間積分 \texttt{MODULE:[SURFCE]}

地表大気間の  TERM00113,TERM00113 のフラックスを計算し,
地中の熱の伝導を考慮して地表でのエネルギーバランスを
implicit 解法を用いて解く.
これによって, 地表面温度 TERM00114 が診断的に求められ
地中温度の TERM00115 での値
TERM00116
が求められる.
さらに,第1層の大気の予報変数の時間変化率
TERM00117,TERM00117 を求める.

積雪・融雪過程が考慮され, 
積雪量の TERM00118 での値 TERM00119 が求められ,
地中の水の移動を考慮して
地中水分 TERM00120 が求められる.

また, 海洋混合層モデルを用いた場合には,
海洋の温度ならびに海氷厚の 
TERM00121 での値が時間積分によって求められる.

\item 放射・鉛直拡散による時間変化の評価 \texttt{MODULE:[VFTND2, RADTND, FLXCOR]}

放射フラックスおよび鉛直拡散を総合した
大気の各予報変数の時間変化率 \\
TERM00122,TERM00122 を求める \texttt{MODULE:[VFTND2]}.
さらに, その中から放射による寄与を分離する \texttt{MODULE:[RADTND]}.
これはモデルでは直接利用しないが,
データ出力の便宜のために行なう.

これらの計算においては, implicit 法を用いているため,
地表温度および大気予報変数の変化による
フラックスの変化を考慮に入れている.
そのことを勘定に入れて収支が合うフラックスを
 \texttt{MODULE:[FLXCOR]} で計算する.
これもデータ出力の便宜のためである.

\item 重力波抵抗の評価 \texttt{MODULE:[GRAVTY]}

地形起源の重力波による大気の運動量の変化を計算し,
鉛直拡散による TERM00123,TERM00123 の時間変化率
TERM00124,TERM00124 に加える.

\item 気圧変化項の評価 

降水と蒸発による気圧の変化を考慮し,
気圧の変化項 TERM00125 を求める.

\item 物理過程の時間積分 \texttt{MODULE:[GDINTG]}

以上で求められた放射, 鉛直拡散, 地表過程, 重力波抵抗等による
大気の各予報変数の時間変化率
TERM00126,TERM00126 を用いて,
TERM00127 での値を時間積分によって求める.

\item 乾燥対流調節 \texttt{MODULE:[DADJST]}

求められた TERM00128,TERM00128 が乾燥対流に対して不安定の場合
乾燥対流調節を施す.

\end{enumerate}

以上の手続きにより,
TERM00129 での予報変数の値 \\
TERM00130,TERM00130  \\
が求められる.

\item 時間フィルタ \texttt{MODULE:[TFILT]}

leap frog による計算モードの発生を抑えるため,
時間フィルタを施す.
TERM00131,TERM00131 
の3つの時刻のデータを用いて平滑操作を行なった結果を
TERM00132 におきかえる操作を各変数について行なう.
(実際には, \texttt{MODULE:[TFILT]} の段階では
TERM00133 の情報は消去されているため,
この操作は2段階で行なう.
第1段階の操作は力学過程の中で行なっている.)

\end{enumerate}
 % 渡部先生が書く
	% 基本設定
	
\subsection{基本的な設定}

ここでは, モデルの基本的な設定を示す.

\subsubsection{座標系}

座標系は, 基本的に,
経度 $\lambda$, 緯度 $\varphi$, 正規化気圧 $\sigma = p/p_S$ 
($p_S(\lambda,\varphi)$ は地表気圧)
を用い, それぞれは直交するとして扱う.
ただし, 地中の鉛直座標は $z$ を用いる.

経度は等間隔に離散化される \Module{ASETL}.
\begin{equation}
  \lambda_i = 2 \pi \frac{i-1}{I}  \;\;\; i = 1, \ldots I-1
\end{equation}

緯度は力学の項で述べる Gauss 緯度 $\varphi_j$ であり \Module{ASETL},
Gauss-Legendre 積分公式から導かれる.
これは, $\mu = \sin \varphi$ を引数とする
J 次の Legendre 多項式の 0 点である \Module{GAUSS}. 

J が大きい場合には, 近似的に,
\begin{equation}
  \varphi_j =  \pi ( \frac{1}{2}- \frac{j-1/2}{J} ) \;\;\; j = 1, \ldots J-1
\end{equation}

通常, 経度・緯度の格子間隔はほぼ等しく $J = I/2$ と取る. 
これは, スペクトル法の三角形切断に基づく.

正規化気圧 $\sigma$ は, 大気の鉛直構造を良く表現するように,
不等間隔で適当に離散化される \Module{ASETS}.
後で力学の項で述べるように, 層の境界の値
$\sigma_{k-1/2}$ を $k = 1 \ldots K+1$ で定義してから,
%
\begin{equation}
 \sigma_k = \left\{ \frac{1}{1+\kappa}
                     \left( \frac{  \sigma^{\kappa +1}_{k-1/2}
                                  - \sigma^{\kappa +1}_{k+1/2}      }
                                  { \sigma_{k-1/2} - \sigma_{k+1/2} }
                     \right)
              \right\}^{1/\kappa}
\end{equation}
によって層を代表する $\sigma$ を求める.
図\ref{a-setup:level}に, 標準的に用いられる 20層のレベルを示す.

\begin{figure}[hbtp]
  \begin{center}
    \epsfile{file=vert-cord.ps,width=80mm}
  \end{center}
  \caption{標準的に用いられるレベル}
  \label{a-setup:level}
\end{figure}

各予報変数は全て, $(\lambda_i, \varphi_j, \sigma_k)$
または $(\lambda_i, \varphi_j, z_l)$ の格子上で定義される.
(地中のレベル $z_l$ については物理過程の項で述べる.)

時間方向には, 等間隔 $\Delta t$ で離散化され,
予報方程式の時間積分が行なわれる.
ただし, 時間積分の安定性が損なわれるおそれのあるときには
$\Delta t$ は変化しうる.

\subsubsection{物理定数}

基本的な物理定数を以下に示す \Module{APCON}.

\begin{tabular}{llll}
地球半径         & $a$         & m                    & 6.37 $\times 10^6$ \\
重力加速度       & $g$         & ms$^-2$              & 9.8                \\
大気定圧比熱     & $C_p$       & J kg$^{-1}$ K$^{-1}$ & 1004.6             \\
大気気体定数     & $R$         & J kg$^{-1}$ K$^{-1}$ & 287.04             \\
水の蒸発潜熱     & $L$         & J kg$^{-1}$          & 2.5 $\times 10^6$  \\
水蒸気定圧比熱   & $C_v$       & J kg$^{-1}$ K$^{-1}$ & 1810.              \\
水の気体定数     & $R_v$       & J kg$^{-1}$ K$^{-1}$ & 461.               \\
液体水の密度     & $d_{H_2O}$  & J kg$^{-1}$ K$^{-1}$ & 1000.              \\
0$^{\circ}$での
飽和蒸気         & $e^*$(273K) & Pa                   & 611                \\
Stefan Bolzman
定数             & $\sigma_{SB}$  
                               & W m$^{-2}$ K$^{-4}$  & 5.67 
                                                          $\times 10^{-8}$ \\
K\'{a}rman 定数  & $k$         &                      & 0.4                \\
氷の融解潜熱     & $L_M$       & J kg$^{-1}$          & 3.4 $\times 10^5$  \\
水の氷点         & $T_M$       & K                    & 273.15 \\
水の定圧比熱     & $C_w$       & J kg$^{-1}$          & 4200.  \\
海水の氷点       & $T_I$       & K                    & 271.35 \\
氷の定圧比熱比   & $C_I  = C_w - L_M/T_M$
                               &                      & 2397.              \\
水蒸気分子量比   & $\epsilon  = R/R_v$
                               &                      & 0.622              \\
仮温度の係数     & $\epsilon_v = \epsilon^{-1} - 1$
                               &                      & 0.606              \\
比熱と気体定数の比
                 & $\kappa = R/C_p$
                               &                      & 0.286              \\
\end{tabular}


	% Computational flow of dynamical core
	\hypertarget{computational-flow-of-dynamical-core}{%
\subsection{Computational flow of dynamical
core}\label{computational-flow-of-dynamical-core}}

In this section, calculations of dynamical component based on coding are
summarized. \texttt{{[}module\ name(file\ name){]}}

\hypertarget{overview-of-dynamical-core}{%
\subsubsection{Overview of dynamical
core}\label{overview-of-dynamical-core}}

\begin{enumerate}
\def\labelenumi{\arabic{enumi}.}
\item
  calculate dynamical term \texttt{{[}DYNTRM(dterm.F){]}}

  1.1 calculate volticity and divergence on wave space and get grid
  value. \texttt{{[}G2W,\ W2G(xdsphe.F){]}}

  1.2 diagnose stream function and potential velocity
  \texttt{{[}DYNTRM(dterm.F){]}}

  1.3 diagnose surface pressure advection, its tendency \& vertical flow
  \texttt{{[}PSDOT(dgdyn.F){]}}

  1.4 calculate factor for hydrostatic eq. \& interporation of
  temprature on Hybrid coord. \texttt{{[}CFACT(dcfct.F){]}}

  1.5 calculate vartual temprature \texttt{{[}VIRTMD(dvtmp.F){]}}

  1.6 calculate temperature advection \texttt{{[}GRTADV(dgdyn.F){]}}

  1.7 calculate momentum advection \texttt{{[}GRUADV(dgdyn.F){]}}

  1.8 spectral transform of tendency terms \texttt{{[}G2W(xdsphe.F){]}}
\item
  Time integration of equation \texttt{DYNSTP(dstep.F)}

  2.1 calculate tracer transport \texttt{{[}TRACEG(dtrcr.F){]}}

  2.2 time integration on wave space \texttt{{[}TINTGR(dintg.F){]}}

  2.3 time integration of tracer terms \texttt{{[}GTRACE(dtrcr.F){]}}

  2.4 time filter \texttt{{[}DADVNC(dadvn.F){]}}

  2.5 get horizontal wind of grid value from wave space
  \texttt{{[}W2G(xdsphe.F){]}}

  2.6 correction of pressure-level diffusion
  \texttt{{[}CORDIF(ddifc.F){]}}

  2.7 correction of horizontal friction heating
  \texttt{{[}CORFRC(ddifc.F){]}}
\end{enumerate}

	% 基本方程式
	\hypertarget{mechanical-processes.}{%
\section{Mechanical Processes.}\label{mechanical-processes.}}

\hypertarget{basic-equations.}{%
\subsection{Basic Equations.}\label{basic-equations.}}

\hypertarget{basic-equations.-1}{%
\subsubsection{Basic Equations.}\label{basic-equations.-1}}

The basic equation is , Sphere (\(\lambda,\varphi\),
\(\lambda,\varphi\)), at \(\sigma\) coordinates It is a system of
primitive equations and It is given as follows ( Haltiner and Williams ,
1980 ).

\begin{enumerate}
\def\labelenumi{\arabic{enumi}.}
\tightlist
\item
  a series of equations
\end{enumerate}

\begin{eqnarray}
  \frac{\partial \pi}{\partial t}
    + \mathbf{v}_{H} \cdot \nabla_{\sigma} \pi
     =  - \nabla_{\sigma} \cdot \mathbf{v}_{H}
          - \frac{\partial \dot{\sigma}}{\partial \sigma}
\end{eqnarray}

\begin{verbatim}
 > <span id="mass" label="mass" label="mass">\\l.lt[mass]&lt ;/span>
\end{verbatim}

\begin{enumerate}
\def\labelenumi{\arabic{enumi}.}
\setcounter{enumi}{1}
\tightlist
\item
  hydrostatic pressure formula
\end{enumerate}

\begin{eqnarray}
  \frac{\partial \Phi}{\partial \sigma} = - \frac{RT_v}{\sigma}
\end{eqnarray}

lt;/span\textgreater{}

\begin{enumerate}
\def\labelenumi{\arabic{enumi}.}
\setcounter{enumi}{2}
\tightlist
\item
  equation of motion
\end{enumerate}

\begin{eqnarray}
  \frac{\partial \zeta}{\partial t}
     =   \frac{1}{a\cos\varphi}
            \frac{\partial A_v}{\partial \lambda}
          - \frac{1}{a\cos \varphi}
            \frac{\partial}{\partial \varphi} ( A_u \cos\varphi )
          - {\mathcal D}(\zeta)
\end{eqnarray}

\begin{verbatim}
 > <span id="Vorticity" label="Vorticity">\centric\centric\centric\centric\centric\centric\lopen}&lt ;/span>
\end{verbatim}

\begin{eqnarray}
  \frac{\partial D}{\partial t}
     =    \frac{1}{a\cos\varphi}
            \frac{\partial A_u}{\partial \lambda}
          + \frac{1}{a\cos\varphi}
            \frac{\partial }{\partial \varphi} ( A_v \cos\varphi )
          - \nabla^{2}_{\sigma}
           ( \Phi + R \bar{T} \pi + E )
          - {\mathcal D}(D)
\end{eqnarray}

\begin{verbatim}
 > <span id="divergence" label="divergence">\blaze[divergence]&lt ;/span>
\end{verbatim}

\begin{enumerate}
\def\labelenumi{\arabic{enumi}.}
\setcounter{enumi}{3}
\item
  thermodynamic equation

  \begin{quote}
  \blaze[Heat Power]\&lt ;/span\textgreater{}
  \end{quote}
\end{enumerate}

\begin{eqnarray}
  \frac{\partial T}{\partial t}
     =  - \frac{1}{a\cos\varphi}
               \frac{\partial uT'}{\partial \lambda}
          - \frac{1}{a}
               \frac{\partial }{\partial \varphi} ( vT' \cos\varphi )
          + T' D  \\
        - \dot{\sigma}
              \frac{\partial T }{\partial \sigma}
          + \kappa T \left( \frac{\partial \pi}{\partial t}
                            + \mathbf{v}_{H} \cdot \nabla_{\sigma} \pi
                            + \frac{ \dot{\sigma} }{ \sigma }
                     \right)
          + \frac{Q}{C_{p}}
          + \frac{Q_{diff}}{C_{p}}
          - {\mathcal D}(T)
\end{eqnarray}

\begin{enumerate}
\def\labelenumi{\arabic{enumi}.}
\setcounter{enumi}{4}
\item
  water vapor formula

  \begin{quote}
  \blazer\blazer.com \textgreater{} \textless span id=``Water Vapor''
  label="Water Vapor\cleaner\cleaner.com lt;/span\textgreater{}
  \end{quote}
\end{enumerate}

\begin{eqnarray}
  \frac{\partial q}{\partial t}
   =  - \frac{1}{a\cos\varphi}
               \frac{\partial uq}{\partial \lambda}
          - \frac{1}{a\cos\varphi}
               \frac{\partial }{\partial \varphi} (vq \cos\varphi)
          + q D  \\
        - \dot{\sigma} \frac{\partial q }{\partial \sigma}
          + S_{q}
          - {\mathcal D}(q)
\end{eqnarray}

Here,

\begin{eqnarray}
\theta  \equiv  T \left( p/p_{0} \right)^{-\kappa} \\
\kappa  \equiv  R/C_{p} \\
  \Phi  \equiv  gz \\
   \pi  \equiv  \ln p_{S} \\
%
 \dot{\sigma}  \equiv   \frac{d \sigma}{d t} \\
%
     T_v  \equiv  T ( 1+\epsilon_v q ) \\
     T  \equiv   \bar{T}(\sigma) + T^{\prime} \\
%
 \zeta  \equiv  \frac{1}{a \cos\varphi }
                    \frac{\partial v}{\partial \lambda}
             -    \frac{1}{a \cos\varphi }
                    \frac{\partial }{\partial \varphi}
                    ( u \cos\varphi ) \\
%
     D  \equiv  \frac{1}{a \cos\varphi }
                    \frac{\partial u}{\partial \lambda}
             +    \frac{1}{a \cos\varphi }
                    \frac{\partial }{\partial \varphi}
                    ( v \cos\varphi ) \\
%
    A_u  \equiv   ( \zeta + f ) v
             - \dot{\sigma} \frac{\partial u}{\partial \sigma}
             - \frac{RT^{\prime}}{a\cos\varphi}
                  \frac{\partial \pi}{\partial \lambda}
             + {\mathcal F}_x \\
%
    A_v  \equiv  - ( \zeta + f ) u
             - \dot{\sigma} \frac{\partial v}{\partial \sigma}
             - \frac{RT^{\prime}}{a}
                  \frac{\partial \pi}{\partial \varphi}
             + {\mathcal F}_y \\
%
     E  \equiv   \frac{u^{2}+v^{2}}{2} \\
%
 \mathbf{v}_{H} \cdot \nabla
        \equiv  \frac{u}{a \cos \varphi}
         \left( \frac{\partial }{\partial \lambda} \right)_{\sigma}
     + \frac{v}{a}
         \left( \frac{\partial }{\partial \varphi} \right)_{\sigma}
            \\
  \nabla^{2}_{\sigma}  
        \equiv  
               \frac{1}{a^{2}\cos^2\varphi}
                 \frac{\partial^{2} }{\partial \lambda^{2}}
             + \frac{1}{a^{2}\cos\varphi}
                 \frac{\partial }{\partial \varphi}
                 \left[ \cos\varphi
                       \frac{\partial }{\partial \varphi} \right]  .
\end{eqnarray}

\begin{quote}
\protect\hypertarget{Vorticityux20definition}{}{Drumming{[}Vorticity
definition }.
\end{quote}

\begin{quote}
\protect\hypertarget{divergentux20definitions}{}{\blaze{[}divergent
definitions }.
\end{quote}

\begin{quote}
\blaze[Section B\blaze]\&lt ;/span\textgreater{}
\end{quote}

\begin{quote}
\textbackslash\textbackslash.3F.3F.3F.3F.3F.3F.3F.3F.3F.3F.3F.3F.3F.3F.3F.3F.3F.3F.3F.3F.3F.3F.3F.3F.3F..3F.3F..3F.3F..3F.3F..3F.3F.\textbackslash.3F.3F..3F..3F..3F.3F.\textbackslash.3F.3F.\textbackslash.3F.3F.3F..3F.3F..3F..3F.3F.3F..3F..3F.3F..3F..3F..3F.3F..3F..3F.3F..
;/span\textgreater{}
\end{quote}

\begin{quote}
\textbackslash{}\lopen[E section]\&lt ;/span\textgreater{}
\end{quote}

\({\mathcal D}(\zeta), {\mathcal D}(D), {\mathcal D}(T), {\mathcal D}(q)\)
is the horizontal diffusion term,
\({\mathcal F}_\lambda, {\mathcal F}_\varphi\) are forces due to
small-scale kinetic processes (treated as `physical processes'), The
\(Q\) is a process of `physical processes' such as radiation,
condensation, and small-scale kinetic processes Heating and temperature
changes, The \(S_q\) is a system of `physical processes' such as
condensation and small-scale kinetic processes It is a water vapor
source term. Also, \(Q_{diff}\) is a frictional heat,

\begin{eqnarray}
  Q_{diff}
 = - \mathbf{v} \cdot  ( \frac{\partial \mathbf{v}}{\partial t} )_{diff} .
\end{eqnarray}

\$( \frac{\partial \mathbf{v}}{\partial t} )\_\{diff\} \$ is , It is a
time-varying term of \(u,v\) due to horizontal and vertical diffusion.

\hypertarget{boundary-conditions.}{%
\subsubsection{Boundary Conditions.}\label{boundary-conditions.}}

The boundary conditions for lead-direct current are

\begin{eqnarray}
  \dot{\sigma} = 0  \ \ \ at \ \ \sigma = 0 , \ 1 .
\end{eqnarray}

It is. The time-varying surface pressure equation and Diagnostic Formula
for determining the vertical velocity (\(\dot{\sigma}\)) in the
\(\sigma\) system

\begin{eqnarray}
   \frac{\partial \pi}{\partial t}
   = - \int_{0}^{1} \mathbf{v}_{H} \cdot \nabla_{\sigma} \pi d \sigma
     - \int_{0}^{1} D  d \sigma ,
\end{eqnarray}

\begin{quote}
\protect\hypertarget{barometricux20pressureux20trend}{}{\blindness }.
\end{quote}

\begin{eqnarray}
   \dot{\sigma}
   = - \sigma
     \frac{\partial \pi}{\partial t}
     - \int_{0}^{\sigma} D d \sigma
     - \int_{0}^{\sigma}
         \mathbf{v}_{H} \cdot \nabla_{\sigma} \pi d \sigma ,
\end{eqnarray}

\begin{quote}
\protect\hypertarget{verticalux20speed}{}{\blazing speed }.
\end{quote}

is led.

	% 力学:鉛直離散化
	
\subsection{鉛直離散化}

Arakawa and Suarez(1983) に従って, 
基礎方程式を鉛直方向に差分によって離散化する.
このスキームは次のような特徴をもつ.
%
\begin{itemize}
\item 全領域積分した質量を保存
\item 全領域積分したエネルギーを保存
\item 全領域積分の角運動量を保存
\item 全質量積分した温位を保存
\item 静水圧の式がlocalにきまる(下層の高度が上層の温度に依存しない)
\item 水平方向に一定の, ある特定の温度分布について,
      静水圧の式が正確になり, 気圧傾度力が0になる.
\item 等温位大気はいつまでも等温位に留まる
\end{itemize}      

\subsubsection{レベルのとりかた}

下の層から上へと層の番号をつける.
$\zeta,D,T,q$の物理量は整数レベル(layer)で定義されるとする.
一方, $\dot{\sigma}$ は半整数レベル(level)で定義される.
%
まず, 半整数レベルでの$\sigma$の値
$\sigma_{k-1/2}, (k=1,2,\ldots K)$
を定義する.
%
ただし, レベル$\frac{1}{2}$は下端($\sigma=1$),
レベル$K+\frac{1}{2}$は上端($\sigma=0$)とする.

整数レベルの$\sigma$の値
$\sigma_k, (k=1,2,\ldots K)$
は次の式から求められる.
%
\begin{equation}
 \label{しぐま定義}
 \sigma_k = \left\{ \frac{1}{1+\kappa}
                     \left( \frac{  \sigma^{\kappa +1}_{k-1/2}
                                  - \sigma^{\kappa +1}_{k+1/2}      }
                                  { \sigma_{k-1/2} - \sigma_{k+1/2} }
                     \right)
              \right\}^{1/\kappa}
\end{equation}
%
さらに,
\begin{equation}
  \label{しぐま厚さ}
  \Delta \sigma_k \equiv \sigma_{k-1/2} - \sigma_{k+1/2}
\end{equation}
を定義しておく.

\subsubsection{鉛直離散化表現}

各方程式の離散化表現は次のようになる.

\begin{enumerate}
\item 連続の式, 鉛直速度

\begin{equation}
  \frac{\partial \pi}{\partial t}
 = - \sum_{k=1}^{K} ( D_k + \Dvect{v}_k \cdot \nabla \pi ) 
       \Delta  \sigma_k
\end{equation}
%
\begin{equation}
  \dot{\sigma}_{k-1/2}
 = - \sigma_{k-1/2} \frac{\partial \pi}{\partial t}
   - \sum_{l=k}^{K} ( D_l + \Dvect{v}_l \cdot \nabla \pi )          
       \Delta  \sigma_l
\end{equation}
%
\begin{equation}
  \dot{\sigma}_{1/2} = \dot{\sigma}_{K+1/2} = 0
\end{equation}
   
\item 静水圧の式

\begin{eqnarray}
 \Phi_{1} & = & \Phi_{s} + C_{p} ( \sigma_{1}^{-\kappa} - 1  ) T_{v,1} \\
          & = & \Phi_{s} + C_{p} \alpha_{1} T_{v,1} \nonumber
\end{eqnarray}
%
\begin{eqnarray}
 \Phi_k - \Phi_{k-1} 
  & = & C_{p}
   \left[ \left( \frac{ \sigma_{k-1/2} }{ \sigma_k } \right)^{\kappa}
          - 1 \right] T_{v,k} 
       + C_{p}
   \left[ 1- 
         \left( \frac{ \sigma_{k-1/2} }{ \sigma_{k-1} } \right)^{\kappa}
              \right] T_{v,k-1} \\
  & =  &  C_{p} \alpha_k T_{v,k} + C_{p} \beta_{k-1} T_{v,k-1}
               \nonumber
\end{eqnarray}
%
ここで,
%
\begin{eqnarray}
 \label{静水圧係数}
 \alpha_k  & = & \left( \frac{ \sigma_{k-1/2} }
                               { \sigma_k } \right)^{\kappa} -1 \\
 \beta_k   & = & 1- \left( \frac{ \sigma_{k+1/2} }
                               { \sigma_k } \right)^{\kappa} .
\end{eqnarray}

\item 運動方程式

\begin{equation}
  \label{渦度結局}
  \frac{\partial \zeta_k}{\partial t} 
        =   \frac{1}{a\cos\varphi} 
            \frac{\partial (A_v)_k}{\partial \lambda}
          - \frac{1}{a\cos\varphi} 
            \frac{\partial }{\partial \varphi} (A_u \cos\varphi)_k
          - {\cal D}(\zeta_k) 
\end{equation}
%
\begin{equation}
  \frac{\partial D}{\partial t} 
        =   \frac{1}{a\cos\varphi} 
            \frac{\partial (A_u)_k}{\partial \lambda}
          + \frac{1}{a\cos\varphi} 
            \frac{\partial }{\partial \varphi} (A_v \cos\varphi)_k
          - \nabla^{2}_{\sigma}
           ( \Phi_k + C_{p} \hat{\kappa}_k \bar{T}_k \pi 
             + (\mbox{\sl KE})_k )
          - {\cal D}(D_k) 
\end{equation}
%
ここで,
%
\begin{eqnarray}
  (A_u)_k
   & = & ( \zeta_k + f ) v_k 
             - \frac{1}{2 \Delta \sigma_k} 
             [   \dot{\sigma}_{k-1/2} ( u_{k-1} - u_k   )
               + \dot{\sigma}_{k+1/2} ( u_k   - u_{k+1} ) ]
           \nonumber \\
   &   &     - \frac{C_{p} \hat{\kappa}_k T_{v,k}'}{a\cos\varphi} 
                  \frac{\partial \pi}{\partial \lambda} 
             + {\cal F}_x
\end{eqnarray}
%
\begin{eqnarray}
  (A_v)_k
   & = & - ( \zeta_k + f ) u_k 
             - \frac{1}{2 \Delta \sigma_k} 
             [   \dot{\sigma}_{k-1/2} ( v_{k-1} - v_k   )
               + \dot{\sigma}_{k+1/2} ( v_k   - v_{k+1} ) ]
           \nonumber \\
   &   &     - \frac{C_{p} \hat{\kappa}_k T_{v,k}'}{a} 
               \frac{\partial \pi}{\partial \varphi} 
             + {\cal F}_y
\end{eqnarray}
%
\begin{eqnarray}
   \label{はっとかっぱ}
   \hat{\kappa}_k 
 &   =  &     \frac{  \sigma_{k-1/2}(   \sigma^{\kappa}_{k-1/2} 
                                    - \sigma^{\kappa}_k      ) 
                  + \sigma_{k+1/2}(   \sigma^{\kappa}_k 
                                    - \sigma^{\kappa}_{k+1/2}  ) }
                 { \sigma^{\kappa}_k
                     ( \sigma_{k-1/2} - \sigma_{k+1/2} )         } 
         \nonumber   \\
 & = & \frac{ \sigma_{k-1/2} \alpha_k + \sigma_{k+1/2} \beta_k }
            { \Delta \sigma_k                                  } 
\end{eqnarray}

\begin{equation}
T'_{v,k} = T_{v,k} - \bar{T}_k
\end{equation}

\item 熱力学の式

\begin{eqnarray}
  \frac{\partial T_k}{\partial t}
    & = & - \frac{1}{a\cos\varphi}
               \frac{\partial u_k T'_k}{\partial \lambda}
          - \frac{1}{a\cos\varphi}
               \frac{\partial }{\partial \varphi} (v_k T'_k \cos\varphi)
          + H_k \nonumber \\
    &   & + \frac{Q_k}{C_{p}}
          + \frac{(Q_{diff})_k}{C_p} 
          - {\cal D}(T_k) \nonumber \\
\end{eqnarray}
%
ここで,
\begin{eqnarray}
   H_k 
    & \equiv & T_k' D_k
              - \frac{1}{\Delta \sigma_k} 
             [   \dot{\sigma}_{k-1/2} ( \hat{T}_{k-1/2} - T_k   )
               + \dot{\sigma}_{k+1/2} ( T_k   - \hat{T}_{k+1/2} ) ]
               \nonumber \\
    &   & + \left\{ \alpha_k
                    \left[ \sigma_{k-1/2} \Dvect{v}_k \cdot \nabla \pi
                          - \sum_{l=k}^{K} 
                           ( D_l + \Dvect{v}_l \cdot \nabla \pi )
                            \Delta  \sigma_l
                    \right]
             \right.   \nonumber \\
    &   &   + \left. \beta_k
                     \left[ \sigma_{k+1/2} \Dvect{v}_k \cdot \nabla \pi
                          - \sum_{l=k+1}^{K} 
                           ( D_l + \Dvect{v}_l \cdot \nabla \pi )
                            \Delta  \sigma_l
                    \right]
              \right\} 
              \frac{1}{\Delta \sigma_k} T_{v,k}  \nonumber \\
%
    & = & T_k' D_k 
          - \frac{1}{\Delta \sigma_k} 
             [   \dot{\sigma}_{k-1/2} ( \hat{T}_{k-1/2} - T_k   )
               + \dot{\sigma}_{k+1/2} ( T_k   - \hat{T}_{k+1/2} ) ]
               \nonumber \\
    &   & + \hat{\kappa}_k \Dvect{v}_k \cdot \nabla \pi T_{v,k} 
               \nonumber \\
    &   & - \alpha_k \sum_{l=k}^{K} 
                           ( D_l + \Dvect{v}_l \cdot \nabla \pi )
                            \Delta  \sigma_l 
                            \frac{T_{v,k}}{\Delta \sigma_k} 
               \nonumber \\
    &   & - \beta_k \sum_{l=k+1}^{K} 
                           ( D_l + \Dvect{v}_l \cdot \nabla \pi )
                            \Delta  \sigma_l 
                            \frac{T_{v,k}}{\Delta \sigma_k} 
\end{eqnarray}
%
\begin{eqnarray} 
  \hat{T}_{k-1/2}
 &  = & \frac{ \left[ \left( \frac{ \sigma_{k-1/2} }
                               { \sigma_k } \right)^{\kappa}
                  - 1 \right] \sigma_{k-1}^{\kappa} T_k 
          + \left[ 1- 
                   \left( \frac{ \sigma_{k-1/2} }
                               { \sigma_{k-1} } \right)^{\kappa}
                      \right] \sigma_k^{\kappa} T_{k-1}         }
          { \sigma_{k-1}^{\kappa} - \sigma_k^{\kappa}           } \\
 &  = & a_k T_k + b_{k-1} T_{k-1}
\end{eqnarray}
%
\begin{eqnarray}
  \label{温度補間係数}
  a_k & = & \alpha_k 
              \left[ 1- \left( \frac{ \sigma_k }{ \sigma_{k-1} }
                        \right)^{\kappa} \right]^{-1}   \\
  b_k & = & \beta_k 
              \left[ \left( \frac{ \sigma_k }{ \sigma_{k+1} } 
                     \right)^{\kappa} - 1 \right]^{-1} .  
\end{eqnarray}

\item 水蒸気の式

\begin{equation}
  \label{q結局}
  \frac{\partial q_k}{\partial t}
      =   - \frac{1}{a\cos\varphi} 
               \frac{\partial u_k q_k}{\partial \lambda}
          - \frac{1}{a\cos\varphi}
               \frac{\partial }{\partial \varphi} ( v_k q_k\cos\varphi)
          + R_k 
          + S_{q,k}
          - {\cal D}(q_k) 
\end{equation}
%
\begin{equation}
R_k  =  q_k D_k 
       - \frac{1}{2 \Delta \sigma_k} 
             [   \dot{\sigma}_{k-1/2} ( q_{k-1} - q_k   )
               + \dot{\sigma}_{k+1/2} ( q_k   - q_{k+1} ) ]
\end{equation}

\end{enumerate}

	% 力学:水平離散化
	\subsection{水平離散化}

水平方向の離散化は
スペクトル変換法を用いる(Bourke, 1988).
経度, 緯度に関する微分の項は直交関数展開によって評価し,
一方, 非線型項は格子点上で計算する.

\subsubsection{スペクトル展開}

展開関数系としては球面上の Laplacianの固有関数系である
球面調和関数 TERM00000,TERM00000 を用いる.
ただし, TERM00001 である.
TERM00002 は次のような方程式を満たし,
%
\begin{quote}
EQ=00000.
\end{quote}
%
Legendre 陪関数 TERM00003 を用いて次のように書き表される.
%
\begin{quote}
EQ=00001.
\end{quote}
%
ただし, TERM00004 である.

球面調和関数による展開は \footnotemark ,
\begin{quote}
EQ=00002.
\end{quote}
と書くと,
%
\footnotetext{ここでは三角形切断を用いている.
              平行四辺形切断の場合には, 
              TERM00005
              と置き換えること.}
%
\begin{quote}
EQ=00003.
\label{球面展開}
\end{quote}
%
その逆変換は,
\begin{quote}
EQ=00017.\\
\label{展開係数}
EQ=00017.
\end{quote}
%
のように表される.
%
積分を和に置き換えて評価する際には,
TERM00006 積分については Gauss の台形公式を,
TERM00007 積分については Gauss-Legendre 積分公式を用いる.
TERM00008 は Gauss 緯度, TERM00009 は Gauss 荷重である.
また, TERM00010 は等間隔の格子である.

スペクトル展開を用いれば,
微分を含む項の格子点値は次のように求められる.
%
\begin{quote}
EQ=00004.
\label{気圧x}
\end{quote}
%
\begin{quote}
EQ=00005.
\label{気圧y}
\end{quote}
%
さらに,
TERM00011, TERM00012 のスペクトル成分から, 
TERM00013,TERM00013 の格子点値が以下のように得られる.
%
\begin{quote}
EQ=00006.
\label{Uを求める}
\end{quote}
%
\begin{quote}
EQ=00007.
\label{Vを求める}
\end{quote}

方程式の移流項に現れる微分は,
次のように求められる.
%
\begin{quote}
\label{A積分}
\nonumber
EQ=00018.\\
\nonumber
EQ=00018.\\
EQ=00018.
\end{quote}
%
\begin{quote}
\label{B積分}
\nonumber
EQ=00019.\\
\nonumber
EQ=00019.\\
EQ=00019.
\end{quote}
%
さらに,
\begin{quote}
EQ=00008.
\end{quote}
を TERM00014 の項の評価のために用いる.

\subsubsection{水平拡散項}

水平拡散項は, 次のように TERM00015 の形で入れる.
%
\begin{quote}
EQ=00009.
\label{水平拡散}
\end{quote}
%
\begin{quote}
EQ=00010.
\end{quote}
%
\begin{quote}
EQ=00011.
\end{quote}
%
\begin{quote}
EQ=00012.
\end{quote}
%
この水平拡散項は計算の安定化のための意味合いが強い.
小さなスケールに選択的な水平拡散を表すため,
TERM00016 としては, 4 TERM00017 16を用いる.
ここで, 渦度および発散の拡散についている余分な項は,
TERM00018 の剛体回転の項が減衰しないことを表したものである.

\subsubsection{方程式のスペクトル表現}


\begin{enumerate}
\item 連続の式

\begin{quote}
\nonumber
EQ=00020.\\
EQ=00020.
\end{quote}
%
ここで,
\begin{quote}
EQ=00013.
\end{quote}

\item 運動方程式

\begin{quote}
\nonumber
EQ=00021.\\
\nonumber
EQ=00021.\\
EQ=00021.
\end{quote}
%
\begin{quote}
\nonumber
EQ=00022.\\
\nonumber
EQ=00022.\\
\nonumber
EQ=00022.\\
EQ=00022.
\end{quote}
%
ただし,
%
\begin{quote}
EQ=00014.
\end{quote}

\item 熱力学の式

\begin{quote}
\nonumber
EQ=00023.\\
\nonumber
EQ=00023.\\
\nonumber
EQ=00023.\\
EQ=00023.
\end{quote}
%
ただし,
%
\begin{quote}
EQ=00015.
\end{quote}



\item 水蒸気の式

\begin{quote}
\nonumber
EQ=00024.\\
\nonumber
EQ=00024.\\
\nonumber
EQ=00024.\\
EQ=00024.
\end{quote}

ただし,
%
\begin{quote}
EQ=00016.
\end{quote}



\end{enumerate}


	% 力学:時間離散化
	\subsection{時間積分}

時間差分スキームは基本的に leap frog である.
ただし, 拡散項および物理過程の項は後方差分もしくは前方差分とする.
計算モードを抑えるために時間フィルター(Asselin, 1972)を用いる.
さらに TERM00254 を大きくとるために,
重力波の項に semi-implicit の手法を適用する(Bourke, 1988).

\subsubsection{leap frog による時間積分と時間フィルター}

移流項等の時間積分スキームとして leap frog を用いる.
水平拡散項には TERM00255 の後方差分を使用する.
また, 拡散項の疑似 TERM00256 面補正と水平拡散による摩擦熱の項とは
補正として扱い, TERM00257 の前方差分となる.
物理過程の項(TERM00258,TERM00258)は,
やはり TERM00259 の前方差分を使用する.
(ただし, 鉛直拡散の時間変化項の計算に関しては後方差分的な取扱いをする.
詳しくは物理過程の章を参照のこと.)

各予報変数を代表して TERM00260 と表すと,
%
\begin{verbatim}
EQ=00064.
\end{verbatim}
%
TERM00261 は移流項等,
TERM00262 は水平拡散項である.

TERM00263 には, 
疑似等 TERM00264 面拡散と水平拡散による摩擦熱(TERM00265)の補正
および物理過程(TERM00266)の項が加えられ,
TERM00267 となる.
%
\begin{verbatim}
EQ=00065.
\end{verbatim}

leap frog における計算モードの除去のために 
Asselin(1972) の時間フィルターを毎ステップ適用する.
すなわち, 
%
\begin{verbatim}
EQ=00066.
\end{verbatim}
%
と TERM00268 を求める.
TERM00269 としては標準的に 0.05 を使用する. 

\subsubsection{semi-implicit 時間積分}

力学の計算では, 基本的に leap frog を用いるが,
一部の項を implicit 扱いで計算する.
ここで, implicit は, 台形 implicit を考える.
ベクトル量 TERM00270 に関して,
TERM00271 での値を TERM00272,
TERM00273 での値を TERM00274,
TERM00275 での値を TERM00276 と書くと,
台形 implicit とは,
TERM00277 を
用いて評価した時間変化項をを用いて解くことにあたる.
%
今, {\boldmath q} の時間変化項として,
leap forg 法で扱う項 A と 台形 implicit 法で扱う項 B とに分けて考える.
A は {\boldmath q} に対して非線形であるが, B は線形であるとする.
すなわち,
%
\begin{verbatim}
EQ=00067.
\end{verbatim}
%
ただし, TERM00278 は正方行列である. すると,
TERM00279
と書けば,
\begin{verbatim}
EQ=00068.
\end{verbatim}
%
これは, 行列演算で簡単に解くことができる.

\subsubsection{semi-implicit 時間積分の適用}

そこで, この方法を適用し, 線形重力波の項を implicit 扱いする.
これにより, 時間ステップ TERM00280 を小さくとることができる.

方程式系において, TERM00281 であるような静止場を基本場とする
線型重力波項とそれ以外の項(添字 TERM00282 を付ける)に分離する.
鉛直方向のベクトル表現
TERM00283, TERM00284 を用いて,
%
\begin{verbatim}
EQ=00069.
\end{verbatim}
%
\begin{verbatim}
EQ=00070.
\end{verbatim}
%
\begin{verbatim}
EQ=00102.
\end{verbatim}

ここで, 非重力波項は,
%
\begin{verbatim}
EQ=00071.
\end{verbatim}
%
\begin{verbatim}
EQ=00072.
\end{verbatim}
%
\begin{verbatim}
EQ=00073.
\end{verbatim}
%
\begin{verbatim}
EQ=00074.
\end{verbatim}
%
\begin{verbatim}
EQ=00103.
EQ=00103.
EQ=00103.
EQ=00103.
EQ=00103.
EQ=00103.
EQ=00103.
\end{verbatim}
\begin{verbatim}
EQ=00075.
\end{verbatim}

ここで, 重力波項のベクトルおよび行列(下線で表示)は,
%
\begin{verbatim}
EQ=00076.
\end{verbatim}
%
\begin{verbatim}
EQ=00077.
\end{verbatim}
%
\begin{verbatim}
EQ=00078.
\end{verbatim}
%
\begin{verbatim}
EQ=00079.
\end{verbatim}
%
\begin{verbatim}
EQ=00080.
\end{verbatim}
%
\begin{verbatim}
EQ=00081.
\end{verbatim}
%
\begin{verbatim}
EQ=00082.
\end{verbatim}
%
ここで, 例えば TERM00302 は,
TERM00303 が成り立つとき 1, そうでないとき 0 となる関数である.

次のような表現を使用して,
%
\begin{verbatim}
EQ=00083.
\end{verbatim}
%
\begin{verbatim}
EQ=00104.
EQ=00104.
\end{verbatim}
%
方程式系に semi-implicit 法を適用すると,
%
\begin{verbatim}
EQ=00084.
\end{verbatim}
%
\begin{verbatim}
EQ=00085.
\end{verbatim}
%
\begin{verbatim}
EQ=00086.
\end{verbatim}


すると, 
%
\begin{verbatim}
EQ=00105.
EQ=00105.
EQ=00105.
EQ=00105.
\end{verbatim}

球面調和関数展開を用いているので,
\[
EQ=00106.
\]
であり上式を TERM00327 について解くことができる.
%
その後,
%
\begin{verbatim}
EQ=00087.
\end{verbatim}
%
および, (109), (111)
により TERM00329 における値 TERM00330
が求められる.

\subsubsection{時間スキームの特性と時間ステップの見積り}

移流型方程式 
\begin{verbatim}
EQ=00088.
\end{verbatim}
において,  leap frog で離散化した場合の安定性を考える.
今, 
\begin{verbatim}
EQ=00089.
\end{verbatim}
と置き差分化すると, 上式は,
\begin{verbatim}
EQ=00090.
\end{verbatim}
となる.
ここで,
\begin{verbatim}
EQ=00091.
\end{verbatim}
とすると,
\begin{verbatim}
EQ=00092.
\end{verbatim}
この解は TERM00331 とおいて,
\begin{verbatim}
EQ=00093.
\end{verbatim}

この絶対値は
\begin{verbatim}
EQ=00094.
\end{verbatim}
であり, TERM00332 の場合には, TERM00333 となり,
時間とともに絶対値が指数的に大きくなる解となる.
これは計算が不安定であることを示す.

一方, TERM00334 の場合は TERM00335 であるため,
計算は中立である.
ただし, TERM00336 の値として2つの解があり,
そのうち一方は, TERM00337 としたときに
TERM00338 であるが, 
他方は TERM00339 となる.
これは, 時間的に大きく振動する解を示す.
このモードは計算モードと呼ばれ, 
leap frog 法の問題点の一つである.
このモードは時間フィルターを施すことによって
減衰させることができる.

TERM00340 の条件は,
水平離散化の格子間隔 TERM00341 が与えられている場合には
それによって TERM00342 の最大値が
\begin{verbatim}
EQ=00095.
\end{verbatim}
となることより,
\begin{verbatim}
EQ=00096.
\end{verbatim}
となる.
スペクトルモデルの場合は, 最大波数 TERM00343 により,
地球半径を TERM00344 として,
\begin{verbatim}
EQ=00097.
\end{verbatim}
これが安定性の条件である.

積分の安定性を保証するには,
TERM00345 としては, もっとも速い移流・伝播の速度をとり,
それによって決まる TERM00346 よりも小さな時間ステップを用いればよい.
semi-implicit を用いない場合には, 重力波の伝播速度
(TERM00347) が安定性の基準となるが,
semi-implicit を用いた場合には, 通常, 東西風による移流が
制限要因となる.
従って, TERM00348 は TERM00349 を東西風の最大値として,
\begin{verbatim}
EQ=00098.
\end{verbatim}
を満たすようにとる.
実際にはこれに安全のための係数をかけたものを用いる.

\subsubsection{時間積分の開始における取扱い}

AGCM で計算されたものではない, 
適当な初期値から始める場合には, モデルに整合的な
TERM00350 および TERM00351 の2つの時刻の物理量を与えることはできない.
しかし, TERM00352 の値として不整合な値を与えると,
大きな計算モードが発生する.

そこで, まず TERM00353 として, TERM00354 の時間ステップで
\begin{verbatim}
EQ=00099.
\end{verbatim}
を求め, さらに, TERM00355 の時間ステップで,
\begin{verbatim}
EQ=00100.
\end{verbatim}
そして, 本来の時間ステップで,
\begin{verbatim}
EQ=00101.
\end{verbatim}
として, 以後普通に leap frog で計算を行なうようにすると,
計算モードの発生が抑えられる.

	% トレーサー
	\include{d-tracer}
	% カプラー
	\include{AO-couple}
	\hypertarget{configuration-of-model-grid}{%
\subsection{Configuration of model
grid}\label{configuration-of-model-grid}}

The atmospheric and oceanic models of MIROC are independent and run on
different computational nodes. The node where the atmospheric model is
executed is called the atmospheric node, and the node where the ocean
model is executed is called the ocean node. In the atmospheric node, the
land surface model, sea surface model, and river model are executed as
sub-models. Information exchange between the atmosphere and the oceans
is performed through the sea surface model in the atmosphere node.
Information such as sea surface temperature and sea ice concentration of
the ocean model in the ocean node is converted to the grid of the sea
surface model so that it can be treated as a boundary condition of the
model in the atmosphere node. On the other hand, the heat, freshwater,
and momentum fluxes calculated on the grid of the sea surface model in
the atmospheric node are converted to the grid of the ocean model and
sent to the ocean node. These series of data communication and
conversion are done by the exchanger. The flux coupler stores data such
as boundary conditions, heat and freshwater fluxes calculated by the sea
surface model and land surface model, and distributes them to each model
as needed. In general, the flux coupler also includes the function of
the exchanger, but this document describes it separately.

\hypertarget{horizontal-grid-of-model}{%
\subsection{Horizontal grid of model}\label{horizontal-grid-of-model}}

The horizontal grid of MIROC is defined as the atmospheric grid, the
land grid, the river grid, and the sea surface grid for each model in
the atmospheric node. The sea surface grid in the atmospheric node is
different from the horizontal grid of the ocean model in the ocean node.
The land surface grid and the sea surface grid are the horizontal grid
of the atmospheric model divided equally into north-south and east-west
directions. The number of divisions can be set arbitrarily for each
grid. However, the number of divisions for the sea surface grid must be
divisible by the number of divisions for the land surface model. The
river grid can be the same as the atmospheric grid or an equal
latitude/longitude interval grid. The horizontal grid of the ocean model
uses horizontal general curvilinear Cartesian coordinates, so it is not
necessary to use the same coordinate system as the atmospheric model.
The exchange of data between the atmospheric model and the ocean model
is performed by using an exchanger, which is prepared in advance with
information on the location, number, area, and vector rotation of the
ocean grid that overlaps with the sea surface grid of the atmospheric
model.

\hypertarget{definition-of-land-sea-distribution}{%
\subsection{Definition of land-sea
distribution}\label{definition-of-land-sea-distribution}}

The land-sea distribution in MIROC is prioritized by the land-sea
distribution defined by the ocean model. While one grid in the ocean
model is defined by land or sea only, the land and ocean grids in the
atmospheric model are determined in proportion to the land and sea to be
consistent with the ocean model's land-sea distribution.

\(SA\) : area of the atmospheric grid, \(SL _ {ij}\) : area of the land
grid, \(SO _ {ij}\) : area of the sea surface grid, \(FLND^{atm}\),
\(FLND^{land} _ {ij}\), \(FLND^{oc} _ {ij}\) : percentage of land
surface is occupied by each grid. Then, following equation is satisfied.

\begin{eqnarray} SA*FLND^{atm} = \sum _ {j=1}^{jldiv}\sum _ {i=1}^{ildiv}(SL _ {ij}*FLND^{land} _ {ij}) = \sum _ {j=1}^{jodiv}\sum _ {i=1}^{iodiv}(SO _ {ij}*FLND^{oc } _{ij}) \end{eqnarray}

where, (ildiv,jldiv) is the number of east-west and north-south
divisions of the land surface grid, and (iodiv,jodiv) is the number of
east-west and north-south divisions of the sea surface grid. In the land
surface grid, if even a small amount of land is defined to exist,
boundary values such as land cover are required.

	% 力学:拡散項等
	% \include{d-diff} % 元々存在しない?
	% 力学:まとめ
	
\subsection{力学部分のまとめ}

ここでは, これまでの記述と重複するが,
力学過程部で行なわれる計算を列挙する.

\subsubsection{力学部分の計算の概要}

力学過程は, 以下のような順序で計算が行なわれる.

\begin{enumerate}
\item 水平風の渦度・発散への変換   \texttt{MODULE:[UV2VDG(dvect)]}
\item 仮温度の計算              \texttt{MODULE:[VIRTMD(dvtmp)]}
\item 気圧傾度項の計算           \texttt{MODULE:[HGRAD(dvect)]}
\item 鉛直流の診断的計算         \texttt{MODULE:[GRDDYN/PSDOT(dgdyn)]}
\item 移流による時間変化項 \texttt{MODULE:[GRDDYN(dgdyn)]}
\item 予報変数のスペクトルへの変換 \texttt{MODULE:[GD2WD(dg2wd)]}
\item 時間変化項のスペクトルへの変換 \texttt{MODULE:[TENG2W(dg2wd)]}
\item スペクトル値時間積分 \texttt{MODULE:[TINTGR(dintg)]}
\item 予報変数の格子点値への変換 \texttt{MODULE:[GENGD(dgeng)]}
\item 疑似等 TERM00000 面拡散補正   \texttt{MODULE:[CORDIF(ddifc)]}
\item 拡散による摩擦熱の考慮    \texttt{MODULE:[CORFRC(ddifc)]}
\item 質量の保存の補正          \texttt{MODULE:[MASFIX(dmfix)]}
\item (物理過程)             \texttt{MODULE:[PHYSCS(padmn)]}
\item (時間フィルター)        \texttt{MODULE:[TFILT(aadvn)]}
\end{enumerate}

\subsubsection{水平風の渦度・発散への変換}

水平風の格子点値 TERM00001,TERM00001 
から渦度・発散の格子点値 TERM00002,TERM00002 を求める.
まず, 渦度・発散のスペクトル
TERM00003,TERM00003 を求める,
\begin{verbatim}
EQ=00033.
EQ=00033.
\end{verbatim}
\begin{verbatim}
EQ=00034.
EQ=00034.
\end{verbatim}
それをさらに, 
\begin{verbatim}
EQ=00000.
\end{verbatim}
等を用いて格子点値に変換する.

\subsubsection{仮温度の計算}

仮温度 TERM00005 は, 
\begin{verbatim}
EQ=00035.
\end{verbatim}
ただし, TERM00006 であり, 
TERM00007 は水蒸気の気体定数
(461 TERM00008TERM00009)
TERM00010 は空気の気体定数
(287.04 TERM00011TERM00012)
である.

\subsubsection{気圧傾度項の計算}

気圧傾度項 TERM00013 は,
まず, TERM00014 を
\begin{verbatim}
EQ=00001.
\end{verbatim}
でスペクトル表現に直してから,
\begin{verbatim}
EQ=00002.
\end{verbatim}
\begin{verbatim}
EQ=00003.
\end{verbatim}

\subsubsection{鉛直流の診断的計算}

気圧変化項, および鉛直流,
\begin{verbatim}
EQ=00004.
\end{verbatim}
%
\begin{verbatim}
EQ=00005.
\end{verbatim}
%
ならびにその非重力波成分を計算する.
%
\begin{verbatim}
EQ=00006.
\end{verbatim}
%
\begin{verbatim}
EQ=00007.
\end{verbatim}

\subsubsection{移流による時間変化項}

運動量移流項:
\begin{verbatim}
EQ=00036.
EQ=00036.
\end{verbatim}
%
\begin{verbatim}
EQ=00037.
EQ=00037.
\end{verbatim}
\begin{verbatim}
EQ=00008.
\end{verbatim}

温度移流項:
\begin{verbatim}
EQ=00009.
\end{verbatim}
\begin{verbatim}
EQ=00010.
\end{verbatim}
%
\begin{verbatim}
EQ=00038.
EQ=00038.
EQ=00038.
EQ=00038.
EQ=00038.
EQ=00038.
\end{verbatim}

水蒸気移流項:
\begin{verbatim}
EQ=00011.
\end{verbatim}
\begin{verbatim}
EQ=00012.
\end{verbatim}
%
\begin{verbatim}
EQ=00013.
\end{verbatim}

\subsubsection{予報変数のスペクトルへの変換}

(1) および
(2) を用いて

TERM00024,TERM00024 を
渦度・発散のスペクトル表現
TERM00025,TERM00025 に変換する.
さらに,
温度 TERM00026, 比湿 TERM00027, 
TERM00028 を
\begin{verbatim}
EQ=00014.
\end{verbatim}
でスペクトル表現に変換する.

\subsubsection{時間変化項のスペクトルへの変換}

渦度の時間変化項
\begin{verbatim}
EQ=00039.
EQ=00039.
EQ=00039.
\end{verbatim}
%
発散の時間変化項の非重力波成分
\begin{verbatim}
EQ=00040.
EQ=00040.
EQ=00040.
EQ=00040.
\end{verbatim}
%
温度の時間変化項の非重力波成分
\begin{verbatim}
EQ=00041.
EQ=00041.
EQ=00041.
\end{verbatim}
%
水蒸気の時間変化項
\begin{verbatim}
EQ=00042.
EQ=00042.
EQ=00042.
\end{verbatim}

\subsubsection{スペクトル値時間積分}

行列形式の方程式
\begin{verbatim}
EQ=00043.
EQ=00043.
EQ=00043.
EQ=00043.
\end{verbatim}
%
を LU 分解を用いて解くことによって 
TERM00038 を求め,
%
\begin{verbatim}
EQ=00015.
\end{verbatim}
%
\begin{verbatim}
EQ=00016.
\end{verbatim}

%
によって
TERM00044,
TERM00045 
を求めて, TERM00046 におけるスペクトルの値を計算する.
\begin{verbatim}
EQ=00044.
EQ=00044.
EQ=00044.
EQ=00044.
EQ=00044.
\end{verbatim}

\subsubsection{予報変数の格子点値への変換}


渦度・発散のスペクトル値 TERM00047,TERM00047 から
水平風速の格子点値 TERM00048,TERM00048 を求める.
\begin{verbatim}
EQ=00017.
\end{verbatim}
%
\begin{verbatim}
EQ=00018.
\end{verbatim}

さらに,
\begin{verbatim}
EQ=00019.
\end{verbatim}
などによって, TERM00052,TERM00052 を求め,
\begin{verbatim}
EQ=00045.
\end{verbatim}
を計算する.

\subsubsection{疑似等 TERM00053 面拡散補正}

水平拡散は 等 TERM00054 面上で適用されるが,
山岳の傾斜の大きな領域では, 山を上る方向に水蒸気が輸送され,
山頂部での偽の降水をもたらすなどの問題を起こす.
それを緩和するために, 等 TERM00055 面の拡散に近くなるような
補正を TERM00056,TERM00056 について入れる.

\begin{verbatim}
EQ=00046.
EQ=00046.
EQ=00046.
\end{verbatim}
%
であるから,
\begin{verbatim}
EQ=00020.
\end{verbatim}
などととする.
TERM00057 は, TERM00058 のスペクトル値 TERM00059 に
拡散係数のスペクトル表現をかけたものを
格子の値に変換して用いる.

\subsubsection{拡散による摩擦熱の考慮}

拡散による摩擦熱は,
\begin{verbatim}
EQ=00021.
\end{verbatim}
と見積もられる.
したがって,
\begin{verbatim}
EQ=00022.
\end{verbatim}

\subsubsection{質量の保存の補正}

スペクトル法による取扱いは,
TERM00060 の全球積分は丸め誤差を除いて保存するが,
質量, すなわち TERM00061 の全球積分の保存は保証されない.
また, スペクトルの波数打ちきりにともない,
水蒸気の格子点値に負の値が出ることがある.
これらの事情から, 
乾燥大気の質量と水蒸気, 雲水の質量を保存させ,
さらに負の水蒸気量となる領域を除去するための補正を行なう.

まず, 力学の計算の最初に \texttt{MODULE:[FIXMAS]},
水蒸気, 雲水の各成分の全球積分値 TERM00062,TERM00062 を計算しておく.
\begin{verbatim}
EQ=00047.
EQ=00047.
\end{verbatim}
また, 計算の最初のステップで
乾燥質量 TERM00063 を計算し, 記憶する.
\begin{verbatim}
EQ=00048.
\end{verbatim}

力学計算の終りには \texttt{MODULE:[MASFIX]},
以下のような手順で補正を行なう.
\begin{enumerate}
\item まず, 負の水蒸気量となる格子点について,
      直下の格子点から水蒸気を分配して,
      負の水蒸気を除去する.
      TERM00064 であるとすると,
      \begin{verbatim}
EQ=00049.
EQ=00049.
\end{verbatim}
      ただし, これは TERM00065 となる場合にのみ行なう.

\item 次に上の手続きで除去されなかった格子点について値を 0 とする.

\item 全球積分値 TERM00066 を計算し,
      これが TERM00067 と一致するように,
      全球の水蒸気量に一定割合をかける.

      \begin{verbatim}
EQ=00023.
\end{verbatim}
      
\item 乾燥空気質量の補正を行なう.
      同様に TERM00068 を計算し,

      \begin{verbatim}
EQ=00024.
\end{verbatim}

\end{enumerate}

\subsubsection{水平拡散とレーリー摩擦}

水平拡散の係数をスペクトル表現すると,

\begin{verbatim}
EQ=00025.
\end{verbatim}
%
\begin{verbatim}
EQ=00026.
\end{verbatim}
%
\begin{verbatim}
EQ=00027.
\end{verbatim}

TERM00069 は レーリー摩擦係数である.
レーリー摩擦係数は
\begin{verbatim}
EQ=00028.
\end{verbatim}
のようなプロファイルで与える.
ただし,
\begin{verbatim}
EQ=00029.
\end{verbatim}
と近似する.
標準値は, TERM00070,
TERM00071 (TERM00072 : モデルの最上レベル),
TERM00073 m,
TERM00074 m である.

\subsubsection{時間フィルター}


leap frog における計算モードの除去のために 
Asselin(1972) の時間フィルターを毎ステップ適用する.
%
\begin{verbatim}
EQ=00030.
\end{verbatim}
%
と TERM00075 を求める.
次のステップの力学過程で用いる TERM00076 としては,
この TERM00077 を用いる.
TERM00078 としては標準的に 0.05 を使用する. 

実際には
まず, 予報変数の格子点値への変換 \texttt{MODULE:[GENGD]} の箇所で,
\begin{verbatim}
EQ=00031.
\end{verbatim}
を求めておき, 物理過程の処理が終わり
TERM00079 の値が確定した後で \texttt{MODULE:[TFILT]} で,
\begin{verbatim}
EQ=00032.
\end{verbatim}
とする.

	% 物理過程:イントロ
	%\hypertarget{physics}{%
\section{Physics}\label{physics}}

General introduction for physics
 % 渡部先生が書く
	% 物理過程:cumulus scheme
	\hypertarget{cumulus-scheme}{%
\subsection{Cumulus scheme}\label{cumulus-scheme}}

\hypertarget{outline-of-cumulus-scheme}{%
\subsubsection{Outline of cumulus
scheme}\label{outline-of-cumulus-scheme}}

The Chikira scheme (Chikira and Sugiyama 2010) has been adopted since
version 5 of MIROC. It represents updrafts, downdrafts, their
detrainment and compensating downward motion over the surrounding area
as well as microphysical processes associated with updrafts and
downdrafts.

The updraft is based on an entraining plume model, where the mass flux
increases upward due to lateral entrainment. The detrainment occurs only
at the cloud top which is defined as the neutral buoyancy level of the
updraft air parcel. The lateral entrainment rate is formulated in terms
of buoyancy and vertical velocity of the air parcel at each level
following Gregory (2001). The momentum transport is formulated following
Gregory et al.~(1997).

The cloud base mass fluxes are determined by the prognostic convective
kinetic energy closure proposed by Arakawa and Xu (1990) and Xu (1991,
1993), which was adopted in the prognostic Arakawa--Schubert scheme
(Randall and Pan 1993; Pan 1995; Randall et al.~1997; Pan and Randall
1998). The convective kinetic energy increases by buoancy and decreases
by dissipation.

The cloud types are spectrally represented according to the updraft
vertical velocity at the cloud base. Larger (smaller) vertical
velocities give smaller (larger) entrainment rates which result in
higher (lower) cloud tops. The cloud base is diagnosed as the lifting
condensation level of the air parcel at the lowest model layer.

The scheme has a simple downdraft model, where a part of the
precipitation caused by the updrafts evaporates and forms the cold air
which enters into the downdrafts. The detrainment of the downdraft mass
fluxes occurs at the neutral buoyancy level and near the surface.

The interaction of the updrafs and downdrafts with the surrounding
environment is formulated following Arakawa and Schubert (1974). The
areal fractions of the updrafts and downdrafts are assumed to be
sufficiently small and the grid-mean prognostic variables are supposed
to be the same as those over the environmental area, which are changed
by the detrainment of the updrafts and downdrafts, the compensating
subsidence and the evaporation and sublimation of the precipitation
associated with the updrafts.

The input variables to this scheme are temperature \(T\), specific
humidity \(q\), cloud liquid water \(q_l\), cloud ice \(q_i\), zonal
wind \(u\), meridional wind \(v\), all tracers including aerosols and
greenhouse gases, height \(z\), pressure \(p\), and cloud cover \(C\).
The scheme gives the tendencies of \(T\), \(q_v\), \(q_l\), \(q_i\),
\(u\), \(v\), \(C\) and all the tracers. The vertical profiles of the
rainfall and snowfall fluxes, cloud liquid water, cloud ice and cloud
fraction associated with the updrafts are also output as diagnostic
variables.

The procedure of the calculations is given as follows along with the
names of the subroutines.

\begin{enumerate}
\def\labelenumi{\arabic{enumi}.}
\tightlist
\item
  calculation of cloud base \texttt{CUMBAS}.
\item
  calculation of in-cloud properties \texttt{CUMUP}.
\item
  calculation of cloud base mass flux \texttt{CUMBMX}.
\item
  calculation of cloud mass flux, detrainment, and precipitation
  \texttt{CUMFLX}.
\item
  diagnosis of cloud water and cloud cover by cumulus \texttt{CUMCLD}.
\item
  calculation of tendencies by detrainment \texttt{CLDDET}.
\item
  calculation of freezing, melting, evaporation, sublimation, and
  downdraft mass flux \texttt{CUMDWN}.
\item
  calculation of tendencies by compensating subsidence \texttt{CLDSBH}.
\item
  calculation of cumulus momentum transport \texttt{CUMCMT}.
\item
  calculation of tracer updraft \texttt{CUMUPR}.
\item
  calculation of tracer downdraft \texttt{CUMDNR}.
\item
  calculation of tracer subsidence \texttt{CUMSBR}.
\item
  fixing tracer mass \texttt{CUMFXR} .
\end{enumerate}

\hypertarget{interaction-between-cumulus-ensemble-and-large-scale-environment}{%
\subsubsection{Interaction between cumulus ensemble and large-scale
environment}\label{interaction-between-cumulus-ensemble-and-large-scale-environment}}

Following Arakawa and Schubert (1974), the equations for tendencies of
the grid-mean variables are written as

\begin{eqnarray}
 \frac{\partial \bar{h}}{\partial t} = M \frac{\partial \bar{h}}{\partial z} + \sum_j D_j \left[ h_j(z_{T,j}) - \bar{h} \right],
\end{eqnarray}

\begin{eqnarray}
 \frac{\partial \bar{q}}{\partial t} = M\frac{\partial \bar{q}}{\partial z} + \sum_j D_j \left[ q_j(z_{T,j}) - \bar{q} \right],
\end{eqnarray}

where \(M\), \(D\), \(h\) denote total mass flux, detrainment mass flux
and moist static energy. \(q\) is a substitute for \(q_v\), \(q_l\) and
\(q_i\) and any tracers which are calculated in the same way. \(z_T\) is
the height of the updraft. The hats indicate in-cloud properties, the
overbars grid-mean. The subscripts \(j\) are an index for the updraft
types.

The total mass flux \(M\) and detrainment \(D\) are defined as

\begin{eqnarray}
M(z) = \sum_j M_{u,j} + M_d \, ,
\end{eqnarray}

\begin{eqnarray}
 D_j(z) = M_{u,j}(z_{T,j}) \delta (z-z_{T,j})
\end{eqnarray}

respectively, where \(M_u\) and \(M_d\) denote mass fluxes of updraft
and downdraft respectively. The updraft mass flux is formulated as

\begin{eqnarray}
 M_{u,j}(z) = M_{B,j} \, \eta_j(z)
\end{eqnarray}

where \(M_B\) and \(\eta\) are the updraft mass flux at its cloud base
and normalized mass flux.

\hypertarget{cloud-base}{%
\subsubsection{Cloud base}\label{cloud-base}}

The cloud base is determined as the lifting condensation level of the
air at the lowest model layer. It is defined as the smallest \(z\) which
satisfies

\begin{eqnarray}
  \bar{q_t}(z_1) \geq \bar{q_v}^* + \frac{\gamma}{L_v(1+\gamma)} \left[\bar{h}(z_1)-\bar{h}^*(z) \right]\,,
\end{eqnarray}

where \(q_t\) denotes total water, \(L_v\) the latent heat of
vaporization, \(z_1\) the height of the lowest model layer at the full
level and

\begin{eqnarray}
 \gamma \equiv \frac{L_v}{C_p}\left(\frac{\partial \bar{q}^*}{\partial \bar{T}}\right)_{\bar{p}}.
\end{eqnarray}

\(C_p\) denotes the specific heat of dry air at constant pressure and
the stars indicate saturation values.

The normalized mass flux below the cloud base is given by
\(\eta = (z/z_B)^{1/2}\) for all of the updraft types where \(z_B\)
denotes the cloud base height.

\hypertarget{updraft-velocity-and-entrainment-rate}{%
\subsubsection{Updraft velocity and entrainment
rate}\label{updraft-velocity-and-entrainment-rate}}

The entrainment rate is defined by

\begin{eqnarray}
 \epsilon = \frac{1}{M_u}\frac{\partial M_u}{\partial z}
\end{eqnarray}

and allowed to vary vertically. Based on the formulation of Gregory
(2001), the updraft velocity is calculated by

\begin{eqnarray}
 \frac{1}{2}\frac{\partial \hat{w}^2}{\partial z} = aB - \epsilon \hat{w}^2 \label{p-cum.1}
\end{eqnarray}

where \(w\) and \(B\) are the vertical velocity and the buoyancy of
updraft air parcel respectively. \(a\) is a dimensionless constant
parameter ranging from 0 to 1 and represents a ratio of buoyancy force
used to accelerate the updraft velocity. The hats indicate the values of
the updraft. The second term on the right-hand side represents reduction
in the upward momentum of the air parcel through the entrainment. Here
and hereafter, the equation number corresponds to that in Chikira and
Sugiyama (2010).

Then it is assumed that

\begin{eqnarray}
 \epsilon \hat{w}^2 \simeq C_\epsilon a B,
\end{eqnarray}

where \(C_\epsilon\) is a dimensionless constant parameter ranging from
0 to 1. This formulation denotes that a certain fraction of the
buoyancy-generated energy is reduced by the entrainment, which is
identical to the fraction used to accelerate the entrained air to the
updraft velocity. Thus, the entrainment rate is written as

\begin{eqnarray}
 \epsilon = C_\epsilon\frac{aB}{\hat{w}^2}. \label{p-cum.2}
\end{eqnarray}

Eqs. (\ref{p-cum.1}) and (\ref{p-cum.2}) lead to

\begin{eqnarray}
 \frac{1}{2}\frac{\partial \hat{w}^2}{\partial z} = a(1 - C_\epsilon) B
\end{eqnarray}

which shows that \(\hat{w}\) is continuously accelerated upward when
buoyancy is positive. Many CRM and LES results show, however, that
updraft velocity is often reduced if the parcel approaches its cloud
top. For this reason, adding an additional term, we use

\begin{eqnarray}
 \frac{1}{2}\frac{\partial \hat{w}^2}{\partial z} = a(1 - C_\epsilon) B - \frac{1}{z_0}\frac{\hat{w}^2}{2}\label{p-cum.4}
\end{eqnarray}

where the last term denotes that the energy of the updraft velocity is
relaxed to zero with a height scale \(z_0\). Eq. (\ref{p-cum.4}) is
discretized as

\begin{eqnarray}
 \frac{1}{2}\frac{\hat{w}^2_{k+1/2} - \hat{w}^2_{k-1/2}}{\Delta z_k} = a(1 - C_\epsilon) B_k - \frac{1}{z_0}\frac{\hat{w}_{k+1/2}^2}{2} \label{p-cum.A5}
\end{eqnarray}

where \(k\) is an index of full levels and \(k+1/2\) and \(k-1/2\)
indicate the upper and lower sides of the half levels. \(\Delta z\) is
the depth of the model layer. Note that the equation is solved for
\(\hat{w}^2\) rather than \(\hat{w}\).

The buoyancy of the cloud air parcel is determined by

\begin{eqnarray}
 B  =   \frac{g}{\bar{T}} ( \hat{T}_v - \bar{T}_v )
\end{eqnarray}

\begin{eqnarray}
 \simeq g \left\{ \frac{\hat{h} - \bar{h}^*}{C_p \bar{T}(1 + \gamma)} + \varepsilon(\hat{q_v}-\bar{q_v}) - \left[ (\hat{q_l}+\hat{q_i}) - (\bar{q_l}+\bar{q_i}) \right] \right\}
\end{eqnarray}

where \(g\) and \(T_v\) denote gravity and virtual temperature
respectively. \(\varepsilon = R_v/R_d - 1\) where \(R_v\) and \(R_d\)
are the gas constants for water vapor and dry air respectively.

\(\hat{w}\), \(B\) and \(\epsilon\) are calculated for each of the
updraft types separately, but we omit the subscript \(j\) for
convenience.

\hypertarget{normalized-mass-flux-and-updraft-properties}{%
\subsubsection{Normalized mass flux and updraft
properties}\label{normalized-mass-flux-and-updraft-properties}}

The properties of the updraft are determined by

\begin{eqnarray}
 \frac{\partial \eta \hat{h}}{\partial z} = \epsilon \eta \bar{h} + Q_i, \label{p-cum.5}
\end{eqnarray}

\begin{eqnarray}
 \frac{\partial \eta \hat{q_t}}{\partial z} = \epsilon \eta \bar{q_t} - P \label{p-cum.6}
\end{eqnarray}

and

\begin{eqnarray}
 \frac{\partial \eta}{\partial z} = \epsilon \eta, \label{p-cum.7}
\end{eqnarray}

where \(Q_i\) and \(P\) denote heating by liquid-ice transition and
precipitation respectively. All the other variables such as temperature,
specific humidity, and liquid and ice cloud water are computed from
these quantities. Tracers are calculated by a method identical to that
for \(\hat{q}_t\).

Equation (\ref{p-cum.7}) leads to

\begin{eqnarray}
 \frac{\partial \ln \eta}{\partial z} = \epsilon.
\end{eqnarray}

Then, \(\eta\) and \(\epsilon\) are discretized as

\begin{eqnarray}
 \frac{\ln \eta_{k+1/2} - \ln \eta_{k-1/2}}{\Delta z_k} = \epsilon_k. \label{p-cum.A1}
\end{eqnarray}

Note that this discrete form leads to an exact solution if \(\epsilon\)
is vertically constant. Also, \(\eta\) is finite as far as \(\epsilon\)
is. For \(\epsilon_k,\) a maximum value of \(4 \times 10^{-3} m^{-1}\)
is applied.

Equations (\ref{p-cum.5}) and (\ref{p-cum.6}) are written as

\begin{eqnarray}
 \frac{\partial \eta \hat{h}}{\partial z} = E \bar{h} + Q_i,
\end{eqnarray}

\begin{eqnarray}
 \frac{\partial \eta \hat{q}_t}{\partial z} = E \bar{q}_t -P
\end{eqnarray}

respectivuly, where \(E = \epsilon\eta\). These equations are
discretized as

\begin{eqnarray}
 \frac{\eta_{k+1/2} \hat{h}_{k+1/2} - \eta_{k-1/2} \hat{h}_{k-1/2}}{\Delta z_k} = E_k \bar{h}_k + {Q_{i,k}}  \label{p-cum.A2}
\end{eqnarray}

\begin{eqnarray}
 \frac{\eta_{k+1/2} \hat{q}_{t,k+1/2} - \eta_{k-1/2} \hat{q}_{t,k-1/2}}{\Delta z_k} = E_k {\bar{q}_{t,k}} - P_k  \label{p-cum.A3}
\end{eqnarray}

Considering the relation that
\(\partial \eta/\partial z = \epsilon\eta\), we descretize \(E_k\) as

\begin{eqnarray}
 E_k = \frac{\eta_{k+1/2} - \eta_{k-1/2}}{\Delta z_k}  \label{p-cum.A4}
\end{eqnarray}

Note that conservation of mass, energy, and water is guaranteed with
Eqs. (\ref{p-cum.A1})--(\ref{p-cum.A4}). This set of equations
leads to exact solutions of \(\hat{h}\) under the special case that
\(\epsilon\) and \(\bar{h}\) are vertically constant and \(Q_i\) is
zero. From Eqs. (\ref{p-cum.A1}), (\ref{p-cum.A2}), and
(\ref{p-cum.A4}), assuming \(Q_i\) is zero,

\begin{eqnarray}
 \hat{h}_{k+1/2} = e^{-\epsilon_k \Delta z_k} \hat{h}_{k - 1/2} + (1 - e^{-\epsilon_k \Delta z_k}) \bar{h}_k,
\end{eqnarray}

which shows that \(\hat{h}_{k+1/2}\) is a linear interpolation between
\(\hat{h}_{k - 1/2}\) and \(\bar{h}_k\). Thus, the stability of
\(\hat{h}\) is guaranteed. The same property applies to \(\hat{q}_t\) as
well, if \(P\) is zero.

These calculations are made for each of the updraft types separately,
but we omit the subscript \(j\) for convenience.

\hypertarget{spectral-representation}{%
\subsubsection{Spectral representation}\label{spectral-representation}}

Following the spirit of the Arakawa--Schubert scheme, updraft types are
spectrally represented. Different values of cloud-base updraft
velocities are given from the minimum to the maximum values with a fixed
interval. The minimum and maximum values are set to 0.1 and 1.4
\(m s^{-1}\), with an interval of 0.1 \(m s^{-1}\).

Then, the updraft properties are calculated upward with Eqs.
(\ref{p-cum.2}), (\ref{p-cum.4}), (\ref{p-cum.5}),
(\ref{p-cum.6}), and (\ref{p-cum.7}). This upward calculation
continues even if the buoyancy is negative as long as the updraft
velocity is positive. If the velocity becomes negative at some level,
the air parcel detrains at the neutral buoyancy level which is below and
closest to the level. That is, the scheme automatically judges whether
the rising parcel can penetrate the negative buoyancy layers when there
is a positive buoyancy layer above. The effect of the convective
inhibition (CIN) near cloud base is also represented by this method.
Note, however, that an effect of overshooting above cloud top is not
represented for simplicity (i.e., detrainment never occurs above cloud
top).

\hypertarget{cloud-base-mass-flux}{%
\subsubsection{Cloud-base mass flux}\label{cloud-base-mass-flux}}

The cloud-base mass flux is determined with the prognostic convective
kinetic energy closure proposed by Arakawa and Xu (1990). That is, the
cloud kinetic energy for each of the updraft types is explicitly
predicted by

\begin{eqnarray}
 \frac{\partial K}{\partial t} = AM_B - \frac{K}{\tau_p}\,,  \label{p-cum.8}
\end{eqnarray}

where \(K\) and \(A\) are the cloud kinetic energy and cloud work
function respectively, and \(\tau_p\) denotes a time scale of
dissipation. The cloud work function \(A\) is defined as

\begin{eqnarray}
 A \equiv \int_{z_B}^{z_T} B \eta \,dz\,.
\end{eqnarray}

The cloud kinetic energy is linked with \(M_B\) by

\begin{eqnarray}
 K = \alpha M_B^2.  \label{p-cum.9}
\end{eqnarray}

The cloud-base mass flux is then solved for each of the updraft types.

\hypertarget{microphysics}{%
\subsubsection{Microphysics}\label{microphysics}}

The method to obtain temperature and specific humidity of in-cloud air
from moist static energy is identical to that in Arakawa and Schubert
(1974). The ratio of precipitation to the total amount of condensates
generated from cloud base to a given height \(z\) is formulated as

\begin{eqnarray}
 F_p(z) = 1 - e^{-(z - z_B - z_0)/z_p},
\end{eqnarray}

where \(z_0\) and \(z_p\) are tuning parameters.

The ratio of cloud ice to cloud condensate is determined simply by a
linear function of temperature,

\begin{eqnarray}
 F_i(T) = \begin{cases} 1 & T \leq T_1 \\ (T_2 - T)/(T_2 - T_1) & T_1 < T < T_2 \\ 0  & T \geq T_2 \end{cases}
\end{eqnarray}

where \(T_1\) and \(T_2\) are set to 258.15 and 273.15 K. The ratio of
snowfall to precipitation is also determined by this function.

From the conservation of condensate static energy,
\(C_p T + gz + L_v q - L_i q_i\) where \(L_i\) is the latent heat of
fusion, for a cloud parcel, \(Q_i\) in Eq. (\ref{p-cum.5}) is
written as

\begin{eqnarray}
 Q_i = L_i \left(\frac{\partial \eta \hat{q}_i}{\partial z} - \epsilon\eta\bar{q}_i\right)
\end{eqnarray}

and discretized as

\begin{eqnarray}
 {Q_i}_k = L_i \left(\frac{\eta_{k+1/2} \hat{q}_{i,k+1/2} - \eta_{k-1/2} \hat{q}_{i,k-1/2}}{\Delta z_k} - E_k \bar{q}_{i,k} \right)
\end{eqnarray}

Strictly, the ratio of the cloud ice to the cloud condensate should be
recalculated after the modification of temperature by \(Q_i\) and the
iterations of the calculation are required; however, it is omitted for
simplicity.

Melting and freezing of precipitation occurs depending on wet-bulb
temperature of large-scale environment and cumulus mass flux.

\hypertarget{evaporation-sublimation-and-downdraft}{%
\subsubsection{Evaporation, sublimation and
downdraft}\label{evaporation-sublimation-and-downdraft}}

A part of precipitation is evaporated at each level as

\begin{eqnarray}
 E_v = a_e (\bar{q}_w - \bar{q}) \left(\frac{P}{V_T}\right),
\end{eqnarray}

where \(E_v\), \(q_w\) and \(V_T\) are the mass of evaporation per a
unit volume and time, wet-bulb saturated specific humidity and terminal
velocity of precipitation respectively \(a_e\) is a constant. Downdraft
mass flux \(M_d\) is generated as

\begin{eqnarray}
 \frac{\partial M_d}{\partial z} = -b_e \bar{\rho} (\bar{T}_w - \bar{T}) P,
\end{eqnarray}

where \(\rho\) and \(T_w\) are density and wet-bulb temperature,
respectively; \(b_e\) is a constant. Properties of downdraft air are
determined by budget equations and the detrainment occurs at neutral
buoyancy level and below cloud base.

If the precipitation is composed of both rain and snow, the rain (snow)
is evaporated (sublimated) in the same ratio as the ratio of rain (snow)
to the total precipitation when the precipitation evaporates to produce
downdrafts.

\hypertarget{cloudiness}{%
\subsubsection{Cloudiness}\label{cloudiness}}

Fractional cloudiness of the updrafts \(C_u\) used in the radiation
scheme is diagnosed by

\begin{eqnarray}
 C_u = \frac{C_\mathrm{max} - C_\mathrm{min}}{\ln M_\mathrm{max} - \ln M_\mathrm{min}}(\ln \sum_j M_{u,j} - \ln M_\mathrm{min}) + C_\mathrm{min},
\end{eqnarray}

where \(C_\mathrm{max}\), \(C_\mathrm{min}\), \(M_\mathrm{max}\),
\(M_\mathrm{min}\) are the maximum and minimum values of the cloudiness
and cumulus mass flux respectively.

The grid mean liquid cloud mixing ratio in the updrafts is given by

\begin{eqnarray}
 l_c = \frac{\beta C_u}{M} \sum_j \hat{q}_{l,j} M_{u,j},
\end{eqnarray}

where \(\beta\) is a dimensionless constant. The grid mean ice cloud
mixing ratio is determined similarly.

\hypertarget{cumulus-momentum-transport}{%
\subsubsection{Cumulus Momentum
Transport}\label{cumulus-momentum-transport}}

Following Gregory et al.~(1997), the zonal and meridional velocities of
the updrafts are calculated as

\begin{eqnarray}
 \frac{\partial \eta \hat{u}}{\partial z} = \epsilon \eta \bar{u} + C_m \eta \frac{\partial \bar{u}}{\partial z},
\end{eqnarray}

where \(C_m\) is a constant from 0 to 1 representing the effect of
pressure. This equation can be rewitten as

\begin{eqnarray}
 \frac{\partial \eta \hat{u}}{\partial z} = (1-C_m) \epsilon \eta \bar{u} + C_m \frac{\partial \eta \bar{u}}{\partial z},
\end{eqnarray}

and is discretized as

\begin{eqnarray}
  \frac{\eta_{k+1/2} \hat{u}_{k+1/2} - \eta_{k-1/2} \hat{u}_{k-1/2}}{\Delta z_k} = (1-C_m) E_k \bar{u}_k
          + C_m \frac{\eta_{k+1/2} \bar{u}_{k+1/2} - \eta_{k-1/2} \bar{u}_{k-1/2}}{\Delta z_k}.
\end{eqnarray}

The horizontal velocities of the downdrafts are calculated similarly.
The tendencies of zonal and meridional velocities by the cumulus
momentum transport (CMT) are calculated as

\begin{eqnarray}
 \left(\frac{\partial u}{\partial t}\right)_{\mathrm{CMT},k} = -g\frac{(\rho\overline{u'w'})_{k+1/2} - (\rho\overline{u'w'})_{k-1/2}}{\Delta p_k},
\end{eqnarray}

\begin{eqnarray}
 \left(\frac{\partial v}{\partial t}\right)_{\mathrm{CMT},k} = -g\frac{(\rho\overline{v'w'})_{k+1/2} - (\rho\overline{v'w'})_{k-1/2}}{\Delta p_k}
\end{eqnarray}

respectively, where \(\rho\overline{u'w'}\) and \(\rho\overline{v'w'}\)
are total zonal and meridional momentum fluxes respectively and
\(\Delta p_k = p_k - p_{k+1}\).

	% 物理過程:shallow convection scheme
	% \RequirePackage{plautopatch}
% \documentclass[platex, dvipdfmx]{article}
% %\title{浅い積雲のパラメタリゼーション}
% %\date{2021.1.22}
% \usepackage[dvipdfmx]{hyperref}
% \usepackage{amsmath,here} %数式、図の位置指定
% \usepackage{pxjahyper}

\begin{document}
\subsection{Shallow Convection Scheme}\label{shallow-convection-scheme}

% ___
% \_ \
%   \ \
%    \ \
%     \ \
%     _\ \_
%     \____\
\subsubsection{Overview of shallow convection}\label{overview-of-shallow-convection}
Shallow convection is the most frequent type of convective cloud in the tropics and subtropics Its impact on climate through the energy budget due to atmospheric radiation is considered important (Stevens, 2005).
Shallow convection is responsible for transporting the boundary layer air to the free atmosphere. It is often not accompanied by precipitation and is characterized by the fact that precipitation-induced downdraft does not reach the surface as in deep convection.

This section briefly describes the vertical structure of the boundary layer favorable for shallow convection.
When the ground surface is heated by sunlight or cold air flows in from above, the energy of convective instability is dissipated by turbulence in the bottom of the atmosphere, forming a mixed layer with a nearly uniform vertical distribution of 
temperature and water vapor at a thickness of about 600 m to 800 m from the surface.
At the upper end of the mixed layer, there is a transition layer of weakly stable stratification, which is the height at which water vapor in updraft begins to condense (lifting condensation level, LCL).
Above LCL, the temperature decreases according to the moist adiabatic lapse rate, and the updraft is observed as clouds. Above the level of free convection (LFC), the cloud continues to grow while mixing with surrounding air. 
The growth of these convective clouds is limited by the temperature inversion layer at the lower end of the free atmosphere, and the cloud tops are often located about 2 km from the surface.

In the former versions of MIROC, A cumulus parameterization proposed by Chikira and Sugiyama (2010) deals with multiple cloud types including shallow cumulus and deep convective clouds. However, it tends to overestimate low-level cloud amounts.
To cope with this bias and improve the performance for reproducing current climate, the shallow convection scheme is introduced from the 6th version of MIROC (Tatebe et al., 2019, Ogura et al., 2017, Ogura, 2015).
The source code in concern (pshcn.F) consists of SUBROUTINE:[PSHCN] and SUBROUTINE:[DISTANCE]. The input values for SUBROUTINE:[PSHCN] are temperature, water vapor mixing ration, and liquid water mixing ratio, ice mixing ratio.
It predicts liquid water potential temperature, total water mixing ratio, ice mixing ratio, and horizontal components of wind in response to vertical transport. It also diagnoses cloud fraction and precipitation.
SUBROUTINE:[DISTANCE], which is called inside SUBROUTINE:[PSHCN], calculates the degree of buoyancy-induced updraft and mixing with the environment.
Since the variables diagnosed in the cumulus scheme (SUBROUTINE:[CUMLUS]) are referenced to determine the conditions for shallow convection, SUBROUTINE:[PSHCN] is required to be run after SUBROUTINE:[CUMULUS], followed by the diagnosis of cloud fraction.
On the other hand, it should be run before the land surface process SUBROUTINE:[SURFCE] because precipitation by convection is referenced in the land surface and ocean models.

% ________
% \______ \
%   _____\ \
%   \  _____\
%    \ \    __
%     \ \___\ \
%      \_______\
\subsubsection{Basics of cloud model}\label{basics-of-cloud-model}
Subgrid clouds are modeled based on the frameworks proposed by Bretherton et al. (2004) and Park and Bretherton (2009).
This scheme employs a simple plume model for cloud to calculate vertical transport of conserved variables and precipitation due to updraft.
An ensemble of shallow convection in a horizontal grid, which is expressed as a single updraft plume, is supposed to experience horizontal mixing with the environment
(entrainment/detrainment). The flux of vertical transport of mass is assumed in the following form:
\begin{equation}\label{def_Mu}
    \rho \overline {w' \psi '}\approx M_u (\psi_u-\overline{\psi}) ,
\end{equation}
where $M_u=\rho_u\sigma_u w_u$ is mass flux of updraft ($\rho_u$,$\sigma_u$, and $w_u$ stand for density in updraft, area fraction of updraft in a grid, and vertical velocity, respectively),
$\psi_u$ is a conserved variable transported by convection (e.g. liquid water potential temperature, total water mixing ratio, horizontal components of momentum) in updraft,
$\overline{\psi}$ denotes the average value in the environmental field of the same conserved value.
The effects of vertical transport due to shallow convection are represented by determining the vertical profiles of unknown values $M_u$ and $\psi_u$.
Flux of mass and conserved values are diagnosed as
\begin{align}
    \frac{\partial M_u}{\partial z} &= E - D \label{zprof_Mu}\\
    \frac{\partial}{\partial z} (\psi_u M_u) &= X_\psi + S_\psi M_u,\label{zprof_psi}
\end{align}
where $X_\psi$ represents horizontal mixing with environmental air, and $S_\psi$ is source term. $E$ and $D$ are rates of entrainment and detrainment, which are described in fractional from
\begin{align}
    E &=\epsilon M_u \label{fracE}\\
    D &=\delta M_u. \label{fracD}
\end{align}
Substituting $\overline{\psi}$ for grid value and assuming the horizontal mixing term as $X_{\psi}=E \overline{\psi} - D\psi_u$ results in
\begin{align}
    \frac{\partial M_u}{\partial z} &= M_u (\epsilon - \delta) \label{zprof_Mu'}\\
    \frac{\partial \psi_u}{\partial z} &= \epsilon(\overline{\psi} - \psi_u) + S_{\psi}. \label{zprof_psi'}
\end{align}
In MIROC6, changes in liquid water potential temperature due to precipitation and the effect of subgrid pressure gradient on horizontal momentum are included in $S_{\psi}$.
Consequently, equations (\ref{zprof_Mu'}) and (\ref{zprof_psi'}) results in a closure problem of two parameters $\delta$ and $\epsilon$.
By determining $\delta$ and $\epsilon$ by the formulation described in section \ref{diagnosing-vertical-profile-of-updraft-mass-flux} 
and solving differential equations along with boundary condition at cloud base, vertical profiles of $M_u$ and $\psi_u$ are calculated.

% ________
% \  ____ \
%  \_\ __\ \
%      \___ \
%     __   \ \
%     \ \___\ \
%      \_______\
\subsubsection{Computation in PSHCN}\label{computation-in-PSHCN}

The effect of convective updraft is calculated as follows.
\begin{itemize}
    \item Liquid water potential temperature $\theta_l$ and total water $q_t$ are diagnosed from input temperature $T$, water vapor mixing ratio $q_v$, liquid water mixing ratio $q_l$, ice mixing ratio $q_i$, 
    \item Updraft mass flux at cloud base is diagnosed.
    \item Height of cloud base is diagnosed.
    \item Presence of shallow convection is determined.
    \item Vertical profiles of $M_u$, $\theta_l$, $q_t$, horizontal wind components $u$ and $v$ are diagnosed.
    \item $\theta_l$, $q_t$, $q_i$, $u$, $v$, liquid water temperature $T_l$ are predicted.
    \item $T$, $q_v$, and $q_l$ are diagnosed according to $T_l$ and $q_t$.
\end{itemize}

\paragraph{Lower boundary condition: diagnosis of cloud base mass flux}\label{lower-boundary-condition}

The mass flux at cloud base is formulated as it depends on turbulent kinetic energy (TKE) in boundary layer and convective inhibition (CIN) at the top of boundary layer.

Firstly, the vertical profile of updraft velocity is supposed to fulfill 
\begin{equation}\label{zprof_wu}
    \frac{1}{2}\frac{\partial}{\partial z}w_u^2=aB_u-b\epsilon w_u^2
\end{equation}
all over the layers with shallow convection. $B_u$ means updraft buoyancy, $a$ and $b$ are empirical parameters. 
The first term of the right-hand side of (\ref{zprof_wu}) is acceleration by buoyancy, and the second term represents drag by entrainment.
By assuming no entrainment below LFC and integrating (\ref{zprof_wu}) from cloud base to LFC, The critical value of upward velocity for updraft plume to reach LFC, $w_c$, can be determined

\begin{equation}\label{wc}
    w_c = \sqrt{2a(CIN)}.
\end{equation}
Updrafts that exceed this critical value penetrates from cloud base.

Computation of CIN is based of Appendix C of Bretherton et al., 
\begin{align}\label{def_CIN}
    CIN = [B_u(p_{base}) + B_u(p_{LCL})]\frac{p_{LCL}-p_{base}}{g(\rho_{LCL}+\rho_{base})} + B_u(p_{LCL})\frac{p_{LFC}-p_{LCL}}{g(\rho_{LFC}+\rho_{LCL})}.
\end{align}
In the following, subscript $\mathit{base}$ represents the value at the top of mixing layer. 
In MIROC6, for simplicity, $B_u(p_{base})$ is set to zero.

Secondly, to obtain the information of vertical velocity at cloud base, the statistical distribution of $w$ is assumed to follow Gaussian distribution
\begin{equation}\label{distr_w}
    f(w) = \frac{1}{2\pi k_f e_{avg}}\exp\left[ -\frac{w^2}{2k_fe_{avg}}\right]
\end{equation}
with variance equal to $k_f e_{avg}$, where $e_{avg}$ is average TKE diagnosed in turbulent and vertical diffusion scheme.
$k_f$ is an empirical parameter describing the partitioning of TKE between horizontal and vertical motions at the subcloud layer inversion, whose recommended value based on large eddy simulation is 0.5.

By taking average of vertical velocity above the critical value $w_c$, cloud base mass flux $M_{u,base}$ is diagnosed as
\begin{equation}\label{Mubase}
    M_{u,base}=\overline{\rho_{base}}\int_{w_c}^{\infty}wf(w)dw =\overline{\rho_{base}}\sqrt{\frac{k_f e_{avg}}{2\pi}}\exp\left[-\frac{w_c^2}{2k_fe_{avg}}\right],
\end{equation}
where $\overline{\rho_{base}}$ is density at LFC.
This mass flux is larger for larger boundary layer TKE and smaller for larger CIN.

\paragraph{Diagnosing hight of cloud base}\label{diagno-height-of-cloud-base}

The cloud base height is set between the top of the boundary layer and the LCL. The larger the CIN is, the lower the cloud base becomes.
The top of boundary layer is diagnosed as the level with maximum vertical gradient of relative humidity. 
Let $z_{Hi}$ be the higher of this level and LCL, and $z_{Lo}$ be the lower, then the cloud base altitude $z_{base}$ is set
\begin{equation}\label{zbase}
    z_{base} = z_{Hi} - (z_{Hi}-z_{Lo})\frac{CIN-CIN_{Lo}}{CIN_{Hi} - CIN_{Lo}}.
\end{equation}
$CIN_{Hi}$ and $CIN_{Lo}$ are coefficients which satisfy $CIN_{Lo}\le CIN \le CIN_{Hi}$ for a typical value of CIN.

\paragraph{Determination of the presence of shallow convection}\label{presence-of-shallow-convection}

For each horizontal column, whether shallow convection occurs is determined with following criteria.
\begin{itemize}
    \item If estimated inversion strength (EIS; Wood and Bretherton, 2006) exceeds a certain threshold,
    the environmental field is judged to be dominated by stratocumulus clouds, and shallow convection is not generated.
    This criterion is introduced because the vertical resolution of climate models does not sufficiently represent the thin and strong inversion layer over the boundary layer,
    and underestimates CIN, which leads to an overestimation of shallow convection.
    EIS is estimated by
     $EIS=\theta_{700}-\theta_{0}-\Gamma_m^{850}(z_{700}-LCL)$
    where $\theta_{700}$ and $\theta_0$ are potential temperature at 700hPa and surface, $\Gamma_m^{850}$ is moist adiabatic lapse rate at 850hPa, 
    and $z_{700}$ is height of 700hPa.
    \item If the intensity of cumulus convection diagnosed by SUBROUTINE:[CUMULUS] exceeds a threshold, the environmental 
    field is supposed to be dominated by deep convection and shallow convection is not generated.
    \item If the areal fraction of shallow convection is under a threshold, computation of shallow convection is omitted.
 \end{itemize}

\paragraph{Diagnosing vertical profile of updraft mass flux}\label{diagnosing-vertical-profile-of-updraft-mass-flux}

For the grid boxes that contain shallow convection, entrainment and detrainment is calculated using the value of $\psi_u$ at cloud base and $M_{u,base}$.
Fractional entrainment and detrainment are computed based on the framework of buoyancy sorting suggested by Kain and Fritsch (1990).
In a layer of thickness $\delta z$, equal parts $\epsilon_0 M_u \delta z$ of updraft and environmental air are involved in the lateral mixing process that creates a spectrum of mixtures.
This yields a total mixing mass flux $2\epsilon_0 M_u \delta z$, with fractional mixing rate $\epsilon_0=c_0/H$ ($c_0$ is a certain empirical coefficient and $H$ is the height from surface).
In the mixed air, there exists states with probability density $\chi$ such that the air from the environmental field occupies a proportion $\chi$. Here, for simplicity of calculation, 
it is considered that the state from pure moist air ($\chi=0$) to pure environmental air ($\chi=1$) is distributed with uniform probability (Kain-Fritsch scheme assumes Gaussian distribution).
Based on the buoyancy force on the mixed air, the entrainment or detrainment is determined. SUBROUTINE:[DISTANCE] is called in SUBROUTINE:[PSHCN].
The output variables in this subroutine are liquid water potential temperature (THETLU) and bool value for entrainment or detrainment (JUDGE).

The occurrence of entrainment is judged as follows. Firstly, if the updraft air is not saturated, entrainment is not assumed to occur.
Nextly, with virtual potential energy in the environmental field ($\overline{\theta_v}$) and updraft ($\theta_{vu}$), buoyancy force on the parcel is defined:
\begin{equation}\label{buoy_u}
    B_u = g\frac{\theta_{vu} - \overline{\theta_{v}}}{ \overline{\theta_v}}
\end{equation}
and entrainment occurs when the buoyancy on parcel is positive.
Furthermore, even when the buoyancy is negative, entrainment occurs if the parcel can travel longer than a certain eddy mixing distance $l_c=c_1 H$, where $c_1=0.1$ is an empirical constant, 
chosen to optimize the trade-cumulus case. This criterion corresponds to the critical buoyancy value
\begin{equation}\label{buoy_c}
    B_c = -\frac{1}{2}\frac{w_u^2}{l_c}    
\end{equation}
and otherwise, all the mixed air is detrained.
Therefore, Once the critical value of the mixing state $\chi_c$ is obtained, which allows the updraft to rise a distance $l_c$ under negative buoyancy, 
the air in the environmental field entrained into the cloud and the air in the updraft that is detrained can be determined as follows
\begin{align}
    M_u\epsilon&=2\epsilon_0 M_u\int_0^{\chi_c}\chi q(\chi) d\chi = \epsilon_0 M_u \chi_c^2 \label{flux_entre}\\
    M_u\delta&=2\epsilon_0 M_u\int_{\chi_c}^{1}(1-\chi) q(\chi) d\chi = \epsilon_0 M_u (1-\chi_c)^2. \label{flux_detre}
\end{align}
Thus, letting
\begin{align}
    \epsilon&=\epsilon_0\chi_c^2 \label{Etilde}\\
    \delta&=\epsilon_0(1-\chi_c)^2, \label{Dtilde}
\end{align}
equatinons (\ref{zprof_Mu'}) and (\ref{zprof_psi'}) are expressed as follows
\begin{align}
    \frac{1}{M_u}\frac{\partial M_u}{\partial z} &= \epsilon - \delta = \epsilon_0(2\chi_c - 1) \label{zprof_Mu_param}\\
    \frac{\partial \psi_u}{\partial z} &= \epsilon (\overline{\psi}-\psi_u) + S_{\psi} = \epsilon_0\chi_c^2(\overline{\psi}-\psi_u) + S_{\psi}, \label{zprof_psi_param}
\end{align}
where $\chi_c$ is computed based on virtual potential temperature of mixed air
\begin{equation}\label{virt_pot_t}
    \theta_v(\chi)=\theta_{vu}+\chi\left[ \beta(\overline{\theta_l}-\theta_{lu})-\left(\frac{\beta L}{c_p\Pi}-\theta_u\right)(\overline{q_t}-q_{tu})\right]   
\end{equation}
(Bretherton et al., 2004). $\beta$ is a thermodynamic parameter which depends on temperature and pressure defined by Randall (1980), 
$\theta_{lu}$ is liquid water potential temperature in updraft, $\theta_u$ is updraft potential temperature,$\overline{q_t}$ is total water mixing ratio of environment,
$q_{tu}$ is total water mixing ratio of updraft, $L$ is latent heat of vaporization,$c_p$ is specific heat capacity of dry air at constant pressure, and $\Pi$ is the Exner function.

Consequently, the governing equations (\ref{zprof_wu}), (\ref{zprof_Mu_param}), and (\ref{zprof_psi_param}) for vertical profiles of $w_u$, $M_u$, and $\psi_u$ are obtained.
These equations are discretized and integrated upward one layer at a time using the lower boundary condition in section \ref{lower-boundary-condition} to yield the vertical profile of each variables.

Afterward, from liquid water potential temperature and total water mixing ratio, liquid water mixing ratio $q_l$ and water vapor mixing ratio $q_v$ are diagnosed.
The cloud water that exceeds a threshold is disposed as rainwater $q_r$, and liquid water potential temperature is updated according to the amount of $q_r$. This corresponds to $S_\psi$ in (\ref{virt_pot_t}).

The formulation of the vertical flux in this scheme is equal to the assumption that the updraft is not large enough to replace all of the air in a grid box in the time step $\Delta t$.
Therefore, the following limiter is imposed to prevent numerical instability when diagnosing mass flux of the updraft.
\begin{equation}\label{Mu_limit}
    M_u = min.\left(M_u, \frac{\rho\Delta z}{\Delta t}\right)    
\end{equation}

% \begin{thebibliography}{99}
%     \bibitem{B04} Bretherton, C. S., J. R. McCaa, and H. Grenier, 2004: 
%         A new parameterization for shallow cumulus convection and its application to marine subtropical cloud-topped boundary layers. Part I: description and 1D results. 
%         \textit{Mon. Wea. Rev.}, 132, 864-882.
%     \bibitem{KF90} Kain, J. S., and J. M. Fritsch, 1990: 
%         A one-dimensional entraining/detraining plume model and its application in convective parameterization.
%         \textit{J. Atmos. Sci.}, 47, 2784-2802.
%     \bibitem{PB09} Park, S. and C. S. Bretherton, 2009: 
%         The University of Washington shallow convection and moist turbulence schemes and their impact on 
%         climate simulations with the Community Atmosphere Model. \textit{J. Clim}, 22, 3449-3469.
%     \bibitem{R80} Randall, D. A., 1980: 
%         Conditional instability of the first kind upside-down. \textit{J. Atmos. Sci.}, 37, 125–130.
%     \bibitem{SouseiRep} Ogura, T., 2015: 
%         Implementation of a shallow convection parameterization. \textit{Program for Risk Information on Climate Change, 
%         Progress report 2014}, 126-131.
%     \bibitem{WB06} Wood, R. and C. S. Bretherton, 2006: 
%         On the relationship between stratiform low cloud cover and lower-tropospheric stability. 
%         \textit{J. Clim}, 19, 6425–6432.
%     \bibitem{S05} Stevens, B., 2005: 
%         Atmospheric Moist Convection. \textit{Annu. Rev. Earth Planet. Sci.}, 33, 605-643.
%     \bibitem{Tatebe19} Tatebe, H., T. Ogura, T. Nitta, et al., 2019: 
%         Description and basic evaluation of simulated mean state, internal variability, and climate sensitivity in MIROC6.
%         \textit{Geosci. Model Dev.}, 12, 2727-2765.
%     \bibitem{Ogura17} Ogura, T., H. Shiogama, M. Watanabe, et al., 2017:
%         Effectiveness and limitations of parameter tuning in reducing biases of top-of-atmosphere radiation and clouds in MIROC version 5.
%         \textit{Geosci. Model Dev.}, 10, 4647-4664.
%     \bibitem{CS10}Chikira, M. and M. Sugiyama, 2010: 
%         A cumulus parameterization with state-dependent entrainment rate. Part I: Description and sensitivity to temperature and humidity profiles.
%         \textit{J. Atmos. Sci.}, 67, 2171–2193
% \end{thebibliography}

\end{document}
	% 物理過程:大規模凝結
	\include{Hotta_pmlsc}
	% 物理過程:雲微物理
	\include{Hotta_pcldphys}
	% 物理過程:放射
	\subsection{放射フラックス}

\subsubsection{放射フラックス計算の概要}

CCSR/NIES AGCM の放射計算スキームは, 
Discrete Ordinate Method および 
k-Distribution Method に基づいて作成されたものである.
気体および雲・エアロゾルによる
太陽放射および地球放射の吸収・射出・散乱過程を考慮し,
放射フラックスの各レベルでの値を計算する.
主な入力データは, 気温 $T$, 比湿 $q$, 雲水量 $l$, 雲量 $C$ であり,
出力データは, 上向きおよび下向きの放射フラックス, $F^-, F^+$,
および上向き放射フラックスの地表温度に対する微分係数
$dF^-/dT_g$ である.

計算は複数の波長域に分けて行なわれる.
各波長域は k-distribution 法に基づき,
さらに複数のサブチャネルに分かれる.
気体吸収としては, 
H$_2$O, CO$_2$, O$_3$, N$_2$O, CH$_4$ のバンド吸収と,
H$_2$O, O$_2$, O$_3$ の連続吸収
およびCFCの吸収を取り入れている.
また, 散乱としては, 気体のレーリー散乱と
雲・エアロゾル粒子による散乱を取り入れている.

計算手順の概略は以下の通りである(括弧内はサブルーチン名).
%
\begin{enumerate}
\item 大気温度からプランク関数を計算する \Module{PLANKS}.
\item 各サブチャネルにおける,
      気体吸収による光学的厚さを計算する \Module{PTFIT}.
\item 連続吸収およびCFCの吸収による
      光学的厚さを計算する \Module{CNTCFC}.
\item レーリー散乱および粒子散乱の
      光学的厚さと散乱モーメントを計算する \Module{SCATMM}.
\item 散乱の光学的厚さと太陽天頂角から, 
      海面のアルベドを求める \Module{SSRFC}.
\item 各サブチャネルごとに,
      プランク関数を光学的厚さで展開する \Module{PLKEXP}.
\item 各サブチャネルごとに,
      各層の透過係数, 反射係数, 放射源関数を計算する \Module{TWST}
\item adding 法によって, 各層の境界での
      放射フラックスを計算する \Module{ADDING}
\end{enumerate}

雲の partial の被覆率を考慮するために,
各層の透過係数, 反射係数, 放射源関数は
雲に覆われた場合と雲がない場合とを別々に計算し,
雲量の重みをかけて平均をとる.
また, 積雲の雲量の考慮も行なっている.
さらに, adding も複数回行ない, 晴天放射フラックスを計算する.

\subsubsection{波長域とサブチャネル}

放射フラックス計算の基本は,
Beer-Lambert の法則
\begin{equation}
  F^\lambda(z) = F^\lambda(0) exp (-k^\lambda z)
\end{equation}
に表される. $F^\lambda$ は波長 $\lambda$ の放射フラックス密度であり.
$k^\lambda$ は吸収係数である.
加熱率にかかわる放射フラックスを計算するためには,
波長に対する積分操作が必要である.
%
\begin{equation}
  F(z) = \int F^\lambda(z) d \lambda 
 = \int F^\lambda(0) exp (-k^\lambda z) d \lambda
 \label{p-rad:beer}
\end{equation}
%
しかし, 気体分子による放射の吸収・射出は,
分子の吸収線構造により, 非常に複雑な波長依存性を持つため,
この積分を精密に評価することは容易ではない.
その比較的精密な計算を容易に行なうために考案された方法が
k-distribution 法である.
ある波長域の中で, 吸収係数 $k$ の,
$\lambda$ に関する密度関数 $F(k)$ を考え,
(\ref{p-rad:beer}) を
\begin{equation}
 \int F^\lambda(0) exp (-k^\lambda z) d \lambda 
 \simeq \int \bar{F}^k(0) exp (-k z) F(k) dk
\end{equation}
で近似する. ここで, $bar{F}^k(0)$ は
$z=0$ における, この波長域で吸収係数 $k$ をもつ
波長で平均したフラックスである.
この式は, $\bar{F}_k, F(k)$ が$k$の
比較的滑らかな関数であれば, 
\begin{equation}
 \int F^\lambda(0) exp (-k^\lambda z) d \lambda 
 \simeq \sum \bar{F}^i(0) exp (-k^i z) F^i
 \label{p-rad:beer-kd}
\end{equation}
のように, 指数関数項の有限個(サブチャネル)の足しあわせで
比較的精密に計算可能である.
この方法はさらに,
吸収と散乱を同時に考慮することが容易であるという利点を持つ.

CCSR/NIES AGCM では,
放射パラメータデータを変えることにより
いろいろな波長分割数で計算を行なうことができる.
現在標準で用いられるものでは,
波長域は18に分割されている.
さらに各波長域は1から6個のサブチャネル(上式の$i$に対応)に分割され,
全体で37チャネルとなる.
波長域は, 波数(cm$^{-1}$)で
50, 250, 400, 550, 770, 990, 1100, 1400, 2000,
2500, 4000, 14500, 31500, 33000, 34500, 36000, 43000, 46000, 50000
で分割されている.

\subsubsection{プランク関数の計算 \Module{PLANKS}}

各波長域で積分したプランク関数 $\overline{B}^w(T)$ は,
以下の式で評価する.

\begin{equation}
  \overline{B}^w(T) 
   = \lambda^{-2} T \exp \left\{ \sum_{n=0}{4} B^w_n (\bar{\lambda}^w T)^{-n}
                         \right\}
\end{equation}

$\bar{\lambda}^w$ は波長域の平均波長,
$B^w_n$ は function fitting によって定められたパラメータである.
これは, 各層の大気温度$T_l$, 各層の境界の大気温度$T_{l+1/2}$
と地表面温度$T_g$に対し計算する.

以下, 波長域に関する添字 $w$ は基本的に省略する.

\subsubsection{気体吸収による光学的厚さの計算 \Module{PTFIT}}

気体吸収による光学的厚さは, 添字 $m$ を分子の種類として,
以下のようになる. 

\begin{equation}
  \tau^g = \sum_{m=1}{N_m} k^{(m)} C^{(m)}
\end{equation}

ここで, $k^{(m)}$ は分子$m$の吸収係数であり, サブチャネルごとに異なる.

\begin{equation}
 k^{(m)} = \exp\left\{ \sum_{i=0}{N_i} \sum_{j=0}{N_j} A^{(m)}_{ij}
                   (\ln p)^{i} (T-T_{STD})^{j}
               \right\}
\end{equation}

という形で, 温度$T$(K), 気圧$p$(hPa) の関数として与えられる.
$C^{(m)}$ は, mol cm$^{-2}$ で表した層の中の気体の量であり,
体積混合比$r$(単位ppmv)から,
\begin{equation}
  C = 1\times 10^{-5} \frac{p}{R_u T} \Delta z \cdot r
\end{equation}
と計算できる. 
ただし, $R_u$ はモルあたりの気体定数(8.31 J mol$^{-1}$ K$^{-1}$)であり,
気層の厚さ $\Delta z$ の単位は km である.
また, ppmv での体積混合比$r$は, 
質量混合比$q$から, 
\begin{equation}
  r = 10^6 R^{(m)}/R^{(air)} q = 10^6 M^{(air)}/M^{(m)}
\end{equation}
によって換算できる.
$R^{(m)},R^{(air)}$ は
それぞれ対象分子と大気の質量あたりの気体定数,
$M^{(m)},M^{(air)}$ は
それぞれ対象分子と大気の平均分子量である.

この計算は, サブチャネルごと, 各層ごとに行なう.

\subsubsection{連続吸収およびCFCの吸収による光学的厚さ \Module{CNTCFC}}

H$_2$O の連続吸収による光学的厚さ$\tau^{H_2O}$は,
ダイマーによるものを考え, 
基本的に水蒸気の体積混合比の二乗に比例した形で評価する.
\begin{equation}
\tau^{H_2O} = ( A^{H_2O} + f(T) \hat{A}^{H_2O} ) (r^{H_2O})^2 \rho \Delta z
\end{equation}

$\hat{A}$ の項にかかる $f(T)$ は, 
ダイマーの吸収の温度依存性を表す.
さらに, 通常の気体吸収を無視する波長帯においては,
水蒸気の体積混合比の一乗に比例する寄与を取り入れる.

O$_2$ の連続吸収は, 混合比一定と仮定して,
\begin{equation}
\tau^{O_2} = A^{O_2} \rho \Delta z
\end{equation}
としている.

O$_3$の連続吸収は, 混合比$r^{O_3}$を用い, 温度依存性を取り入れて,
\begin{equation}
\tau^{O_3} = \sum_{n=0}{2} A^{O_3}_n r^{O_3} \frac{T}{T_{STD}}^n \rho \Delta z
\end{equation}

CFCの吸収は, $N_m$ 種類の CFC を考えて,
\begin{equation}
\tau^{CFC} = \sum_{m=1}{N_m} A^{CFC}_m r^{(m)} \rho \Delta z
\end{equation}

これらの光学的厚さの総和を$\tau^{CON}$ とする.
\begin{equation}
 \tau^{CON} =  \tau^{H_2O} + \tau^{O_2} + \tau^{O_3} + \tau^{CFC} 
\end{equation}

この計算は各波長域ごと, 各層ごとに行なう.

\subsubsection{散乱の光学的厚さと散乱モーメント \Module{SCATMM}}

レーリー散乱および粒子消散の(散乱と吸収を含めた)光学的厚さは
\begin{equation}
\tau^{s} 
 = \left( e^R + \sum_{p=1}{N_p} e^{(p)}_m r^{(p)}\right) \rho \Delta z
\end{equation}
ここで, $e^R$ はレーリー散乱の消散係数,
$e^{(p)}$ は粒子$p$の消散係数,
$r^{(p)}$ は標準状態に換算した
粒子$p$の体積混合比である.

ここで, 雲水の質量混合比 $l$ から
雲粒の標準状態換体積混合比(ppmv)への換算は以下のようになる.
\begin{equation}
  r = 10^6 \frac{p_{STD}}{R T_{STD}}/\rho_w
\end{equation}
ただし, $\rho_w$ は雲粒の密度である.

一方, 光学的厚さのうち散乱に起因する部分 $\tau_s^s$ は,
\begin{equation}
\tau_s^{s} 
 = \left( s^R + \sum_{p=1}{N_p} s^{(p)}_m r^{(p)}\right) \rho \Delta z
\end{equation}
ここで, $s^R$ はレーリー散乱の散乱係数,
$s^{(p)}$ は粒子$p$の散乱係数である.

また, 規格化された散乱のモーメント 
$g$ (非対称因子) および $f$ (前方散乱因子)は,
\begin{equation}
g = \frac{1}{\tau_s} \left[
    \left( g^R + \sum_{p=1}{N_p} g^{(p)}_m r^{(p)}\right) \rho \Delta z
    \right]
\end{equation}
\begin{equation}
f = \frac{1}{\tau_s} \left[ 
    \left( f^R + \sum_{p=1}{N_p} f^{(p)}_m r^{(p)}\right) \rho \Delta z
    \right]
\end{equation}
ここで, $g^R, f^R$ はレーリー散乱の散乱モーメント,
$g^{(p)}, f^{(p)}$ は粒子$p$の散乱モーメントである.

この計算は各波長域ごと, 各層ごとに行なう.

\subsubsection{海面のアルベド \Module{SSRFC}}

海面のアルベド $\alpha_s$ は散乱の光学的厚さを鉛直に足し合わせたもの
$<\tau^{s}>$ および太陽入射角ファクタ $\mu_0$ を用いて,
\begin{equation}
  \alpha_s = \exp\left\{ \sum_{i,j} C_{ij} {\cal T}^j {\mu_0}^j \right\}
\end{equation}
のように表される.
ただし,
\begin{equation}
 {\cal T} = ( 4 <\tau^{s}>/\mu )^{-1}
\end{equation}
である.

この計算は各波長域ごとに行なう.

\subsubsection{光学的厚さの総計}

気体バンド吸収, 連続吸収, レーリー散乱, 粒子散乱・吸収を
全て考慮した光学的厚さは, 
%
\begin{equation}
  \tau = \tau^g + \tau^{CON} + \tau^{s}
\end{equation}
%
となる. ここで, $\tau^g$ はサブチャネルごとに異なるため,
サブチャネルごと, 各層ごとに計算を行なう.

\subsubsection{プランク関数の展開 \Module{PLKEXP}}

各層の中で, プランク関数 $B$ を
\begin{equation}
  B(\tau') = b_0 + b_1 \tau' + b_2 \left(\tau'\right)^2
\end{equation}
のように展開して表現し, 展開係数 $b_0, b_1, b,2$ を求める.
ここで, $B(0)$ として
各層の上端(上の層との境界)での$B$を,
$B(\tau)$として, 各層の下端(下の層との境界)での$B$を,
$B(\tau/2)$として, 各層の代表レベルでの$B$を用いる.
\begin{eqnarray}
  b_0 & = & B(0) \nonumber \\
  b_1 & = & ( 4B(\tau/2) - B(\tau) - 3B(0) )/\tau  \\
  b_2 & = & 2 ( B(\tau) + B(0) - 2B(\tau/2) )/\tau^2  \nonumber
\end{eqnarray}

この計算は, サブチャネルごと, 各層ごとに行なう.

\subsubsection{各層の透過・反射係数, 放射源関数 \Module{TWST}}

これまで求められた, 光学的厚さ$\tau$, 散乱の光学的厚さ$\tau^s$,
散乱モーメント$g, f$, プランク関数の展開係数$b_0, b_1, b_2$,
太陽入射角ファクタ $\mu_0$ を用いて,
均一な層を仮定し, 2ストリーム近似で
透過係数$R$, 反射係数$T$, 下方向への放射源関数$\epsilon^+$,
上方向への放射源関数$\epsilon^-$を求める.

単一散乱アルベド$\omega$は,
\begin{equation}
  \omega = \tau_s^s/\tau
\end{equation}
前方散乱因子 $f$ による寄与を
補正した光学的厚さ$\tau^*$,
単一散乱アルベド$\omega^*$, 非対称因子$g^*$ は,
\begin{eqnarray}
  \tau^* & = & \frac{\tau}{1-\omega f} \\
  \omega^* & = & \frac{(1-f)\omega}{1-\omega f}   \\
  g^* & = & \frac{g-f}{1-f}  
\end{eqnarray}

これから, 規格化された散乱の位相関数として,
\begin{eqnarray}
  \hat{P}^\pm   & = & \omega^* {W^-}^2 \left( 1 \pm 3g^* \mu \right)/2 \\
  \hat{S}_s^\pm & = & \omega^* W^-     \left( 1 \pm 3g^* \mu \mu_0 \right)/2
\end{eqnarray}
ただし, $\mu$ は2ストリームの方向余弦であり,
\begin{equation}
  \mu \equiv \left\{ \begin{array}{ll}
                   1/\sqrt{3} \; \; \; & 可視・近赤外域 \\
                   1/1.66     \; \; \; & 赤外域
                    \end{array}
             \right.
\end{equation}
\begin{equation}
  W^- \equiv \mu^{-1/2}
\end{equation}

さらに,
\begin{eqnarray}
  X & = & \mu^{-1} - (\hat{P}^+ - \hat{P}^- ) \\
  Y & = & \mu^{-1} - (\hat{P}^+ + \hat{P}^- ) \\
  \hat{\sigma}_s^{\pm} & = & \hat{S}_s^+ \pm \hat{S}_s^- \\
  \lambda & = & \sqrt{XY}
\end{eqnarray}
を用いると, 反射率$R$および透過率$T$は以下のようになる.
\begin{eqnarray}
 \frac{A^+{\tau^*}}{A^-{\tau^*}}
  & = & \frac{X (1+e^{-\lambda\tau^*}) - \lambda (1-e^{-\lambda\tau^*})}
             {X (1+e^{-\lambda\tau^*}) + \lambda (1-e^{-\lambda\tau^*})} \\
 \frac{B^+{\tau^*}}{B^-{\tau^*}}
  & = & \frac{X (1-e^{-\lambda\tau^*}) - \lambda (1+e^{-\lambda\tau^*})}
             {X (1-e^{-\lambda\tau^*}) + \lambda (1+e^{-\lambda\tau^*})}
\end{eqnarray}
\begin{eqnarray}
  R & = &  \frac{1}{2} \left(  \frac{A^+{\tau^*}}{A^-{\tau^*}} 
                             + \frac{B^+{\tau^*}}{B^-{\tau^*}} \right) \\
  T & = &  \frac{1}{2} \left(  \frac{A^+{\tau^*}}{A^-{\tau^*}} 
                             - \frac{B^+{\tau^*}}{B^-{\tau^*}} \right)
\end{eqnarray}

次にまず, プランク関数起源の放射源関数を求める.
\begin{equation}
  \hat{b}_n = 2 \pi (1-\omega^*) W^- b_n \; \; \; n=0,1,2 
\end{equation}
から, 放射源関数の展開係数が求められ,
\begin{eqnarray}
  D_2^\pm & = & \frac{\hat{b}_2}{Y} \\
  D_1^\pm & = & \frac{\hat{b}_1}{Y} \mp  \frac{2 \hat{b}_2}{XY} \\
  D_0^\pm & = & \frac{\hat{b}_0}{Y} + \frac{2 \hat{b}_2}{XY^2} 
                \mp  \frac{\hat{b}_1}{XY} \\
\end{eqnarray}
\begin{eqnarray}
  D^\pm(0)      & = & D_0^pm \\
  D^\pm(\tau^*) & = & D_0^pm + D_1^pm \tau^* + D_2^pm {\tau^*}^2
\end{eqnarray}
によりプランク関数起源の放射源関数 $\hat{\epsilon}_A^\pm$ は,
\begin{eqnarray}
  \hat{\epsilon}_A^- & = & D^-(0) - R D^+(0) - T D^-(\tau^*) \\
  \hat{\epsilon}_A^+ & = & D^+(0) - T D^+(0) - R D^-(\tau^*)
\end{eqnarray}

一方,  太陽入射起源の放射源関数は,
\begin{equation}
  Q\gamma = \frac{X\hat{\sigma}_s^+ + \mu_0^{-1} \hat{\sigma}_s^-}
                 {\lambda^2 - \mu_0^{-2} }
\end{equation}
より,
\begin{equation}
  V_s^\pm = \frac{1}{2} \left[
             Q\gamma \pm \left( \frac{Q\gamma}{\mu X} 
                                + \frac{\hat{\sigma}_s^-}{X} \right)
                        \right]
\end{equation}
を用いることにより, 以下の様になる.
\begin{eqnarray}
  \hat{\epsilon}_S^- & = & V_s^- - R V_s^+ - T V_s^- e^{-\tau^*/\mu_0} \\
  \hat{\epsilon}_S^+ & = & V_s^+ - T V_s^+ - R V_s^- e^{-\tau^*/\mu_0}
\end{eqnarray}

この計算は, サブチャネルごと, 各層ごとに行なう.

\subsubsection{各層の放射源関数の組み合わせ}

プランク関数起源と太陽入射起源の
両者を合わせた放射源関数は
\begin{equation}
  \epsilon^\pm  = 
  \epsilon_A^\pm + \hat{\epsilon}_S^\pm e^{-<\tau^*>/\mu_0} F_0 \\
\end{equation}
となる. ただし, $<\tau^*>$ は大気上端から
いま考慮している層の上端までの
$\tau^*$ を合計した光学的厚さであり, 
$F_0$ はいま考慮している波長域における入射フラックスである.
すなわち, $e^{-<\tau^*>/\mu_0} F_0$ は
いま考慮している層の上端での入射フラックスである.
%
この計算は実際には, 
\begin{equation}
  e^{-<\tau^*>/\mu_0} = \Pi' e^{-\tau^*/\mu_0}
\end{equation}
のように行なう. $\Pi'$ は大気最上層から
今考えている層の1つ上の層までの積を表す.

この計算は, サブチャネルごと, 各層ごとに行なう.

\subsubsection{各層境界での放射フラックス \Module{ADDING}}

各層の透過係数$R_l$, 反射係数$T_l$, 放射源関数$\epsilon^\pm_l$
が全ての層$l$で求められると,
adding 法を用いて各層境界での放射フラックスを求めることができる.
これは, 2つの層の$R,T,\epsilon$ がわかっていると,
2つの層を合成した層全体の$R,T,\epsilon$ が簡単な計算により
求められることを利用したものである.
均質な層では, 上から入射した場合の反射率, 透過率と
下から入射した場合の反射率, 透過率とは同じであるが,
複数の層を合成した不均質な層では異なるため,
上から入射した場合の反射率, 透過率 $R^+, T^+$ と
下から入射した場合の反射率, 透過率 $R^-, T^-$ とを区別する.
今, 上の層1 と 下の層2 でこれら
$R^\pm_1, T\pm_1, \epsilon^\pm_1,
 R^\pm_2, T\pm_2, \epsilon^\pm_2$ が既知であると,
合成した層での値
$R^\pm_{1,2}, T\pm_{1,2}, \epsilon^\pm_{1,2}$ は
以下のようになる.
\begin{eqnarray}
  R^+_{1,2} & = & R^+_1 + T^-_1 ( 1- R^+_2 R^-_1 )^{-1} R^+_2 T^+_1 \\
  R^-_{1,2} & = & R^-_2 + T^+_2 ( 1- R^+_1 R^-_2 )^{-1} R^-_1 T^-_2 \\
  T^+_{1,2} & = & T^+_2 ( 1- R^+_1 R^-_2 )^{-1} T^+_1 \\
  T^-_{1,2} & = & T^-_1 ( 1- R^+_1 R^-_2 )^{-1} T^-_2 \\
  \epsilon^+_{1,2} & = & \epsilon^+_2 
    + T^+_2 ( 1- R^+_2 R^-_1 )^{-1} ( R^-_1 \epsilon^-_2 + \epsilon^+_1 ) \\
  \epsilon^-_{1,2} & = & \epsilon^-_1 
    + T^-_1 ( 1- R^+_2 R^-_1 )^{-1} ( R^+_2 \epsilon^+_1 + \epsilon^-_2 ) 
\end{eqnarray}

上から1, 2, \ldots $N$ 層まであるとする. 
ただし, 地表を一つの層と考え, 第$N$層とする.
第$n$層から$N$層までを1つの層と考えたときの反射率, 放射源関数
$R^+_{n,N}, \epsilon^-_{n,N}$ を考えると,
\begin{eqnarray}
  R^+_{n,N} & = & R^+_n 
      + T^-_n ( 1- R^+_{n+1,N} R^-_n )^{-1} R^+_{n+1,N} T^+_n \\
  \epsilon^-_{n,N} & = & \epsilon^-_n
    + T^-_n ( 1- R^+_{n,N} R^-_n )^{-1} 
      ( R^+_{n,N} \epsilon^+_n + \epsilon^-_{n,N} ) 
\end{eqnarray}
これは, 地表での値
\begin{eqnarray}
  R^+_{N,N} & = &  R^+_N = 2 {W^+}^2 \alpha_s \\
  \epsilon^-_{N,N} & = &  \epsilon^-_N 
    = W^+ \left( 2 \alpha_s \mu_0 e^{-<\tau^*>/\mu_0} F_0 
                 + 2 \pi (1-\alpha_s) B_N 
          \right)
\end{eqnarray}
から出発して, 順次 $n=N-1, \ldots 1$ で解くことができる.
ただし,
\begin{equation}
  W^+ \equiv \mu^{1/2}
\end{equation}


次に, 第1層から第$n$ 層までを1つの層と考えたときの反射率, 放射源関数
$R^-_{1,n}, \epsilon^+_{1,n}$ を考えると,
\begin{eqnarray}
  R^-_{1,n} & = & R^-_n 
      + T^+_n ( 1- R^+_{1,n-1} R^-_n )^{-1} R^-_{1,n-1} T^-_n \\
  \epsilon^+_{1,n} & = & \epsilon^+_n
    + T^+_n ( 1- R^+_{1,n-1} R^-_n )^{-1} 
      ( R^-_{1,n-1} \epsilon^-_n + \epsilon^+_{1,n-1} ) 
\end{eqnarray}
となり, これも $R^-_{1,1} = R^-_1, \epsilon^+_{1,1} = \epsilon^+_1$
から出発して順次 $n=2, \ldots N$ で解くことができる.

これらを用いると,
層$n$と$n+1$の境界における下向きのフラックス $u^+_{n,n+1}$
および上向きのフラックス $u^-_{n,n+1}$ は,
$1\sim n$ 層を組み合わせた層と
$n+1\sim N$ 層を組み合わせた層の2つの層の間の問題に還元され,
\begin{eqnarray}
 u^+_{n+1/2} = (1-R^-_{1,n} R^+_{n+1,N})^{-1}
    (\epsilon^+_{1,n} + R^-_{1,n} \epsilon^-_{n+1,N} ) \\
 u^-_{n+1/2} = R^+_{n+1,N}  u^+_{n,n+1} + \epsilon^-_{n+1,N}
\end{eqnarray}
と書き表すことができる.
ただし, 大気上端でのフラックスは,
\begin{eqnarray}
 u^+_{1/2} & = & 0 \\
 u^-_{1/2} & = & \epsilon^-_{1,N}
\end{eqnarray}

最後にこのフラックスはスケールされたものであるので,
再スケーリングを行ない, さらに直達太陽入射を加えて
放射フラックスを求める.

\begin{eqnarray}
  F^+_{n+1/2} & = & \frac{W^+}{\bar{W}} u^+_{n+1/2} 
                + \mu_0 e^{-<\tau^*>_{1,n}/\mu_0} F_0 \\\\
  F^-_{n+1/2} & = & \frac{W^+}{\bar{W}} u^-_{n+1/2} \\
\end{eqnarray}

この計算は, サブチャネルごとに行なう.

\subsubsection{フラックスの足し込み}

各層のサブチャネルごとの放射フラックス$F^\pm_c$が求められると,
それをサブチャネルの代表する波長幅に対応する
重み$w_c$をかけて足し合わせることにより,
波長積分のフラックスが求められる.

\begin{equation}
  \bar{F}^\pm = \sum_c w_c F^\pm
\end{equation}

実際には, 短波長域(太陽光領域), 
長波長域(地球放射領域)に分けて足し合わせる.
また, 短波長域の一部(波長$0.7\mu$より短い領域)の
地表での下向きフラックスを PAR (光合成活性放射)として得る.

\subsubsection{フラックスの温度微分}

地表面温度を implicit で解くために,
上向きフラックスの地表面温度に対する微分項
$dF^-/dT_g$ を計算する.
そのために, $T_g$より1K高い温度に対する値 
$\overline{B}^w(T_g+1)$ も求め, それを用いて
adding 法によるフラックスの計算をやりなおし,
元の値との差を $dF^-/dT_g$ とする.
これは長波長域(地球放射領域)のみ意味のある値となる.

\subsubsection{雲量の取扱い}

CCSR/NIES AGCM では,
1つの格子の中での雲の水平方向の被覆率を考慮している.
雲は以下の2種類である.
\begin{enumerate}
\item 層雲. 大規模凝結スキーム \Module{LSCOND} で診断される.
      各層($n$)ごとに格子平均の雲水量 $l^l_n$ と
      水平被覆率(雲量) $C^l_n$ が定義される.      
\item 積雲. 積雲対流スキーム \Module{CUMLUS} で診断される.
      各層($n$)ごとに格子平均の雲水量 $l^c_n$ が定義されるが,
      水平被覆率(雲量) $C^c$ は鉛直に一定とする.
\end{enumerate}
これらの取扱いにおいて, 層雲は鉛直にランダムに重なり合うと仮定し,
積雲は上下層で常に同じ領域を占めると仮定する
(その領域の中に限れば雲量は0もしくは1であるとする).
そのために, 以下のように計算を行なう.

\begin{enumerate}
\item レーリーおよび粒子散乱・吸収の光学的厚さ等
      $\tau^s, \tau_s^s, g, f$ を,
      \begin{enumerate}
      \item 雲水量$l^l_n/C^l_n$の雲が存在する場合(層雲)
      \item 雲の全くない場合
      \item 雲水量$l^c_n/C^c$の雲が存在する場合(積雲)
      \end{enumerate}
      について計算する.

\item 各層の反射係数, 透過係数, 
      放射源関数(プランク関数起源, 日射起源)を
      上の3つの場合についてそれぞれ計算する.
      雲なしの場合の値を
      $R^\circ$, 層雲のある場合を$R^l$, 積雲のある場合を
      $R^c$ などとする.

\item 各層の反射係数, 透過係数, 
      放射源関数を, 層雲の雲量の重み$C^l$をつけて平均する.
      平均したものを $\bar{}$をつけて表すと,
      \begin{eqnarray}
        \bar{R} & = & ( 1 - C^l ) R^\circ + C^l R^l \\
        \bar{T} & = & ( 1 - C^l ) T^\circ + C^l T^l \\
        \bar{\epsilon} & = & 
            ( 1 - C^l ) \epsilon_A^\circ + C^l \epsilon_A^l \\        
          & + & 
            \left[ ( 1 - C^l ) \epsilon_S^\circ + C^l \epsilon_S^l \right] 
            e^{-\overline{<\tau^*>}/\mu_0} F_0 
      \end{eqnarray}
      とする. ただし,
      \begin{equation}
        e^{-\overline{<\tau^*>}/\mu_0} 
        = \Pi' \left[ ( 1 - C^l ) e^{-\tau^{*\circ}/\mu_0} 
                       + C_l e^{-\tau^{*l}/\mu_0} \right]
      \end{equation}
      である. 
      また,
      \begin{eqnarray}
        \epsilon^\circ & = & \epsilon_A^\circ +
                             \epsilon_S^\circ 
                              e^{-<\tau^{*\circ}>/\mu_0} F_0 \\
        \epsilon^c     & = & \epsilon_A^c +
                             \epsilon_S^c 
                              e^{-<\tau^{*c}>/\mu_0} F_0        
      \end{eqnarray}
      も求める.

\item 平均の特性値($\bar{R}$など)を用いた場合,
      雲なしの特性値($R^\circ$など)を用いた場合,
      積雲の特性値($R^c$など)を用いた場合について,
      それぞれadding によってフラックス
      $\bar{F}, F^\circ, F^c$ を求める.
      
\item 最終的に求めるフラックスは
      \begin{equation}
        F = ( 1 - C^c ) \bar{F} + C^c F^c
      \end{equation}
      ($F^\circ$ は cloud radiative forcing の見積りのために
       計算している)

\end{enumerate}

\subsubsection{入射フラックスと入射角 \Module{SHTINS}}

入射フラックス $F_0$ は,
太陽定数を $F_{00}$, 
太陽地球間の距離の, 
その時間平均値との比を $r_s$ とすると.
%
\begin{equation}
F_0 = F_00 r_s^-2 
\end{equation}
ここで, $r_s$ は以下のように求める.
%
\begin{equation}
  M \equiv 2 \pi ( d - d_0 ) 
\end{equation}
として,
\begin{equation}
  r_s = a_0 - a_1 \cos M - a_2 \cos 2M - a_3 \cos 3M
\end{equation}
ただし, $d$ は年初から日単位で表した時刻である.

また,入射角は以下のように求める.
太陽の角度位置 $\omega_s$ を
\begin{equation}
  \omega_s = M + b_1 \sin M + b_2 \sin 2M + b_3 \sin 3M
\end{equation}
として,  太陽の赤緯 $\delta_s$ は,
\begin{equation}
  \sin \delta_s = \sin \epsilon \sin ( \omega_s - \omega_0 ) 
\end{equation}
%
すると, 入射角ファクタ $\mu = \cos \zeta$ ($\zeta$ は天頂角)は,
\begin{equation}
\mu = \cos \zeta = \cos \varphi \cos \delta_s \cos h
                 + \sin \varphi \sin \delta_s
\end{equation}
$\varphi$ は緯度,
$h$は時角(地方時から $\pi$ を引いたもの)である.

以上において, 地球軌道の離心率を$e$とすると(Katayama, 1974),
\begin{eqnarray}
   a_0 & = &  1 + e^2 \\
   a_1 & = &  e - 3/8 e^3 - 5/32 e^5 \\
   a_2 & = &  1/2 e^2 - 1/3e^4 \\
   a_3 & = &  3/8 e^3 - 135/64^5 \\
   b_1 & = & 2e - 1/4 e^3 + 5/96 e^5 \\
   b_2 & = & 5/4 e^2 - 11/24 e^4 \\
   b_3 & = & 13/12 e^3 - 645/940 e^5 \\
\end{eqnarray}

年平均日射を与えることも可能である.
この場合, 年平均入射量および年平均入射角は, 
近似的に次のようになる.
%
\begin{equation}
\overline{F} = F_{00}/\pi
\end{equation}
%
\begin{equation}
\overline{\mu} \simeq 0.410 + 0.590 \cos^2 \varphi .
\end{equation}

\subsubsection{その他の留意点}

\begin{enumerate}
\item 放射の計算は通常, 毎ステップ行なうわけではない.
      そのために, 放射フラックスをセーブしておき, 
      放射計算をしない時刻にはそれを用いる.
      その際, 短波放射に関しては,
      次回の計算時刻との間の可照時間($\mu_0>0$である時間)の割合$f$と
      可照時間内で平均した太陽入射角ファクタ $\bar{\mu_0}$を用いて
      フラックス $\bar{F}$ を求め,
      \begin{equation}
        F =  f \frac{\mu_0}{\bar{\mu_0}} \bar{F}
      \end{equation}
      とする.


\item 雲水は, 温度に依存して, 
      水雲粒および氷雲粒として扱われる.
      氷雲として扱われる割合$f_I$は,
      \begin{equation}
        f_I = \frac{ T_0 - T }{ T_0 - T_1 }
      \end{equation}
      (ただし, 最大値1, 最小値0) である. また,
      $T_0 = 273.15\mbox{K}, T_1 = 258.15\mbox{K}$ とする.

\end{enumerate}





	% 物理過程:拡散フラックス
	\hypertarget{turbulence-scheme}{%
\subsection{Turbulence Scheme}\label{turbulence-scheme}}

A turbulence scheme represents the effect of subgrid-scale turbulence on the grid-mean prognostic variables. It calculates the vertical diffusion of momentum, heat, water and other tracers. The
Mellor-Yamada-Nakanishi-Niino scheme (MYNN scheme; Nakanishi 2001; Nakanishi and Niino 2004) has been used as a turbulence scheme in MIROC since its version 5, which is an improved version of the
Mellor-Yamada scheme (Mellor 1973; Mellor and Yamada 1974; Mellor and Yamada 1982). Its closure level is 2.5. Level 3 is available, however it was not adopted as a standard option, since we could not
gain large benefits despite its much greater computational costs.

In the MYNN scheme, liquid water potential temperature \(\theta_l\) and total water \(q_w\) are used as key variables, which are defined as

\begin{eqnarray} \theta_l \equiv \left(T - \frac{L_v}{C_p}q_l - \frac{L_v+L_f}{C_p}q_i \right) \left(\frac{p_s}{p}\right)^{\frac{R_d}{C_p}}, \end{eqnarray}

\begin{eqnarray} q_w \equiv q_v+q_l+q_i, \end{eqnarray}

where \(T\) and \(p\) are temperature and pressure; \(q_v\), \(q_l\) and \(q_i\) are specific humidity, liquid water content, and ice water content respectively; \(C_p\) and \(R_d\) are specific heat
at constant pressure and gas constant of dry air respectively; \(L_v\) and \(L_f\) are latent heat of vaporization and fusion per unit mass respectively. \(p_s\) is \(1000hPa\). These variables
conserve for the phase change of water.

In the level 2.5, the scheme predicts the time evolution of twice turbulent kinetic energy as a prognostic variable, which is defined by

\begin{eqnarray}q^2 \equiv \langle u^2 + v^2 + w^2 \rangle\end{eqnarray}

where \(u\), \(v\), and \(w\) are velocities in zonal, meridional and vertical directions respectively. In this chapter, uppercase letters represent grid-mean variables and lowercase counterparts the
deviation from the grid-mean. \(\langle \ \rangle\) denotes an ensemble mean. In the Level 3, \(\langle {\theta_l}^2 \rangle\), \(\langle {q_w}^2 \rangle\), \(\langle \theta_l q_w \rangle\) are also
predicted, but we skip the details here.

The outline of the computational procedures is given as follows along with the names of the subroutines. All the subroutines listed here are written in a Fortran source code of pvdfm.F.

\begin{enumerate}
\def\labelenumi{\arabic{enumi}.}
\tightlist
\item
  Calculation of friction velocity and the Obukhov length
\item
  Calculation of buoyancy coefficients {[}SUBROUTINE:\texttt{VDFCND}{]}
\item
  Calculation of stability functions of the Level 2 {[}SUBROUTINE:\texttt{VDFLEV2}{]}
\item
  Calculation of planetary boundary layer depth {[}SUBROUTINE:\texttt{PBLHGT}{]}
\item
  Calculation of master length scale {[}SUBROUTINE:\texttt{VDFMLS}{]}
\item
  Calculation of diffusion coefficients, vertical fluxes and their derivatives {[}SUBROUTINE:\texttt{VDFLEV3}{]}
\item
  Calculation of production and dissipation terms of twice turbulent kinetic energy {[}SUBROUTINE:\texttt{VDFLEV3}{]}
\item
  Calculation of tendencies of prognostic variables with implicit scheme
\end{enumerate}

\hypertarget{surface-layer}{%
\subsubsection{Surface Layer}\label{surface-layer}}

The friction velocity \(u_*\) and the Obukhov length \(L_M\) are given as

\begin{eqnarray}u_*=\left({\langle uw \rangle_g}^2+{\langle vw \rangle_g}^2 \right)^\frac{1}{4},\end{eqnarray}

\begin{eqnarray}L_M=-\frac{\Theta_{v,g} {u_*}^3}{kg \langle w\theta_v \rangle_g},\end{eqnarray}

where the subscript \(g\) indicates the values near the surface \(\Theta_v\) and \(\theta_v\) denote virtual potential temperature, \(k\) the von Kármán constant, and \(g\) the acceleration of
gravity. The values of the lowest model layer is used for \(\Theta_{v,g}\).

\hypertarget{calculation-of-the-buoyancy-coefficients}{%
\subsubsection{Calculation of the Buoyancy Coefficients}\label{calculation-of-the-buoyancy-coefficients}}

The buoyancy-production term in the prognostic equation of the twice turbulent kinetic energy contains \(\langle w\theta_v \rangle\). Following Mellor and Yamada (1982), we assume the probability
distribution of \(\theta_l\) and \(q_w\) in a given grid and rewrite this term as

\begin{eqnarray}\langle w\theta_v \rangle=\beta_\theta \langle w\theta_l \rangle + \beta_q \langle wq_w \rangle.\end{eqnarray}

However, note that unlike Mellor and Yamada (1982) and Nakanishi and Niino (2004), the probability distribution assumed here is not Gaussian. It is triangular documented in the PDF-based prognostic
large-scale condensation scheme (Watanabe et al.~2009). In this case, the buoyancy coefficients, \(\beta_\theta\) and \(\beta_q\) are written as

\begin{eqnarray}\beta_\theta=1+\epsilon Q_w-(1+\epsilon)Q_l-Q_i-\tilde{R}abc,\end{eqnarray}

\begin{eqnarray}\beta_q=\epsilon \Theta +\tilde{R}ac,\end{eqnarray}

where \(\epsilon=R_v/R_d-1\). \(R_v\) is the gas constant for water vapor, and

\begin{eqnarray}a=\left(1+\frac{L_v}{C_p}\left.\frac{\partial Q_s}{\partial T}\right|_{T=T_l}\right)^{-1},\end{eqnarray}

\begin{eqnarray}b=\frac{T}{\Theta}\left.\frac{\partial Q_s}{\partial T}\right|_{T=T_l},\end{eqnarray}

\begin{eqnarray}c=\frac{\Theta}{T}\frac{L_v}{C_p}\left[1+\epsilon Q_w-(1+\epsilon)Q_l-Q_i\right]-(1+\epsilon)\Theta,\end{eqnarray}

\begin{eqnarray}\tilde{R}=R\left\{1-a\left[Q_w-Q_s(T_l)\right]\frac{Q_l}{2\sigma_s}\right\}-\frac{{Q_l}^2}{4{\sigma_s}^2},\end{eqnarray}

\begin{eqnarray}{\sigma_s}^2=\langle {q_w}^2 \rangle -2b \langle \theta_l q_w \rangle + b^2\langle {\theta_l}^2 \rangle,\end{eqnarray}

where \(R\) and \(Q_l\) are cloud amount and liquid water computed from the probability distribution in the grids, respectively, and \(Q_s\) is saturation water vapor.

\hypertarget{stability-functions-for-the-level-2}{%
\subsubsection{Stability Functions for the Level 2}\label{stability-functions-for-the-level-2}}

It is known that the Mellor-Yamada Level 2.5 scheme fails to capture the behavior of growing turbulence realistically (Helfand and Labraga 1988). Thus, the MYNN scheme first calculates the twice
turbulent kinetic energy of the Level2 \({q_2}^2\), and then make a correction to the diffusion when \(q<q_2\), i.e., the turbulence is in a growing phase. The stability functions of the level 2,
\(S_{H2}\) and \(S_{M2}\), required for the calculation of \(q_2\), are represented by

\begin{eqnarray}S_{H2}=S_{HC}\frac{Rf_c-Rf}{1-Rf},\end{eqnarray}

\begin{eqnarray}S_{M2}=S_{MC}\frac{R_{f1}-Rf}{R_{f2}-Rf}S_{H2},\end{eqnarray}

where \(Rf\) is the flux Richardson number and calculated as

\begin{eqnarray}Rf=R_{i1}\left[Ri+R_{i2}-(Ri^2-R_{i3}Ri+{R_{i2}}^2)^{1/2}\right].\end{eqnarray}

Here, \(Ri\) is the gradient Richardson number represented by

\begin{eqnarray}Ri=\frac{g}{\Theta}\left(\beta_\theta \frac{\partial \Theta_l}{\partial z}+\beta_q \frac{\partial Q_w}{\partial z}\right) \Bigg/ \left[ \left(\frac{\partial U}{\partial z}\right)^2+\left(\frac{\partial V}{\partial z}\right)^2 \right].\end{eqnarray}

The other symbols indicate quantities independent of the environmental field, which are given as follows.

\begin{eqnarray}S_{HC}=3A_2(\gamma_1+\gamma_2),\end{eqnarray}

\begin{eqnarray}S_{MC}=\frac{A_1}{A_2}\frac{F_1}{F_2},\end{eqnarray}

\begin{eqnarray}Rf_c=\frac{\gamma_1}{\gamma_1+\gamma_2},\end{eqnarray}

\begin{eqnarray}R_{f1}=B_1\frac{\gamma_1-C_1}{F_1},\end{eqnarray}

\begin{eqnarray}R_{f2}=B_1\frac{\gamma_1}{F_2},\end{eqnarray}

\begin{eqnarray}R_{i1}=\frac{1}{2S_{Mc}},\end{eqnarray}

\begin{eqnarray}R_{i2}=R_{f1}S_{MC},\end{eqnarray}

\begin{eqnarray}R_{i3}=4R_{f2}S_{MC}-2R_{i2},\end{eqnarray}

where

\begin{eqnarray}A_1=B_1\frac{1-3\gamma_1}{6},\end{eqnarray}

\begin{eqnarray}A_2=A_1\frac{\gamma_1-C_1}{\gamma_1 Pr},\end{eqnarray}

\begin{eqnarray}C_1=\gamma_1-\frac{1}{3A_1{B_1}^{\frac{1}{3}}},\end{eqnarray}

\begin{eqnarray}F_1=B_1(\gamma_1-C_1)+2A_1(3-2C_2)+3A_2(1-C_2)(1-C_5),\end{eqnarray}

\begin{eqnarray}F_2=B_1(\gamma_1+\gamma_2)-3A_1(1-C_2),\end{eqnarray}

\begin{eqnarray}\gamma_2=\frac{B_2}{B_1}\left(1-C_3\right)+\frac{2A_1}{B_1}\left(3-2C_2\right),\end{eqnarray}

and

\begin{eqnarray}
(Pr,\gamma_1,B_1,B_2,C_2,C_3,C_4,C_5)=(0.74,0.235,24.0,15.0,0.7,0.323,0.0,0.2).
\end{eqnarray}

\hypertarget{master-length-scale}{%
\subsubsection{Master Length Scale}\label{master-length-scale}}

\hypertarget{original-formulation-by-nakanishi-2001}{%
\paragraph{Original Formulation by Nakanishi (2001)}\label{original-formulation-by-nakanishi-2001}}

Nakanishi (2001) proposed the following formulation for the master length scale \(L\).

\begin{eqnarray}\frac{1}{L}=\frac{1}{L_S}+\frac{1}{L_T}+\frac{1}{L_B} \label{p-dif.1}, \end{eqnarray}

where \(L_S\), \(L_T\), \(L_B\) represent length scales in the surface layer, convective boundary layer, and stably stratified layer respectively. These length scales are formulated as

\begin{eqnarray}
L_S=\left\{
    \begin{array}{lr}
      kz/3.7 &\zeta\ge 1\\
      kz/(2.7+\zeta) &0\le\zeta< 1\\
      kz(1-\alpha_4\zeta)^{0.2} &\zeta< 0,
    \end{array}
  \right.
\end{eqnarray}

\begin{eqnarray}L_T=\alpha_1\frac{\displaystyle \int_0^\infty{qz}\,dz}{\displaystyle \int_0^\infty{q}\,dz},\end{eqnarray}

\begin{eqnarray}
L_B=\left\{
    \begin{array}{ll}
      \alpha_2 q/N &\partial\Theta_v/\partial z> 0 \quad\rm{and}\quad\zeta\ge 0\\
      \left[\alpha_2+\alpha_3(q_c/L_TN)^{1/2}\right]q/N &\partial\Theta_v/\partial z> 0 \quad\rm{and}\quad\zeta< 0\\
      \infty &\partial\Theta_v/\partial z\le 0,
    \end{array}
  \right.
\end{eqnarray}

where \(\zeta\equiv z/L_M\) is a height normalized by the Monin-Obukhov length \(L_M\), \(N\equiv\left[(g/\Theta)(\partial\Theta_v/\partial z)\right]^{1/2}\) is the Brunt-Väisälä frequency and
\(q_c\equiv [(g/\Theta)\langle w\theta_v \rangle_gL_T]^{1/3}\) is a velocity scale in the convective boundary layer.

\hypertarget{modifications-in-miroc}{%
\paragraph{Modifications in MIROC}\label{modifications-in-miroc}}

The above formulation works well when the domain of the model is limited to the planetary boundary layer (PBL) and its surrounding area. However, if the the upper troposphere is included, the
formulation gives inappropriate behaviors depending on the conditions: e.g.~\(L_T\), the length scale of the convective boundary layer, is used in the free atmosphere, and the turbulent kinetic energy
in the free atmosphere is taken into account in the calculation of \(L_T\).

In order to avoid such misbehaviors, the top height of the convective boundary layer \(H_{PBL}\) is estimated in MIROC and we consider that the region below
\(h=\left[(F_H H_{PBL})^2+H_0^2\right)^{1/2}\) is the one where the PBL-derived turbulence is dominant. Here, we adopted \(F_H=1.5\) and \(H_0=500\)m.

Below the altitude \(h\), equation (\ref{p-dif.1}) is used as the master length scale, but the vertical range of the integration in \(L_T\) is modified as

\begin{eqnarray}L_T=\alpha_1\frac{\displaystyle \int_0^h{qz}\,dz}{\displaystyle \int_0^h{q}\,dz},\end{eqnarray}

and then the master length scale above \(h\) is represented as

\begin{eqnarray}\frac{1}{L}=\frac{1}{L_S}+\frac{1}{L_A}+\frac{1}{L_{max}}\end{eqnarray}

where \(L_A=\alpha_5\,q/N\) is a length scale of air parcel vertically transported by turbulence in a stably stratified layer. \(\alpha_5\) represents the effect of dissipation set to \(0.53\).
\(L_{max}=500\)m gives the upper limit of \(L\).

\hypertarget{estimation-of-the-top-height-of-the-convective-boundary-layer}{%
\paragraph{Estimation of the Top Height of the Convective Boundary Layer}\label{estimation-of-the-top-height-of-the-convective-boundary-layer}}

Based on Holtslag and Boville (1993), \(H_{PBL}\) is estimated using the bulk Richardson number \(Ri_B\) given as

\begin{eqnarray}Ri_B=\frac{[g/\Theta_v(z_1)][\Theta_v(z_k)-\Theta_{v,g}](z_k-z_g)}{[U(z_k)-U(z_1)]^2+[V(z_k)-V(z_1)]^2+F_u{u_*}^2},\end{eqnarray}

where \(z_k\) is the altitude of a \(k\)-th model layer from the bottom at full level, \(z_1\) the altitude of the lowest layer at full level, \(z_g\) the altitude of the surface. \(F_u\) is a
dimensionless tuning parameter, and

\begin{eqnarray}\Theta_{v,g}=\Theta_v(z_1)+F_b \frac{\langle w\theta_v \rangle_g}{w_m},\end{eqnarray}

\begin{eqnarray}w_m=u_*/\phi_m,\end{eqnarray}

\begin{eqnarray}\phi_m=\left(1-15\frac{z_s}{L_M}\right)^{-\frac{1}{3}},\end{eqnarray}

where \(z_s\) is the altitude of the surface layer assumed to be \(0.1H_{PBL}\). \(F_b\) is a dimensionless tuning parameter.

\(Ri_B\) is successively calculated from \(k=2\) upward, and then if \(Ri_B\) exceeds \(0.5\) for the first time, it is linearly interpolated between this level and the level immediately below it. The
height satiffying \(Ri_B=0.5\) is used as \(H_{PBL}\). Since \(H_{PBL}\) is necessary for the calculation of \(z_s\), we first calculate \(z_s\) using a temporary value of \(H_{PBL}=z_1-z_g\), from
which we calculate the first guess of \(H_{PBL}\). Then we use this value for the recalculation of \(z_s\), and then it is used for the final estimate of \(H_{PBL}\).

\hypertarget{calculation-of-diffusion-coefficients}{%
\subsubsection{Calculation of Diffusion Coefficients}\label{calculation-of-diffusion-coefficients}}

\hypertarget{twice-turbulent-kinetic-energy-of-level-2}{%
\paragraph{Twice Turbulent Kinetic Energy of Level 2}\label{twice-turbulent-kinetic-energy-of-level-2}}

The twice turbulent kinetic energy of the level 2, \({q_2}^2\), is calculated from the following equation, which neglects the time derivative, advection and diffusion terms in the prognostic equation
of the twice turbulent kinetic energy.

\begin{eqnarray} P_s + P_b - \varepsilon = 0 \label{p-dif.2}, \end{eqnarray}

where \(P_s\) and \(P_b\) denote the production terms by shear and buoyancy respectively. \(\varepsilon\) is the dissipation term. \(P_s\) and \(P_b\) are written as

\begin{eqnarray} P_s = -\langle wu \rangle \frac{\partial U}{\partial z} - \langle wv \rangle \frac{\partial V}{\partial z}, \end{eqnarray}

\begin{eqnarray} P_b = \frac{g}{\Theta}\langle w\theta_v \rangle, \end{eqnarray}

respectively. In the level 2 of the MYNN scheme, these are represented as

\begin{eqnarray} P_s = LqS_{M2} \left[ \left(\frac{\partial U}{\partial z}\right)^2 + \left(\frac{\partial V}{\partial z}\right)^2 \right] \label{p-dif.3}, \end{eqnarray}

\begin{eqnarray} P_b = LqS_{H2} \frac{g}{\Theta}\left[ \beta_\theta \frac{\partial \Theta_l}{\partial z} + \beta_q \frac{\partial Q_w}{\partial z} \right] \label{p-dif.4}, \end{eqnarray}

\begin{eqnarray} \varepsilon = \frac{q^3}{B_1 L} \label{p-dif.5}. \end{eqnarray}

From (\ref{p-dif.2}), (\ref{p-dif.3}), (\ref{p-dif.4}), and (\ref{p-dif.5}), \(q_2^2\) is calculated by

\begin{eqnarray}{q_2}^2=B_1L^2\left\{S_{M2}\left[\left(\frac{\partial U}{\partial z}\right)^2+\left(\frac{\partial V}{\partial z}\right)^2\right]+S_{H2}\frac{g}{\Theta}\left(\beta_\theta \frac{\partial \Theta_l}{\partial z}+\beta_q \frac{\partial Q_w}{\partial z}\right)\right\}.\end{eqnarray}

\hypertarget{stability-functions-of-the-level-2.5}{%
\paragraph{Stability Functions of the Level 2.5}\label{stability-functions-of-the-level-2.5}}

When \(q<q_2\), i.e., the turbulence is in a growing phase, the stability functions of the Level 2.5 for momentum and heat, \(S_M\) and \(S_H\) respectively, are calculated using \(\alpha=q/q_2\)
introduced by Helfand and Labraga (1998) as

\begin{eqnarray}S_M=\alpha S_{M2},\end{eqnarray}

\begin{eqnarray}S_H=\alpha S_{H2}.\end{eqnarray}

When \(q \geq q_2\), \(S_M\) and \(S_H\) are calculated as

\begin{eqnarray}S_M=A_1\frac{E_3-3C_1 E_4}{E_2 E_4+E_5 E_3},\end{eqnarray}

\begin{eqnarray}S_H=A_2\frac{E_2+3C_1 E_5}{E_2 E_4+E_5 E_3},\end{eqnarray}

where

\begin{eqnarray}E_1=1-3A_2B_2(1-C_3)G_H,\end{eqnarray}

\begin{eqnarray}E_2=1-9A_1A_2(1-C_2)G_H,\end{eqnarray}

\begin{eqnarray}E_3=E_1+9{A_2}^2(1-C_2)(1-C_5)G_H,\end{eqnarray}

\begin{eqnarray}E_4=E_1-12A_1A_2(1-C_2)G_H,\end{eqnarray}

\begin{eqnarray}E_5=6{A_1}^2G_M,\end{eqnarray}

\begin{eqnarray}G_M=\frac{L^2}{q^2}\left[\left(\frac{\partial U}{\partial z}\right)^2+\left(\frac{\partial V}{\partial z}\right)^2\right],\end{eqnarray}

\begin{eqnarray}G_H=-\frac{L^2}{q^2}\frac{g}{\Theta}\left(\beta_\theta \frac{\partial \Theta_l}{\partial z}+\beta_q \frac{\partial Q_w}{\partial z}\right).\end{eqnarray}

The above formulas appear to be different from those in Nakanishi (2001), but are equivalent and can be computed with a smaller computational cost.

\hypertarget{calculation-of-diffusion-coefficients-1}{%
\paragraph{Calculation of Diffusion Coefficients}\label{calculation-of-diffusion-coefficients-1}}

The diffusion coefficients for momentum, twice turbulent kinetic energy, heat and water are represented by

\begin{eqnarray}K_M=LqS_M,\end{eqnarray}

\begin{eqnarray}K_q=3LqS_M,\end{eqnarray}

\begin{eqnarray}K_H=LqS_H,\end{eqnarray}

\begin{eqnarray}K_w=LqS_H,\end{eqnarray}

respectively.

\hypertarget{calculation-of-fluxes}{%
\paragraph{Calculation of Fluxes}\label{calculation-of-fluxes}}

The vertical fluxes for \(U\), \(V\), \(q^2\), \(C_pT\) and \(Q_w\) at half levels are calculated as

\begin{eqnarray}F_{u,k-1/2}=-\rho_{k-1/2}K_{M,k-1/2}\frac{U_{k}-U_{k-1}}{\Delta z_{k-1/2}},\end{eqnarray}

\begin{eqnarray}F_{v,k-1/2}=-\rho_{k-1/2}K_{M,k-1/2}\frac{V_{k}-V_{k-1}}{\Delta z_{k-1/2}},\end{eqnarray}

\begin{eqnarray}F_{q,k-1/2}=-\rho_{k-1/2}K_{q,k-1/2}\frac{{q^2}_ {k}-{q^2}_ {k-1}}{\Delta z_{k-1/2}},\end{eqnarray}

\begin{eqnarray}F_{T,k-1/2}=-\rho_{k-1/2}K_{H,k-1/2}\,C_p\Pi_{k-1/2}\frac{\Theta_{l,k}-\Theta_{l,k-1}}{\Delta z_{k-1/2}},\end{eqnarray}

\begin{eqnarray}F_{w,k-1/2}=-\rho_{k-1/2}K_{w,k-1/2}\frac{Q_{w,k}-Q_{w,k-1}}{\Delta z_{k-1/2}},\end{eqnarray}

respectively, where \(\rho\) denotes density and \(\Pi\) the Exner function. In order to perform time integration with an implicit scheme, the derivative of each of the vertical fluxes is also
calculated as

\begin{eqnarray}\frac{\partial F_{u,k-1/2}}{\partial U_{k-1}}=\frac{\partial F_{v,k-1/2}}{\partial V_{k-1}}=-\frac{\partial F_{u,k-1/2}}{\partial U_{k}}=-\frac{\partial F_{v,k-1/2}}{\partial V_{k}}=\rho_{k-1/2}K_{M,k-1/2}\frac{1}{\Delta z_{k-1/2}},\end{eqnarray}

\begin{eqnarray}\frac{\partial F_{q,k-1/2}}{\partial {q^2}_ {k-1}}=-\frac{\partial F_{q,k-1/2}}{\partial {q^2}_ {k}}=\rho_{k-1/2}K_{q,k-1/2}\frac{1}{\Delta z_{k-1/2}},\end{eqnarray}

\begin{eqnarray}\frac{\partial F_{T,k-1/2}}{\partial T_{k-1}}=\rho_{k-1/2}K_{H,k-1/2}C_p\frac{\Pi_{k-1/2}}{\Pi_{k-1}}\frac{1}{\Delta z_{k-1/2}},\end{eqnarray}

\begin{eqnarray}\frac{\partial F_{T,k-1/2}}{\partial T_{k}}=-\rho_{k-1/2}K_{H,k-1/2}C_p\frac{\Pi_{k-1/2}}{\Pi_{k}}\frac{1}{\Delta z_{k-1/2}},\end{eqnarray}

\begin{eqnarray}\frac{\partial F_{w,k-1/2}}{\partial Q_{w,k-1}}=-\frac{\partial F_{w,k-1/2}}{\partial Q_{w,k}}=\rho_{k-1/2}K_{w,k-1/2}\frac{1}{\Delta z_{k-1/2}},\end{eqnarray}

where \(\Delta z_{k-1/2}=z_k-z_{k-1}\). The fluxes for other tracers are also calculated in the same way using \(K_w\).

\hypertarget{calculation-of-turbulent-variables}{%
\subsubsection{Calculation of Turbulent Variables}\label{calculation-of-turbulent-variables}}

\hypertarget{calculation-of-twice-turbulent-kinetic-energy}{%
\paragraph{Calculation of Twice Turbulent Kinetic Energy}\label{calculation-of-twice-turbulent-kinetic-energy}}

The prognostic equation for \(q^2\) is expressed as

\begin{eqnarray} \frac{d q^2}{dt}=-\frac{1}{\rho}\frac{\partial F_q}{\partial z}+2\left(P_s+P_b-\varepsilon\right). \end{eqnarray}

In the Level 2.5, \(P_s,P_b,\varepsilon\) are written as

\begin{eqnarray}P_s=Lq S_M \left[\left(\frac{\partial U}{\partial z}\right)^2+\left(\frac{\partial V}{\partial z}\right)^2\right],\end{eqnarray}

\begin{eqnarray}P_b=Lq S_H \frac{g}{\Theta}\left(\beta_\theta \frac{\partial \Theta_l}{\partial z}+\beta_q \frac{\partial Q_w}{\partial z}\right),\end{eqnarray}

\begin{eqnarray}\varepsilon=\frac{q^3}{B_1L}.\end{eqnarray}

Advection terms are calculated using the tracer transport routines in the dynamical core. The turbulence scheme calculates the time evolution by the diffusion, production and dissipation terms with an
implicit scheme.

\hypertarget{diagnosis-of-variance-and-covariance}{%
\paragraph{Diagnosis of Variance and Covariance}\label{diagnosis-of-variance-and-covariance}}

The prognostic equations for \(\langle {\theta_l}^2 \rangle,\langle {q_w}^2 \rangle,\langle \theta_l q_w \rangle\) are expressed as

\begin{eqnarray}
\frac{d\left\langle{\theta_l}^{2}\right\rangle}{d t}=-\frac{\partial}{\partial z}\left\langle w \theta_{l}^{2}\right\rangle-2\left\langle w \theta_{l}\right\rangle \frac{\partial \Theta_{l}}{\partial z}-2 \varepsilon_{\theta l},
\end{eqnarray}

\begin{eqnarray}
\frac{d\left\langle {q_w}^{2}\right\rangle}{d t}=-\frac{\partial}{\partial z}\left\langle w q_{w}^{2}\right\rangle-2\left\langle w q_{w}\right\rangle \frac{\partial Q_{w}}{\partial z}-2 \varepsilon_{q w},
\end{eqnarray}

\begin{eqnarray}
\frac{d\left\langle\theta_{l} q_{w}\right\rangle}{d t}=-\frac{\partial}{\partial z}\left\langle w \theta_{l} q_{w}\right\rangle-\left\langle w q_{w}\right\rangle \frac{\partial \Theta_{l}}{\partial z}-\left\langle w \theta_{l}\right\rangle \frac{\partial Q_{w}}{\partial z}-2 \varepsilon_{\theta q}.
\end{eqnarray}

In the Level 2.5, the time derivative, advection, and diffusion terms in these equations are neglected assuming the following local balances.

\begin{eqnarray} -\left\langle w \theta_{l}\right\rangle \frac{\partial \Theta_{l}}{\partial z}-\varepsilon_{\theta l} = 0 \label{p-dif.6},\end{eqnarray}

\begin{eqnarray} -\left\langle w q_{w}\right\rangle \frac{\partial Q_{w}}{\partial z}-\varepsilon_{q w} = 0 \label{p-dif.7},\end{eqnarray}

\begin{eqnarray} -\left\langle w q_{w}\right\rangle \frac{\partial \Theta_{l}}{\partial z}-\left\langle w \theta_{l}\right\rangle \frac{\partial Q_{w}}{\partial z}-2 \varepsilon_{\theta q} = 0 \label{p-dif.8}.\end{eqnarray}

In the Level 2.5 of the MYNN scheme, \(-\left\langle w \theta_{l}\right\rangle\), \(-\left\langle w q_{w}\right\rangle\), \(\varepsilon_{\theta l}\), \(\varepsilon_{q w}\), \(\varepsilon_{\theta q}\)
are represented as

\begin{eqnarray} -\left\langle w \theta_{l}\right\rangle = LqS_H \frac{\partial \Theta_{l}}{\partial z} \label{p-dif.9},\end{eqnarray}

\begin{eqnarray} -\left\langle w q_{w}\right\rangle = LqS_H \frac{\partial Q_{w}}{\partial z} \label{p-dif.10},\end{eqnarray}

\begin{eqnarray}
\varepsilon_{\theta l}=\frac{q}{B_{2} L}\left\langle\theta_{l}^{2}\right\rangle \label{p-dif.11},
\end{eqnarray}

\begin{eqnarray}
\varepsilon_{q w}=\frac{q}{B_{2} L}\left\langle q_{w}^{2}\right\rangle \label{p-dif.12},
\end{eqnarray}

\begin{eqnarray}
\varepsilon_{\theta q}=\frac{q}{B_{2} L}\left\langle\theta_{l} q_{w}\right\rangle \label{p-dif.13}.
\end{eqnarray}

from (\ref{p-dif.6})-(\ref{p-dif.13}), \(\langle {\theta_l}^2 \rangle\), \(\langle {q_w}^2 \rangle\), \(\langle \theta_l q_w \rangle\) can be diagnosed as

\begin{eqnarray}\langle {\theta_l}^2 \rangle =B_2L^2S_H\left(\frac{\partial \Theta_l}{\partial z}\right)^2,\end{eqnarray}

\begin{eqnarray}\langle {q_w}^2 \rangle =B_2L^2S_H\left(\frac{\partial Q_w}{\partial z}\right)^2,\end{eqnarray}

\begin{eqnarray}\langle \theta_l q_w \rangle =B_2L^2S_H\frac{\partial \Theta_l}{\partial z}\frac{\partial Q_w}{\partial z}.\end{eqnarray}

\hypertarget{treatment-in-the-bottom-layer}{%
\paragraph{Treatment in the Bottom Layer}\label{treatment-in-the-bottom-layer}}

Since the lowest model layer corresponds to the surface layer where values of physical variables rapidly change in the vertical direction, the following Monin-Obukhov similarity theory is used to
accurately evaluate the vertical gradient of the variables.

\begin{eqnarray} \frac{\partial M}{\partial z} = \frac{u_*}{kz}\phi_m \label{p-dif.14},\end{eqnarray}

\begin{eqnarray} \frac{\partial \Theta}{\partial z} = \frac{\theta_*}{kz}\phi_h \label{p-dif.15},\end{eqnarray}

\begin{eqnarray} \frac{\partial Q_v}{\partial z} = \frac{q_{v*}}{kz}\phi_h \label{p-dif.16},\end{eqnarray}

where \(M\) is the horizontal wind velocity for the horizontal axis aligned to the direction of the horizontal wind in the surface layer. \(\phi_m\) and \(\phi_h\) are the dimensionless gradient
functions for momentum and heat respectively. \(\theta_*\) and \(q_{v*}\) are the scales of potential temperature and water vapor in the surface layer respectively, and satisfy the following
relationships.

\begin{eqnarray} \langle wm \rangle_g = -u_*^2 \label{p-dif.17},\end{eqnarray}

\begin{eqnarray} \langle w\theta \rangle_g = -u_*\theta_* \label{p-dif.18},\end{eqnarray}

\begin{eqnarray} \langle wq_v \rangle_g = -u_*q_{v*} \label{p-dif.19},\end{eqnarray}

where \(m\) is the deviation of \(M\) from the grid mean. Using \(M\) and \(m\), the production term of the turbulence kinetic energy is written as

\begin{eqnarray} P_s + P_b = \langle wm \rangle \frac{\partial M}{\partial z} + \frac{g}{\Theta} \langle w\theta_v \rangle. \end{eqnarray}

Using (\ref{p-dif.14}), (\ref{p-dif.17}) and the definition of the Obukhov length, it is rewritten as

\begin{eqnarray} P_s + P_b = \frac{u_*^3}{kz_1}\left[\phi_m\left(\zeta_1\right)-\zeta_1\right], \end{eqnarray}

where \(\zeta_1\) is \(\zeta\) at the full level of the lowest model layer.

Assuming that the effect of cloud particles are negligible in the surface layer, \(\langle {\theta_l}^2\rangle\), \(\langle {q_w}^2\rangle\), \(\langle \theta_lq_w\rangle\) is calculated
diagnostically from (\ref{p-dif.6})-(\ref{p-dif.8}), (\ref{p-dif.11})-(\ref{p-dif.13}), (\ref{p-dif.15}), (\ref{p-dif.16}), (\ref{p-dif.18}), and (\ref{p-dif.19})
as

\begin{eqnarray}\langle {\theta_l}^2\rangle=\frac{\phi_h\left(\zeta_1\right)}{u_*kz_1}{\langle w\theta \rangle_g}^2 \bigg/ \frac{q}{B_2L}, \end{eqnarray}

\begin{eqnarray}\langle {q_w}^2\rangle=\frac{\phi_h\left(\zeta_1\right)}{u_*kz_1}{\langle wq_v\rangle_g}^2 \bigg/ \frac{q}{B_2L}, \end{eqnarray}

\begin{eqnarray}\langle \theta_lq_w\rangle=\frac{\phi_h\left(\zeta_1\right)}{u_*kz_1}\langle w\theta \rangle_g\langle wq_v \rangle_g \bigg/ \frac{q}{B_2L}. \end{eqnarray}

\(\phi_m\) and \(\phi_h\) are formulated following Businger et al.~(1971) as

\begin{eqnarray}
\phi_m(\zeta)=\left\{
    \begin{array}{lr}
      1+\beta_1\zeta, &\zeta\ge 0\\
      \left(1-\gamma_1\zeta\right)^{-1/4}, &\zeta< 0\\
    \end{array}
  \right.
\end{eqnarray}

\begin{eqnarray}
\phi_h(\zeta)=\left\{
    \begin{array}{lr}
      \beta_2+\beta_1\zeta, &\zeta\ge 0\\
      \beta_2\left(1-\gamma_2\zeta\right)^{-1/2}, &\zeta< 0\\
    \end{array}
  \right.
\end{eqnarray}

\begin{eqnarray}(\beta_1,\beta_2,\gamma_1,\gamma_2)=(4.7,0.74,15.0,9.0).\end{eqnarray}

\hypertarget{time-integration-with-implicit-scheme}{%
\subsubsection{Time Integration with Implicit Scheme}\label{time-integration-with-implicit-scheme}}

\hypertarget{tendency-of-q2}{%
\paragraph{\texorpdfstring{Tendency of \(q^2\)}{Tendency of q\^{}2}}\label{tendency-of-q2}}

The prognostic equation for \(q^2\) is discretized as

\begin{eqnarray} \left(\frac{q^2_{k,n+1}-q^2_{k,n}}{\Delta t}\right)_{\text{turb}} = -\frac{1}{\rho_k\Delta z_k}\left(F_{q,k+1/2,n+1}-F_{q,k-1/2,n+1}\right) +2\left( P_{s,k,n} + P_{b,k,n} - \frac{q_{k,n}}{B_1L}q^2_{k,n+1}\right), \label{p-dif.20} \end{eqnarray}

where \(n\) and \(n+1\) indicate the current and next time steps respectively, and \(\Delta z_k \equiv z_{k+1/2}-z_{k-1/2}\). The subscript \emph{turb} indicates the calculation by the turbulence
scheme and the advection term is omitted. \(F_q\) at \(n+1\) is computed by

\begin{eqnarray} F_{q,k-1/2,n+1} = F_{q,k-1/2,n} + \frac{\partial F_{q,k-1/2}}{\partial q^2_k}(q^2_{k,n+1}-q^2_{k,n}) +  \frac{\partial F_{q,k-1/2}}{\partial q^2_{k-1}}(q^2_{k-1,n+1}-q^2_{k-1,n}). \label{p-dif.21} \end{eqnarray}

With a definition of \begin{eqnarray}\mu_k = \left(\frac{q^2_{k,n+1}-q^2_{k,n}}{\Delta t}\right)_{\text{turb}},\end{eqnarray}

(\ref{p-dif.20}) and (\ref{p-dif.21}) lead to

\begin{eqnarray}
 X_{1,k}\,\mu_{k+1}+X_{2,k}\,\mu_k+X_{3,k}\,\mu_{k-1} = Y_k, \label{p-dif.22}
\end{eqnarray}

where

\begin{align}
 X_{1,k} &= \frac{\partial F_{q,k+1/2}}{\partial q^2_{k+1}} \Delta t, \\
 X_{2,k} &= \rho_k \Delta z_k \left(1+2\frac{q_{k,n}}{B_1 L}\Delta t \right) + \left( \frac{\partial F_{q,k+1/2}}{\partial q^2_k} - \frac{\partial F_{q,k-1/2}}{\partial q^2_k} \right)\Delta t, \\
 X_{3,k} &= -\frac{\partial F_{q,k-1/2}}{\partial q^2_{k-1}} \Delta t, \\
 Y_k &= F_{q,k-1/2,n} - F_{q,k+1/2,n} + 2\rho_k \Delta z_k \left( P_{s,k,n} + P_{b,k,n} - \frac{q^3_{k,n}}{B_1 L} \right).
\end{align}

(\ref{p-dif.22}) makes the following matrix equation,

\begin{eqnarray}
\left(\begin{array}{lllllll}
 X_{2,K}   & X_{3,K}   & 0         & 0         & 0         & \cdots    & 0       \\
 X_{1,K-1} & X_{2,K-1} & X_{3,K-1} & 0         & 0         & \cdots    & 0       \\
 0         & X_{1,K-2} & X_{2,K-2} & X_{3,K-2} & 0         & \cdots    & 0       \\
 \vdots    &           &           & \ddots    &           &           & \vdots  \\
 0         & \cdots    & 0         & X_{1,3}   & X_{2,3}   & X_{3,3}   & 0       \\
 0         & \cdots    & 0         & 0         & X_{1,2}   & X_{2,2}   & X_{3,2} \\
 0         & \cdots    & 0         & 0         & 0         & X_{1,1}   & X_{2,1}
\end{array}\right)
\left(\begin{array}{l}
 \mu_K \\
 \mu_{K-1} \\
 \mu_{K-2} \\
 \vdots \\
 \mu_3 \\
 \mu_2 \\
 \mu_1
\end{array}\right)
= \left(
\begin{array}{l}
 Y_K \\
 Y_{K-1} \\
 Y_{K-2} \\
 \vdots \\
 Y_3 \\
 Y_2 \\
 Y_1
\end{array}
\right), \label{p-dif.23}
\end{eqnarray}

where the subscript \(K\) denote the index for the top model layer. (\ref{p-dif.23}) is solved for \(\mu_k\) using the LU decomposition.

\hypertarget{tendencies-of-the-other-prognostic-variables}{%
\paragraph{Tendencies of the Other Prognostic Variables}\label{tendencies-of-the-other-prognostic-variables}}

Letting \(\psi\) be a substitute for \(u\), \(v\), \(T\), \(q_w\), the tendency of \(\psi\) is calculated by

\begin{eqnarray} \left(\frac{\psi_{k,n+1}-\psi_{k,n}}{\Delta t}\right)_{\text{turb}} = -\frac{1}{\rho_k\Delta z_k}\left(F_{\psi,k+1/2,n+1}-F_{\psi,k-1/2,n+1}\right), \end{eqnarray}

where

\begin{eqnarray} F_{\psi,k-1/2,n+1} = F_{\psi,k-1/2,n} + \frac{\partial F_{\psi,k-1/2}}{\partial \psi_k}(\psi_{k,n+1}-\psi_{k,n}) +  \frac{\partial F_{\psi,k-1/2}}{\partial \psi_{k-1}}(\psi_{k-1,n+1}-\psi_{k-1,n}). \end{eqnarray}

These equations lead to (\ref{p-dif.23}) again and computed with the LU decomposition, but \(\mu_k\), \(X_{1,k}\), \(X_{2,k}\), \(X_{3,k}\) and \(Y_k\) are redefined as

\begin{align}
 \mu_k &= \left(\frac{\psi_{k,n+1}-\psi_{k,n}}{\Delta t}\right)_{\text{turb}}, \\
 X_{1,k} &= \frac{\partial F_{\psi,k+1/2}}{\partial \psi_{k+1}} \Delta t, \\
 X_{2,k} &= \rho_k \Delta z_k + \left( \frac{\partial F_{\psi,k+1/2}}{\partial \psi_k} - \frac{\partial F_{\psi,k-1/2}}{\partial \psi_k} \right)\Delta t, \\
 X_{3,k} &= -\frac{\partial F_{\psi,k-1/2}}{\partial \psi_{k-1}} \Delta t, \\
 Y_k &= F_{\psi,k-1/2,n} - F_{\psi,k+1/2,n}.
\end{align}

	% 物理過程:地表フラックス
	% 
\subsection{地表フラックス}

\subsubsection{地表フラックススキームの概要}

地表フラックススキームは, 
接地境界層における乱流輸送による
大気地表間の物理量のフラックスを評価する.
主な入力データは, 風速 TERM00850,TERM00850, 気温 TERM00851, 比湿 TERM00852 であり,
出力データは, 運動量, 熱, 水蒸気の鉛直フラックスと
implicit 解を得るための微分値である.

バルク係数は Louis(1979), Louis {\em et al.}(1982) に従って求める. 
ただし, 運動量と熱に対する粗度の違いを考慮した補正を行なっている. 

計算手順の概略は以下の通りである.
\begin{enumerate}
\item 大気の安定度として
      Richardson 数を計算する.
\item Richardson 数からバルク係数を計算する \texttt{MODULE:[PSFCL]}.
\item バルク係数からフラックスとその微分を計算する.
\item 必要であれば, 求められたフラックスを用いて
      海面の粗度効果・自由対流の効果・風速補正を考慮した後に,
      もう一度計算を行なう.
\end{enumerate}

\subsubsection{フラックス計算の基本式}

地表フラックス TERM00853,TERM00853 は
バルク係数 TERM00854,TERM00854 を用いて
次のように表される.
%
\begin{verbatim}
EQ=00328.
\end{verbatim}
\begin{verbatim}
EQ=00329.
\end{verbatim}
\begin{verbatim}
EQ=00330.
\end{verbatim}
\begin{verbatim}
EQ=00331.
\end{verbatim}
%
ただし, TERM00859 は可能蒸発量である.
実蒸発量の計算は「地表過程」ならび
「大気地表系の拡散型収支式の解法」の節で述べる.

\subsubsection{Richardson 数}

大気地表間の安定度の基準となる,
バルクRichardson数 TERM00860 は
%
\begin{verbatim}
EQ=00332.
\end{verbatim}
ここで, 
\begin{verbatim}
EQ=00333.
\end{verbatim}
は補正ファクターで, 補正前のバルク Richardson数から近似的に求めるが, 
ここでは計算方法は略す. 

\subsubsection{バルク係数}

バルク係数 TERM00861,TERM00861 は
Louis(1979), Louis {\em et al.}(1982) に従って求める. 
ただし, 運動量と熱に対する粗度の違いを考慮した補正を行なっている. 
すなわち, 運動量, 熱, 水蒸気に対する粗度を
それぞれ TERM00862,TERM00862 とすると
一般に TERM00863,TERM00863 であるが, 熱, 水蒸気についても
TERM00864 の高さからのフラックスに対するバルク係数
TERM00865, TERM00866 をまず求め, その後に補正する. 
%
\begin{verbatim}
EQ=00334.
\end{verbatim}
%
\begin{verbatim}
EQ=00335.
\end{verbatim}
\begin{verbatim}
EQ=00336.
\end{verbatim}
%
\begin{verbatim}
EQ=00337.
\end{verbatim}
\begin{verbatim}
EQ=00338.
\end{verbatim}

TERM00867,TERM00867 は
中立時の(TERM00868 からのフラックスに対する)バルク係数で,
%
\begin{verbatim}
EQ=00339.
\end{verbatim}

補正ファクター TERM00869 は, 
\begin{verbatim}
EQ=00340.
\end{verbatim}
であるが, 計算方法は略す. 
係数は, TERM00870,TERM00870 である. 

バルク係数の TERM00871 依存性を図示すると,
図\ref{p-sflx:cm}, 図\ref{p-sflx:ch}のようになる.

\begin{figure}[htbp]
  \begin{center}
    \epsfile{file=sflx-cm.ps,width=70mm}
    \caption{運動量に対する粗度}
    \label{p-sflx:cm}
  \end{center}
\end{figure}
\begin{figure}[htbp]
  \begin{center}
    \epsfile{file=sflx-ch.ps,width=70mm}
    \caption{熱に対する粗度. TERM00872 の場合}
    \label{p-sflx:ch}
  \end{center}
\end{figure}

\subsubsection{フラックスの計算}

これにより, フラックスが計算される.
%
\begin{verbatim}
EQ=00341.
\end{verbatim}
\begin{verbatim}
EQ=00342.
\end{verbatim}
\begin{verbatim}
EQ=00343.
\end{verbatim}
\begin{verbatim}
EQ=00344.
\end{verbatim}

微分項は, 以下のようになる.
\begin{verbatim}
EQ=00345.
\end{verbatim}
\begin{verbatim}
EQ=00346.
\end{verbatim}
\begin{verbatim}
EQ=00347.
\end{verbatim}
\begin{verbatim}
EQ=00348.
\end{verbatim}
\begin{verbatim}
EQ=00349.
\end{verbatim}

ここで, 注意したいのは,
TERM00882 はこの時点では求められていない量であることである.
表皮温度は, 
地表熱バランスの条件
\begin{verbatim}
EQ=00350.
\end{verbatim}
を満たすように決まる.
この時点では, TERM00883 としては前の時間ステップにおけるものを使って評価する.
地表バランスを満たす本当のフラックスの値は,
地表過程と結合してこの式を解いてから定まる.
その意味で, 上のフラックスに TERM00884 をつけておいた.

\subsubsection{海面における取扱い}

海面では, Miller et al.(1992) に従い, 以下の2つの効果を考慮している.
\begin{itemize}
\item 風速が弱いときに自由対流運動が卓越すること
\item 海面の粗度が風速によって変化すること
\end{itemize}

自由対流運動の効果は, 浮力フラックス TERM00885 を計算し,
\begin{verbatim}
EQ=00351.
\end{verbatim}
TERM00886 のときに,
\begin{verbatim}
EQ=00352.
\end{verbatim}
\begin{verbatim}
EQ=00353.
\end{verbatim}
とすることで考慮する.  TERM00888 は混合層の厚さのスケールに対応する.
現在の標準値は TERM00889 m である.
% この自由対流運動の効果は, 海面以外でも考慮している.

海面の粗度変化は, 摩擦速度 TERM00890
\begin{verbatim}
EQ=00354.
\end{verbatim}
を用いて,
\begin{verbatim}
EQ=00355.
EQ=00355.
EQ=00355.
\end{verbatim}
のように評価する. TERM00891 TERM00892 TERM00893 は
大気の動粘性係数であり, 
他の係数の標準値は
TERM00894,TERM00894,
TERM00895,TERM00895,
TERM00896,TERM00896 である.

以上の計算では, TERM00897,TERM00897 が必要であるため,
逐次近似計算を行なう.

\subsubsection{風速の補正}

一般に粗度の大きな地表では, 粗度の小さな地表に比べて
運動量の下向き輸送が効率的であるためにその直上の風が弱く,
粗度による TERM00898 の違いを風速の違いによって打ち消す効果が働く.

モデルにおいて地表フラックス計算に渡される風速は
力学過程の時間積分によって計算された値であり,
スペクトル展開によって平滑化された値となっている.
そのために, 海面と陸面など, 粗度の大きく違う地表が
小さなスケールで混在している領域では, 
この補償効果がうまく表現できない.
そのため, 一度運動量フラックスを計算し,
大気最下層の風速をそれによって補正してから
もういちど運動量・熱・水のフラックスを計算しなおす.

\subsubsection{風速の最小値}

小規模運動の効果を考え,
地表フラックスの算出の際の地表風速
TERM00899 の最小値を設定する.
現在の標準値は, 各フラックスに共通で
3m/s である.

 % 廃止?
	% 物理過程:地表モデル
	\begin{itemize}
\tightlist
\item
  \ref{1-surface-flux}
\item
  \ref{11-sea-surface-flux-ocnflx}

  \begin{itemize}
  \tightlist
  \item
    \ref{12-sea-surface-conditions-ocnbcs}

    \begin{itemize}
    \tightlist
    \item
      \ref{121-sea-surface-albedo-for-visible-seaalb}
    \item
      \ref{122-sea-surface-roughness-seaz0f}
    \end{itemize}
  \item
    \ref{13-sea-surface-flux-sfcflx}

    \begin{itemize}
    \tightlist
    \item
      \ref{131-bulk-factor-blkcof}
    \end{itemize}
  \item
    \ref{14-radiation-flux-at-sea-surface-radsfc}
  \item
    \ref{15-sea-surface-heat-balance-ocnslv}
  \end{itemize}
\end{itemize}

\hypertarget{surface-flux}{%
\section{1 Surface Flux}\label{surface-flux}}

Until CCSR/NIES AGCM, both land surface and sea surface were treated as
one of the atmospheric physical processes, but after MIROC3 (Hasumi and
Emori, 2004), land surface processes became independent as MATSIRO.
However, since MIROC3
(\href{https://ccsr.aori.u-tokyo.ac.jp/~hasumi/miroc_description.pdf}{Hasumi
and Emori, 2004}), land surface processes have been separated into
MATSIRO. In \texttt{SUBROUTINE:{[}SURFCE{]}} in pgsfc.F,
\texttt{ENTRY:{[}OCNFLX{]}} (in \texttt{SUBROUTINE:{[}OCEAN{]}} of
pgocn.F) is called for the sea surface, and \texttt{ENTRY:{[}LNDFLX{]}}
(in \texttt{SUBROUTINE:{[}MATSIRO{]}} of matdrv.F) is called for the
land surface, respectively. This chapter describes sea surface
processes, which are still treated within the framework of atmospheric
physical processes (MIROC6). For the land surface processes, please
refer to
\href{https://github.com/integrated-land-simulator/model_description}{Description
of ILS}.

\href{https://github.com/MIROC-DOC/model_description/blob/coupler_iwakiri/draft/AO-coupler.md}{カップラーのセクション}とmerge予定。

\begin{itemize}
\tightlist
\item
  Outputs
\end{itemize}

\setlength\LTleft{0pt}\setlength\LTright{0pt}\begin{longtable}[]{@{}lllll@{}}
\toprule\relax
Meaning & Presentation & Variable & dimension & unit \\
\midrule\relax
\endhead
upward long wave & & RFLXLG & IJLSDIM & \\
upward short wave & & RFLXSG & IJLSDIM & \\
\bottomrule
\end{longtable}

\begin{itemize}
\tightlist
\item
  Inputs
\end{itemize}

\setlength\LTleft{0pt}\setlength\LTright{0pt}\begin{longtable}[]{@{}lllll@{}}
\toprule\relax
Meaning & Presentation & Variable & dimension & unit \\
\midrule\relax
\endhead
surface downward radiation & & RFSFCD & IJSDIM, NRALB & \\
cos(solar zenith) & \(cos(\theta)\) & RCOSZ & IJSDIM & {[}-{]} \\
rainfall (cumulus convection scheme) & & GPRCC & IJSDIM, NTR & \\
rainfall (Large scale condensation scheme) & & GPRCL & IJSDIM, NTR & \\
snowfall (cumulus convection scheme) & & GSNWC & IJSDIM, NTR & \\
snowfall (cLarge scale condensation scheme) & & GSNWL & IJSDIM, NTR & \\
u wind & \(u\) & GDU & IJSDIM, KMAX & {[}m/s{]} \\
v wind & \(v\) & GDV & IJSDIM, KMAX & {[}m/s{]} \\
temperature & \(T\) & GDT & IJSDIM, KMAX & {[}K{]} \\
humidity & \(q\) & GDQ & IJSDIM, KMAX, NTR & {[}kg/kg{]} \\
pressure & \(P\) & GDP & IJSDIM, KMAX+1 & \\
pressure (half level) & & GDPM & IJSDIM, KMAX+1 & \\
altitude (half level) & & GDZM & IJSDIM, KMAX+1 & \\
time & & TIME & & \\
dt for implicit & & DELTP & & \\
time step (interval) & & DELTI & & \\
\bottomrule
\end{longtable}

The only 1st layer is practically handed to the surface schemes.

\hypertarget{sea-surface-flux-ocnflx}{%
\section{\texorpdfstring{1.1 Sea surface flux
\texttt{{[}OCNFLX{]}}}{1.1 Sea surface flux {[}OCNFLX{]}}}\label{sea-surface-flux-ocnflx}}

Sea surface processes provide the boundary conditions at the lower end
of the atmosphere through the exchange of momentum, heat, and water
fluxes between the atmosphere and the surface. In
\texttt{ENTRY:{[}OCNFLX{]}}, the following procedure is used to deal
with sea surface processes.

\begin{enumerate}
\def\labelenumi{\arabic{enumi}.}
\tightlist
\item
  prepare variables for sea ice extent and no ice extent, respectively,
  using sea ice concentration.
\item
  Determine the surface boundary conditions.
\item
  Calculate the flux balance.
\item
  Calculate the radiation budget at the sea surface.
\item
  Calculate the deposition by CHASER.
\item
  solve the heat balance at the sea surface and update the surface
  temperature and each flux value.
\end{enumerate}

\setlength\LTleft{0pt}\setlength\LTright{0pt}\begin{longtable}[]{@{}lllll@{}}
\toprule\relax
Meaning & Presentation & Variable & dimension & unit \\
\midrule\relax
\endhead
u wind of the 1st layer of the atmosphere & \(u_a\) & GAUA & IJOSDM &
{[}m/s{]} \\
v wind of the 1st layer of the atmosphere & \(v_a\) & GAVA & IJOSDM &
{[}m/s{]} \\
temperature of the 1st layer of the atmosphere & \(T_a\) & GATA & IJOSDM
& {[}K{]} \\
humidity of the 1st layer of the atmosphere & \(q_a\) & GAQA & IJOSDM &
{[}kg/kg{]} \\
pressure of the 1st layer of the atmosphere & \(P_a\) & GAPA & IJOSDM
& \\
surface pressure Ps & \(P_s\) & GAPS & IJOSDM & \\
surface height & & GAZS & IJOSDM & \\
surface radiation fluxes & & RSFCD & IJOSDM & \\
cos(solar zenith) & \(cos(\theta)\) & RCOSZ & IJOSDM & {[}-{]} \\
\bottomrule
\end{longtable}

If use CHASER, variables below are also needed.

\setlength\LTleft{0pt}\setlength\LTright{0pt}\begin{longtable}[]{@{}lllll@{}}
\toprule\relax
Meaning & Presentation & Variable & dimension & unit \\
\midrule\relax
\endhead
Henry const & & EH & IJOSDM & \\
precipitation flux (cumulus convection scheme) & \(Pr_c\) & PFLXC &
IJOSDM & \\
precipitation flux (large scale condensation scheme) & \(Pr_l\) & PFLXL
& IJOSDM & \\
latitude & & LLAT & IJOSDM & \\
\bottomrule
\end{longtable}

Practically, precipitation flux from 2 schemes are treated together.

\begin{eqnarray}
    Pr = Pr_c + Pr_l
\end{eqnarray}

In the sea ice area (\(L=1\)), the surface temperature \(T_s\) is the
sea ice surface temperature \(T_{ice}\). However, if \(T_{ice}\) is
higher than \(T_{melt}=0\), then \(T_{melt}\) is used.

\begin{eqnarray}
    T_s = min(T_{ice},T_{melt})
\end{eqnarray}

The sea ice bottom temperature \(T_b\) is assumed to be the sea surface
temperature \(T_{o(1)}\).

\begin{eqnarray}
    T_b = T_{o(1)}
\end{eqnarray}

The amount of sea ice \(W_{ice}\) and the amount of snow on it
\(W_{snow}\) are converted per unit area by considering \(R_{ice}\) and
used in the calculation. However, a limiter \(\epsilon\) is provided to
prevent the values from becoming too small.

\begin{eqnarray}
R_{ice} =\mathrm{max}( R_{ice,orginal}, \epsilon)
\end{eqnarray}

In the ice-free region (\(L=2\)), the surface temperature \(T_s\) and
sea ice bottom temperature \(T_b\) are assumed to be the sea temperature
temperature \(T_{o(1)}\).

\begin{eqnarray}
    T_s = T_b = T_{o(1)}
\end{eqnarray}

The evaporation efficiency ise set to 1 for both \(L=1, 2\).

If the sea ice concentration \(R_{ice}\) is not given, it can be
diagnosed simply from the sea ice volume \(W_{ice}\) in
\texttt{ENTRY:{[}OCNICR{]}}.

\begin{eqnarray}
R_{ice} = \mathrm{min}\Big(\sqrt{\frac{\mathrm{max}(W_{ice},0)}{W_{ice,c}}},1.0\Big)
\end{eqnarray}

The standard gives the amount of sea ice per area as
\(W_{ice,c}=300 \mathrm{[kg/m^2]}\).

\hypertarget{sea-surface-conditions-ocnbcs}{%
\subsection{\texorpdfstring{1.2 Sea Surface Conditions
\texttt{{[}OCNBCS{]}}}{1.2 Sea Surface Conditions {[}OCNBCS{]}}}\label{sea-surface-conditions-ocnbcs}}

\begin{itemize}
\tightlist
\item
  Output variables
\end{itemize}

\setlength\LTleft{0pt}\setlength\LTright{0pt}\begin{longtable}[]{@{}lllll@{}}
\toprule\relax
Meaning & Presentation & Variable & dimension & unit \\
\midrule\relax
\endhead
surface albedo & \(\alpha\) & GRALB & IJLODM, NRDIR, NRBND & -- \\
surface roughness & \(r_0\) & GRZ0 & IJLODM, NTYZ0 & -- \\
heat flux & \(G\) & FOGFLX & IJLODM & -- \\
heat diffusion coefficient & \(\frac{\partial G}{\partial T}\) & DGFDS &
IJLODM & -- \\
\bottomrule
\end{longtable}

\begin{itemize}
\tightlist
\item
  Input variables
\end{itemize}

\setlength\LTleft{0pt}\setlength\LTright{0pt}\begin{longtable}[]{@{}lllll@{}}
\toprule\relax
Meaning & Presentation & Variable & dimension & unit \\
\midrule\relax
\endhead
surface temperature & \(T_s\) & GRTS & IJLODM & \(\mathrm{[K]}\) \\
ice base temperature & \(T_b\) & GRTB & IJLODM & \(\mathrm{[K]}\) \\
lake ice amount & \(Ic\) & GRICE & IJLODM & \(\mathrm{[kg/m^2]}\) \\
snow amount & \(Sn\) & GRSNW & IJLODM & \\
ice concentration & \(R_{ice}\) & GRICR & IJLODM & {[}-{]} \\
u wind of the 1st layer of the atmosphere & \(u_a\) & GDUA & IJLODM &
\(\mathrm{[m/s]}\) \\
v wind of the 1st layer of the atmosphere & \(v_a\) & GDVA & IJLODM &
\(\mathrm{[m/s]}\) \\
cos(solar zenith) & \(\mathrm{cos}(\theta)\) & RCOSZ & IJLODM &
{[}-{]} \\
\bottomrule
\end{longtable}

In \texttt{ENTRY{[}OCNBCS{]}} in \texttt{SUBROUTINE:{[}OCNSUB{]}},
surface albedo and roughness are calculated. They are calculated
supposing ice-free conditions, then modified.

First, let us consider the sea albedo. The sea level \(\alpha_{(d,b)}\),
\(b=1,2,3\) represent the visible, near-infrared, and infrared
wavelength bands, respectively. Also, \(d=1,2\) represents direct and
scattered light, respectively. The albedo for the visible bands are
calculated in \texttt{SUBROUTINE\ {[}SEAALB{]}}, supposing ice-free
conditions. The albedo for near-infrared is set to same as the visible
one. The albedo for infrared is uniformly set to a constant value.

The grid-averaged albedo, taking into account the sea ice concentration
\(R_{ice}\), is

\begin{eqnarray}
    \alpha = \alpha -R_{ice} \alpha_{ice}
\end{eqnarray}

\(\alpha_{ice}\) is given by the standard as
\(\alpha_{ice,1}=0.5,\alpha_{ice,2}=0.5,\alpha_{ice,3}=0.05\). 4.

In addition, we want to consider the effect of snow cover. Here, we
consider the albedo modification by temperature. The standard threshold
values for snow temperature are \(T_{al,2}=258.15 \mathrm{[K]}\) and
\(T_{al,1}=273.15 \mathrm{[K]}\). The snow albedo changes linearly with
temperature change from \(\alpha_{snow,1}=0.75\) to
\(\alpha_{ snow,2}\). Let the coefficient \(\tau_{snow}\), which is
\(0\le \tau \le 1\).

\begin{eqnarray}
\tau_{snow} = \frac{T_s - T_{al,1}}{T_{al,2}-T_{al,1}}
\end{eqnarray}

Update the snow albedo \(\alpha_{snow}\) as

\begin{eqnarray}
    \alpha_{snow} = \alpha_{snow,0} + \tau_{snow}(\alpha_{snow,2}-\alpha_{snow,1})
\end{eqnarray}

Second, let us consider the sea surface roughness. The roughnesses of
for momentum, heat and vapor are calculated in \texttt{{[}SEAZ0F{]}},
supposing the ice-free conditions.

When the sea ice exists (\(L=1\)), each roughness is modified to take
into account the sea concentration \(R_{ice}\).

\begin{eqnarray}
    z_{0,M} = z_{0,M} + ( z_{0,ice,M} - z_{0,M})  \alpha_{ice}
\end{eqnarray}

\begin{eqnarray}
    z_{0,H} = z_{0,H} + ( z_{0,ice,H} - z_{0,H})  \alpha_{ice}
\end{eqnarray}

\begin{eqnarray}
    z_{0,E} = z_{0,E} + ( z_{0,ice,E} - z_{0,E})  \alpha_{ice}
\end{eqnarray}

Here, \(r_{0,ice,*}\) is roughness of sea ice, \(\alpha_{ice}\) is the
sea ice concentration.

When the snow even exists,

\begin{eqnarray}
    z_{0,M} = z_{0,M} + ( z_{0,snow,M} - z_{0,M})  \alpha_{snow}
\end{eqnarray}

\begin{eqnarray}
    z_{0,H} = z_{0,H} + ( z_{0,snow,H} - z_{0,H})  \alpha_{snow}
\end{eqnarray}

\begin{eqnarray}
    z_{0,E} = z_{0,E} + ( z_{0,snow,E} - z_{0,E})  \alpha_{snow}
\end{eqnarray}

Here, \(r_{0,snow,*}\) is roughness of sea ice, \(\alpha_{snow}\) is the
sea ice concentration.

Third, let us consider the conductivity of ice.

When sea ice exists (\(L=1\)), the thermal conductivity
\(k_{ice}^\star\) of sea ice is obtained by using \(D_{f,ice}\) (thermal
diffusivity of sea ice) and sea ice density \(\sigma_{ice}\).

\begin{eqnarray}
k_{ice}^\star = \frac{D_{f,ice}}{\mathrm{max}(R_{ice}/\sigma_{ice}, \epsilon)}
\end{eqnarray}

The calculated thermal conductivity is modified to \(k_{ice}\) to take
into account that it varies with snow cover.

\begin{eqnarray}
h_{snow} = \mathrm{min}(
    \mathrm{max}(
    R_{snow}/\sigma_{snow}),\epsilon
        ),h_{snow,max}
        )
\end{eqnarray}

\begin{eqnarray}      
k_{ice} = k_{ice}^\star (1-R_{ice}) + \frac{D_{ice}}{1+\| D_{ice}/D_{snow} \cdot h_{snow} \|} R_{ice}
\end{eqnarray}

where \(h_{snow}\) is the snow depth, \(R_{snow}\) is the snow area
fraction, \(\sigma_{snow}\) is the snow density, \(h_{snow,max}\) is the
maximum snow depth, and \(D_{snow}\) is the thermal diffusivity of snow.

Therefore, the heat conduction flux and its derivative are

\begin{eqnarray}
 G = k_{ice} (T_b - T_s)
\end{eqnarray}

\begin{eqnarray}
 \frac{\partial G}{\partial T} = k_{ice}
\end{eqnarray}

Note that in the ice-free region (\(L=2\))

\begin{eqnarray}
G=k_{ocn}
\end{eqnarray}

where \(k_{ocn}\) is the heat flux in the sea temperature layer. Here,
\(k_{ocn}\) is the heat flux in the sea temperature layer.

\hypertarget{sea-surface-albedo-for-visible-seaalb}{%
\subsubsection{\texorpdfstring{1.2.1 Sea Surface Albedo for Visible
\texttt{{[}SEAALB{]}}}{1.2.1 Sea Surface Albedo for Visible {[}SEAALB{]}}}\label{sea-surface-albedo-for-visible-seaalb}}

\begin{itemize}
\tightlist
\item
  Inputs
\end{itemize}

\setlength\LTleft{0pt}\setlength\LTright{0pt}\begin{longtable}[]{@{}lllll@{}}
\toprule\relax
Meaning & Presentation & Variable & dimension & unit \\
\midrule\relax
\endhead
cos(solar zenith) & \(cos(\theta)\) & COSZ & IJLODM & {[}-{]} \\
\bottomrule
\end{longtable}

\begin{itemize}
\tightlist
\item
  Outputs
\end{itemize}

\setlength\LTleft{0pt}\setlength\LTright{0pt}\begin{longtable}[]{@{}lllll@{}}
\toprule\relax
Meaning & Presentation & Variable & dimension & unit \\
\midrule\relax
\endhead
sea surface albedo (direct, diffuse) & \(\alpha_{L(d)}\) & GALB & IJLODM
,2 & {[}-{]} \\
\bottomrule
\end{longtable}

For sea surface level albedo \(\alpha_{L(d)}\), \(d=1,2\) represents
direct and scattered light, respectively.

Using the solar zenith angle at latitude \(\theta\), the albedo for
direct light is presented by

\begin{eqnarray}
    \alpha_{L(1)} = e^{(C_3A^* + C_2) A^* +C_1}
\end{eqnarray}

where

\begin{eqnarray}
    A = \mathrm{min}(\mathrm{max}(\mathrm{cos}(\theta),0.03459),0.961)
\end{eqnarray}

On the other hand, the albedo for scattered light is uniformly set to a
constant parameter.

\begin{eqnarray}
    \alpha_{L(2)} = 0.06
\end{eqnarray}

\hypertarget{sea-surface-roughness-seaz0f}{%
\subsubsection{\texorpdfstring{1.2.2 Sea Surface Roughness
\texttt{{[}SEAZ0F{]}}}{1.2.2 Sea Surface Roughness {[}SEAZ0F{]}}}\label{sea-surface-roughness-seaz0f}}

\begin{itemize}
\tightlist
\item
  Outputs
\end{itemize}

\setlength\LTleft{0pt}\setlength\LTright{0pt}\begin{longtable}[]{@{}lllll@{}}
\toprule\relax
Meaning & Presentation & Variable & dimension & unit \\
\midrule\relax
\endhead
surface roughness for momentum & \(r_{0,M}\) & GRZ0M & IJLODM & -- \\
surface roughness for heat & \(r_{0,H}\) & GRZ0H & IJLODM & -- \\
surface roughness for vapor & \(r_{0,E}\) & GRZ0E & IJLSDIM & -- \\
\bottomrule
\end{longtable}

\begin{itemize}
\tightlist
\item
  Inputs
\end{itemize}

\setlength\LTleft{0pt}\setlength\LTright{0pt}\begin{longtable}[]{@{}lllll@{}}
\toprule\relax
Meaning & Presentation & Variable & dimension & unit \\
\midrule\relax
\endhead
u wind of the 1st layer of the atmosphere & \(u_a\) & GDUA & IJLODM &
{[}m/s{]} \\
v wind of the 1st layer of the atmosphere & \(v_a\) & GDVA & IJLODM &
{[}m/s{]} \\
\bottomrule
\end{longtable}

The roughness variation of the sea surface is determined by the friction
velocity \(u^\star\)

\begin{eqnarray}
u^{\star} = \sqrt{C_{M_0} ({u_a}^2  +{v_a}^2)}
\end{eqnarray}

We perform successive approximation calculation of \({C_{M_0}}\),
because \(F_u,F_v,F_\theta,F_q\) are required.

\begin{eqnarray}
    r_{0,M} = z_{0,M_0} + z_{0,M_R} + \frac{z_{0,M_R} {u^\star }^2 }{g} + \frac{z_{0,M_S}\nu }{u^\star}
\end{eqnarray}

\begin{eqnarray}
    r_{0,H} = z_{0,H_0} + z_{0,H_R} + \frac{z_{0,H_R} {u^\star }^2 }{g} + \frac{z_{0,H_S}\nu }{u^\star}
\end{eqnarray}

\begin{eqnarray}
    r_{0,E} = z_{0,E_0} + z_{0,E_R} + \frac{z_{0,E_R} {u^\star }^2 }{g} + \frac{z_{0,E_S}\nu }{u^\star}
\end{eqnarray}

Here, \(\nu = 1.5 \times 10^{-5} \mathrm{[m^2/s]}\) is the kinetic
viscosity of the atmosphere. \(z_{0,M},z_{0,H}\) and \(z_{0,E}\) are
surface roughness for momentum, heat, and vapor, respectively.
\(z_{0,M_0},z_{0,H_0}\) and \(z_{0,E_0}\) are base, and rough factor
(\(z_{0,M_R},z_{0,M_R}\) and \(z_{0,E_R}\)) and smooth factor
(\(z_{0,M_S},z_{0,M_S}\) and \(z_{0,E_S}\)) are taken into account.

\hypertarget{sea-surface-flux-sfcflx}{%
\subsection{\texorpdfstring{1.3 Sea Surface Flux
\texttt{{[}SFCFLX{]}}}{1.3 Sea Surface Flux {[}SFCFLX{]}}}\label{sea-surface-flux-sfcflx}}

The surface flux scheme evaluates the physical quantity fluxes between
the atmospheric surfaces due to turbulent transport in the boundary
layer. The main input are wind speed (\(u_a, v_a\)), temperature
(\(T_a\)), and specific humidity (\(q_s\)) from the 1st layer of the
atmosphere. The output are the vertical fluxes and the differential
values (for obtaining implicit solutions) of momentum, heat, and water
vapor.

Surface fluxes (\(F_u, F_v, F_\theta, F_q\)) are expressed using the
bulk coefficients (\(C_M, C_H, C_E\)) as follows

\begin{eqnarray}
    F_u  =  - \rho C_M |{\mathbf{v}}| u
\end{eqnarray}

\begin{eqnarray}
    F_v  =  - \rho C_M |{\mathbf{v}}| v
\end{eqnarray}

\begin{eqnarray}
    F_\theta  = \rho c_p C_H |{\mathbf{v}}| ( \theta_g - \theta )
\end{eqnarray}

\begin{eqnarray}
    F_q^P =  \rho C_E |{\mathbf{v}}| ( q_g - q_a )
\end{eqnarray}

Note that \(F_q^P\) is the possible evaporation flux.

The turbulent fluxes at the sea surface are solved by bulk formulae as
follows. Then, by solving the surface energy balance, the ground surface
temperature (\(T_s\)) is updated, and the surface flux values with
respect to those values are also updated. The solutions obtained here
are temporary values. In order to solve the energy balance by
linearizing with respect to \(T_s\), the differential with respect to
\(T_s\) of each flux is calculated beforehand.

\begin{itemize}
\tightlist
\item
  Momentum flux
\end{itemize}

\begin{eqnarray}
 \tau_x = - \rho C_{M}|V_a| u_a
\end{eqnarray}

\begin{eqnarray}
 \tau_y = - \rho C_{M}|V_a| v_a
\end{eqnarray}

where \(\tau_x\) and \(\tau_y\) are the momentum fluxes (surface stress)
of the zonal and meridional directions, respectively.

\begin{itemize}
\tightlist
\item
  Sensible heat flux
\end{itemize}

\begin{eqnarray}
 H_s = c_p \rho C_{Hs}|V_a| (T_s - (P_s/P_a)^{\kappa}T_a)
\end{eqnarray}

where \(H_s\) is the sensible heat flux from the sea surface;
\(\kappa = R_{air} / c_p\) and \(R_{air}\)are the gas constants of air;
and \(c_p\) is the specific heat of air.

\begin{itemize}
\tightlist
\item
  Bare sea surface evaporation flux
\end{itemize}

\begin{eqnarray}
\hat{F}q^P_{1/2} = \rho_{1/2} C_E |{\mathbf{v}}_1| \left( q^\star(T_0) - q_1 \right)
\end{eqnarray}

\hypertarget{bulk-factor-blkcof}{%
\subsubsection{\texorpdfstring{1.3.1 Bulk factor
\texttt{{[}BLKCOF{]}}}{1.3.1 Bulk factor {[}BLKCOF{]}}}\label{bulk-factor-blkcof}}

The bulk Richardson number (\(R_{iB}\)), which is used as a benchmark
for the stability between the atmospheric surfaces, is

\begin{eqnarray}
R_{iB} =
            \frac{ \frac{g}{\theta_s} (\theta_1 - \theta(z_0))/z_1 }
              { (u_1/z_1)^2                                  }
       = \frac{g}{\theta_s}
         \frac{T_1 (p_s/p_1)^\kappa - T_0 }{u_1^2/z_1} f_T
\end{eqnarray}

Here, \(g\) is the gravitational accerelation, \(\theta_s\)
(\(\Theta_0\) in MATSIRO description) is the basic potential
temperature, \(T_1\) is the atmospheric temperature of the 1st layer,
\(T_0\) is the surface surface temperature, \(p_s\) is the surface
pressure, \(p_1\) is the pressure of the 1st layer, \$\kappa \$ is the
Karman constant, and

\begin{eqnarray}
f_T = (\theta_1 - \theta(z_0))/(\theta_1 - \theta_0)
\end{eqnarray}

The bulk coefficients of \(C_M,C_H,C_E\) are calculated according to
\href{./papers/Louis1979_Article_AParametricModelOfVerticalEddy.pdf}{Louis
(1979)} and
\href{./papers/Louis1982_a_short_history_of_the_operational_pbl_parameterization_at_ecmwf.pdf}{Louis
{\emph{et al.}}(1982)}. However, corrections are made to take into
account the difference between momentum and heat roughness. If the
roughnesses for momentum, heat, and water vapor are set to
\(z_{0,M}, z_{0,H}, z_{0,E}\), respectively, the results are generally
\(z_{0,M} > z_{0,H}, z_{0,E}\), but the bulk coefficients for heat and
water vapor for the fluxes from the height of \(z_{0,M}\) are also set
to \(\widetilde{C_H}\), \(\widetilde{C_E}\) first, and then corrected.

\begin{eqnarray}
    C_M = \left\{
      \begin{array}{lr}
      C_{0,M} [ 1 + (b_M/e_M)  R_{iB} ]^{-e_M}
            &,
          R_{iB} \geq 0 \\
      C_{0,M} \left[ 1 - b_M R_{iB} \left( 1+ d_M b_M C_{0,M}
                                  \sqrt{\frac{z_1}{z_{0,M}}| R_{iB}|} \,
                                  \right)^{-1} \right]     
          &,
          R_{iB} < 0 \\
      \end{array} \right.
\end{eqnarray}

\begin{eqnarray}
    \widetilde{C_H} = \left\{
      \begin{array}{lr}
      \widetilde{C_{0,H}} [ 1 + (b_H/e_H) R_{iB} ]^{-e_H}
            &,
          R_{iB} \geq 0 \\
      \widetilde{C_{0,H}} \left[ 1 - b_H R_{iB}
                                  \left( 1+ d_H b_H \widetilde{C_{0,H}}
                                  \sqrt{\frac{z_1}{z_{0,M}}| R_{iB}|} \,
                                  \right)^{-1} \right]
             &,     
          R_{iB} < 0 \\
      \end{array} \right.
\end{eqnarray}

\begin{eqnarray}
    C_H = \widetilde{C_H} f_T
\end{eqnarray}

\begin{eqnarray}
    \widetilde{C_E} = \left\{
      \begin{array}{lr}
      \widetilde{C_{0,E}} [ 1 + (b_E/e_E) R_{iB} ]^{-e_E}
            &,
          R_{iB} \geq 0 \\
      \widetilde{C_{0,E}} \left[ 1 - b_E R_{iB}
                                  \left( 1+ d_E b_E \widetilde{C_{0,E}}
                                  \sqrt{\frac{z_1}{z_{0,M}}| R_{iB}|} \,
                                  \right)^{-1} \right]      
          &,
          R_{iB} < 0 \\
      \end{array} \right.
\end{eqnarray}

\begin{eqnarray}
    C_E = \widetilde{C_E} f_q
\end{eqnarray}

\(C_{0M}, \widetilde{C_{0H}}, \widetilde{C_{0E}}\) is the bulk
coefficient (for fluxes from \(z_{0M}\)) at neutral,

\begin{eqnarray}
    C_{0M}  =  \widetilde{C_{0H}}  =  \widetilde{C_{0E}}  =
       \frac{k^2}{\left[\ln \left(\frac{z_1}{z_{0M}}\right)\right]^2 }
\end{eqnarray}

Correction Factor \(f_q\) is ,

\begin{eqnarray}
  f_q = (q_1 - q(z_0))/(q_1 - q^{\ast}(\theta_0))
\end{eqnarray}

but the method of calculation is omitted. The coefficients of Louis
factors are \(( b_M, d_M, e_M ) = ( 9.4, 7.4, 2.0 )\),
\(( b_H, d_H, e_H ) = ( b_E, d_E, e_E ) = ( 9.4, 5.3, 2.0 )\).

is a correction factor, which is approximated from the uncorrected bulk
Richardson number, but we abbreviate the calculation here.

\hypertarget{radiation-flux-at-sea-surface-radsfc}{%
\subsection{\texorpdfstring{1.4 Radiation Flux at Sea Surface
\texttt{{[}RADSFC{]}}}{1.4 Radiation Flux at Sea Surface {[}RADSFC{]}}}\label{radiation-flux-at-sea-surface-radsfc}}

For the ground surface albedo \(\alpha_{(d,b)}\), \(b=1,2\) represent
the visible and near-infrared wavelength bands, respectively. Also,
\(d=1,2\) are direct and scattered, respectively. For the downward
shortwave radiation \(SW^\downarrow\) and upward shortwave radiation
\(SW^\uparrow\) incident on the earth's surface, the direct and
scattered light together are

\begin{eqnarray}
    SW^\downarrow = SW^\downarrow_{(1,1)}+SW^\downarrow_{(1,2)}+SW^\downarrow_{(2,1)}+SW^\downarrow_{(2,2)} \\
SW^\uparrow = SW^\downarrow_{(1,1)}\cdot\alpha_{(1,1)}+SW^\downarrow_{(1,2)}\cdot\alpha_{(1,2)}+SW^\downarrow_{(2,1)}\cdot\alpha_{(2,1)}+SW^\downarrow_{(2,2)}\cdot\alpha_{(2,2)}
\end{eqnarray}

\hypertarget{sea-surface-heat-balance-ocnslv}{%
\subsection{\texorpdfstring{1.5 Sea Surface Heat Balance
\texttt{{[}OCNSLV{]}}}{1.5 Sea Surface Heat Balance {[}OCNSLV{]}}}\label{sea-surface-heat-balance-ocnslv}}

The comments for some variables say ``soil'', but this is because the
program was adapted from a land surface scheme, and has no particular
meaning.

\begin{itemize}
\tightlist
\item
  Outputs
\end{itemize}

\setlength\LTleft{0pt}\setlength\LTright{0pt}\begin{longtable}[]{@{}lllll@{}}
\toprule\relax
Meaning & Presentation & Variable & dimension & unit \\
\midrule\relax
\endhead
surface water flux & \(W_{free/ice}\) & WFLUXS & IJLODM,2 & -- \\
upward long wave & \(LW^\uparrow\) & RFLXLU & IJLODM & -- \\
flux balance & \(F\) & SFLXBL & IJLODM & -- \\
\bottomrule
\end{longtable}

\begin{itemize}
\tightlist
\item
  Inputs variables
\end{itemize}

\setlength\LTleft{0pt}\setlength\LTright{0pt}\begin{longtable}[]{@{}lll@{}}
\toprule\relax
Meaning & Presentation & Variable \\
\midrule\relax
\endhead
sensible heat flux coefficent & \(\frac{\partial H}{\partial T_s}\) &
DTFDS \\
latent heat flux coefficient & \(\frac{\partial E}{\partial T_s}\) &
DQFDS \\
surface heat flux coefficient & \(\frac{\partial G}{\partial T_s}\) &
DGFDS \\
downward SW radiation & \(SW^\downarrow\) & RFLXSD \\
upward SW radiation & \(SW^\uparrow\) & RFLXLU \\
downward LW radiation & \(LW^\downarrow\) & RFLXLD \\
sea surface albedo & \(\alpha\) & GRALBL \\
sea ice concentration & \(R_{ice}\) & GRICR \\
\bottomrule
\end{longtable}

\begin{itemize}
\tightlist
\item
  Modified in this subroutine
\end{itemize}

\setlength\LTleft{0pt}\setlength\LTright{0pt}\begin{longtable}[]{@{}lllll@{}}
\toprule\relax
Meaning & Presentation & Variable & dimension & unit \\
\midrule\relax
\endhead
surface temperature & \(T_s\) & GDTS & IJLODM & -- \\
surface heat flux from \texttt{seaBC} & \(G\) & GFLUXS & IJLODM & -- \\
sensible heat flux & \(H\) & TFLUXS & IJLODM & -- \\
latent heat flux & \(E\) & QFLUXS & IJLDSM & -- \\
\bottomrule
\end{longtable}

\begin{itemize}
\tightlist
\item
  Others (appeared in texts)
\end{itemize}

\setlength\LTleft{0pt}\setlength\LTright{0pt}\begin{longtable}[]{@{}lllll@{}}
\toprule\relax
Meaning & Presentation & Variable & dimension & unit \\
\midrule\relax
\endhead
sea surface albedo for shortwave radiation (ice-free) & \(\alpha_S\) &
-- & {[}-{]} & -- \\
the Stefan-Boltzmann constant & \(\sigma\) & STB & -- & -- \\
\bottomrule
\end{longtable}

Reference:
\href{https://ccsr.aori.u-tokyo.ac.jp/~hasumi/COCO/coco4.pdf}{Hasumi,
2015, Appendices A}

Downward radiative fluxes are not directly dependent on the condition of
the sea surface, and their observed values are simply specified to drive
the model. Shotwave emission from the sea surface is negligible, so the
upward part of the shortwave radiative flux is accounted for solely by
reflection of the incoming downward flux. Let \(\alpha _S\) be the sea
surface albedo for shortwave radiation. The upward shortwave radiative
flux is represented by

\begin{eqnarray}
    SW^\uparrow = - \alpha_S SW^\downarrow
\end{eqnarray}

On the other hand, the upward longwave radiative flux has both
reflection of the incoming flux and emission from the sea surface. Let
\(\alpha\) be the sea surface albedo for longwave radiation and
\(\epsilon\) be emissivity of the sea surface relative to the black body
radiation. The upward shortwave radiative flux is represented by

\begin{eqnarray}
    LW^\uparrow = - \alpha LW^\downarrow + \epsilon \sigma T_s ^4
\end{eqnarray}

where \(\sigma\) is the Stefan-Boltzmann constant and \(T_s\) is surface
temperature. If sea ice exists, snow or sea ice temperature is
considered by fractions. When radiative equilibrium is assumed,
emissivity becomes identical to co-albedo:

\begin{eqnarray}
    \epsilon = 1 - \alpha
\end{eqnarray}

The net surface flux is presented by

\begin{eqnarray}
    F^*=H + (1-\alpha)\sigma T_s^4 + \alpha LW^\uparrow - LW^\downarrow +SW^\uparrow - SW^\downarrow        
\end{eqnarray}

The heat flux into the sea surface is presented, with the surface heat
flux calculated in \texttt{PSFCM}

\begin{eqnarray}
    G^* = G - F^*
\end{eqnarray}

Note that \(G^*\) is downward positive.

The temperature derivative term is

\begin{eqnarray}
    \frac{\partial G^*}{\partial T_s} = \frac{\partial G}{\partial T_s}+\frac{\partial H}{\partial T_s}+\frac{\partial R}{\partial T_s}
\end{eqnarray}

When the sea ice exists, the sublimation flux is considered

\begin{eqnarray}
    G_{ice} = G^* - l_s E
\end{eqnarray}

The temperature derivative term is

\begin{eqnarray}
    \frac{\partial G_{ice}}{\partial T_s}=\frac{\partial G^*}{\partial T_s} + l_s\frac{\partial E}{\partial T_s}
\end{eqnarray}

Finally, we can update the surface temperature with the sea ice
concentration with
\(\Delta T_s=G_{ice} ( \frac{\partial G_{ice}}{\partial T_s})^{-1}\)

\begin{eqnarray}
    T_s = T_s +R_{ice} \Delta T_s
\end{eqnarray}

Then, the sensible and latent heat flux on the sea ice is updated.

\begin{eqnarray}
    E_{ice} = E + \frac{\partial E}{\partial T_s}\Delta T_s
\end{eqnarray}

\begin{eqnarray}
    H_{ice} = H + \frac{\partial H}{\partial T_s}\Delta T_s
\end{eqnarray}

When the sea ice does not existed, otherwise, the evaporation flux is
added to the net flux.

\begin{eqnarray}
    G_{free}=F^* + l_cE
\end{eqnarray}

Finally each flux is updated.

For the sensible heat flux, the temperature change on the sea ice is
considered.

\begin{eqnarray}
    H=H+ R_{ice}  H_{ice}
\end{eqnarray}

Then, the heat used for the temperature change is saved.

\begin{eqnarray}
    F = R_{ice} H_{ice}
\end{eqnarray}

For the upward longwave radiative flux, the temperature change on the
sea ice is considered.

\begin{eqnarray}
    LW^\uparrow=LW^\uparrow +  4\frac{\sigma}{T_s}R_{ice}  \Delta T_s
\end{eqnarray}

For the surface heat flux, the sea ice concentration is considered.

\begin{eqnarray}
    G=(1-R_{ice})G_{free} + R_{ice}G_{ice}
\end{eqnarray}

For the latent heat flux, the sea ice concentration is considered.

\begin{eqnarray}
    E=(1-R_{ice})E + R_{ice}E_{ice}
\end{eqnarray}

Each term above are saved as freshwater flux.

\begin{eqnarray}
    W_{free} = (1-R_{ice}) E
\end{eqnarray}

\begin{eqnarray}
    W_{ice} = R_{ice} E_{ice}
\end{eqnarray}

	% 物理過程:implicit 解法
	% 
\subsection{大気・地表系の拡散型収支式の解法}

\subsubsection{基本的な解法}

放射, 鉛直拡散, 接地境界層・地表過程は
一括して一部の項を implicit 扱いで
時間変化項を計算し, 最後に時間積分を行なう.
ベクトル量 {\boldmath q} の時間変化項として,
Euler 法で扱う項 TERM00000 と implicit 法で扱う項 TERM00001 とに分けて考える.
%
\begin{verbatim}
EQ=00000.
\end{verbatim}
%
一般の場合にこれを解くことは困難であるが,
近似的に TERM00002 を線形化することにより解くことが可能となる.
\begin{verbatim}
EQ=00001.
\end{verbatim}
のように行列 TERM00003 を用いて線形化する.
ここで,
\begin{verbatim}
EQ=00002.
\end{verbatim}
である. 
すると, 
TERM00004
と書けば,
\begin{verbatim}
EQ=00003.
\end{verbatim}
%
これは, 行列演算で原理的には簡単に解くことができる.

\subsubsection{基本方程式}

放射, 鉛直拡散, 接地境界層・地表過程の
方程式は, 基本的に以下のように表わされる.
%
\begin{verbatim}
EQ=00051.
EQ=00051.
EQ=00051.
EQ=00051.
EQ=00051.
\end{verbatim}
%
ここで, TERM00005,TERM00005 は
鉛直拡散による, それぞれ TERM00006,TERM00006
の鉛直上向きフラックス密度である.
また, TERM00007 は放射による
鉛直上向きエネルギーフラックス密度である.

大気は, TERM00008 を座標系で離散化される.
風速, 気温等は層 TERM00009 で定義される.
フラックスは, 層の境界 TERM00010 で定義される.
下層から上層に TERM00011 が増大する.
また, TERM00012, 
TERM00013 である.
TERM00014 座標は, 鉛直1次元過程を考えている限りは, 
定数(TERM00015)倍の違いを除いては TERM00016 座標と同じであると考えてよい.
ここで,
\begin{verbatim}
EQ=00004.
\end{verbatim}
\begin{verbatim}
EQ=00005.
\end{verbatim}
と書く.

\subsubsection{implicit時間差分}

鉛直拡散項などの線形化できる項に関しては, implicit法を用いる.
拡散係数なども予報変数に依存するが,
その係数は最初に求めるだけで, 反復的に解くことはしない.
ただし, 安定性の向上のために時間ステップの扱いを工夫している(後述). 

例えば, 離散化した TERM00017 の方程式(\ref{u-eq.orig})は,
%
\begin{verbatim}
EQ=00006.
\end{verbatim}
ここで, TERM00018 は時間ステップである.
TERM00019 等は, TERM00020 の関数であるから, その依存性を線形化して,
%
\begin{verbatim}
EQ=00007.
\end{verbatim}

従って, TERM00021 と置くと,
%
\begin{verbatim}
EQ=00008.
\end{verbatim}
%
すなわち,
\begin{verbatim}
EQ=00009.
\end{verbatim}

これは, 以下のような行列形式で書くことができる.
%
\begin{verbatim}
EQ=00010.
\end{verbatim}
%
\begin{verbatim}
EQ=00011.
\end{verbatim}
%
これを LU 分解などの方法で解けば良い.
通常は, TERM00022 は3重対角となるので, 簡単に解ける.
解いた後は, (\ref{u-flux.next})を用いて
この方法にコンシステントなフラックスを計算しておく.
TERM00023 についても全く同様である.

\subsubsection{implicit時間差分の結合}

気温, 比湿, 地中温度については, 前節のように簡単にはいかない.
%
\begin{verbatim}
EQ=00052.
EQ=00052.
\end{verbatim}
%
\begin{verbatim}
EQ=00012.
\end{verbatim}
%
\begin{verbatim}
EQ=00013.
\end{verbatim}
%
ここで, 上記の式における TERM00024, TERM00025 は
TERM00026, TERM00027 から取っていることに注意. なぜならば,
地表でのフラックスは, 以下のようになるからである.
\begin{verbatim}
EQ=00014.
\end{verbatim}
\begin{verbatim}
EQ=00015.
\end{verbatim}
\begin{verbatim}
EQ=00016.
\end{verbatim}
ここで, 地表面表皮温度を TERM00030 とすると,
TERM00031, TERM00032 (飽和比湿), TERM00033 である.
これらは, 全て, TERM00034 に依存する.
また, TERM00035 は, 全ての TERM00036 での値が TERM00037 に依存する. 

(\ref{u-matrix}) と同様に, 行列 TERM00038,TERM00038 を用いて
(\ref{deq-theta}), (\ref{deq-q}), (\ref{deq-g}) を書き直すと, 
%
 TERM00039 (TERM00040,TERM00040 について) または TERM00041 (TERM00042 について) のとき, 
%
  \begin{verbatim}
EQ=00053.
EQ=00053.
\end{verbatim}

\begin{verbatim}
EQ=00017.
\end{verbatim}

\begin{verbatim}
EQ=00018.
\end{verbatim}
%
ただし, 
\begin{verbatim}
EQ=00019.
\end{verbatim}
\begin{verbatim}
EQ=00020.
\end{verbatim}
\begin{verbatim}
EQ=00021.
\end{verbatim}

 TERM00043 (TERM00044,TERM00044 について) または TERM00045 (TERM00046 について) のとき, 
%
  \begin{verbatim}
EQ=00054.
EQ=00054.
EQ=00054.
\end{verbatim}
%
\begin{verbatim}
EQ=00022.
\end{verbatim}
%
\begin{verbatim}
EQ=00055.
EQ=00055.
EQ=00055.
\end{verbatim}
%
ただし, 
\begin{verbatim}
EQ=00023.
\end{verbatim}
\begin{verbatim}
EQ=00024.
\end{verbatim}
\begin{verbatim}
EQ=00025.
\end{verbatim}
%
ただし, (\ref{comb-g})は, 地表面のバランスの条件
\begin{verbatim}
EQ=00026.
\end{verbatim}
を, 土壌温度の式の TERM00047 の場合として扱ったもので, 
(\ref{deq-g})の表式には含まれていないことに注意. 

これら,
(\ref{comb-theta2}), (\ref{comb-q2}), (\ref{comb-g2}), 
(\ref{comb-theta}), (\ref{comb-q}), (\ref{comb-g})
を連立すると, TERM00048 個の未知変数に関して, 
同数の方程式があるので, 解くことができる.
実際の解法としては, LU分解を用いて行なうことができる.

解いた後は, 
(\ref{u-flux.next}) と同様に,
コンシステントなフラックスを求めておく.

\subsubsection{時間差分の結合式の解法}

(\ref{comb-theta}) などは, 以下のように書ける.
%
\begin{verbatim}
EQ=00027.
\end{verbatim}
ここで, TERM00049,TERM00049
の項は地表フラックスに伴う項,
その他は鉛直拡散に伴う項である.
%
ここで, 上下を逆にして行列で表現すると, 以下のようになる.
%
\begin{verbatim}
EQ=00028.
\end{verbatim}

いま, 表記の簡単のために, TERM00050 とする.  以後の議論は, これによって一般性
を失うことはない.
%
\begin{verbatim}
EQ=00029.
\end{verbatim}
%
ここで,
TERM00051,TERM00051 としたときの式,
(地表でのフラックス交換を考えない場合にあたる)
を LU 分解で解くことを考える.
%
\begin{verbatim}
EQ=00030.
\end{verbatim}

LU 分解すると,
%
\begin{verbatim}
EQ=00031.
\end{verbatim}
%
これから, 
%
\begin{verbatim}
EQ=00032.
\end{verbatim}
%
を TERM00052 について解き(TERM00053 から出発すれば, 簡単に解ける), 
それから,
%
\begin{verbatim}
EQ=00033.
\end{verbatim}
%
を, TERM00054 について, TERM00055 から出発して順に解くことができる.

TERM00056,TERM00056 だと, LU 分解は, 
%
\begin{verbatim}
EQ=00034.
\end{verbatim}
%
これから, 
%
\begin{verbatim}
EQ=00035.
\end{verbatim}
%
だが, これと, (\ref{solve-z}) を見比べると, 以下の関係があることがわかる.
%
\begin{verbatim}
EQ=00036.
\end{verbatim}
%
これを用いると,
%
\begin{verbatim}
EQ=00037.
\end{verbatim}
%
となる. すなわち,
%
\begin{verbatim}
EQ=00038.
\end{verbatim}
%
ここで, TERM00057 および, TERM00058 は,
TERM00059 とおいた式(\ref{solve-0}),
すなわち, 地表フラックスの項を考慮しない式で
LU分解を行なうことによって得られることに注意したい.
これらの項の物理的な意味としては,
地表面とのフラックス交換過程において,
大気全体が, 熱容量 TERM00060 を持ち, 
上からフラックス TERM00061 が
供給される1つの層とみなすことができることを示す.

(\ref{comb-theta2})および(\ref{comb-theta}), 
(\ref{comb-q2})および(\ref{comb-q}), 
(\ref{comb-g2})および(\ref{comb-g})  のそれぞれにおいて
(\ref{solve-1}) に対応する式が得られ, 以下のようになる.

\begin{verbatim}
EQ=00039.
\end{verbatim}
\begin{verbatim}
EQ=00040.
\end{verbatim}
\begin{verbatim}
EQ=00041.
\end{verbatim}

従って, 上の3式を連立させれば,
未知変数 TERM00062,TERM00062 を解くことができる.
これらが解ければ, 後は
(\ref{solve-x}) を順次 TERM00063,TERM00063 と解くことができる.
%
その後, 得られた温度にコンシステントなフラックスを
\begin{verbatim}
EQ=00056.
EQ=00056.
\end{verbatim}
として計算する.
%
ここでは, TERM00064 が一般行列の場合を示したが,
実際には3重対角行列となるので, さらに簡単である.

プログラム中においては,
\texttt{MODULE:[VFTND1(pimtx.F)]} で大気部分について,
\texttt{MODULE:[GNDHT1(pggnd.F)]} で地中部分について, LU分解解法の前半
(TERM00065 を求めるところ)を行ない, 
\texttt{MODULE:[SLVSFC(pgslv.F)]} において, TERM00066 の方程式を解き,
TERM00067,TERM00067 を求めている.
その後, \texttt{MODULE:[GNDHT2(pggnd.F)]} において
LU分解解法の後半を行ない, 地中について温度変化率を解き, 
収支が合うようにフラックスを補正する.
また, \texttt{MODULE:[VFTND2(pimtx.F)]} において大気中について
温度変化率を解き, 
\texttt{MODULE:[FLXCOR(pimtx.F)]} でフラックスを補正している.

\subsubsection{時間差分の結合式}

TERM00068,TERM00068 を求める結合式は, 
以下の様に条件を変えながら 3回解く. 
\begin{enumerate}
\item 地表湿潤度 TERM00069 を 1 として解く. 地表温度は変数. 
\item \texttt{MODULE:[GNDBET]} で得られた地表湿潤度で解く. 
      地表温度は変数.
\item \texttt{MODULE:[GNDBET]} で得られた地表湿潤度で解く. 
      融雪等の場合, 地表温度は氷点に固定. 
\end{enumerate}

1 回目の計算は, 可能蒸発量 TERM00070 を見積もるために行なわれる.
(地表湿潤度が小さいときに, モデルのエネルギーバランスから
得られた TERM00071 を用いて, 可能蒸発量を
TERM00072
として診断すると, 非現実的な大きな値になってしまう.)
可能蒸発量は,
\begin{verbatim}
EQ=00057.
\end{verbatim}
となる.
添字 TERM00073 は, 補正後の意味で, 
これが得られた温度等にコンシステントなフラックスである.
    
2 回目以降の計算では, 
\begin{enumerate}
\item 1回めの計算で求めた可能蒸発量に
      地表湿潤度(蒸発効率) TERM00074 を
      かけたものを蒸発量 TERM00075 とする.
      \begin{verbatim}
EQ=00042.
\end{verbatim}

\item 蒸発量 TERM00076 は
      \begin{verbatim}
EQ=00043.
\end{verbatim}
      で求められるものとして, 
      改めてエネルギーバランスを解き直す.
\end{enumerate}
の2通りの蒸発量の計算方法を用いることができる
(標準では1.の方法を用いる).
3 回目の計算は, 融雪, 融氷のときや, 混合層海洋で海氷が生成するときに
地表温度を氷点などに固定してエネルギーバランスを解くために行なう. 
このとき, 融雪などの水の相変化に使われるエネルギー量が診断的に求まり, 
後で融雪量などを計算する際に用いられる. 

結合式の具体的な形は以下の様である. 
%
\begin{verbatim}
EQ=00058.
EQ=00058.
\end{verbatim}
%
ここで, TERM00078,TERM00078 および TERM00079,TERM00079 は, 
LU分解解法の前半を行なって得られる, 行列およびベクトルの成分である. 
地表面が雪または氷に覆われているときには, 潜熱 TERM00080 の代わりに
昇華の潜熱 TERM00081 を用いる. TERM00082 は水の融解の潜熱である. 
%
ただし, 第2回めの計算で, 
蒸発の見積もりとして第一の方法を用いた場合は, 以下のようになる.
\begin{verbatim}
EQ=00059.
EQ=00059.
\end{verbatim}

3 回目の計算で, 地表温度を固定した場合の結合式は, 
\begin{verbatim}
EQ=00060.
EQ=00060.
\end{verbatim}
ここで, TERM00083 は固定する温度までの変化率で, 
\begin{verbatim}
EQ=00044.
\end{verbatim}
TERM00084 は, 融雪, 融氷の場合は 273.15K, 
海氷の生成の場合は 271.15K である. 
また, 蒸発計算の第2の方法を用いる場合には,
同様に TERM00085 のかわりに TERM00086 を用い,
TERM00087 の微分項を 0 として計算する.
このとき, 
\begin{verbatim}
EQ=00061.
EQ=00061.
\end{verbatim}
で計算される TERM00088 は地表エネルギーバランスで, 
水の相変化に使われる分のエネルギーである. 

\subsubsection{implicit 時間差分における時間ステップの扱い}

鉛直拡散項の時間差分には implicit 法を用いているが, 
一般に拡散係数が非線形であり, この係数を explicit に評価している
ことにより, 数値不安定の問題が生じ得る. 
安定性の向上のために, Kalnay and Kanamitsu (19??) にならって
時間ステップの扱いを工夫している. 

簡単化のために以下の常微分方程式を例に取って説明する. 
\begin{verbatim}
EQ=00045.
\end{verbatim}
係数 TERM00089 が非線形性を表す. 
係数のみ explicit に評価して素直に implicit 差分化すると次式のようになる. 
\begin{verbatim}
EQ=00046.
\end{verbatim}
ところが, ここでは 2ステップ先の TERM00090 の値 TERM00091 を考えて, 
\begin{verbatim}
EQ=00047.
\end{verbatim}
\begin{verbatim}
EQ=00048.
\end{verbatim}
とする. 
一般に, (\ref{modify-fd1}), (\ref{modify-fd2}) のようにする方が
(\ref{normal-fd})よりも安定性が良いことが知られている. 

(\ref{modify-fd1}), (\ref{modify-fd2}) を, 時間変化率を求める
形に書き直すと以下を得る. 
\begin{verbatim}
EQ=00049.
\end{verbatim}
\begin{verbatim}
EQ=00050.
\end{verbatim}
すなわち, 時間変化率を求める際の時間ステップには, 
時間積分のステップの 2 倍を用いる. 

 % 担当者なし
	% 物理過程:重力波抵抗
	% 
\subsection{重力波抵抗}

\subsubsection{重力波抵抗スキームの概要}

重力波抵抗スキームは,
サブグリッドスケールの地形によって励起される
重力波の上方への運動量フラックスを表現し,
その収束に伴う水平風の減速を計算する.
主な入力データは, 東西風 TERM00000, 南北風 TERM00001, 気温 TERM00002, であり,
出力データは東西風と南北風の時間変化率,
TERM00003,TERM00003, である.

計算手順の概略は以下の通りである.
%
\begin{enumerate}
\item 地表面での運動量フラックスを
      地表高度の分散, 
      最下層での水平風速, 成層安定度などから求める.
\item 運動量フラックスを持つ重力波の上方への伝播を考える.
      運動量フラックスが臨界フルード数から決まる
      臨界フラックスを越える場合には,
      砕波が起こってフラックスはその臨界値となるとする.
\item 運動量フラックスの各層での収束に応じた
      水平風の時間変化を計算する.
\end{enumerate}

\subsubsection{局所フルード数と運動量フラックスの関係}

地表起源の重力波による
水平運動量の鉛直フラックスを考えると,
ある高度でのフラックス TERM00004 と
局所フルード数 TERM00005 との間には,
\begin{quote}
EQ=00000.
\label{p-grav:tau-fl}
\end{quote}
の関係が成り立つ.
ここで, TERM00006 は
ブラントバイサラ振動数, 
TERM00007 は大気の密度, 
TERM00008 は風速, TERM00009 は地表高度の波打ちの水平スケールに対応する
比例定数である.
これから,
\begin{quote}
EQ=00001.
\label{p-grav:fl-tau}
\end{quote}

局所フルード数 TERM00010 は,
ある値, 臨界フルード数 TERM00011 を越えることができないとする.
(\ref{p-grav:tau-fl}) から計算される
局所フルード数が臨界フルード数 TERM00012 を越える場合には
重力波は過飽和となり,
臨界フルード数に対応する運動量フラックスまで
フラックスは減少する.

\subsubsection{地表での運動量フラックス}

地表面で励起される重力波による
水平運動量の鉛直フラックスの大きさ TERM00013 は,
ただし, 地表での局所フルード数 
TERM00014 を
(\ref{p-grav:fl-tau}) に代入することにより,
%
\begin{quote}
EQ=00002.
\end{quote}
%
と見積られる.
ここで, 
TERM00015 は地表風速,
TERM00016,TERM00016 はそれぞれ地表付近の大気の
安定度と密度である.
TERM00017 はサブグリッドの地表高度変化の指標であり,
地表高度の標準偏差 TERM00018 に等しいとする.

ここで, 地表での局所フルード数 
TERM00019 が 臨界フルード数
TERM00020 を越えるときは, 
運動量フラックスは TERM00021 を(\ref{p-grav:fl-tau}) に代入した値に
抑えられるとする.
すなわち,
\begin{quote}
EQ=00003.
\end{quote}

\subsubsection{上層での運動量フラックス}

レベル TERM00022 での運動量フラックス TERM00023 が
求められているとする.
TERM00024 は, 飽和が起こらないときには
TERM00025 に等しい.
この運動量フラックス TERM00026 が,
TERM00027 レベルでの臨界フルード数から計算される運動量フラックス
を上回るときには, TERM00028 層内で砕波が起こり,
運動量フラックスは臨界に対応するフラックスまで減少する.

\begin{quote}
EQ=00004.
\end{quote}

ただし TERM00029 は,
各層での風速ベクトルの,
最下層の水平風の方向に対する射影成分の大きさであり,
\begin{quote}
EQ=00005.
\end{quote}

\subsubsection{運動量収束による水平風の時間変化の大きさ}

水平風の射影成分 TERM00030 の時間変化率は,
\begin{quote}
EQ=00006.
\end{quote}
%
によって求められる.すなわち,
%
\begin{quote}
EQ=00007.
\end{quote}
%
これを用いて,
東西風と南北風の時間変化率は以下のように計算される.
\begin{quote}
EQ=00008.\\
EQ=00008.
\end{quote}

\subsubsection{その他の留意点}

\begin{enumerate}
\item 最下層の風速が小さく TERM00031 のとき,また,
      地表の起伏が小さく TERM00032 のときは, 
      地表で重力波が励起されないと仮定する.
\end{enumerate}

 % 担当者なし
	% 物理過程:対流調節
	% \hypertarget{drying-convection-regulation}{%
\subsection{Drying convection
regulation}\label{drying-convection-regulation}}

\hypertarget{overview-of-drying-convective-regulation}{%
\subsubsection{Overview of Drying Convective
Regulation}\label{overview-of-drying-convective-regulation}}

Drying convection control , Convective instability in the stratum
between two successive levels, In other words, if the temperature decay
rate is greater than the dry adiabatic decay rate The temperature
reduction rate is adjusted to the dry adiabatic reduction rate. Water
vapor and other substances are mixed in at this time. The main input
data are temperature \(T\) and specific humidity \(q\), The output data
is the adjusted air temperature \(T\) and specific humidity \(q\).

Essentially, if vertical diffusion is efficient, then The vertical
convective instability should be basically removed. However, it may be
in short supply in the stratosphere, A convection adjustment has been
added to stabilize the calculation.

\hypertarget{drying-convection-regulation-procedures.}{%
\subsubsection{Drying convection regulation
procedures.}\label{drying-convection-regulation-procedures.}}

The conditions for convective instability in the layers \((k-1,k)\) are

\begin{eqnarray}
\frac{T_{k-1} - T_{k}}{p_{k-1} - p_{k}} 
  > \frac{R}{C_p} \bar{T_{k-1/2}}
  = \frac{R}{C_p}
    \frac{\Delta p_{k-1} T_{k-1} + \Delta p_{k} T_{k}}
         {\Delta p_{k-1} + \Delta p_{k}} 
\end{eqnarray}

Namely,

\begin{eqnarray}
 S = T_{k-1} - T_{k}
     - \frac{R}{C_p} 
        \frac{\Delta p_{k-1} T_{k-1} + \Delta p_{k} T_{k}}
         {\Delta p_{k-1} + \Delta p_{k}} 
       (p_{k-1} - p_{k})
   > 0 
\end{eqnarray}

\begin{quote}
\protect\hypertarget{p-adj:cond}{}{p-adj{[}p-adj:cond\en{]}}.
\end{quote}

is a condition.

When this is satisfied ,

\begin{eqnarray}
T_{k-1}  \leftarrow  \frac{\Delta p_{k}}{\Delta p_{k-1} + \Delta p_{k}} S \\
T_{k}  \leftarrow  \frac{\Delta p_{k-1}}{\Delta p_{k-1} + \Delta p_{k}} S 
\end{eqnarray}

to compensate for the temperature. Furthermore,

\begin{eqnarray}
q_{k-1}, q_{k} \leftarrow
     \frac{\Delta p_{k-1} q_{k-1} + \Delta p_{k} q_{k}}
          {\Delta p_{k-1} + \Delta p_{k}} 
\end{eqnarray}

to average the values of specific humidity etc. in the two layers.

When you do this, The layers above and below it may become unstable.
That's why, Repeating this operation from the lower level to the upper
level. Repeat until there is no more layer of convective instability.
However, considering the calculation error and so on, (as a condition of
\protect\hyperlink{p-adj:condo}{\textbackslash p{[}p-adj:condo{]}}), It
is considered to have converged if S is less than or equal to some small
finite value that is not zero.

Currently, the standard adjustment is between the second and third layer
from the bottom and above.
 % 廃止
	% 参考文献
	% 
\section*{参考文献}

\begin{description}

\item Asselin, R. A., 1972:
      Frequency filter for time integrations.
      {\it Mon. Wea. Rev.}, {\bf 100}, 487–490.

\item Arakawa, A. and W.H. Schubert, 1974:
      Interactions of cumulus cloud ensemble with the large-scale
      environment. Part I. {\em J. Atmos. Sci.,\/} {\bf 31,} 671--701.

\item Arakawa A., Suarez M.J., 1983:
      Vertical differencing of the primitive equations
      in sigma coodinates.
      {\it Mon. Weather Rev.}, {\bf 111}, 34--45.

\item Blackadar, A. K., 1962:
      The vertical distribution of wind and turbulent exchange in neutral atmosphere.
      {\it J. Geophys. Res.}, {\bf 67}, 3095–3102.

\item Bourke, W., 1988:
      Spectral methods in global climate and weather prediction models.
      {\it  in Physically-Based Modelling and Simulation of Climate
              and Climatic Change. Part I.}, 169--220., Kluwer.

\item Haltiner, G.J. and R.T. Williams,  1980:
      Numerical Prediction and Dynamic Meteorology (2nd ed.),
      John Wiley \& Sons, 477pp.

\item Katayama A., 1972:
      A Simplified Scheme for Computing Radiative Transfer
      in the Troposphere, Numerical Simulation of Weather and Climate.
      {\it Technical Report No. 6, University of California, Los Angeles}, 77 pp.

\item Kondo J., 1993:
      A new bucket model for predicting water content
      in the surface soil layer.
      {\it J. Japan Soc. Hydrol. Water Res.}, {\bf 6}, 344-349. (in Japanese)

\item Le Treut H. and Z.-X. Li, 1991:
      Sensitivity of an atmospheric general circulation model to
      prescribed SST changes: feedback effects associated with the
      simulation of cloud optical properties.
      {\it Climate Dynamics}, {\bf 5}, 175-187.

\item Louis, J., 1979:
      A parametric model of vertical eddy fluxes in the
      atmosphere.
      {\it Bound. Layer Meteor.}, {\bf 17}, 187--202.

\item Louis, J., M. Tiedtle, J.-F. Geleyn, 1982:
      A short history of the PBL parameterization at ECMWF.
      {\it Workshop on Planetary Boundary layer Parameterization},
      59-80, ECMWF, Reading U.K.

\item McFarlane, N. A., 1987:
      The effect of orographically excited gravity wave drag
      on the general circulation of
      the lower stratosphere and troposphere.
      {\itJ. Atmos. Sci.}, {\bf 44}, 1775–1800.

\item Miller, M.J., A.C.M. Beljaars and T.N. Palmer, 1992:
      The sensitivity of the ECMWF model
      to the parameterization of evaporation from the tropical oceans.
      {\it J. Climate}, {\bf 5}, 418-434.

\item Moorthi S. and M.J. Suarez, 1992:
      Relaxed Arakawa-Scubert: A parameterization of moist convection
      for general circulation models.
      {\em Mon. Weather Rev.,\/} {\bf 120} 978--1002.

\item Mellor, G.L. and T. Yamada, 1974:
      A hierarchy of turbulence closure models
      for planetary boundary layers.
      {\it J. Atmos. Sci.}, {\bf 31}, 1791--1806.

\item Mellor, G.L. and T. Yamada, 1982:
      Development of a turbulence closure
      model for geophysical fluid problems.
      {\it Rev. Geophys Atmos. Phys.}, {\bf 20}, 851--875.

\item Nakajima T. and M. Tanaka, 1986:
      Matrix formulation for the transfer of solar radiation
      in a plane-parallel scattering atmosphere.
      {\it J. Quant. Spectrosc. Radiat. Transfer}, {\bf 35}, 13-21.

\end{description}
 % まだ作っていない
	\hypertarget{references-dynamics}{%
\subsection{References (dynamics)}\label{references-dynamics}}

\begin{enumerate}
\def\labelenumi{\arabic{enumi}.}
\item
  Arakawa, A., and C. S. Konor, 1996: Vertical Differencing of the
  Primitive Equations Based on the Charney--Phillips Grid in Hybrid
  $\sigma$-p Vertical Coordinates. Monthly Weather Review, 124,
  511--528.
\item
  Asselin, R., 1972: Frequency Filter for Time Integrations. Monthly
  Weather Review, 100, 487--490.
\item
  Bourke, W., 1988: Spectral methods in global climate and prediction
  models. In ``Physically-based modeling and simulation of climate and
  climatic change (Part 1)''. M. E. Schlesinger, Ed., Kluwer Academic
  Publisher, 169-222.
\item
  Colella, P., and P. R. Woodward, 1984: The Piecewise Parabolic Method
  (PPM) for Gas-Dynamical Simulations. Journal Of Computational Physics,
  54, 174-201.
\item
  Haltiner, G J and T Williams, 1980: Numerical Prediction and Dynamic
  Meteorology. Second Edition, John Wiley and Sons, 477pp.
\item
  Lin, S.-J., and R. B. Rood, 1996: Multidimensional flux-form
  semi-Lagrangian transport schemes. Monthly Weather Review, 124,
  2046--2070.
\item
  Mesinger, F., and A. Arakawa, 1976: Numerical methods used in
  atmospheric models, GARP Publications Series 17, WMO-ICSU Joint
  Organising Committee, 64pp.
\item
  Williams, P. D., 2009: A proposed modification to the Robert--Asselin
  time filter. Monthly Weather Review, 137, 2538--2546.
\end{enumerate}


	%
	%
\end{document}

