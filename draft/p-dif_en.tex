\hypertarget{turbulence-scheme}{%
\subsection{Turbulence scheme}\label{turbulence-scheme}}

The turbulence scheme represents the effect of subgrid-scale turbulence
on the grid-averaged quantities. The turbulence scheme accounts for the
vertical diffusion of momentum, heat, water and other tracers. The
Mellor-Yamada-Nakanishi-Niino scheme (the MYNN scheme; Nakanishi 2001;
Nakanishi and Niino 2004), an improved version of the Mellor-Yamada
scheme (Mellor 1973; Mellor and Yamada 1974; Mellor and Yamada 1982),
has been used as the turbulence scheme in MIROC since version 5. Closure
level is 2.5. Level 3 is also available, but it is a non-standard option
because it does not provide a performance gain worth the increase in
computation.

In the MYNN scheme, liquid water potential temperature \(\theta_l\) and
total water \(q_w\) are used as thermodynamic variables and are defined
as follows, respectively. These are conserved quantities that do not
depend on the phase change of water.

\begin{eqnarray} \theta_l \equiv \left(T - \frac{L_v}{C_p}q_l - \frac{L_v+L_f}{C_p}q_i \right) \left(\frac{p_s}{p}\right)^{\frac{R_d}{C_p}} \end{eqnarray}

\begin{eqnarray} q_w \equiv q_v+q_l+q_i \end{eqnarray}

where \(T\) and \(p\) are temperature and pressure; \(q_v\), \(q_l\),
and \(q_i\) are specific humidity, cloud water, and cloud ice; \(C_p\)
and \(R_d\) are specific heat at constant pressure and gas constant of
dry air; and \(L_v\) and \(L_f\) are latent heat of vaporization and per
unit mass, respectively. \(p_s\) is 1000 hPa.

In Level 2.5, the amount of kinetic energy of turbulence multiplied by
two is a forecast variable, and its time evolution is also calculated
within this scheme. This value is defined by

\begin{eqnarray}q^2 \equiv \langle u^2 + v^2 + w^2 \rangle\end{eqnarray}

where \(u\), \(v\), and \(w\) are velocities in the zonal, meridional,
and vertical directions, respectively. Hereafter in this chapter,
uppercase variables will represent grid mean quantities, and the
lowercase variables will represent the deviation from them.
\(\langle \ \rangle\) denotes an ensemble mean. For Level 3,
\(\langle {\theta_l}^2 \rangle\), \(\langle {q_w}^2 \rangle\),
\(\langle \theta_l q_w \rangle\) are also forecast variables, but the
details are not explained here.

The outline of the calculation procedure is given as follows along with
the names of the subroutines.

\begin{enumerate}
\def\labelenumi{\arabic{enumi}.}
\tightlist
\item
  calculation of the friction velocity and the Monin-Obukhov length
\item
  calculation of the buoyancy coefficients in consideration of partial
  condensation {[}\texttt{VDFCND}{]}
\item
  calculation of the stability functions in Level 2
  {[}\texttt{VDFLEV2}{]}
\item
  calculation of the depth of the planetary boundary layer
  {[}\texttt{PBLHGT}{]}
\item
  calculation of the master turbulent length scale {[}\texttt{VDFMLS}{]}
\item
  calculation of the diffusion coefficients, and the vertical fluxes and
  their derivatives {[}\texttt{VDFLEV3}{]}
\item
  calculation of the generation and dissipation terms of turbulent flow
  {[}\texttt{VDFLEV3}{]}
\item
  computation of implicit time integration of prognostic variables
\end{enumerate}

\hypertarget{surface-boundary-layer}{%
\subsubsection{Surface boundary layer}\label{surface-boundary-layer}}

The friction velocity \(u_*\) and the Monin-Obukhov length \(L_M\) are
given as follows.

\begin{eqnarray}u_*=\left({\langle uw \rangle_g}^2+{\langle vw \rangle_g}^2 \right)^\frac{1}{4}\end{eqnarray}

\begin{eqnarray}L_M=-\frac{\Theta_{v,g} {u_*}^3}{kg \langle w\theta_v \rangle_g}\end{eqnarray}

where the subscript \(g\) indicates that the value is near the surface
of the earth, and the value of the lowest layer of the model is used.
\(\Theta_v\) and \(\theta_v\) denote virtual potential temperature,
\(k\) the Von Karman constant, and \(g\) the gravitational acceleration.

\hypertarget{diagnosis-of-the-buoyancy-coefficients}{%
\subsubsection{Diagnosis of the buoyancy
coefficients}\label{diagnosis-of-the-buoyancy-coefficients}}

The calculation of the buoyancy term appearing in the turbulence
equation requires the value of \(\langle w\theta_v \rangle\). Following
Mellor (1982), this term can be written as

\begin{eqnarray}\langle w\theta_v \rangle=\beta_\theta \langle w\theta_l \rangle + \beta_q \langle wq_w \rangle\end{eqnarray}

by assuming a probability distribution in the grid of \(\theta_l\),
\(q_w\). However, unlike Mellor (1982) and Nakanishi and Niino (2004),
the probability distribution is not Gaussian, but triangular in shape as
given by the PDF-based prognostic large-scale condensation scheme
(Watanabe et al.~2008). The buoyancy coefficients \(\beta_\theta\),
\(\beta_q\) are written as follows.

\begin{eqnarray}\beta_\theta=1+\epsilon Q_w-(1+\epsilon)Q_l-Q_i-\tilde{R}abc\end{eqnarray}

\begin{eqnarray}\beta_q=\epsilon \Theta +\tilde{R}ac\end{eqnarray}

where \(\epsilon=R_v/R_d-1\). \(R_d\) and \(R_v\) are the gas constants
for dry air and water vapor, respectively. Also,

\begin{eqnarray}a=\left(1+\frac{L_v}{C_p}\left.\frac{\partial Q_s}{\partial T}\right|_{T=T_l}\right)^{-1}\end{eqnarray}

\begin{eqnarray}b=\frac{T}{\Theta}\left.\frac{\partial Q_s}{\partial T}\right|_{T=T_l}\end{eqnarray}

\begin{eqnarray}c=\frac{\Theta}{T}\frac{L_v}{C_p}\left[1+\epsilon Q_w-(1+\epsilon)Q_l-Q_i\right]-(1+\epsilon)\Theta\end{eqnarray}

\begin{eqnarray}\tilde{R}=R\left\{1-a\left[Q_w-Q_s(T_l)\right]\frac{Q_l}{2\sigma_s}\right\}-\frac{{Q_l}^2}{4{\sigma_s}^2}\end{eqnarray}

\begin{eqnarray}{\sigma_s}^2=\langle {q_w}^2 \rangle -2b \langle \theta_l q_w \rangle + b^2\langle {\theta_l}^2 \rangle\end{eqnarray}

where \(R,Q_l\) are the amount of cloud and liquid water diagnosed from
the probability distribution in the grid, respectively, and \(Q_s\) is
the amount of saturated water vapor.

\hypertarget{stability-functions-in-the-level-2}{%
\subsubsection{Stability functions in the Level
2}\label{stability-functions-in-the-level-2}}

It is known that the Mellor-Yamada Level 2.5 scheme fails to capture the
behavior of growing turbulence realistically (Helfand and Labraga 1988).
Therefore, the MYNN scheme first calculates the kinetic energy of
turbulence at Level2, \({q_2}^2/2\), where the local equilibrium is
assumed, and then applies a correction when \(q<q_2\), i.e., the
turbulence is in the growth phase. The stability functions
\(S_{H2},S_{M2}\) of Level 2, which are required for the calculation of
\(q_2\), can be obtained as follows.

\begin{eqnarray}S_{H2}=S_{HC}\frac{Rf_c-Rf}{1-Rf}\end{eqnarray}

\begin{eqnarray}S_{M2}=S_{MC}\frac{R_{f1}-Rf}{R_{f2}-Rf}S_{H2}\end{eqnarray}

where \(Rf\) denotes the flux Richardson number which is given as
follows.

\begin{eqnarray}Rf=R_{i1}\left(Ri+R_{i2}-\sqrt{Ri^2-R_{i3}Ri+R_{i4}}\right)\end{eqnarray}

\(Ri\) is the gradient Richardson number calculated as follows.

\begin{eqnarray}Ri=\frac{g}{\Theta}\left(\beta_\theta \frac{\partial \Theta_l}{\partial z}+\beta_q \frac{\partial Q_w}{\partial z}\right) \Bigg/ \left[ \left(\frac{\partial U}{\partial z}\right)^2+\left(\frac{\partial V}{\partial z}\right)^2 \right]\end{eqnarray}

The other symbols are quantities that are independent of the
environmental field and are given as follows.

\begin{eqnarray}S_{HC}=3A_2(\gamma_1+\gamma_2)\end{eqnarray}

\begin{eqnarray}S_{MC}=\frac{A_1}{A_2}\frac{F_1}{F_2}\end{eqnarray}

\begin{eqnarray}Rf_c=\frac{\gamma_1}{\gamma_1+\gamma_2}\end{eqnarray}

\begin{eqnarray}R_{f1}=B_1\frac{\gamma_1-C_1}{F_1}\end{eqnarray}

\begin{eqnarray}R_{f2}=B_1\frac{\gamma_1}{F_2}\end{eqnarray}

\begin{eqnarray}R_{i1}=\frac{1}{2S_{Mc}}\end{eqnarray}

\begin{eqnarray}R_{i2}=R_{f1}S_{MC}\end{eqnarray}

\begin{eqnarray}R_{i3}=4R_{f2}S_{MC}-2R_{i2}\end{eqnarray}

\begin{eqnarray}R_{i4}={R_{i2}}^2\end{eqnarray}

where

\begin{eqnarray}A_1=B_1\frac{1-3\gamma_1}{6}\end{eqnarray}

\begin{eqnarray}A_2=A_1\frac{\gamma_1-C_1}{\gamma_1 Pr}\end{eqnarray}

\begin{eqnarray}C_1=\gamma_1-\frac{1}{3A_1{B_1}^{\frac{1}{3}}}\end{eqnarray}

\begin{eqnarray}F_1=B_1(\gamma_1-C_1)+2A_1(3-2C_2)+3A_2(1-C_2)(1-C_5)\end{eqnarray}

\begin{eqnarray}F_2=B_1(\gamma_1+\gamma_2)-3A_1(1-C_2)\end{eqnarray}

\begin{eqnarray}\gamma_2=\frac{B_2}{B_1}\left(1-C_3\right)+\frac{2A_1}{B_1}\left(3-2C_2\right)\end{eqnarray}

and

\begin{eqnarray}
(Pr,\gamma_1,B_1,B_2,C_2,C_3,C_4,C_5)=(0.74,0.235,24.0,15.0,0.7,0.323,0.0,0.2)
\end{eqnarray}

\hypertarget{master-turbulent-length-scale}{%
\subsubsection{Master turbulent length
scale}\label{master-turbulent-length-scale}}

\hypertarget{formulation-by-nakanishi-2001}{%
\paragraph{Formulation by Nakanishi
(2001)}\label{formulation-by-nakanishi-2001}}

Nakanishi (2001) proposed the following formula as the master length
scale \(L\).

\begin{eqnarray}\frac{1}{L}=\frac{1}{L_S}+\frac{1}{L_T}+\frac{1}{L_B} \label{1} \end{eqnarray}

\(L_S, L_T, L_B\) represent the length scales in the surface layer,
convective boundary layer, and stably stratified layer, respectively,
and are formulated as follows.

\begin{eqnarray}
L_S=\left\{
    \begin{array}{lr}
      kz/3.7, &\zeta\ge 1\\
      kz/(2.7+\zeta), &0\le\zeta< 1\\
      kz(1-\alpha_4\zeta)^{0.2}, &\zeta< 0\\
    \end{array}
  \right.
\end{eqnarray}

\begin{eqnarray}L_T=\alpha_1\frac{\displaystyle \int_0^\infty{qz}\,dz}{\displaystyle \int_0^\infty{q}\,dz}\end{eqnarray}

\begin{eqnarray}
L_B=\left\{
    \begin{array}{ll}
      \alpha_2 q/N, &\partial\Theta_v/\partial z> 0 \quad\rm{and}\quad\zeta\ge 0\\
      \left[\alpha_2+\alpha_3\sqrt{q_c/L_TN}\right]q/N, &\partial\Theta_v/\partial z> 0 \quad\rm{and}\quad\zeta< 0\\
      \infty, &\partial\Theta_v/\partial z\le 0\\
    \end{array}
  \right.
\end{eqnarray}

where \(\zeta\equiv z/L_M\) is the height normalized by the
Monin-Obukhov length \(L_M\),
\(N\equiv\left[(g/\Theta)(\partial\Theta_v/\partial z)\right]^{1/2}\) is
the Brunt-Väisälä frequency, and
\(q_c\equiv [(g/\Theta)\langle w\theta_v \rangle_gL_T]^{1/3}\) is the
velocity scale in the convective boundary layer.

\hypertarget{modifications-in-the-implementation-for-miroc}{%
\paragraph{Modifications in the implementation for
MIROC}\label{modifications-in-the-implementation-for-miroc}}

The above formulation in Nakanishi (2001) is appropriate when the domain
of the model is limited to the atmospheric boundary layer and its
peripheral region. However, when the model includes the upper
troposphere, problems such as follows may arise depending on the
conditions: \(L_T\), the length scale of the convective boundary layer,
is used in the free atmosphere, and the turbulent energy in the free
atmosphere is included as \(q\) in the calculation of \(L_T\).

Therefore, for implementation in MIROC, the top height of the convective
boundary layer \(H_{PBL}\) is estimated and the region below
\(h=\sqrt{(F_H H_{PBL})^2+H_0^2}\) is considered as the region where
boundary-layer turbulence is dominant. Here, \(F_H=1.5\) and
\(H_0=500\)m.

Below the altitude \(h\), equation (1) is used as the master length
scale, but in \(L_T\), the range of integration is modified as follows.

\begin{eqnarray}\frac{1}{L}=\frac{1}{L_S}+\frac{1}{L_A}+\frac{1}{L_{max}}\end{eqnarray}

where \(L_A=\alpha_5\,q/N\) is the length scale when an air mass moves
vertically due to turbulence in stable stratification. \(\alpha_5\)
represents the effect of dissipation and \(\alpha_5=0.53\).
\(L_{max}=500\)m gives the upper limit of \(L\).

\hypertarget{estimation-of-the-top-height-of-the-convection-boundary-layer}{%
\paragraph{Estimation of the top height of the convection boundary
layer}\label{estimation-of-the-top-height-of-the-convection-boundary-layer}}

Based on Holtslag and Boville (1993), the estimate of \(H_{PBL}\) is
calculated using the bulk Richardson number \(Ri_B\) given as follows.

\begin{eqnarray}Ri_B=\frac{[g/\Theta_v(z_1)][\Theta_v(z_k)-\Theta_{v,g}](z_k-z_g)}{[U(z_k)-U(z_1)]^2+[V(z_k)-V(z_1)]^2+F_u{u_*}^2}\end{eqnarray}

where \(z_k\) is the full level altitude of the kth layer from the
bottom, \(z_1\) is the full level altitude of the lowest layer of the
model, and \(z_g\) is the surface altitude. \(F_u\) is a dimensionless
tuning parameter. Also,

\begin{eqnarray}\Theta_{v,g}=\Theta_v(z_1)+F_b \frac{\langle w\theta_v \rangle_g}{w_m}\end{eqnarray}

\begin{eqnarray}w_m=u_*/\phi_m\end{eqnarray}

\begin{eqnarray}\phi_m=\left(1-15\frac{z_s}{L_M}\right)^{-\frac{1}{3}}\end{eqnarray}

where \(z_s\) is the altitude of the surface layer, and
\(z_s=0.1H_{PBL}\). \(F_b\) is a dimensionless tuning parameter.

\(Ri_B\) is calculated in turn from \(k=2\) upward, and is linearly
interpolated between the layer where \(Ri_B>0.5\) for the first time and
the layer immediately below it. The height where \(Ri_B=0.5\) exactly is
used as \(H_{PBL}\). Since \(H_{PBL}\) is required for the calculation
of \(z_s\), \(H_{PBL}\) is first calculated using \(z_s\) with the
temporary value \(H_{PBL}=z_1-z_g\) substituted, and then the true
\(H_{PBL}\) is recalculated using \(z_s\) with this \(H_{PBL}\)
substituted.

\hypertarget{calculation-of-diffusion-coefficients}{%
\subsubsection{Calculation of diffusion
coefficients}\label{calculation-of-diffusion-coefficients}}

\hypertarget{turbulent-kinetic-energy-in-the-level-2}{%
\paragraph{Turbulent kinetic energy in the Level
2}\label{turbulent-kinetic-energy-in-the-level-2}}

The turbulent kinetic energy of level 2, \({q_2}^2/2\), is calculated
from the following equation, which neglects the time derivative,
advection, and diffusion terms in the time evolution equation of the
turbulent kinetic energy.

\begin{eqnarray} P_s + P_b - \varepsilon = 0 \label{2} \end{eqnarray}

where \(P_s\), \(P_b\), \(\varepsilon\) denote the generation term by
shear, the generation term by buoyancy, and the dissipation term,
respectively. \(P_s\), \(P_b\) are represented as follows.

\begin{eqnarray} P_s = -\langle wu \rangle \frac{\partial U}{\partial z} - \langle wv \rangle \frac{\partial V}{\partial z} \end{eqnarray}

\begin{eqnarray} P_b = \frac{g}{\Theta}\langle w\theta_v \rangle \end{eqnarray}

In the Level 2 of the MYNN scheme, they are written as follows.

\begin{eqnarray} P_s = LqS_{M2} \left[ \left(\frac{\partial U}{\partial z}\right)^2 + \left(\frac{\partial V}{\partial z}\right)^2 \right] \label{3} \end{eqnarray}

\begin{eqnarray} P_b = LqS_{H2} \frac{g}{\Theta}\left[ \beta_\theta \frac{\partial \Theta_l}{\partial z} + \beta_q \frac{\partial Q_w}{\partial z} \right] \label{4} \end{eqnarray}

\begin{eqnarray} \varepsilon = \frac{q^3}{B_1 L} \label{5} \end{eqnarray}

From (2), (3), (4), and (5), \({q_2}^2\) is calculated as follows.

\begin{eqnarray}{q_2}^2=B_1L^2\left\{S_{M2}\left[\left(\frac{\partial U}{\partial z}\right)^2+\left(\frac{\partial V}{\partial z}\right)^2\right]+S_{H2}\frac{g}{\Theta}\left(\beta_\theta \frac{\partial \Theta_l}{\partial z}+\beta_q \frac{\partial Q_w}{\partial z}\right)\right\}\end{eqnarray}

\hypertarget{stability-functions-in-the-level-2.5}{%
\paragraph{Stability functions in the Level
2.5}\label{stability-functions-in-the-level-2.5}}

When \(q<q_2\), i.e., the turbulence is in the growth phase, the
stability functions of level 2.5, \(S_M\) and \(S_H\), are calculated as
follows using the coefficient \(\alpha=q/q_2\) introduced by Helfand and
Labraga (1998).

\begin{eqnarray}S_M=\alpha S_{M2},\quad S_H=\alpha S_{H2}\end{eqnarray}

On the other hand, when \(q \geq q_2\), \(S_M\) and \(S_H\) are
calculated as follows. The following equations differ from those in
Nakanishi (2001) in the description method, but gives equivalent results
with less computation.

\begin{eqnarray}S_M=A_1\frac{E_3-3C_1 E_4}{E_2 E_4+E_5 E_3}\end{eqnarray}

\begin{eqnarray}S_H=A_2\frac{E_2+3C_1 E_5}{E_2 E_4+E_5 E_3}\end{eqnarray}

where

\begin{eqnarray}E_1=1-3A_2B_2(1-C_3)G_H\end{eqnarray}

\begin{eqnarray}E_2=1-9A_1A_2(1-C_2)G_H\end{eqnarray}

\begin{eqnarray}E_3=E_1+9{A_2}^2(1-C_2)(1-C_5)G_H\end{eqnarray}

\begin{eqnarray}E_4=E_1-12A_1A_2(1-C_2)G_H\end{eqnarray}

\begin{eqnarray}E_5=6{A_1}^2G_M\end{eqnarray}

\begin{eqnarray}G_M=\frac{L^2}{q^2}\left[\left(\frac{\partial U}{\partial z}\right)^2+\left(\frac{\partial V}{\partial z}\right)^2\right]\end{eqnarray}

\begin{eqnarray}G_H=-\frac{L^2}{q^2}\frac{g}{\Theta}\left(\beta_\theta \frac{\partial \Theta_l}{\partial z}+\beta_q \frac{\partial Q_w}{\partial z}\right)\end{eqnarray}

\hypertarget{calculation-of-diffusion-coefficients-1}{%
\paragraph{Calculation of diffusion
coefficients}\label{calculation-of-diffusion-coefficients-1}}

The diffusion coefficients \(K_M\), \(K_q\), \(K_H\), and \(K_w\) for
wind speed, turbulent energy, heat, and water are calculated as follows
from \(S_M,S_H\).

\begin{eqnarray}K_M=LqS_M\end{eqnarray}

\begin{eqnarray}K_q=3LqS_M\end{eqnarray}

\begin{eqnarray}K_H=LqS_H\end{eqnarray}

\begin{eqnarray}K_w=LqS_H\end{eqnarray}

\hypertarget{calculation-of-fluxes}{%
\paragraph{Calculation of fluxes}\label{calculation-of-fluxes}}

The vertical flux \(F\) of each physical quantity is calculated as
follows.

\begin{eqnarray}F_{u,k-1/2}=-\rho_{k-1/2}K_{M,k-1/2}\frac{U_{k}-U_{k-1}}{\Delta z_{k-1/2}}\end{eqnarray}

\begin{eqnarray}F_{v,k-1/2}=-\rho_{k-1/2}K_{M,k-1/2}\frac{V_{k}-V_{k-1}}{\Delta z_{k-1/2}}\end{eqnarray}

\begin{eqnarray}F_{q,k-1/2}=-\rho_{k-1/2}K_{q,k-1/2}\frac{{q^2}_ {k}-{q^2}_ {k-1}}{\Delta z_{k-1/2}}\end{eqnarray}

\begin{eqnarray}F_{T,k-1/2}=-\rho_{k-1/2}K_{H,k-1/2}\,C_p\Pi_{k-1/2}\frac{\Theta_{l,k}-\Theta_{l,k-1}}{\Delta z_{k-1/2}}\end{eqnarray}

\begin{eqnarray}F_{w,k-1/2}=-\rho_{k-1/2}K_{w,k-1/2}\frac{Q_{w,k}-Q_{w,k-1}}{\Delta z_{k-1/2}}\end{eqnarray}

where \(\rho\) is density and \(\Pi\) is the Exner function. In order to
perform time integration with implicit scheme, the derivative of each
vertical flux is also obtained as follows.

\begin{eqnarray}\frac{\partial F_{u,k-1/2}}{\partial U_{k-1}}=\frac{\partial F_{v,k-1/2}}{\partial V_{k-1}}=-\frac{\partial F_{u,k-1/2}}{\partial U_{k}}=-\frac{\partial F_{v,k-1/2}}{\partial V_{k}}=\rho_{k-1/2}K_{M,k-1/2}\frac{1}{\Delta z_{k-1/2}}\end{eqnarray}

\begin{eqnarray}\frac{\partial F_{q,k-1/2}}{\partial {q^2}_ {k-1}}=-\frac{\partial F_{q,k-1/2}}{\partial {q^2}_ {k}}=\rho_{k-1/2}K_{q,k-1/2}\frac{1}{\Delta z_{k-1/2}}\end{eqnarray}

\begin{eqnarray}\frac{\partial F_{T,k-1/2}}{\partial T_{k-1}}=\rho_{k-1/2}K_{H,k-1/2}C_p\frac{\Pi_{k-1/2}}{\Pi_{k-1}}\frac{1}{\Delta z_{k-1/2}}\end{eqnarray}

\begin{eqnarray}\frac{\partial F_{T,k-1/2}}{\partial T_{k}}=-\rho_{k-1/2}K_{H,k-1/2}C_p\frac{\Pi_{k-1/2}}{\Pi_{k}}\frac{1}{\Delta z_{k-1/2}}\end{eqnarray}

\begin{eqnarray}\frac{\partial F_{w,k-1/2}}{\partial Q_{w,k-1}}=-\frac{\partial F_{w,k-1/2}}{\partial Q_{w,k}}=\rho_{k-1/2}K_{w,k-1/2}\frac{1}{\Delta z_{k-1/2}}\end{eqnarray}

where \(\Delta z_{k-1/2}=z_k-z_{k-1}\). The fluxes for other tracers are
also calculated in the same way using \(K_w\).

\hypertarget{calculation-of-turbulent-variables}{%
\subsubsection{Calculation of turbulent
variables}\label{calculation-of-turbulent-variables}}

\hypertarget{calculation-of-turbulent-kinetic-energy}{%
\paragraph{Calculation of turbulent kinetic
energy}\label{calculation-of-turbulent-kinetic-energy}}

The prognostic equation for \(q^2\) is expressed as follows.

\begin{eqnarray} \frac{d q^2}{dt}=-\frac{1}{\rho}\frac{\partial F_q}{\partial z}+2\left(P_s+P_b-\varepsilon\right) \end{eqnarray}

In the Level 2.5, \(P_s,P_b,\varepsilon\) are written as follows.

\begin{eqnarray}P_s=Lq S_M \left[\left(\frac{\partial U}{\partial z}\right)^2+\left(\frac{\partial V}{\partial z}\right)^2\right]\end{eqnarray}

\begin{eqnarray}P_b=Lq S_H \frac{g}{\Theta}\left(\beta_\theta \frac{\partial \Theta_l}{\partial z}+\beta_q \frac{\partial Q_w}{\partial z}\right)\end{eqnarray}

\begin{eqnarray}\varepsilon=\frac{q^3}{B_1L}\end{eqnarray}

Advection terms are calculated using tracer transport routines in the
dynamics scheme. In the turbulence scheme, the time evolution by
diffusion, generation and dissipation terms of \(q^2\) is calculated by
the implicit scheme.

\hypertarget{diagnosis-of-variance-and-covariance}{%
\paragraph{Diagnosis of variance and
covariance}\label{diagnosis-of-variance-and-covariance}}

The prognostic equations for
\(\langle {\theta_l}^2 \rangle,\langle {q_w}^2 \rangle,\langle \theta_l q_w \rangle\)
are expressed as follows.

\begin{eqnarray}
\frac{d\left\langle{\theta_l}^{2}\right\rangle}{d t}=-\frac{\partial}{\partial z}\left\langle w \theta_{l}^{2}\right\rangle-2\left\langle w \theta_{l}\right\rangle \frac{\partial \Theta_{l}}{\partial z}-2 \varepsilon_{\theta l}
\end{eqnarray}

\begin{eqnarray}
\frac{d\left\langle {q_w}^{2}\right\rangle}{d t}=-\frac{\partial}{\partial z}\left\langle w q_{w}^{2}\right\rangle-2\left\langle w q_{w}\right\rangle \frac{\partial Q_{w}}{\partial z}-2 \varepsilon_{q w}
\end{eqnarray}

\begin{eqnarray}
\frac{d\left\langle\theta_{l} q_{w}\right\rangle}{d t}=-\frac{\partial}{\partial z}\left\langle w \theta_{l} q_{w}\right\rangle-\left\langle w q_{w}\right\rangle \frac{\partial \Theta_{l}}{\partial z}-\left\langle w \theta_{l}\right\rangle \frac{\partial Q_{w}}{\partial z}-2 \varepsilon_{\theta q}
\end{eqnarray}

In the Level 2.5, the time derivative, advection, and diffusion terms in
these equations are ignored, and the following balances are assumed
locally.

\begin{eqnarray} -\left\langle w \theta_{l}\right\rangle \frac{\partial \Theta_{l}}{\partial z}-\varepsilon_{\theta l} = 0 \label{6}\end{eqnarray}

\begin{eqnarray} -\left\langle w q_{w}\right\rangle \frac{\partial Q_{w}}{\partial z}-\varepsilon_{q w} = 0 \label{7}\end{eqnarray}

\begin{eqnarray} -\left\langle w q_{w}\right\rangle \frac{\partial \Theta_{l}}{\partial z}-\left\langle w \theta_{l}\right\rangle \frac{\partial Q_{w}}{\partial z}-2 \varepsilon_{\theta q} = 0 \label{8}\end{eqnarray}

In the Level 2.5 of MYNN scheme,
\(-\left\langle w \theta_{l}\right\rangle\),
\(-\left\langle w q_{w}\right\rangle\), \(\varepsilon_{\theta l}\),
\(\varepsilon_{q w}\), \(\varepsilon_{\theta q}\) are represented as
follows.

\begin{eqnarray} -\left\langle w \theta_{l}\right\rangle = LqS_H \frac{\partial \Theta_{l}}{\partial z} \label{9}\end{eqnarray}

\begin{eqnarray} -\left\langle w q_{w}\right\rangle = LqS_H \frac{\partial Q_{w}}{\partial z} \label{10}\end{eqnarray}

\begin{eqnarray}
\varepsilon_{\theta l}=\frac{q}{B_{2} L}\left\langle\theta_{l}^{2}\right\rangle \label{11}
\end{eqnarray}

\begin{eqnarray}
\varepsilon_{q w}=\frac{q}{B_{2} L}\left\langle q_{w}^{2}\right\rangle \label{12}
\end{eqnarray}

\begin{eqnarray}
\varepsilon_{\theta q}=\frac{q}{B_{2} L}\left\langle\theta_{l} q_{w}\right\rangle \label{13}
\end{eqnarray}

from (6)-(13), \(\langle {\theta_l}^2 \rangle\),
\(\langle {q_w}^2 \rangle\), \(\langle \theta_l q_w \rangle\) can be
diagnosed as follows.

\begin{eqnarray}\langle {\theta_l}^2 \rangle =B_2L^2S_H\left(\frac{\partial \Theta_l}{\partial z}\right)^2\end{eqnarray}

\begin{eqnarray}\langle {q_w}^2 \rangle =B_2L^2S_H\left(\frac{\partial Q_w}{\partial z}\right)^2\end{eqnarray}

\begin{eqnarray}\langle \theta_l q_w \rangle =B_2L^2S_H\frac{\partial \Theta_l}{\partial z}\frac{\partial Q_w}{\partial z}\end{eqnarray}

\hypertarget{treatment-in-the-bottom-layer}{%
\paragraph{Treatment in the bottom
layer}\label{treatment-in-the-bottom-layer}}

Since the lowest layer of the model corresponds to the ground layer
where the vertical gradient of geophysical quantities change rapidly,
the following Monin-Obukhov similarity theory is used to evaluate the
vertical gradient accurately.

\begin{eqnarray} \frac{\partial M}{\partial z} = \frac{u_*}{kz}\phi_m \label{14}\end{eqnarray}

\begin{eqnarray} \frac{\partial \Theta}{\partial z} = \frac{\theta_*}{kz}\phi_h \label{15}\end{eqnarray}

\begin{eqnarray} \frac{\partial Q_v}{\partial z} = \frac{q_{v*}}{kz}\phi_h \label{16}\end{eqnarray}

where \(M\) is the wind speed when the horizontal axis is in the
direction of the horizontal wind in the surface layer. \(\phi_m\) and
\(\phi_h\) are the dimensionless gradient functions for momentum and
heat, respectively. \(\theta_*\), \(q_{v*}\) are the scales of potential
temperature and water vapor in the surface layer, respectively, and
satisfy the following relationships.

\begin{eqnarray} \langle wm \rangle_g = -u_*^2 \label{17}\end{eqnarray}

\begin{eqnarray} \langle w\theta \rangle_g = -u_*\theta_* \label{18}\end{eqnarray}

\begin{eqnarray} \langle wq_v \rangle_g = -u_*q_{v*} \label{19}\end{eqnarray}

\(m\) is the deviation of \(M\) from the grid average. Using \(M\) and
\(m\), the generation term of turbulence kinetic energy can be written
as

\begin{eqnarray} P_s + P_b = \langle wm \rangle \frac{\partial M}{\partial z} + \frac{g}{\Theta} \langle w\theta_v \rangle \end{eqnarray}

Using (14), (17) and the defining equation of the Monin-Obukhov length,
this can be calculated as follows.

\begin{eqnarray} P_s + P_b = \frac{u_*^3}{kz_1}\left[\phi_m\left(\zeta_1\right)-\zeta_1\right] \end{eqnarray}

Here, \(\zeta_1\) is \(\zeta\) at the full level of the lowest layer of
the model.

By assuming that there are no cloud particles in the surface layer,
\(\langle {\theta_l}^2\rangle\), \(\langle {q_w}^2\rangle\),
\(\langle \theta_lq_w\rangle\) can be calculated diagnostically from
(6)-(8), (11)-(13), (15), (16), (18), and (19) as follows.

\begin{eqnarray}\langle {\theta_l}^2\rangle=\frac{\phi_h\left(\zeta_1\right)}{u_*kz_1}{\langle w\theta \rangle_g}^2 \bigg/ \frac{q}{B_2L} \end{eqnarray}

\begin{eqnarray}\langle {q_w}^2\rangle=\frac{\phi_h\left(\zeta_1\right)}{u_*kz_1}{\langle wq_v\rangle_g}^2 \bigg/ \frac{q}{B_2L} \end{eqnarray}

\begin{eqnarray}\langle \theta_lq_w\rangle=\frac{\phi_h\left(\zeta_1\right)}{u_*kz_1}\langle w\theta \rangle_g\langle wq_v \rangle_g \bigg/ \frac{q}{B_2L} \end{eqnarray}

\(\phi_m,\phi_h\) are formulated as follows based on Businger et
al.~(1971).

\begin{eqnarray}
\phi_m(\zeta)=\left\{
    \begin{array}{lr}
      1+\beta_1\zeta, &\zeta\ge 0\\
      \left(1-\gamma_1\zeta\right)^{-1/4}, &\zeta< 0\\
    \end{array}
  \right.
\end{eqnarray}

\begin{eqnarray}
\phi_h(\zeta)=\left\{
    \begin{array}{lr}
      \beta_2+\beta_1\zeta, &\zeta\ge 0\\
      \beta_2\left(1-\gamma_2\zeta\right)^{-1/2}, &\zeta< 0\\
    \end{array}
  \right.
\end{eqnarray}

\begin{eqnarray}(\beta_1,\beta_2,\gamma_1,\gamma_2)=(4.7,0.74,15.0,9.0)\end{eqnarray}

\hypertarget{time-integration-with-implicit-scheme}{%
\subsubsection{Time integration with implicit
scheme}\label{time-integration-with-implicit-scheme}}
