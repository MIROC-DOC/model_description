\textbf{Acknowledgment}

The original first version of this document was written in Japanese by
Dr.~Numaguti (deceased) in 1995 about the former version of the model,
CCSR/NIES AGCM5.4, which was created by himself. Since then, the AGCM
has been significantly upgraded with continuous community efforts, and
it has been periodically renamed by newer version number (e.g.,
CCSR/NIES AGCM5.6, MIROC3, MIROC4, and MIROC5) with almost the same
phases of releases of IPCC's Assessment Reports (TAR, AR4, and AR5).
Even though there are multiple published papers (e.g., K-1 model
developers (2004) for MIROC3 and Watanabe et al.~(2010) for MIROC5)
explained about the different versions of the model, the description
were made only for key points without details, and there were no major
updates of the original descriptive document by Dr.~Numaguti for a long
time.

Here, with the newest release of MIROC series, MIROC6(Tatebe et al.,
2019), it was decided to make an overhaul of the document for the first
time since the late 1990's. This decision was triggered by some
students' simple question: ``Why is this document so obsolete?'' Then,
under the TOUGOU project, some of the PIs organized the MIROC6 AGCM
document writing team in the spring of 2020.The team consists of two
groups; the authors and the supervisors/editors. All of the authors are
doctor course students of the University of Tokyo, who use or are
interested in using MIROC6, and the supervisors/editors are researchers,
who contributed to development of MIROC6 and/or its previous versions.

In the beginning, the team converted the original Japanese document
written in LaTeX format into English Markdown format. Ms.~Kino Kanon,
Ms.~Haruka Hotta, Mr.~Yuki Takano, and Dr.~Fuyuki Saito contributed to
make a semi-automated tool for this conversion. Then the team was
divided into several groups, and each group became in charge of each
section. These groups and corresponding sections are as follows:

\begin{itemize}
\item
  Mr.~Tomoki Iwakiri, Mr.~Masaki Toda, and Mr.~Kazuya Yamazaki
  supervised by Dr.~Fuyuki Saito for Dynamics (Ch.2) and Coupler Scheme
  (Ch.4.1)
\item
  Mr.~Yuki Takano supervised by Dr.~Minoru Chikira for Cumulus
  Scheme(Ch.3.2)
\item
  Mr.~Takuya Jinna supervised by Dr.~Tomoo Ogura for Shallow Convection
  Scheme (Ch.3.3)
\item
  Ms.~Haruka Hotta supervised by Dr.~Takuro Michibata and Dr.~Kentaro
  Suzuki for Large Scale Condensation (Ch.3.4) and Cloud Microphysics
  (Ch.3.5)
\item
  Mr.~Taro Higuchi supervised by Dr.~Takanori Kodama and Dr.~Miho
  Sekiguchi for Radiation Scheme (Ch.3.6)
\item
  Mr.~Taigo Ando supervised by Dr.~Minoru Chikira for Trubulence Scheme
  (Ch.3.7)
\item
  Ms.~Kanon Kino supervised by Dr.~Tatsuo Suzuki for Surface Flux
  Scheme(Ch.3.8)
\end{itemize}

It took about six months to draft, and the whole draft was reviewed by
all supervisors/editors. At last, the rest of the sections were amended
and improved by all. The team thanks the support of ``Integrated
Research Program for Advancing Climate Models (TOUGOU Program)'' from
the Ministry of Education, Culture, Sports, Science, and Technology
(MEXT), Japan.Finally, the team would sincerely express our respect and
condolences to Dr.~Numaguti, the first developer of the model and the
author of the original version of this document.

Hiroaki Tatebe, Masahiro Watanabe, and Kei Yoshimura lead member of the
teamMarch 31, 2021
