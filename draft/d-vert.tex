Table of contents

\begin{itemize}
\tightlist
\item
  Vertical Discretization

  \begin{itemize}
  \tightlist
  \item
    Model levels (数式・説明をハイブリッド化済)
  \item
    Vertical discretization (数式をハイブリッド化済)
  \item
    Differences from \(\sigma\)-coordinate (新規追加)
  \end{itemize}
\end{itemize}

\hypertarget{vertical-discretization}{%
\subsection{Vertical Discretization}\label{vertical-discretization}}

Following Arakawa and Konor (1996) but in the Lorentz grid, the basic
equations are discretized vertically by differences. This scheme has the
following characteristics.

\begin{itemize}
\item
  Save the total integrated mass
\item
  Save the total integrated energy
\item
  Preserving angular momentum for global integration
\item
  Conservation of total mass-integrated potential temperature
\item
  The hydrostatic pressure equation comes down to local (the altitude of
  the lower level is independent of the temperature of the upper level)
\item
  For a given temperature distribution, constant in the horizontal
  direction, the hydrostatic pressure equation becomes accurate and the
  barometric gradient force becomes zero.
\item
  Isothermal atmosphere stays isothermal forever
\end{itemize}

\hypertarget{model-levels}{%
\subsubsection{Model levels}\label{model-levels}}

下の層から上へと層の番号をつける。\(\zeta,D,T,q\)の物理量は整数レベルで定義されるとし、鉛直速度\(\dot{\eta}\)は半整数レベルにおいて定義する。ただし、\(\frac{1}{2}\)は下端(\(\eta=1\))、\(K+\frac{1}{2}\)は上端(\(\eta=0\))である。さらに、半整数レベルにおける気圧\(p\)を以下の式で定義する。

\begin{eqnarray}
p_{k+1/2} = A_{k+1/2} +B_{k+1/2}\,p_s
\end{eqnarray}

よって、\(\sigma\equiv p/p_s\)は以下のように表せる。

\begin{eqnarray}
\sigma_{k+1/2} = \frac{A_{k+1/2}}{p_s} +B_{k+1/2}
\end{eqnarray}

また、基準地表気圧\(p_0=1000\ \mathrm{hPa}\)を用いて\(\eta\)を次の式で定義する。

\begin{eqnarray}
\eta_{k+1/2} = \frac{A_{k+1/2}}{p_0} +B_{k+1/2}
\end{eqnarray}

整数レベルにおける気圧\(p_k, (k=1,2,\ \ldots\ K)\)は次の式で内挿する。

\begin{eqnarray}
 p_k = \left\{ \frac{1}{1+\kappa}
                     \left( \frac{  p^{\kappa +1}_{k-1/2}
                                  - p^{\kappa +1}_{k+1/2}      }
                                  { p_{k-1/2} - p_{k+1/2} }
                     \right)
              \right\}^{1/\kappa}
\end{eqnarray}

さらに、

\begin{eqnarray}
  \Delta\sigma_k \equiv \sigma_{k-1/2} - \sigma_{k+1/2}
\end{eqnarray} \begin{eqnarray}
  \Delta B_k \equiv B_{k-1/2} - B_{k+1/2}
\end{eqnarray}

を定義しておく。

\hypertarget{vertical-discretization-1}{%
\subsubsection{Vertical
discretization}\label{vertical-discretization-1}}

各方程式のハイブリッド座標における離散化表現は次のようになる。

\begin{enumerate}
\def\labelenumi{\arabic{enumi}.}
\tightlist
\item
  連続の式、鉛直速度
\end{enumerate}

\begin{eqnarray}
  \frac{\partial \pi}{\partial t}
 = - \sum_{k=1}^{K} \left\{ D_k \Delta\sigma_k + ({\mathbf{v}}_k \cdot \nabla \pi)\Delta B_k \right\}
\end{eqnarray}

MIROC6.0では、以前のバージョンで用いられていた\(\sigma\)座標系(MIROC6.0でも選択可能である)とできるだけ類似した表式で離散化を行ったため、鉛直速度は\(\dot{\sigma}=m\dot{\eta}/p_s\)で表現している。また、鉛直移流\(\dot{\eta}(\partial/\partial\eta)\)は\(m\dot{\eta}/p_s(\partial/\partial\sigma)\)と等価であり、プログラム中では後者を使用している。

\begin{eqnarray}
  \left(\dot{\sigma}=\right)\frac{(m\dot{\eta})_{k-1/2}}{p_s}
 = - B_{k-1/2} \frac{\partial \pi}{\partial t}
   - \sum_{l=k}^{K}\left\{ D_l \Delta\sigma_l + ({\mathbf{v}}_l \cdot \nabla \pi)\Delta B_l \right\}
\end{eqnarray}

\begin{eqnarray}
  \frac{(m\dot{\eta})_{1/2}}{p_s} = \frac{(m\dot{\eta})_{k+1/2}}{p_s} = 0
\end{eqnarray}

\begin{enumerate}
\def\labelenumi{\arabic{enumi}.}
\setcounter{enumi}{1}
\tightlist
\item
  静水圧の式
\end{enumerate}

\begin{eqnarray}
 \Phi_{1}  =  \Phi_{s} + C_{p} ( \sigma_{1}^{-\kappa} - 1  ) T_{v,1}
\end{eqnarray} \begin{eqnarray}
           =  \Phi_{s} + C_{p} \alpha_{1} T_{v,1} 
\end{eqnarray}

\begin{eqnarray}
 \Phi_k - \Phi_{k-1} 
   =  C_{p}
   \left[ \left( \frac{ p_{k-1/2} }{ p_k } \right)^{\kappa}
          - 1 \right] T_{v,k} 
       + C_{p}
   \left[ 1- 
         \left( \frac{ p_{k-1/2} }{ p_{k-1} } \right)^{\kappa}
              \right] T_{v,k-1}
\end{eqnarray} \begin{eqnarray}
   =    C_{p} \alpha_k T_{v,k} + C_{p} \beta_{k-1} T_{v,k-1}
\end{eqnarray}

ここで、

\begin{eqnarray}
 \alpha_k \equiv \left( \frac{ p_{k-1/2} }
                               { p_k } \right)^{\kappa} -1
\end{eqnarray} \begin{eqnarray} \beta_k \equiv  1- \left( \frac{ p_{k+1/2} }
                               { p_k } \right)^{\kappa} .
\end{eqnarray}

\begin{enumerate}
\def\labelenumi{\arabic{enumi}.}
\setcounter{enumi}{2}
\tightlist
\item
  運動方程式
\end{enumerate}

\begin{eqnarray}
  \frac{\partial \zeta_k}{\partial t} 
        =   \frac{1}{a\cos\varphi} 
            \frac{\partial (A_v)_k}{\partial \lambda}
          - \frac{1}{a\cos\varphi} 
            \frac{\partial }{\partial \varphi} (A_u \cos\varphi)_k
          - {\mathcal D}(\zeta_k) 
\end{eqnarray}

\begin{eqnarray}
  \frac{\partial D}{\partial t} 
        =   \frac{1}{a\cos\varphi} 
            \frac{\partial (A_u)_k}{\partial \lambda}
          + \frac{1}{a\cos\varphi} 
            \frac{\partial }{\partial \varphi} (A_v \cos\varphi)_k
          - \nabla^{2}_{\sigma}
           ( \Phi_k + R\bar{T} \pi 
             + ({\mathit KE})_k )
          - {\mathcal D}(D_k) 
\end{eqnarray}

\begin{eqnarray}
  (A_u)_k
    =  ( \zeta_k + f ) v_k 
             - \left[ \frac{(m\dot{\eta})_{k-1/2}}{p_s} \frac{u_{k-1} - u_k}{\Delta\sigma_{k-1}+\Delta\sigma_k}
               + \frac{(m\dot{\eta})_{k+1/2}}{p_s} \frac{u_k   - u_{k+1}}{\Delta\sigma_{k}+\Delta\sigma_{k+1}} \right]
\end{eqnarray} \begin{eqnarray}
           - \frac{1}{a\cos\varphi} \frac{\partial \pi}{\partial \lambda}(C_p T_{v,k}\hat{\kappa}-R\bar{T})
             + {\mathcal F}_x
\end{eqnarray}

\begin{eqnarray}
  (A_v)_k
    =  - ( \zeta_k + f ) u_k 
             - \left[ \frac{(m\dot{\eta})_{k-1/2}}{p_s} \frac{v_{k-1} - v_k}{\Delta\sigma_{k-1}+\Delta\sigma_k}
               + \frac{(m\dot{\eta})_{k+1/2}}{p_s} \frac{v_k   - v_{k+1}}{\Delta\sigma_{k}+\Delta\sigma_{k+1}} \right]
\end{eqnarray} \begin{eqnarray}
           - \frac{1}{a} \frac{\partial \pi}{\partial \varphi}(C_p T_{v,k}\hat{\kappa}-R\bar{T})
             + {\mathcal F}_y
\end{eqnarray}

\begin{eqnarray}
   \hat{\kappa}_k 
    = \frac{ B_{k-1/2} \alpha_k + B_{k+1/2} \beta_k }
            { \Delta\sigma_k                                  } 
\end{eqnarray}

\begin{enumerate}
\def\labelenumi{\arabic{enumi}.}
\setcounter{enumi}{3}
\tightlist
\item
  熱力学方程式
\end{enumerate}

\begin{eqnarray}
  \frac{\partial T_k}{\partial t}
     =  - \frac{1}{a\cos\varphi}
               \frac{\partial u_k T'_k}{\partial \lambda}
          - \frac{1}{a\cos\varphi}
               \frac{\partial }{\partial \varphi} (v_k T'_k \cos\varphi)
          + H_k
\end{eqnarray} \begin{eqnarray}
        + \frac{Q_k}{C_{p}}
          + \frac{(Q_{diff})_k}{C_p} 
          - {\mathcal D}(T_k)
\end{eqnarray}

ここで、

\begin{eqnarray}
   H_k 
     \equiv  T_k' D_k
              - \left[   \frac{(m\dot{\eta})_{k-1/2}}{p_s} \frac{\hat{T}_{k-1/2} - T_k}{\Delta\sigma_k}
               + \frac{(m\dot{\eta})_{k+1/2}}{p_s} \frac{T_k - \hat{T}_{k+1/2}}{\Delta\sigma_k} \right]
\end{eqnarray} \begin{eqnarray}
        + \left\{ \alpha_k
                    \left[ B_{k-1/2} {\mathbf{v}}_k \cdot \nabla \pi
                          - \sum_{l=k}^{K} 
                           (D_l \Delta \sigma_l + ({\mathbf{v}}_l \cdot \nabla \pi)\Delta B_l)
                    \right]
             \right.
\end{eqnarray} \begin{eqnarray}
          + \left. \beta_k
                     \left[ B_{k+1/2} {\mathbf{v}}_k \cdot \nabla \pi
                          - \sum_{l=k+1}^{K} 
                           (D_l \Delta \sigma_l + ({\mathbf{v}}_l \cdot \nabla \pi)\Delta B_l)
                    \right]
              \right\} 
              \frac{1}{\Delta \sigma_k} T_{v,k}
\end{eqnarray} \begin{eqnarray}
     =  T_k' D_k 
          - \left[ \frac{(m\dot{\eta})_{k-1/2}}{p_s} \frac{\hat{T}_{k-1/2} - T_k}{\Delta \sigma_l}
               + \frac{(m\dot{\eta})_{k+1/2}}{p_s} \frac{T_k - \hat{T}_{k+1/2}}{\Delta \sigma_l} \right]
\end{eqnarray} \begin{eqnarray}
        + \hat{\kappa}_k {\mathbf{v}}_k \cdot \nabla \pi T_{v,k} 
\end{eqnarray} \begin{eqnarray}
        - \alpha_k \sum_{l=k}^{K} 
                           (D_l \Delta \sigma_l + ({\mathbf{v}}_l \cdot \nabla \pi)\Delta B_l)
                            \frac{T_{v,k}}{\Delta \sigma_k} 
\end{eqnarray} \begin{eqnarray}
        - \beta_k \sum_{l=k+1}^{K} 
                           (D_l \Delta \sigma_l + ({\mathbf{v}}_l \cdot \nabla \pi)\Delta B_l)
                            \frac{T_{v,k}}{\Delta \sigma_k}
\end{eqnarray}

\begin{eqnarray}
  \hat{T}_{k-1/2}
   = a_k T_k + b_{k-1} T_{k-1}
\end{eqnarray}

\begin{eqnarray}
  a_k  =  \alpha_k
              \left[ 1- \left( \frac{ p_k }{ p_{k-1} }
                        \right)^{\kappa} \right]^{-1}
\end{eqnarray} \begin{eqnarray}
  b_k  =  \beta_k 
              \left[ \left( \frac{ p_k }{ p_{k+1} } 
                     \right)^{\kappa} - 1 \right]^{-1} .  
\end{eqnarray}

\begin{enumerate}
\def\labelenumi{\arabic{enumi}.}
\setcounter{enumi}{4}
\tightlist
\item
  水蒸気の時間発展方程式
\end{enumerate}

\begin{eqnarray}
  \frac{\partial q_k}{\partial t}
      =   - \frac{1}{a\cos\varphi} 
               \frac{\partial u_k q_k}{\partial \lambda}
          - \frac{1}{a\cos\varphi}
               \frac{\partial }{\partial \varphi} ( v_k q_k\cos\varphi)
          + R_k 
          + S_{q,k}
          - {\mathcal D}(q_k) 
\end{eqnarray}

\begin{eqnarray}
R_k  =  q_k D_k 
       - \frac{1}{2} 
             \left[   \frac{(m\dot{\eta})_{k-1/2}}{p_s} \frac{q_{k-1} - q_k}{\Delta\sigma_k}
               + \frac{(m\dot{\eta})_{k+1/2}}{p_s} \frac{q_k   - q_{k+1}}{\Delta\sigma_k} \right]
\end{eqnarray}

\hypertarget{differences-from-sigma-coordinate}{%
\subsubsection{\texorpdfstring{Differences from
\(\sigma\)-coordinate}{Differences from \textbackslash sigma-coordinate}}\label{differences-from-sigma-coordinate}}

ここではハイブリッド座標を、できるだけ\(\sigma\)座標に近い表式で離散化したため、両者の差は比較的少ない。両者の重要な対応関係を以下に挙げる。

\begin{itemize}
\tightlist
\item
  全層で\(A_{k+1/2}=0\)とすれば、ハイブリッド座標は\(\sigma\)座標に帰着する。
\item
  ハイブリッド座標における\(\Delta B_k\)と\(\Delta \sigma\)は、\(\sigma\)座標系においては両者とも\(\Delta \sigma_k\)に対応する。
\end{itemize}
