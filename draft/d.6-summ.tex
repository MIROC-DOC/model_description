\hypertarget{summary-of-the-dynamical-core}{%
\subsection{Summary of the dynamical
core}\label{summary-of-the-dynamical-core}}

In this section, we enumerate the calculations performed in the
dynamical core, although they overlap with the previous descriptions.

\hypertarget{conversion-of-horizontal-wind-to-vorticity-and-divergence}{%
\subsubsection{Conversion of Horizontal Wind to Vorticity and
Divergence}\label{conversion-of-horizontal-wind-to-vorticity-and-divergence}}

Obtain grid point values of vorticity and divergence from the grid point
values of \(u_{ij}, v_{ij}\) for horizontal wind. First, we obtain the
vorticity and divergence in spectral space, \(\zeta_n^m, D_n^m\),

\begin{eqnarray}
\zeta_n^m  =  \frac{1}{I} \sum_{i=1}^{I} \sum_{j=1}^{J}  
                  \mathrm{i}m v_{ij} \cos\varphi_j {Y_n^{m*}}_{ij}
                \frac{w_j}{a(1-\mu_j^{2})}
           +    \frac{1}{I} \sum_{i=1}^{I} \sum_{j=1}^{J}  
                     u_{ij} \cos\varphi_j (1-\mu_j^2)
                  \frac{\partial }{\partial \mu} {Y_n^{m*}}_{ij}
                 \frac{w_j}{a(1-\mu_j^{2})} \; ,
\end{eqnarray}

\begin{eqnarray}
    D_n^m  =  \frac{1}{I} \sum_{i=1}^{I} \sum_{j=1}^{J}  
                  \mathrm{i}m u_{ij} \cos\varphi_j {Y_n^{m*}}_{ij}
                \frac{w_j}{a(1-\mu_j^{2})}
           -    \frac{1}{I} \sum_{i=1}^{I} \sum_{j=1}^{J}  
                  v_{ij} \cos\varphi_j  (1-\mu_j^2)
                  \frac{\partial }{\partial \mu} {Y_n^{m*}}_{ij}
                 \frac{w_j}{a(1-\mu_j^{2})} ; .
\end{eqnarray}

The grid point value is calculated by

\begin{eqnarray}
  \zeta_{ij}
   =  {\mathcal R}{\mathbf{e}} \sum_{m=-N}^{N} \sum_{n=|m|}^{N}
      \zeta_n^m  {Y_n^m}_{ij} \; ,
\end{eqnarray}

and so on.

Corresponding file \& subroutines:
\texttt{{[}G2Wpush,\ G2Wtrans,\ G2Wshift,\ W2Gpush,\ W2Gtrans,\ W2Gshift\ (xdsphe.F){]}}

\hypertarget{calculating-a-virtual-temperature}{%
\subsubsection{Calculating a virtual
temperature}\label{calculating-a-virtual-temperature}}

virtual Temperature \(T_v\) is ,

\begin{eqnarray}
  T_v = T ( 1 + \epsilon_v q - l ) \; ,
\end{eqnarray}

However, it is \(\epsilon_v = R_v/R - 1\) and \(R_v\) is the gas
constant for water vapor (461 Jkg\(^{-1}\)K\(^{-1}\)) and \(R\) is the
gas constant for air (287.04 Jkg\(^{-1}\)K\(^{-1}\)).

Corresponding file \& subroutines: \texttt{{[}VIRTMD\ (dvtmp.F){]}}
\#\#\# Calculating the pressure gradient term

The pressure gradient term \(\nabla \pi = \frac{1}{p_S} \nabla p_S\) is
first used to define the \(\pi_n^m\)

\begin{eqnarray}
  \pi_n^m  =  \frac{1}{I} \sum_{i=1}^{I} \sum_{j=1}^{J}  
               (\ln {p_S})_{ij} {Y_n^{m *}}_{ij}  w_j \; ,
\end{eqnarray}

to a spectral representation and then ,

\begin{eqnarray}
   \frac{1}{a \cos \varphi}
   \left( \frac{\partial \pi}{\partial \lambda} \right)_{ij}
     =
   \frac{1}{a \cos \varphi}
        {\mathcal R}{\mathbf{e}} \sum_{m=-N}^{N} \sum_{n=|m|}^{N}
       \mathrm{i}m \tilde{X}_n^m {Y_n^m}_{ij}  \; ,
\end{eqnarray}

\begin{eqnarray}
   \frac{1}{a}
   \left( \frac{\partial \pi}{\partial \varphi} \right)_{ij}
     =  
   \frac{1}{a \cos \varphi}
       {\mathcal R}{\mathbf{e}} \sum_{m=-N}^{N} \sum_{n=|m|}^{N}
       \pi_n^m
       ( 1-\mu^{2} ) \frac{\partial }{\partial \mu} {Y_n^m}_{ij}  \; .
\end{eqnarray}

Corresponding file \& subroutine: \texttt{{[}PSDOT\ (dgdyn.F){]}}

\hypertarget{diagnosis-of-vertical-flow}{%
\subsubsection{Diagnosis of vertical
flow}\label{diagnosis-of-vertical-flow}}

Pressure change term, and lead DC,

\begin{eqnarray}
  \frac{\partial \pi}{\partial t}
   = - \sum_{k=1}^{K} \left\{ D_k \Delta\sigma_k + ({\mathbf{v}}_k \cdot \nabla \pi)\Delta B_k \right\}
\end{eqnarray}

\begin{eqnarray}
  \frac{(m\dot{\eta})_{k-1/2}}{p_s}
   = - B_{k-1/2} \frac{\partial \pi}{\partial t}
    - \sum_{l=k}^{K}\left\{ D_l \Delta\sigma_l + ({\mathbf{v}}_l \cdot \nabla \pi)\Delta B_l \right\}
\end{eqnarray}

and its non-gravity components.

\begin{eqnarray}
  \left( \frac{\partial \pi}{\partial t} \right)^{NG}
   =   - \sum_{k=1}^{K} {\mathbf{v}}_{k} \cdot \nabla \pi  
       \Delta B_{k}
\end{eqnarray}

\begin{eqnarray}
  \frac{(m\dot{\eta})^{NG}_{k-1/2}}{p_s}
   = - B_{k-1/2} \left( \frac{\partial \pi}{\partial t} \right)^{NG}
    - \sum_{l=k}^{K} {\mathbf{v}}_{l} \cdot \nabla \pi
       \Delta B_{l}
\end{eqnarray}

Corresponding file and subroutine: \texttt{{[}PSDOT\ (dgdyn.F){]}}

\hypertarget{tendency-terms-due-to-advection}{%
\subsubsection{Tendency terms due to
advection}\label{tendency-terms-due-to-advection}}

Momentum advection term:

\begin{eqnarray}
  (A_u)_k
    =  ( \zeta_k + f ) v_k
             - \left[ \frac{(m\dot{\eta})_{k-1/2}}{p_s} \frac{u_{k-1} - u_k}{\Delta\sigma_{k-1}+\Delta\sigma_k}
               + \frac{(m\dot{\eta})_{k+1/2}}{p_s} \frac{u_k   - u_{k+1}}{\Delta\sigma_{k}+\Delta\sigma_{k+1}} \right]
\end{eqnarray} \begin{eqnarray}
           - \frac{1}{a\cos\varphi} \frac{\partial \pi}{\partial \lambda}(C_p T_{v,k}\hat{\kappa}-R\bar{T})
             + {\mathcal F}_x
\end{eqnarray}

\begin{eqnarray}
  (A_v)_k
    =  - ( \zeta_k + f ) u_k
             - \left[ \frac{(m\dot{\eta})_{k-1/2}}{p_s} \frac{v_{k-1} - v_k}{\Delta\sigma_{k-1}+\Delta\sigma_k}
               + \frac{(m\dot{\eta})_{k+1/2}}{p_s} \frac{v_k   - v_{k+1}}{\Delta\sigma_{k}+\Delta\sigma_{k+1}} \right]
\end{eqnarray} \begin{eqnarray}
           - \frac{1}{a} \frac{\partial \pi}{\partial \varphi}(C_p T_{v,k}\hat{\kappa}-R\bar{T})
             + {\mathcal F}_y
\end{eqnarray}

Temperature advection term:

\begin{eqnarray}
 (u T')_k  = u_k (T_k - \bar{T} )
\end{eqnarray}

\begin{eqnarray}
 (v T')_k  = v_k (T_k - \bar{T} )
\end{eqnarray}

\begin{eqnarray}
   H_k =  T_k' D_k
          - \left[ \frac{(m\dot{\eta})_{k-1/2}}{p_s} \frac{\hat{T}_{k-1/2} - T_k}{\Delta \sigma_l}
               + \frac{(m\dot{\eta})_{k+1/2}}{p_s} \frac{T_k - \hat{T}_{k+1/2}}{\Delta \sigma_l} \right]
\end{eqnarray} \begin{eqnarray}
        + \hat{\kappa}_k {\mathbf{v}}_k \cdot \nabla \pi T_{v,k}
\end{eqnarray} \begin{eqnarray}
        - \alpha_k \sum_{l=k}^{K}
                           (D_l \Delta \sigma_l + ({\mathbf{v}}_l \cdot \nabla \pi)\Delta B_l)
                            \frac{T_{v,k}}{\Delta \sigma_k}
\end{eqnarray} \begin{eqnarray}
        - \beta_k \sum_{l=k+1}^{K}
                           (D_l \Delta \sigma_l + ({\mathbf{v}}_l \cdot \nabla \pi)\Delta B_l)
                            \frac{T_{v,k}}{\Delta \sigma_k}
\end{eqnarray}

Water vapor advection term:

\begin{eqnarray}
 (u q)_k  = u_k q_k
\end{eqnarray}

\begin{eqnarray}
 (v q)_k  = v_k q_k
\end{eqnarray}

\begin{eqnarray}
R_k  =  q_k D_k
       - \frac{1}{2}
             \left[   \frac{(m\dot{\eta})_{k-1/2}}{p_s} \frac{q_{k-1} - q_k}{\Delta\sigma_k}
               + \frac{(m\dot{\eta})_{k+1/2}}{p_s} \frac{q_k   - q_{k+1}}{\Delta\sigma_k} \right]
\end{eqnarray}

Corresponding file \& subroutine
\texttt{{[}GRTADV,\ GRUADV\ (dgdyn.F){]}}

\hypertarget{transformation-of-prognostic-variables-to-spectral-space}{%
\subsubsection{Transformation of prognostic variables to spectral
space}\label{transformation-of-prognostic-variables-to-spectral-space}}

\begin{enumerate}
\def\labelenumi{(\arabic{enumi})}
\setcounter{enumi}{121}
\tightlist
\item
  and (123).
\end{enumerate}

Transform \(u_{ij}^{t-\Delta t}, v_{ij}^{t-\Delta t}\) to a spectral
representation of vorticity and divergence \(\zeta_n^m, D_n^m\).
Furthermore, transforming the temperature \(T^{t-\Delta t}\), specific
humidity \(q^{t-\Delta t}\), and \(\pi = \ln p_S^{t-\Delta t}\) to

\begin{eqnarray}
  X_n^m  =  \frac{1}{I} \sum_{i=1}^{I} \sum_{j=1}^{J}  
               X_{ij} {Y_n^{m *}}_{ij}  w_j \; ,
\end{eqnarray}

to a spectral representation.

Corresponding file \& subroutine:
\texttt{{[}G2Wpush,\ G2Wtrans,\ G2Wshift\ (xdsphe.F){]}}

\hypertarget{transformation-of-tendency-terms-to-spectral-space}{%
\subsubsection{Transformation of tendency terms to spectral
space}\label{transformation-of-tendency-terms-to-spectral-space}}

Tendency Term of Vorticity

\begin{eqnarray}
  \frac{\partial{\zeta_n^m}}{\partial {t}}
    =  \frac{1}{I} \sum_{i=1}^{I} \sum_{j=1}^{J}  
    \mathrm{i}m (A_v)_{ij} \cos \varphi_j
    {Y_n^{m *}}_{ij}
    \frac{w_j}{a(1-\mu_j^{2})}
\\
  +\frac{1}{I} \sum_{i=1}^{I} \sum_{j=1}^{J}  
    (A_u)_{ij} \cos \varphi_j
    (1-\mu_j^2)
    \frac{\partial }{\partial \mu} {Y_n^{m *}}_{ij}
    \frac{w_j}{a(1-\mu_j^{2})}
\end{eqnarray}

The non-gravity wave component of the tendency term of the divergence

\begin{eqnarray}
  \left( \frac{\partial{D_n^m}}{\partial {t}} \right)^{NG}
   =  \frac{1}{I} \sum_{i=1}^{I} \sum_{j=1}^{J}  
          \mathrm{i}m (A_u)_{ij} \cos \varphi_j
          {Y_n^{m *}}_{ij}
         \frac{w_j}{a(1-\mu_j^{2})}
          \\
   -\frac{1}{I} \sum_{i=1}^{I} \sum_{j=1}^{J}  
          (A_v)_{ij} \cos \varphi_j
          (1-\mu_j^2)
          \frac{\partial }{\partial \mu} {Y_n^{m *}}_{ij}
          \frac{w_j}{a(1-\mu_j^{2})}
          \\
   -\frac{n(n+1)}{a^{2}}
         \frac{1}{I} \sum_{i=1}^{I} \sum_{j=1}^{J}  
          \hat{E}_{ij}  {Y_n^{m *}}_{ij} w_j
          \\
\end{eqnarray}

The non-gravity wave component of the tendency term of temperature

\begin{eqnarray}
  \left( \frac{\partial{T_n^m}}{\partial {t}} \right)^{NG}
   =  - \frac{1}{I} \sum_{i=1}^{I} \sum_{j=1}^{J}  
          \mathrm{i}m (u T')_{ij} \cos \varphi_j
          {Y_n^{m *}}_{ij}
         \frac{w_j}{a(1-\mu_j^{2})}
          \\
     + \frac{1}{I} \sum_{i=1}^{I} \sum_{j=1}^{J}  
          (v T')_{ij} \cos \varphi_j
          (1-\mu_j^2)
          \frac{\partial }{\partial \mu} {Y_n^{m *}}_{ij}
          \frac{w_j}{a(1-\mu_j^{2})}
          \\
     + \frac{1}{I} \sum_{i=1}^{I} \sum_{j=1}^{J}  
          \hat{H}_{ij}
          {Y_n^{m *}}_{ij} w_j
\end{eqnarray}

Tendency term of water vapor

\begin{eqnarray}
  \frac{\partial{q_n^m}}{\partial {t}}
   =  - \frac{1}{I} \sum_{i=1}^{I} \sum_{j=1}^{J}  
          \mathrm{i}m (uq)_{ij} \cos \varphi_j
          {Y_n^{m *}}_{ij}
         \frac{w_j}{a(1-\mu_j^{2})}
          \\
     + \frac{1}{I} \sum_{i=1}^{I} \sum_{j=1}^{J}  
          (vq)_{ij} \cos \varphi_j
          (1-\mu_j^2)
          \frac{\partial }{\partial \mu} {Y_n^{m *}}_{ij}
          \frac{w_j}{a(1-\mu_j^{2})}
          \\
     + \frac{1}{I} \sum_{i=1}^{I} \sum_{j=1}^{J}  
          R_{ij}
          {Y_n^{m *}}_{ij} w_j
\end{eqnarray}

Corresponding file \& subroutines:
\texttt{{[}G2Wpush,\ G2Wtrans,\ G2Wshift\ (xdsphe.F){]}}

\hypertarget{time-integration-in-spectral-space}{%
\subsubsection{Time integration in spectral
space}\label{time-integration-in-spectral-space}}

Equations in matrix form

\begin{eqnarray}
      \left\{ ( 1+2\Delta t {\mathcal D}_H )( 1+2\Delta t {\mathcal D}_M )
           \underline{I}  
      - ( \Delta t )^{2}  ( \underline{W} \ \underline{h}
           + (1+2\Delta t {\mathcal D}_M)
             {\mathbf{G}} {\mathbf{C}}^{T} ) \nabla^{2}_{\sigma}
  \right\}
      \overline{ {\mathbf{D}} }^{t}
       \\
  = ( 1+2\Delta t {\mathcal D}_H )( 1-\Delta t {\mathcal D}_M )
       {\mathbf{D}}^{t-\Delta t}
  +\Delta t
         \left( \frac{\partial {\mathbf{D}}}{\partial t} \right)_{NG}  
  \\
  -\Delta t \nabla^{2}_{\sigma}     
                   \left\{  ( 1+2\Delta t {\mathcal D}_H ) {\mathbf{\Phi}}_{S}
                          + \underline{W}
                            \left[ ( 1-2\Delta t {\mathcal D}_H )
                                    {\mathbf{T}}^{t-\Delta t}
                                  + \Delta t
                                      \left( \frac{\partial {\mathbf{T}}}
                                                  {\partial t}     
                                      \right)_{NG} \right]
                   \right.
  \\
                 \left.  \hspace*{20mm}
                          + ( 1+2\Delta t {\mathcal D}_H ) {\mathbf{G}}
                            \left[ \pi^{t-\Delta t}
                                  + \Delta t
                                     \left( \frac{\partial \pi}
                                                 {\partial t}
                                     \right)_{NG}  \right]
                   \right\} .
\end{eqnarray}

Using LU decomposition, \(\bar{D}\) is obtained by solving for

\begin{eqnarray}
  \frac{\partial {\mathbf{T}}}{\partial t}
      =   \left( \frac{\partial {\mathbf{T}}}
                        {\partial t}       \right)_{NG}  
         - \underline{h} {\mathbf{D}}
\end{eqnarray}

\begin{eqnarray}
  \frac{\partial \pi}{\partial t}
      =   \left( \frac{\partial \pi}
                        {\partial t}       \right)_{NG}  
         - {\mathbf{C}} \cdot {\mathbf{D}}
\end{eqnarray}

Calculate the value of the spectrum in
\(\partial {\mathbf{T}}/\partial t\), \(\partial \pi/\partial t\) and
then calculate the value of the spectrum in \(t+\Delta t\) using

\begin{eqnarray}
  \zeta^{t+\Delta t}  =  \left( \zeta^{t-\Delta t}
                                +   2 \Delta t \frac{\partial{\zeta}}{\partial {t}} \right)
                          ( 1 + 2 \Delta t {\mathcal D}_M )^{-1} \\
  D^{t+\Delta t}  =  2 \bar{D} - D^{t-\Delta t}\\
  T^{t+\Delta t}  =  \left( T^{t-\Delta t}
                                +  2 \Delta t  \frac{\partial{T}}{\partial {t}} \right)
                          ( 1 + 2 \Delta t {\mathcal D}_H )^{-1} \\
  q^{t+\Delta t}  =  \left( q^{t-\Delta t}
                                +  2 \Delta t \frac{\partial{q}}{\partial {t}} \right)
                          ( 1 + 2 \Delta t {\mathcal D}_E )^{-1} \\
\pi^{t+\Delta t}  =  \pi^{t-\Delta t}
                                +  2 \Delta t \frac{\partial{\pi}}{\partial {t}}
\end{eqnarray}

Corresponding file \& subroutine: \texttt{{[}TINTGR\ (dintg.F){]}}

\hypertarget{transformation-of-prognostic-variables-to-grid-point-values}{%
\subsubsection{Transformation of prognostic variables to grid point
Values}\label{transformation-of-prognostic-variables-to-grid-point-values}}

Obtain grid values of horizontal wind speed from the spectral values of
vorticity and divergence (\(\zeta_n^m, D_n^m\)) \(u_{ij}, v_{ij}\).

\begin{eqnarray}
  u_{ij}
  =  \frac{1}{\cos \varphi_j}
     {\mathcal R}{\mathbf{e}} \sum_{m=-N}^{N}
                       \sum_{\stackrel{n=|m|}{n \neq 0}}^{N}
    \left\{
             \frac{a}{n(n+1)} \zeta_n^m
            (1-\mu^{2}) \frac{\partial{}}{\partial {\mu}} {Y_n^m}_{ij}
          -  \frac{\mathrm{i}m a}{n(n+1)} D_n^m {Y_n^m}_{ij}
    \right\}
\end{eqnarray}

\begin{eqnarray}
  v_{ij}
  =  \frac{1}{\cos \varphi_j}
     {\mathcal R}{\mathbf{e}} \sum_{m=-N}^{N}
                       \sum_{\stackrel{n=|m|}{n \neq 0}}^{N}
    \left\{
          -  \frac{\mathrm{i}m a}{n(n+1)} \zeta_n^m  {Y_n^m}_{ij}
          -  \frac{a}{n(n+1)} \tilde{D}_n^m
            (1-\mu^{2}) \frac{\partial{}}{\partial {\mu}} {Y_n^m}_{ij}
    \right\}
\end{eqnarray}

Furthermore,

\begin{eqnarray}
  T_{ij}
   =  {\mathcal R}{\mathbf{e}} \sum_{m=-N}^{N} \sum_{n=|m|}^{N}
      T_n^m  {Y_n^m}_{ij} \; ,
\end{eqnarray}

\(T_{ij}, \pi_{ij}, q_{ij}\), and so on,

\begin{eqnarray}
  {p_S}_{ij} = \exp \pi_{ij}
\end{eqnarray}

to calculate.

Corresponding file \& subroutines:
\texttt{{[}W2Gpush,\ W2Gtrans,\ W2Gshift\ (xdsphe.F){]}}

\hypertarget{diffusion-correction-along-pressure-level}{%
\subsubsection{Diffusion Correction along pressure
level}\label{diffusion-correction-along-pressure-level}}

The horizontal diffusion is applied on the surface of \(\eta-\)plane,
but it can cause problems in large slopes, such as transporting water
vapor uphill and causing false precipitation at the top of a mountain.
To mitigate this problem, corrections have been made for \(T,q,l\) to
make the diffusion closer to that of the \(p\) surface, e.g., for
\(T,q,l\).

\begin{eqnarray}
  {\mathcal D}_p (T) = (-1)^{N_D/2} K \nabla^{N_D}_p T  
                \simeq  (-1)^{N_D/2} K \nabla^{N_D}_{\eta} T  
                      - \frac{\partial{\sigma}}{\partial {p}}
                      (-1)^{N_D/2} K \nabla^{N_D}_{\eta} p
                      \cdot \frac{\partial{T}}{\partial {\sigma}}
\end{eqnarray} \begin{eqnarray}
                =      (-1)^{N_D/2} K \nabla^{N_D}_{\eta} T  
                    -  (-1)^{N_D/2} K \nabla^{N_D}_{\eta} \pi
                          \cdot \sigma \frac{\partial{T}}{\partial {\sigma}}
\end{eqnarray} \begin{eqnarray}
                =    {\mathcal D} (T)
                    -  {\mathcal D} (\pi)
                       \sigma \frac{\partial{T}}{\partial {\sigma}}
\end{eqnarray}

So,

\begin{eqnarray}
  T_k \leftarrow  T_k
       -  2 \Delta t
        \sigma_{k} \frac{T_{k+1}-T_{k-1}}{\sigma_{k+1} - \sigma_{k-1}}
        {\mathcal D}(\pi)
\end{eqnarray}

and so on. In \({\mathcal D}(\pi)\), the spectral value of \(\pi\) is
converted to a grid by multiplying the spectral value of \(\pi_n^m\) by
the spectral representation of the diffusion coefficient.

Corresponding file \& subroutine: \texttt{{[}CORDIF\ (ddifc.F){]}}

\hypertarget{frictional-heat-associated-with-diffusion.}{%
\subsubsection{Frictional heat associated with
diffusion.}\label{frictional-heat-associated-with-diffusion.}}

Frictional heat from diffusion is ,

\begin{eqnarray}
  Q_{DIF} = - \left( u_{ij} {\mathcal D}(u)_{ij}
                   + v_{ij} {\mathcal D}(v)_{ij} \right)
\end{eqnarray}

It is estimated that Therefore,

\begin{eqnarray}
  T_k \leftarrow  T_k
       -  \frac{2 \Delta t}{C_p}
           \left( u_{ij} {\mathcal D}(u)_{ij}
                 + v_{ij} {\mathcal D}(v)_{ij} \right)
\end{eqnarray}

Corresponding file \& subroutine: \texttt{{[}CORDIF\ (ddifc.F){]}}

\hypertarget{horizontal-diffusion-and-rayleigh-friction}{%
\subsubsection{Horizontal Diffusion and Rayleigh
Friction}\label{horizontal-diffusion-and-rayleigh-friction}}

The coefficients of horizontal diffusion can be expressed spectrally,

\begin{eqnarray}
 {{\mathcal D}_M}_n^m = K_M
                      \left[ \left( \frac{n(n+1)}{a^2} \right)^{N_D/2}
                                - \left( \frac{2}{a^2} \right)^{N_D/2}
                      \right]
                  + K_R
\end{eqnarray}

\begin{eqnarray}
  {{\mathcal D}_H}_n^m = K_M \left( \frac{n(n+1)}{a^2} \right)^{N_D/2}
\end{eqnarray}

\begin{eqnarray}
  {{\mathcal D}_E}_n^m = K_E \left( \frac{n(n+1)}{a^2} \right)^{N_D/2}
\end{eqnarray}

\(K_R\) is the Rayleigh coefficient of friction. The Rayleigh
coefficient of friction is

\begin{eqnarray}
  K_R = K_R^0 \left[ 1+\tanh \left( \frac{z-z_R}{H_R} \right) \right]
\end{eqnarray}

However, the profile is given in the same way as However,

\begin{eqnarray}
  z = - H \ln \sigma
\end{eqnarray}

The results are approximate to those of \(K_R^0 = {(30day)}^{-1}\) and
\(z_R = -H \ln \sigma_{top}\). The standard values are
\(K_R^0 = {(30day)}^{-1}\), \(z_R = -H \ln \sigma_{top}\)
(\(\sigma_{top}\): top level of the model), \(H = 8000\) m, and
\(H_R = 7000\) m.

Corresponding file \& subroutine \texttt{{[}DSETDF\ (dsetd.F){]}}

\hypertarget{time-filter}{%
\subsubsection{Time Filter}\label{time-filter}}

To reduce numerical mode associated with leap frog scheme, time filter
is applied every time step. MIORC6 used modified Asselin time filter
(Williams, 2009), which is updated version of Asselin(1972) used
previous version of MIROC. Although Asselin time filter attenuate high
frequency physical mode, bringing low accuracy of leap frog scheme,
current time filter succeeded in suppressing it.

Modified Asselin filter is expressed as following equation

\begin{eqnarray}
 \bar{\bar{X}}^t = \bar{X}^t + \nu\alpha[\bar{\bar{X}}^{t-\Delta t} -2 \bar{X}^t + X^{t+\Delta t}]
\end{eqnarray}

\begin{eqnarray}
 \bar{X}^{t+\Delta t} = X^{t+\Delta t} + \nu(1-\alpha)[\bar{\bar{X}}^{t-\Delta t} -2 \bar{X}^t + X^{t+\Delta t}]
\end{eqnarray}

where bar indicates time filter. The parameters set to \(\nu=0.05\),
\(\alpha=0.5\). Assuming \(\alpha=1\), modified Asselin filter is same
as Asselin filter.

In the model, \begin{eqnarray}
 \bar{\bar{X}}^{t*} = (1-\nu\alpha)^{-1}[(1-2\nu\alpha)\bar{X}^t +\nu\alpha \bar{\bar{X}}^{t-\Delta t} ]
\end{eqnarray} is firstly calculated at \texttt{MODULE:\ {[}DADVNC{]}} where
transformation of prognostic variableto grid point values. And then,
\(X^{t-\Delta t}-2X^t\) is stored. When the \(X^{t+\Delta t}\) is
obtained later, time filter conduct at \texttt{MODULE\ {[}TFILT{]}},

\begin{eqnarray}
 \bar{\bar{X}}^{t} = (1-\nu\alpha)\bar{\bar{X}}^{t*} +\nu\alpha X^{t+\Delta t}
\end{eqnarray} \begin{eqnarray}
\bar{X}^{t+\Delta t} = X^{t+\Delta t} + \nu (1-\alpha)[ \bar{\bar{X}}^{t-\Delta t} - 2\bar{X}^{t} + X^{t+\Delta t}]
 \end{eqnarray}

Corresponding file \& subroutine: \texttt{{[}DADVNC\ (dadvn.F){]}}

\hypertarget{correction-for-conservation-of-mass}{%
\subsubsection{Correction for conservation of
mass}\label{correction-for-conservation-of-mass}}

In the spectral method, the global integral of \(\pi = \ln p_S\) is
preserved with rounding errors removed, but the preservation of the
mass, i.e.~the global integral of \(p_S\) is not guaranteed. Moreover, a
wavenumber break in the spectra sometimes results in negative values of
the water vapor grid points. For this reason, we perform a correction to
preserve the masses of dry air, water vapor, and cloud water, and to
remove the regions with negative water vapor content.

Before entering dynamical calculations, \texttt{{[}FIXMAS{]}}, the
global integrals of water vapor and cloud water are calculated for
\(M_q, M_l\).

\begin{eqnarray}
  M_q^0  =  \sum_{ijk} q p_S  \Delta\lambda_i w_j \Delta\sigma_k  \\
  M_l^0  =  \sum_{ijk} l p_S  \Delta\lambda_i w_j \Delta\sigma_k
\end{eqnarray}

In the first step of the calculation, the dry mass \(M_d\) is calculated
and stored.

\begin{eqnarray}
  M_d^0 = \sum_{ijk} (1-q-l) p_S \Delta\lambda_i w_j \Delta\sigma_k
\end{eqnarray}

After exiting dynamical calculation, \texttt{{[}MASFIX{]}}, the
following procedure is followed.

First, negative water vapor is removed by dividing the water vapor from
the grid points immediately below the grid points. Suppose that
\(q_k < 0\) is used,

\begin{eqnarray}
        q_k'      =  0          \\
        q_{k-1}'  =  q_{k-1} + \frac{\Delta p_k}{\Delta p_{k-1}} q_k
\end{eqnarray}

However, this should only be done if it is \(q_{k-1}' \ge 0\).

Next, set the value to zero for the grid points not removed by the above
procedure.

\begin{enumerate}
\def\labelenumi{\arabic{enumi}.}
\setcounter{enumi}{2}
\tightlist
\item
  calculate the global integral value of \(M_q\) and multiply the global
  water vapor content by a fixed percentage so that it is the same as
  that of \(M_q^0\).
\end{enumerate}

\begin{eqnarray}
        q'' = \frac{M_q^0}{M_q} q'
\end{eqnarray}

\begin{enumerate}
\def\labelenumi{\arabic{enumi}.}
\setcounter{enumi}{3}
\tightlist
\item
  correct for dry air mass Likewise calculate \(M_d\),
\end{enumerate}

\begin{eqnarray}
        p_S'' = \frac{M_d^0}{M_d} p_S
\end{eqnarray}

Corresponding file \& subroutine:
\texttt{{[}FIXMAS,\ MASFIX\ (dmfix.F){]}}
