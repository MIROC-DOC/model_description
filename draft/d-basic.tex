\hypertarget{dynamics}{%
\section{Dynamics}\label{dynamics}}

\hypertarget{basic-equations}{%
\subsection{Basic Equations}\label{basic-equations}}

\hypertarget{basic-equations-1}{%
\subsubsection{Basic Equations}\label{basic-equations-1}}

The basic equations are a system of primitive equations at the spherical
(\(\lambda,\varphi\)) and \(\eta\) coordinates, given as follows
(Arakawa and Konor 1996).

\begin{enumerate}
\def\labelenumi{\arabic{enumi}.}
\tightlist
\item
  Continuity equation
\end{enumerate}
\begin{eqnarray}
  \frac{\partial m}{\partial t}
    + \nabla_{\eta} \cdot (m\mathbf{v}_H)+ \frac{\partial (m\dot{\eta})}{\partial \eta} = 0
\end{eqnarray}

\begin{enumerate}
\def\labelenumi{\arabic{enumi}.}
\setcounter{enumi}{1}
\tightlist
\item
  Hydrostatic equation
\end{enumerate}
\begin{eqnarray}
  \frac{\partial \Phi}{\partial \eta} = - \frac{RT_v}{p} m
\end{eqnarray}

\begin{enumerate}
\def\labelenumi{\arabic{enumi}.}
\setcounter{enumi}{2}
\tightlist
\item
  Equation of motion
\end{enumerate}
\begin{eqnarray}
  \frac{\partial \zeta}{\partial t} 
     =   \frac{1}{a\cos\varphi}
            \frac{\partial A_v}{\partial \lambda}
          - \frac{1}{a\cos \varphi}
            \frac{\partial}{\partial \varphi} ( A_u \cos\varphi )
          - {\mathcal D}(\zeta) 
\end{eqnarray}
\begin{eqnarray}
  \frac{\partial D}{\partial t} 
     =    \frac{1}{a\cos\varphi}
            \frac{\partial A_u}{\partial \lambda}
          + \frac{1}{a\cos\varphi}
            \frac{\partial }{\partial \varphi} ( A_v \cos\varphi )
          - \nabla^{2}_{\eta}
           ( \Phi + R \bar{T} \pi + E ) 
          - {\mathcal D}(D) 
\end{eqnarray}

\begin{enumerate}
\def\labelenumi{\arabic{enumi}.}
\setcounter{enumi}{3}
\tightlist
\item
  Thermodynamic equation
\end{enumerate}
\begin{eqnarray}
  \frac{\partial T}{\partial t}
     &=&  - \frac{1}{a\cos\varphi}
               \frac{\partial uT'}{\partial \lambda}
          - \frac{1}{a}
               \frac{\partial }{\partial \varphi} ( vT' \cos\varphi )
          + T' D \\
        &-& \dot{\eta} 
              \frac{\partial T }{\partial \eta}
          + \frac{\kappa T}{\sigma} \left[ B\left( \frac{\partial \pi}{\partial t}
                            + {\mathbf{v}}_{H} \cdot \nabla_{\eta}\pi \right)
                            + \frac{ m\dot{\eta} }{ p_s }
                     \right]
          + \frac{Q}{C_{p}}
          + \frac{Q_{diff}}{C_{p}}
          - {\mathcal D}(T) 
\end{eqnarray}

\begin{enumerate}
\def\labelenumi{\arabic{enumi}.}
\setcounter{enumi}{4}
\tightlist
\item
  Tracers
\end{enumerate}

For any tracer whose mixing ratio is denoted as \(q\),
\begin{eqnarray}
  \frac{\partial q}{\partial t}
   &=&  - \frac{1}{a\cos\varphi}
               \frac{\partial uq}{\partial \lambda}
          - \frac{1}{a\cos\varphi}
               \frac{\partial }{\partial \varphi} (vq \cos\varphi)
          + q D \\
        &-& \dot{\eta} \frac{\partial q }{\partial \eta}
          + S_{q}
          - {\mathcal D}(q) 
\end{eqnarray}

Here,
\begin{eqnarray}
m &\equiv & \left(\frac{\partial p}{\partial \eta}\right)_{p_s}, \\
\theta  &\equiv &  T \left( p/p_{0} \right)^{-\kappa}, \\
\kappa  &\equiv &  R/C_{p}, \\
  \Phi  &\equiv &  gz, \\
   \pi  &\equiv &  \ln p_{S}, \\
 \dot{\eta}  &\equiv &   \frac{\mathrm{d}\eta}{\mathrm{d}t}, \\
     T_v  &\equiv &  T ( 1+\epsilon_v q ), \\
     T  &\equiv &   \bar{T} + T^{\prime}, \\
     \bar{T}&\equiv & 300 \ \mathrm{K}, \\
 \zeta  &\equiv &  \frac{1}{a \cos\varphi }
                    \frac{\partial v}{\partial \lambda} 
             -    \frac{1}{a \cos\varphi }
                    \frac{\partial }{\partial \varphi}
                    ( u \cos\varphi ), \\
     D  &\equiv &  \frac{1}{a \cos\varphi }
                    \frac{\partial u}{\partial \lambda} 
             +    \frac{1}{a \cos\varphi }
                    \frac{\partial }{\partial \varphi}
                    ( v \cos\varphi ), \\
    A_u  &\equiv &   ( \zeta + f ) v
             - \dot{\eta} \frac{\partial u}{\partial \eta} 
             - \frac{RT^{\prime}}{a\cos\varphi} 
                  \frac{\partial \pi}{\partial \lambda} 
             + {\mathcal F}_x, \\
    A_v  &\equiv &  - ( \zeta + f ) u
             - \dot{\eta} \frac{\partial v}{\partial \eta} 
             - \frac{RT^{\prime}}{a}
                  \frac{\partial \pi}{\partial \varphi} 
             + {\mathcal F}_y, \\
     E  &\equiv &   \frac{u^{2}+v^{2}}{2}, \\
 {\mathbf{v}}_{H} \cdot \nabla
        &\equiv &  \frac{u}{a \cos \varphi} 
         \left( \frac{\partial }{\partial \lambda} \right)_{\sigma}
     + \frac{v}{a}
         \left( \frac{\partial }{\partial \varphi} \right)_{\sigma}, \\
  \nabla^{2}_{\eta}  
        &\equiv &  
               \frac{1}{a^{2}\cos^2\varphi} 
                 \frac{\partial^{2} }{\partial \lambda^{2}} 
             + \frac{1}{a^{2}\cos\varphi} 
                 \frac{\partial }{\partial \varphi}
                 \left[ \cos\varphi
                       \frac{\partial }{\partial \varphi} \right].
\end{eqnarray}

\({\mathcal D}(\zeta), {\mathcal D}(D), {\mathcal D}(T), {\mathcal D}(q)\)
are horizontal diffusion terms,
\({\mathcal F}_\lambda, {\mathcal F}_\varphi\) are forces due to
small-scale kinetic processes (treated as `physical processes'), \(Q\)
are forces due to radiation, condensation, small-scale kinetic
processes, etc. Heating and temperature change due to `physical
processes', and \(S_q\) is a water vapor source term due to `physical
processes' such as condensation and small-scale motion. \(Q_{diff}\) is
the heat of friction and
\begin{eqnarray}
  Q_{diff}
 = - {\mathbf{v}} \cdot  \left( \frac{\partial {\mathbf{v}}}{\partial t} \right)_{diff} .
\end{eqnarray}

\(( \frac{\partial {\mathbf{v}}}{\partial t} )_{diff}\) is a
time-varying term of \(u,v\) due to horizontal and vertical diffusion.

\hypertarget{boundary-conditions}{%
\subsubsection{Boundary Conditions}\label{boundary-conditions}}

Upper and lower boundary conditions for the vertical velocity is:
\begin{eqnarray}
  \dot{\eta} = 0  \ \ \ at \ \ \eta = 0 , \ 1 .
\end{eqnarray}
The prognostic equation for \(p_s\) and the diagnostic equation for the
vertical velocity can be derived by integrating the continuity equation
and applying these boundary conditions.
