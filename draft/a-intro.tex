\hypertarget{features-and-structure-of-the-model}{%
\subsection{Features and structure of the
model}\label{features-and-structure-of-the-model}}

\textbf{NOTE: the descriptions in this section are outdated.}

\hypertarget{basic-features-of-the-model.}{%
\subsubsection{Basic Features of the
Model.}\label{basic-features-of-the-model.}}

The CCSR/NIES AGCM is a numerical model for describing the global
three-dimensional atmosphere based on physical laws and calculating the
time evolution of the system as an initial value problem.

The data to be inputted are as follows.

\begin{itemize}
\item
  Initial data for each forecast variable (horizontal wind speed,
  temperature, surface pressure, specific humidity, cloud liquid water
  content, and surface volume)
\item
  Boundary condition data (surface elevation, surface condition, sea
  surface temperature, etc.)
\item
  Various parameters of the model (atmospheric components, physical
  process parameters, etc.)
\end{itemize}

On the other hand, the output looks like the following.

\begin{itemize}
\item
  Data for each forecast parameter and diagnostic parameter, for each
  time or time average
\item
  Initial data to be used for continuous execution (restart data)
\item
  Progress and various diagnostic messages
\end{itemize}

The predictor is the data obtained as a time series by integrating the
differential equation of time evolution, and the diagnostic variable is
the quantity calculated from the predictor, the boundary conditions and
the parameters by some method that does not include time integration.

More specifically, the model basically finds the solution to the
following equations (forecast equations).

\begin{eqnarray}
  \frac{\partial{u}}{\partial {t}}  =  \left( {\mathcal F}_x \right)_D + \left( {\mathcal F}_x \right)_P
   \\
  \frac{\partial{v}}{\partial {t}}  =  \left( {\mathcal F}_y \right)_D + \left( {\mathcal F}_y \right)_P \\
  \frac{\partial{T}}{\partial {t}}  =  \left( Q \right)_D + \left( Q \right)_P \\
  \frac{\partial{p_S}}{\partial {t}}  =  \left( M \right)_D + \left( M \right)_P \\
  \frac{\partial{q}}{\partial {t}}  =  \left( S \right)_D + \left( S \right)_P \\
  \frac{\partial{T_g}}{\partial {t}}  =  \left( Q_g \right)_D + \left( Q_g \right)_P
\end{eqnarray}

Here, \(u,v,T,p_S,q,T_g\) are two-dimensional and three-dimensional
forecast variables such as wind, north-south wind, temperature, surface
pressure, specific humidity, and surface state amount, respectively, and
the right-hand side is a term that causes time variation of each
forecast variable. The terms \({\mathcal F}_x,{\mathcal F}_y,Q,S,Q_g\)
are calculated based on the forecast variables \(u,v,T,p_S,q,T_g\), but
the terms \(u\) and \(v\), such as advection due to the motion of the
atmosphere (the terms of the terminal terminal number 5 in the above
equation), and the terms \(D\), such as cloud and radiation, are not
included in the time-varying variables. There are two main types of
terms, one for each process in the process of The former is called the
mechanical process, and the latter is called the physical process.

The advection term is the main part of the time-varying term in
mechanical processes, and the accurate estimation of the spatial
derivative is important in its calculation. The CCSR/NIES AGCM utilizes
the spherical harmonic expansion to calculate the horizontal
differential term. On the other hand, it is important for physical
processes to be represented in a simple model with parameters
(parameterization), such as energy conversions due to the phase change
of water, radiative absorption and emission, the effects of small-scale
atmospheric motions, and the effects of various processes on the ground
surface.

The time integration of the forecasting equation is done by
approximating the left-hand side of (1) etc. by the difference. For
example,

\begin{eqnarray}
  \frac{\partial{q}}{\partial {t}} \rightarrow \frac{q^{t+\Delta t} - q^{t}}{\Delta t}
\end{eqnarray}

By making ,

\begin{eqnarray}
  q^{t+\Delta t} = q^{t}
       + \Delta t \left[ \left( S \right)_D + \left( S \right)_P  \right]
\end{eqnarray}

where \(S\) is a function of the forecast variables \(u,v,T,p_S,q\).
Although \(S\) is a function of the forecast variables \(u,v,T,p_S,q\),
and so on, there are various time difference schemes that can be used in
this calculation depending on the time of day the forecast variables are
used to evaluate \(S\). The CCSR/NIES AGCM uses the Euler method, which
uses the value of the \(t\) as it is, the leap frog method, which uses
the value of the \(t+\Delta t/2\), and the implicit method, which uses
the (approximate) value of the \(t+\Delta t\).

In the CCSR/NIES AGCM, the time integration of the predictors is done
separately for the mechanical and physical processes. The first term of
the dynamics is basically a leap frog,

\begin{eqnarray}
  \tilde{q}^{t+\Delta t} = q^{t-\Delta t} + 2 \Delta t \left( S \right)_D^{t}
\end{eqnarray}

However, some terms are treated as implicit. with the exception of some
terms which are treated as implicit. In the physical process, based on
the results of integrating the mechanical terms, the Euler and implicit
methods are used together,

\begin{eqnarray}
  q^{t+\Delta t} = \tilde{q}^{t+\Delta t} + 2 \Delta t \left( S \right)_P
\end{eqnarray}

in (8). Note that \(\Delta t\) in (8) is replaced by \(2 \Delta t\).

\hypertarget{model-execution-flow.}{%
\subsubsection{Model Execution Flow.}\label{model-execution-flow.}}

The flow of the model execution is briefly shown below. The entries in
the index are the names of the corresponding subroutine.

\begin{enumerate}
\def\labelenumi{\arabic{enumi}.}
\item
  set the parameters of an experiment, coordinates, etc.
  \texttt{MODULE:{[}SETPAR,PCONST,SETCOR,SETZS{]}}
\item
  read the initial values of the predictor variable
  \texttt{MODULE:{[}RDSTRT{]}}
\item
  start the time step \texttt{MODULE:{[}TIMSTP{]}}
\item
  perform time integration by mechanical processes
  \texttt{MODULE:{[}DYNMCS{]}}
\item
  perform time integration by physical processes
  \texttt{MODULE:{[}PHYSCS{]}}
\item
  advance the time \texttt{MODULE:{[}STPTIM,\ TFILT{]}}
\end{enumerate}

Output the data if necessary \texttt{MODULE:{[}HISTOU{]}}

Output the restart data if necessary \texttt{MODULE:{[}WRRSTR{]}}

\begin{enumerate}
\def\labelenumi{\arabic{enumi}.}
\setcounter{enumi}{8}
\tightlist
\item
  3. Back to
\end{enumerate}

\hypertarget{predictive-variables.}{%
\subsubsection{Predictive variables.}\label{predictive-variables.}}

The predictive variables are as follows. The values in parentheses are
the coordinate system, and \(\lambda,\varphi,\sigma, z\) indicate the
longitude, latitude, dimensionless pressure, \(\sigma\), and vertical
depth, respectively. The values in the square brackets are in units of
the index.

\setlength\LTleft{0pt}\setlength\LTright{0pt}\begin{longtable}[]{@{}lll@{}}
\toprule\relax
Header0 & Header1 & Header2 \\
\midrule\relax
\endhead
east-west wind speed & \(u\) (\(\lambda,\varphi,\sigma\)) &
\(\mathrm{[m/s]}\) \\
north-south wind speed & \(v\) (\(\lambda,\varphi,\sigma\)) &
\(\mathrm{[m/s]}\) \\
atmospheric temperature & \(T\) (\(\lambda,\varphi,\sigma\)) &
\(\mathrm{[K]}\) \\
surface pressure & \(p_S\) (\(\lambda,\varphi\)) & \(\mathrm{[hPa]}\) \\
specific humidity & \(q\) (\(\lambda,\varphi,\sigma\)) &
\(\mathrm{[kg/kg]}\) \\
Cloud water mixing ratio & \(l\) (\(\lambda,\varphi,\sigma\)) &
\(\mathrm{[kg/kg]}\) \\
underground temperature & \(T_g\) (\(\lambda,\varphi,z\)) & {[}K{]} \\
subterranean moisture & \(W_g\) (\(\lambda,\varphi,z\)) &
\(\mathrm{[m^3/m^3]}\) \\
amount of snowfall & \(W_y\) (\(\lambda,\varphi\)) & \(\mathrm{[-]}\) \\
sea-ice thickness & \(h_I\) (\(\lambda,\varphi\)) & \(\mathrm{[-]}\) \\
\bottomrule
\end{longtable}

However, the sea ice thickness is usually only a predictor in the
mixed-layer coupled model. Also, subsurface temperature is not usually a
predictor when the ocean is not covered by sea ice. In the CCSR/NIES
AGCM, \(q\) and \(l\) are not independent variables; in fact, \(q+l\) is
the forecast variable.

Of these quantities, the quantities for the surface and the subsurface,
\(T_g, W_g, W_y, h_I\), store only one step at a time, while the
quantities for the atmosphere, \(u, v, T, p_S, q, l\), need to store two
steps at a time. This is due to the fact that the leap forg method is
used in the time integration of the dynamic process of the quantities
related to the atmosphere.

The quantities of the atmosphere, \(u, v, T, p_S, q, l\), are variables
managed by the main routine,
\texttt{Administration\ of\ the\ Atmosphere\textquotesingle{}{[}AGCM5\textbackslash{}a{]}}.
On the other hand, the quantities relating to the earth's surface and
ground, \(T_g, W_g, W_y, h_I\), do not appear in the main routine, but
are managed by the subroutine \texttt{MODULE:{[}PHYSCS{]}} of the
physical process.

\hypertarget{the-flow-of-time-evolution-of-variables}{%
\subsubsection{The flow of time evolution of
variables}\label{the-flow-of-time-evolution-of-variables}}

We briefly summarize the flow of the model, focusing on the time
evolution of the predictor variables.

\begin{enumerate}
\def\labelenumi{\arabic{enumi}.}
\tightlist
\item
  read the initial value \texttt{MODULE:{[}RDSTRT,PRSTRT{]}}
\end{enumerate}

Initially, the quantities \(u, v, T, p_S, q, l\) for the atmosphere must
be prepared as two sets of quantities in \(t\) and \(t-\Delta t\). These
quantities can be prepared when starting from the output of previous
models, but cannot be prepared when starting from ordinary observations
or climatic values. In that case, we will start from the same value of
the two time steps and start up the calculation using the fine
\(\Delta t\) (see below for details).

Initial values of the atmospheric quantities \(u, v, T, p_S, q, l\) are
read by the \texttt{MODULE:{[}RDSTRT{]}}, which is called by the main
routine. On the other hand, the initial values for the quantities of
\(T_g, W_g, W_y, h_I\) for the earth's surface and ground are read by
\texttt{MODULE:{[}PRSTRT{]}}, which is called by
\texttt{MODULE:{[}PHYSCS{]}}.

\begin{enumerate}
\def\labelenumi{\arabic{enumi}.}
\setcounter{enumi}{1}
\tightlist
\item
  start the time step \texttt{MODULE:{[}TIMSTP{]}}
\end{enumerate}

Forecast variables at time \(t\) (and partly in \(t-\Delta t\))
\(u^{t}, u^{t-\Delta t}, v^{t}, v^{t-\Delta t}, T^{t}, T^{t-\Delta t}, p_S^{t}, p_S^{t-\Delta t}, q^{t}, q^{t-\Delta t}, l^{t}, l^{t-\Delta t}, T_g^{t}, W_g^{t}, W_y^{t}\)
shall be complete.

\begin{verbatim}
 Although $\Delta t$ is basically an externally given parameter, the stability of the calculation is evaluated at regular intervals and $\Delta t$ should be reduced if the calculation is likely to be unstable.
\end{verbatim}

Set the output of the predictor variable \texttt{MODULE:{[}AHSTIN{]}}

In the atmospheric forecast variables, what is usually output is the
value of time \(t\) at this stage,
\(u^{t}, v^{t}, T^{t}, p_S^{t}, q^{t}, l^{t}\). The actual output is
done at the later timing of \texttt{MODULE:{[}HISTOU{]}}, but it is sent
to the buffer at this point.

\begin{enumerate}
\def\labelenumi{\arabic{enumi}.}
\setcounter{enumi}{3}
\tightlist
\item
  mechanical processes \texttt{MODULE:{[}DYNMCS{]}}
\end{enumerate}

Solving for the time variation of the predicted variables due to
dynamical processes.
\(u^{t}, u^{t-\Delta t}, v^{t}, v^{t-\Delta t}, T^{t}, T^{t-\Delta t}, p_S^{t}, p_S^{t-\Delta t}, q^{t}, q^{t-\Delta t}, l^{t}, l^{t-\Delta t}\)
The value of the predicted variable in \(t+\Delta t\), considering only
mechanical processes, based on
\(\hat{u}^{t+\Delta t}, \hat{v}^{t+\Delta t}, \hat{T}^{t+\Delta t}, \hat{p_S}^{t+\Delta t}, \hat{q}^{t+\Delta t}, \hat{l}^{t+\Delta t}\)
Ask for .

\begin{verbatim}
 1. convert to vorticity and divergence `MODULE:[UV2VDG, VIRTMD, HGRAD]`
\end{verbatim}

In order to estimate the change terms of the atmospheric predictors
\(u, v, T, p_S, q, l\) due to mechanical processes, \(u^{t}, v^{t}\) are
first converted to grid values of vorticity and divergence,
\(\zeta^{t},D^{t}\). This is because the dynamics equations are written
in terms of vorticity and divergence. Although this transformation
involves a spatial derivative, it can be performed precisely by using
the spherical harmonic expansion \texttt{MODULE:{[}UV2VDG{]}}.

Furthermore, we calculate the pseudo temperature \(T_v^{t}\) and the
horizontal differential of surface pressure \(\nabla \ln p_S\) using
\texttt{MODULE:{[}VIRTMD{]}}, also using the spherical harmonic
expansion function, and \texttt{MODULE:{[}HGRAD{]}}.

\begin{verbatim}
 2. calculation of the time-varying term by advection `MODULE:[GRDDYN]`
\end{verbatim}

Using the values in \(u, v, T, p_S, q, l\) at \(t\), a part of the
time-varying terms for each atmospheric variable is calculated by using
the values in \(u, v, T, p_S, q, l\) for horizontal and vertical
advection. First, the time-varying terms of vertical velocity
\(\dot{\sigma}\) and \(p_S\) are diagnostically obtained from successive
equations, and then the vertical advection terms of \(u, v, T, q, l\)
are calculated using the time-varying terms. Furthermore, calculate the
horizontal advection fluxes in \(u, v, T, q, l\).

\begin{verbatim}
 3. convert to a spectrum `MODULE:[GD2WD, TENG2W]`
\end{verbatim}

Value of grid points in \(t-\Delta t\) for atmospheric forecast
variables from \(u^{t}, v^{t}, T^{t}, p_S^{t}, q^{t}, l^{t}\) in the
spectral space of spherical harmonic function expansion (but converted
to vorticity divergence)
\(\tilde{\zeta}^{t}, \tilde{D}^{t}, \tilde{T}^{t}, \tilde{\pi}^{t}, \tilde{q}^{t}, \tilde{l}^{t}\)
(but, \(\pi \equiv \ln p_S\)) \texttt{MODULE:{[}GD2WD{]}}.

In addition, the time-varying terms of \(u, v, T, p_S, q, l\) are
expanded into spectra. We also calculate the convergence of the
horizontal advection fluxes by using the derivative in spectral space
and add them to the spectral representation of the time-varying term
\texttt{MODULE:{[}TENG2W{]}}.

With this, most of the time-varying terms in \(\zeta, D, T, \pi, q, l\)
can be obtained as spectral values. However, among the time-varying
terms in \(\zeta, D, T, \pi\), those that depend on the horizontally
diverging \(D\) are not included in the time-varying term at this point
because the time integration is performed by the semi-implicit method.

\begin{verbatim}
 4. time integration `MODULE:[TINTGR]`
\end{verbatim}

Among the time-varying terms in \(\zeta, D, T, \pi\), a term that
depends linearly on the horizontal divergence \(D\) (the gravitational
wave term) is treated by the semi-implicit method, and furthermore, by
implicitly incorporating the horizontal diffusion of
\(\zeta, D, T, q, l\), the mechanical process part of Time integration
is performed. This allows for a spectral representation of the predicted
value of \(t+\Delta t\) considering only the mechanical processes
\(\tilde{\zeta}^{t+\Delta t}, \tilde{D}^{t+\Delta t}, \tilde{T}^{t+\Delta t}, \tilde{\pi}^{t+\Delta t}, \tilde{q}^{t+\Delta t}, \tilde{l}^{t+\Delta t}\)
is required.

\begin{verbatim}
 5. conversion to grid values `MODULE:[GENGD]`
\end{verbatim}

Grid values for the forecast values of \(u, v, T, p_S, q, l\), and
\(t+\Delta t\) considering only mechanical processes from a spectral
representation of the forecast variables
\(\hat{u}^{t+\Delta t}, \hat{v}^{t+\Delta t}, \hat{T}^{t+\Delta t}, \hat{p_S}^{t+\Delta t}, \hat{q}^{t+\Delta t}, \hat{l}^{t+\Delta t}\)
.

\begin{verbatim}
 6. diffusion correction `MODULE:[CORDIF, CORFRC]`
\end{verbatim}

Horizontal diffusion is applied in the plane of \(\sigma\), but in large
slopes, it causes problems such as upward transport of water vapor and
false precipitation at the tops of mountains. To mitigate this problem,
a correction has been added for \(T,q,l\) to be close to the diffusion
of \(p\) surface, such as \texttt{MODULE:{[}CORDIF{]}}.

Also, heat from friction is added to \(\hat{T}\)
\texttt{MODULE:{[}CORFRC{]}}

\begin{verbatim}
 7. mass conservation correction `MODULE:[MASFIX]`
\end{verbatim}

Corrections are made so that the global integral values of \(q\) and
\(l\) are preserved and the negative value of \(q\) is eliminated.
Furthermore, corrections are made so that the mass of dry air remains
constant.

After exiting DYNMCS, the value of the forecaster variable in
\(t-\Delta t\) is discarded and is overwritten by the value of the
forecaster variable in \(t\). The area of the forecaster variable in
\(t\) is replaced by the value of the forecaster variable in
\(t+\Delta t\) which only takes into account the mechanical processes.

\begin{enumerate}
\def\labelenumi{\arabic{enumi}.}
\setcounter{enumi}{4}
\tightlist
\item
  physical process \texttt{MODULE:{[}PHYSCS{]}}
\end{enumerate}

Value of the predicted variables in \(t+\Delta t\) considering only
mechanical processes
\(\hat{u}^{t+\Delta t}, \hat{v}^{t+\Delta t}, \hat{T}^{t+\Delta t}, \hat{p_S}^{t+\Delta t}, \hat{q}^{t+\Delta t}, \hat{l}^{t+\Delta t}\)
and by adding a time-varying term from physical processes to the value
of the predicted variable in \(t+\Delta t\)
\(u^{t+\Delta t}, v^{t+\Delta t}, T^{t+\Delta t}, p_S^{t+\Delta t}, q^{t+\Delta t}, l^{t+\Delta t}\)
Ask for .

\begin{verbatim}
 Calculation of the basic diagnostic variables `MODULE:[PSETUP]`
\end{verbatim}

Find the basic diagnostic variables such as pseudo temperature,
barometric pressure at each level, and altitude.

\begin{verbatim}
 2. cumulus convection, large-scale condensation `MODULE:[CUMLUS, LSCOND]`
\end{verbatim}

Calculates the time-varying terms of \(T, q, l\) due to cumulus
convection and integrates them with \texttt{MODULE:{[}CUMLUS{]}} and the
time integration with \texttt{MODULE:{[}GDINTG{]}} using the term alone.
Also, the time-varying terms of \(T, q, l\) due to large-scale
condensation are found and integrated by \texttt{MODULE:{[}LSCOND{]}},
and the time integration is performed with \texttt{MODULE:{[}GDINTG{]}}
using only the term \texttt{MODULE:{[}GDINTG{]}}. Precipitation due to
cumulus convection and large-scale condensation (\(P_c, P_l\)), cloud
cover (\(C_c, C_l\)), and so on can be obtained. This gives the values
of \(T, q, l\) adjusted for convective condensation processes
(\(\hat{T}^{t+\Delta t,a}, \hat{q}^{t+\Delta t,a}, \hat{l}^{t+\Delta t,a}\)).

\begin{verbatim}
 3. set the surface boundary condition `MODULE:[GNDSFC, GNDALB]`
\end{verbatim}

The state of the earth surface is set according to a given data. The
ground state index, the sea surface temperature, and so on are set
\texttt{MODULE:{[}GNDSFC{]}}. The surface albedo is set according to the
data given by \texttt{MODULE:{[}GNDALB{]}}. (The calculation of the
sea-surface albedo is incorporated into the routine for calculating the
radiation flux.)

\begin{verbatim}
 4. calculation of the radiation flux `MODULE:[RADCON, RADFLX]`
\end{verbatim}

Set the atmospheric composition data for radiation flux calculation
\texttt{MODULE:{[}RADCON{]}}. Usually, ozone is read from a file. The
cloud water and cloud masses are obtained from the cumulus convection
and large-scale condensation methods, but they can also be given here
externally. Using these two files and
\(\hat{T}^{t+\Delta t,a}, \hat{q}^{t+\Delta t,a}\), the differential
coefficients for the surface temperature used in \(F_R\) and the
implicit calculation are calculated with the \(F_R\) and
\texttt{MODULE:{[}RADFLX{]}}.

\begin{verbatim}
 5. calculation of the vertical diffuse flux `MODULE:[VDFFLX, VFTND1]`
\end{verbatim}

Using
\(\hat{u}^{t+\Delta t}, \hat{v}^{t+\Delta t}, \hat{T}^{t+\Delta t,a}, \hat{q}^{t+\Delta t,a}, \hat{l}^{t+\Delta t,a}\),
the fluxes of \(u, v, T, q, l\) by the vertical diffusion process and
the differential coefficient for implicit calculation are calculated
using \texttt{MODULE:{[}VDFFLX{]}}. Furthermore, computes the implicit
solution using the LU decomposition up to the middle of the process
\texttt{MODULE:{[}VFTND1{]}}.

\begin{verbatim}
 6. calculation of surface processes and time integration of underground variables `MODULE:[SURFCE]`
\end{verbatim}

The fluxes of \$u, v, T, q \$ between the surface and the atmosphere are
calculated, and the energy balance at the ground surface is solved by
using an implicit solution considering the conduction of heat in the
ground. This results in obtaining the surface temperature (\(T_s\)) and
the value of the ground temperature (\(T_g^{t+\Delta t}\)) from the
surface temperature (\(t+\Delta t\)) diagnostically. Furthermore, the
time rate of change of the predicted variables of the atmosphere in the
first layer, \(F_{x,1}, F_{y,1}, Q_1, S_1\), can be obtained.

Considering the snow accumulation and snowmelt process, the value of
snow accumulation (\(t+\Delta t\)) is determined and the ground moisture
(\(W_g^{t+\Delta t}\)) is determined by considering the movement of
water in the ground.

When an oceanic mixed-layer model is used, the values of ocean
temperature and sea ice thickness can be obtained by time integration in
\(t+\Delta t\).

\begin{verbatim}
 7. evaluation of time variation due to radial and vertical diffusion `MODULE:[VFTND2, RADTND, FLXCOR]`
\end{verbatim}

Rate of change of each forecast variable for the combined radiative flux
and vertical diffusion Calculates
\({\mathcal F}_x, {\mathcal F}_y, Q, S\) \texttt{MODULE:{[}VFTND2{]}}.
Furthermore, the contribution of radiation is separated from the model
by \texttt{MODULE:{[}RADTND{]}}. This is not used directly in the model,
but is done for the sake of outputting the data.

Since we use the implicit method, we take into account the changes in
fluxes due to changes in surface temperature and atmospheric variables.
The fluxes are calculated by \texttt{MODULE:{[}FLXCOR{]}} to take the
change in the surface temperature and atmospheric variables into
account. This is also for the convenience of the data output.

\begin{verbatim}
 8. evaluation of gravitational wave resistance `MODULE:[GRAVTY]`
\end{verbatim}

The change in atmospheric momentum due to geological origin is
calculated and added to the rates of change of \(u, v\) due to vertical
diffusion of the atmosphere (\({\mathcal F}_x, {\mathcal F}_y\)).

\begin{verbatim}
 9. evaluation of the atmospheric pressure change term
\end{verbatim}

Considering the change in atmospheric pressure due to precipitation and
evaporation, the atmospheric pressure change term \(M\) is obtained.

\begin{verbatim}
 10. time integration of physical processes `MODULE:[GDINTG]`
\end{verbatim}

Using the rates of change of atmospheric variables such as radiation,
vertical diffusion, surface processes, and gravitational wave resistance
calculated above, the values in \(t+\Delta t\) are integrated in time
using \({\mathcal F}_x, {\mathcal F}_y, Q, M, S\).

\begin{verbatim}
 11. drying convection adjustment `MODULE:[DADJST]`
\end{verbatim}

If the calculated values of \(T, q, l\) are unstable with respect to dry
convection, dry convection adjustment is applied.

By the above procedure, the value of the forecast variable in
\(t+\Delta t\)
\(u^{t+\Delta t}, v^{t+\Delta t}, T^{t+\Delta t}, p_S^{t+\Delta t}, q^{t+\Delta t}, l^{t+\Delta t}\)
is required.

\begin{enumerate}
\def\labelenumi{\arabic{enumi}.}
\setcounter{enumi}{5}
\tightlist
\item
  time filter \texttt{MODULE:{[}TFILT{]}}
\end{enumerate}

A time filter is applied to suppress the occurrence of calculation modes
by leap frog. With the time data of
\(u^{t-\Delta t}, u^{t}, u^{t+\Delta t}\), smoothing operations are
applied to each variable to convert them to \(u^{t}\). (Actually, since
the information on \(u^{t-\Delta t}\) is deleted at
\texttt{MODULE:{[}TFILT{]}} stage, this operation is performed in two
steps. The first operation is performed in the mechanical process.)
