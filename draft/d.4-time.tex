\hypertarget{time-integration}{%
\subsection{Time Integration}\label{time-integration}}

The time discretization is essentially the leap frog scheme. However,
backward or forward differences are used for diffusion terms and
physical process terms.
A time filter (Williams, 2009), which is a
modified version of the Asselin time filter (Asselin 1972), is used to
suppress computational modes.
A semi-implicit method is applied to
the gravitational wave term to make the \(\Delta t\) larger (Bourke,
1988).

\hypertarget{time-integration-and-time-filtering-with-leap-frog}{%
\subsubsection{Time Integration and Time Filtering with the Leap
Frog Method}\label{time-integration-and-time-filtering-with-leap-frog}}

We use leap frog as the time integration scheme for advection terms and
other dynamic terms.
A backward difference of \(2 \Delta t\) is used for
the horizontal diffusion term.
The \(p\)-surface correction of the
diffusion term and the frictional heat due to horizontal diffusion
are treated by forward differences of
\(2 \Delta t\). The physical process terms
(\({\mathcal F}_\lambda, {\mathcal F}_\varphi, Q, S_q\)) use the
forward difference of \(2 \Delta t\) (except for the vertical diffusion
term, which uses the forward difference of
\({\mathcal F}_\lambda, {\mathcal F}_\varphi, Q, S_q\)). However, the
calculation of the prognositc veriables of vertical diffusion is treated
as a backward difference. Please refer to the chapter on physical
processes for details.)

Expressing each prognostic variable as \({X}\),
\begin{eqnarray}
  \hat{X}^{t+\Delta t}
    =  \bar{X}^{t-\Delta t}
    + 2 \Delta t
      \dot{X}_{adv}\left( {X}^{t} \right)
    + 2 \Delta t
      \dot{X}_{dif}\left( \hat{X}^{t+\Delta t} \right),
\end{eqnarray}
where \(\dot{X}_{adv}\) is the advection term etc., and
\(\dot{X}_{dif}\) is the horizontal diffusion term.

\(\hat{X}^{t+\Delta t}\) is then corrected for diffusion
(\(\dot{X}_{dis}\) for \(p\)-surface correction and the heat of
friction) and physical processes (\(\dot{X}_{phy}\)), yielding
\({X}^{t+\Delta t}\).
\begin{eqnarray}
  {X}^{t+\Delta t}
    =  \hat{X}^{t+\Delta t}
    + 2 \Delta t
      \dot{X}_{dis}\left( \hat{X}^{t+\Delta t} \right)
    + 2 \Delta t
      \dot{X}_{phy}\left( \hat{X}^{t+\Delta t} \right)
\end{eqnarray}

To damp numerical modes, a time filter (Williams, 2009) is applied to
leap-frog method at every steps.
The time filter is given below, where terms with over bars are filtered.
\begin{eqnarray}
\bar{\bar{X}}^{t} = \bar{X}^{t} + \nu \alpha [\bar{\bar{X}}^{t-\Delta t} - \bar{X}^{t} + X^{t+\Delta t}],
\end{eqnarray} \begin{eqnarray}
\bar{X}^{t+\Delta t} = X^{t+\Delta t} + \nu (1-\alpha) [\bar{\bar{X}}^{t-\Delta t} - 2 \bar{X}^{t} + X^{t+\Delta t}],
\end{eqnarray}
where \(\nu=0.05\) and \(\alpha=0.5\).

\hypertarget{semi-implicit-time-integration}{%
\subsubsection{Semi-Implicit Time
Integration}\label{semi-implicit-time-integration}}

Basically, the leap frog is used for the dynamic processes, but the
trapezoidal implicit scheme is used for some terms. For a vector
quantity \({\mathbf q}\), let us write the value at \(t\) as
\({\mathbf q}\), the value at \(t+\Delta t\) as \({\mathbf q}^+\), and
the value at \(t-\Delta t\) as \({\mathbf q}^-\). Then, in the
trapezoidal implicit scheme, the time change term is evaluated for
\(({\mathbf q}^+ + {\mathbf q}^- )/2\), instead of \({\mathbf q}\) used
in the simple leap forg method. We now divide \({\mathbf q}\) into two
time varying terms, one (\({\mathcal A}\)) for the leap forg method and
the other (\({\mathcal B}\)) for the trapezoidal implicit method. We
assume that (\({\mathcal A}\)) is nonlinear to \({\mathbf q}\), while
(\({\mathcal B}\)) is linear. In other words,
\begin{eqnarray}
  {\mathbf q}^+
      = {\mathbf q}^-
      + 2 \Delta t {\mathcal A}( {\mathbf q}  )
      + 2 \Delta t B (   {\mathbf q}^+
                       + {\mathbf q}^-   )/2,
\end{eqnarray}
where (\({\mathcal B}\)) is a square matrix. Defining
\(\Delta {\mathbf q} \equiv {\mathbf q}^+ - {\mathbf q}\), we get
\begin{eqnarray}
  ( I - \Delta t B ) \Delta {\mathbf q}
      = 2 \Delta t \left( {\mathcal A}({\mathbf q})
                         + B {\mathbf q} \right).
\end{eqnarray}
This can be easily solved by matrix operations.

\hypertarget{applying-the-semi-implicit-time-integration}{%
\subsubsection{Applying the Semi-Implicit Time
Integration}\label{applying-the-semi-implicit-time-integration}}

Here, we apply the semi-implicit method and treat terms associated with
linear gravity waves as implicit, which allows us to increase the time
step \(\Delta t\).

We divide the basic equation into a linear gravity wave term
(\(T=\bar{T}_k\)) with a static field as the basic field and other terms
(with the indices \(NG\)). Using a vector representation for the
vertical direction (\({\mathbf{D}}=\{ D_{k} \}\) and
\({\mathbf{T}}=\{ T_{k} \}\)),
\begin{eqnarray}
   \frac{\partial \pi}{\partial t} &=&
          \left( \frac{\partial \pi}{\partial t} \right)_{NG}
     - {\mathbf{C}} \cdot {\mathbf{D}}, \\
  \frac{\partial {\mathbf{D}}}{\partial t} &=&
          \left( \frac{\partial {\mathbf{D}}}{\partial t} \right)_{NG}
          - \nabla^{2}_{\eta} ( {\mathbf{\Phi}}_{S}
                                  + \underline{W} {\mathbf{T}}
                                  + {\mathbf{G}} \pi )
          - {\mathcal D}_M {\mathbf{D}} , \\
  \frac{\partial {\mathbf{T}}}{\partial t}
      &=&   \left( \frac{\partial {\mathbf{T}}}
                        {\partial t}       \right)_{NG}
         - \underline{h} {\mathbf{D}}
         - {\mathcal D}_H {\mathbf{T}}.
\end{eqnarray}

Here, the non-gravitational wave term is
\begin{eqnarray}
  \left( \frac{\partial \pi}{\partial t} \right)^{NG}
   &=&   - \sum_{k=1}^{K} {\mathbf{v}}_{k} \cdot \nabla \pi
       \Delta B_{k}, \\
  \frac{(m\dot{\eta})^{NG}_{k-1/2}}{p_s}
 &=& - B_{k-1/2} \left( \frac{\partial \pi}{\partial t} \right)^{NG}
   - \sum_{l=k}^{K} {\mathbf{v}}_{l} \cdot \nabla \pi
       \Delta B_{l}, \\
  \left( \frac{\partial D}{\partial t} \right)^{NG}
       &=&   \frac{1}{a\cos\varphi}
            \frac{\partial (A_u)_{k}}{\partial \lambda}
          + \frac{1}{a\cos\varphi}
            \frac{\partial }{\partial \varphi} (A_v \cos\varphi)_k
          - \nabla^{2}_{\eta} \hat{E}_{k}
          - {\mathcal D}(D_{k}), \\
  \left( \frac{\partial T_{k}}{\partial t} \right)^{NG}
      &=&   - \frac{1}{a\cos\varphi}
               \frac{\partial u_k T'_k}{\partial \lambda}
          - \frac{1}{a\cos\varphi}
               \frac{\partial }{\partial \varphi} (v_k T'_k \cos\varphi)
          + \hat{H}_{k}
          - {\mathcal D}(T_{k}), \\
 \hat{H}_k  &=&  T_{k}^{\prime} D_{k} - \left[   \frac{(m\dot{\eta})_{k-1/2}}{p_s} \frac{\hat{T}_{k-1/2} - T_k}{\Delta\sigma_k}
               + \frac{(m\dot{\eta})_{k+1/2}}{p_s} \frac{T_k - \hat{T}_{k+1/2}}{\Delta\sigma_k} \right] \\
         &+& \hat{\kappa}_{k} T_{v,k} {\mathbf{v}}_{k} \cdot \nabla \pi \\
         &-& \frac{\alpha_{k}}{\Delta \sigma_{k} } T_{v,k}
             \sum_{l=k}^{K} {\mathbf{v}}_{l} \cdot \nabla \pi
               \Delta B_{l}
           - \frac{\beta_{k}}{\Delta \sigma_{k} } T_{v,k}
             \sum_{l=k+1}^{K} {\mathbf{v}}_{l} \cdot \nabla \pi
               \Delta B_{l} \\
        &-& \frac{\alpha_{k}}{\Delta \sigma_{k} } T'_{v,k}
             \sum_{l=k}^{K} D_l  \Delta \sigma_{l}
           - \frac{\beta_{k}}{\Delta \sigma_{k} } T'_{v,k}
             \sum_{l=k+1}^{K} D_l  \Delta \sigma_{l}
         + \frac{Q_k + (Q_{diff})_k}{C_p}, \\
  \hat{E}_k &=& E_{k}
            + \sum_{k=1}^{K} W_{kl} ( T_{v,l}-T_{l} ),
\end{eqnarray}
where the vector and matrix of the gravitational wave term (underlined)
are
\begin{eqnarray}
  C_{k} &=& \Delta \sigma_{k}, \\
  W_{kl} &=& C_{p} \alpha_{l} \delta_{k \geq l}
         + C_{p} \beta_{l} \delta_{k-1 \geq l}, \\
  G_{k} &=& R\bar{T}, \\
h_{kl} &=& \frac{\bar{T}}{\Delta\sigma_k}\left[\alpha_k \Delta\sigma_l \delta_{k\ge l}+\beta_k \Delta\sigma_l \delta_{k+1\le l}\right].
\end{eqnarray}
Here, \(\delta_{k \leq l}\) is 1 if \(k \leq l\) is valid and 0
otherwise.\\

We now use the following expressions for time differences:
\begin{eqnarray}
  \delta_{t} {X} &\equiv & \frac{1}{2 \Delta t}
        \left( {X}^{t+\Delta t} - {X}^{t-\Delta t} \right), \\
    \overline{X}^{t} &\equiv & \frac{1}{2} \left( {X}^{t+\Delta t}  + {X}^{t-\Delta t} \right) \\
  &=&  {X}^{t-\Delta t} + \delta_{t} {X} \Delta t.
\end{eqnarray}
Then, applying the semi-implicit method to the system of equations, we get
\begin{eqnarray}
\label{eqn_for_pi}
  \delta_{t} \pi &=&
          \left( \frac{\partial \pi}{\partial t} \right)_{NG}
     - {\mathbf{C}} \cdot \overline{ {\mathbf{D}} }^{t}, \\
  \delta_{t} {\mathbf{D}} &=&
          \left( \frac{\partial {\mathbf{D}}}{\partial t} \right)_{NG}
          - \nabla^{2}_{\eta} ( {\mathbf{\Phi}}_{S}
                                  + \underline{W}
                                     \overline{ {\mathbf{T}} }^{t}
                                  + {\mathbf{G}}
                                  \overline{\pi}^{t} )
          - {\mathcal D}_M ( {\mathbf{D}}^{t-\Delta t}
                         + 2 \Delta t \delta_{t} {\mathbf{D}} ), \\
\label{eqn_for_t}
  \delta_{t} {\mathbf{T}} &=&
        \left( \frac{\partial {\mathbf{T}}}{\partial t} \right)_{NG}
         - \underline{h} \overline{ {\mathbf{D}} }^{t}
         - {\mathcal D}_H ( {\mathbf{T}}^{t-\Delta t}
                        + 2 \Delta t \delta_{t} {\mathbf{T}} ).
\end{eqnarray}

Thus,
\begin{eqnarray}
      & &\left\{ ( 1+2\Delta t {\mathcal D}_H )( 1+2\Delta t {\mathcal D}_M )
           \underline{I}
      - ( \Delta t )^{2}  ( \underline{W} \ \underline{h}
           + (1+2\Delta t {\mathcal D}_M)
             {\mathbf{G}} {\mathbf{C}}^{T} ) \nabla^{2}_{\eta}
  \right\}
      \overline{ {\mathbf{D}} }^{t} \\
  &=& ( 1+2\Delta t {\mathcal D}_H )( 1+\Delta t {\mathcal D}_M )
       {\mathbf{D}}^{t-\Delta t}
  + \Delta t
     \left( \frac{\partial {\mathbf{D}}}{\partial t} \right)_{NG}  \\
  &-&  \Delta t \nabla^{2}_{\eta}
                   \left\{  ( 1+2\Delta t {\mathcal D}_H ) {\mathbf{\Phi}}_{S}
                          + \underline{W}
                            \left[ ( 1+2\Delta t {\mathcal D}_H )
                                    {\mathbf{T}}^{t-\Delta t}
                                  + \Delta t
                                      \left( \frac{\partial {\mathbf{T}}}
                                                  {\partial t}
                                      \right)_{NG} \right]
                   \right. \\
                 &+& \hspace{1.4cm} \left. ( 1+2\Delta t {\mathcal D}_H ) {\mathbf{G}}
                            \left[ \pi^{t-\Delta t}
                                  + \Delta t \left( \frac{\partial \pi}{\partial t}
                                     \right)_{NG}  \right]
                   \right\} .
\end{eqnarray}

Since the spherical harmonic expansion is used, we can rewrite
\(\nabla_{\eta}^2\) as the following:
\begin{eqnarray}
\nabla_{\eta}^2=-\frac{n(n+1)}{a^2},
\end{eqnarray}
which enables us to solve the above equations for
\(\overline{ {\mathbf{D}}_n^m }^{t}\). Then, using (\ref{eqn_for_pi}),
(\ref{eqn_for_t}) and
\(D^{t+\Delta t} = 2\overline{ {\mathbf{D}} }^{t} - D^{t-\Delta t},\)
we can obtain the value of prognostic variables \(\hat{X}^{t+\Delta t}\)
at \(t+\Delta t\).

\hypertarget{time-scheme-properties-and-requiments-for-time-steps}{%
\subsubsection{Time Scheme Properties and Requiments for Time
Steps}\label{time-scheme-properties-and-requiments-for-time-steps}}

Let us consider solving the advection equation with the leap-frog
method:
\begin{eqnarray}
  \frac{\partial{X}}{\partial {t}} = c \frac{\partial{X}}{\partial {x}}.
\end{eqnarray}

Assuming \(X = X_0 \exp(ikx)\), the descretized form of the above
equation becomes:
\begin{eqnarray}
  X^{n+1} = X^{n-1} + 2 i k \Delta t X^n.
\end{eqnarray}
Assuming \(X\) evolves exponentially, we can define \(\lambda\) such
that
\begin{eqnarray}
  \lambda &=& X^{n+1}/X^n = X^n/X^{n-1}, \\
  \lambda^2 &=& 1 + 2 i kc \Delta t \lambda \; .
\end{eqnarray}
Defining \(p \equiv kc \Delta t\), the solution becomes:
\begin{eqnarray}
 \lambda = -i p \pm \sqrt{1-p^2}.
\end{eqnarray}

The absolute value of those solutions are
\begin{eqnarray}
  |\lambda| = \left\{
             \begin{array}{ll}
               1 & |p| \le 1 \\
               p \pm \sqrt{p^2-1} & |p| > 1
             \end{array}
             \right.
\end{eqnarray}
and in the case of \(|p|>1\), we get \(|\lambda| > 1\), and the absolute
value of the solution increases exponentially with time. This indicates
that the computation is unstable.

In the case of \(|p| \le 1\), however, the calculation is neutral since
the value of \(|\lambda| = 1\). However, there are two solutions to
\(\lambda\), one of which, when set to \(\Delta t \rightarrow 1\), leads
to \(\lambda \rightarrow 1\), while the other leads to
\(\lambda \rightarrow -1\), which indicates an osscilating solution.
This mode is called ``computational mode'' and is one of the problems of
the leap frog method. This mode can be damped by applying a time filter
described later.

Given the horizontal grid spacing \(\Delta x\), the maximum value of
\(k\) becomes
\begin{eqnarray}
  \max k = \frac{\pi}{\Delta x}.
\end{eqnarray}
Then, the condition \(|p|=kc \Delta t \le 1\) requires
\begin{eqnarray}
   \Delta t \le \frac{\Delta x}{\pi c}.
\end{eqnarray}

In case of a spectral model, using the Earth's radius \(a\) and the
maximum wavenumber \(N\), the requirement becomes
\begin{eqnarray}
   \Delta t \le \frac{a}{N c},
\end{eqnarray}
which is a condition for the numerical stability.\\

To guarantee the stability of the integration, one needs to take the time step  \(\Delta t\) smaller than that required by the fastest-propagating mode.
If the semi-implicit scheme is not used, the propagation speed of gravity waves, which can be as fast as 300 m$/$s, sets the criterion for stability.
With the gravity waves taken account of by the semi-implicit method, however, the fastest mode usually becomes the maximum easterly wind $U_{\mathrm{max}}$. Therefore,
\begin{eqnarray}
   \Delta t \le \frac{a}{N U_{max}} .
\end{eqnarray}
In practice, this is multiplied by a factor smaller than 1 for further safety.

\hypertarget{handling-of-the-initiation-of-time-integration}{%
\subsubsection{Handling of the Initiation of Time
Integration}\label{handling-of-the-initiation-of-time-integration}}

When starting from an initial condition that is not calculated by
MIROC 6.0 itself, it is not possible to give values of all prognostic variables at two time steps \(t\)
and \(t-\Delta t\) consistently with the model dynamics.
However, giving inconsistent values for \(t-\Delta t\) results in a large computation mode.

To avoid this, a special procedure is followed at the initiation of time integration.
Firstly, assuming \(X^{\Delta t/4} = X^0\), a \(1/4\)-step integration is performed to obtain $X^{\Delta t/2}$:
\begin{eqnarray}
  X^{\Delta t/2} = X^0 + \Delta t/2 \dot{X}^{\Delta t/4}
                 = X^0 + \Delta t/2 \dot{X}^0.
\end{eqnarray}
Then, a \(1/2\)-step integration is performed to yield $X^{\Delta t}$:
\begin{eqnarray}
  X^{\Delta t}   = X^0 + \Delta t \dot{X}^{\Delta t/2}.
\end{eqnarray}
Finally, in the normal time step,
\begin{eqnarray}
  X^{2\Delta t}   = X^0 + 2 \Delta t \dot{X}^{\Delta t}.
\end{eqnarray}
From here on, the leap-frog method is executed in the usual manner.
