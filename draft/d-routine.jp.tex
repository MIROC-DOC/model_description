\hypertarget{ux529bux5b66ux904eux7a0bux306bux304aux3051ux308bux8a08ux7b97ux30d5ux30edux30fc}{%
\subsection{力学過程における計算フロー}\label{ux529bux5b66ux904eux7a0bux306bux304aux3051ux308bux8a08ux7b97ux30d5ux30edux30fc}}

ここでは, 力学過程部で行なわれる計算をコードに基づいて列挙する.

\hypertarget{ux529bux5b66ux90e8ux5206ux306eux8a08ux7b97ux306eux6982ux8981}{%
\subsubsection{力学部分の計算の概要}\label{ux529bux5b66ux90e8ux5206ux306eux8a08ux7b97ux306eux6982ux8981}}

力学過程は, 以下のような順序で計算が行なわれる.

\begin{enumerate}
\def\labelenumi{\arabic{enumi}.}
\item
  力学項の計算 \texttt{{[}DYNTRM(dterm.F){]}}

  1.1 波数空間で渦度・発散などの計算、格子点値の渦度等を取得
  \texttt{{[}G2W,\ W2G(xdsphe.F){]}}

  1.2 流線関数、速度ポテンシャルの診断 \texttt{{[}DYNTRM(dterm.F){]}}

  1.3 地表面気圧の移流項、傾向、鉛直速度の計算
  \texttt{{[}PSDOT(dgdyn.F){]}}

  1.4 ハイブリット座標における静水圧平衡の式、気温の内挿の係数計算
  \texttt{{[}CFACT(dcfct.F){]}}

  1.5 仮温度の計算 \texttt{{[}VIRTMD(dvtmp.F){]}}

  1.6 気温の移流項の計算 \texttt{{[}GRTADV(dgdyn.F){]}}

  1.7. 運動量移流項の計算 \texttt{{[}GRUADV(dgdyn.F){]}}

  1.8. 時間変化項のスペクトル変換 \texttt{{[}G2W(xdsphe.F){]}}
\item
  方程式の時間積分 \texttt{DYNSTP(dstep.F)}

  2.1 トレーサー輸送の計算 \texttt{{[}TRACEG(dtrcr.F){]}}

  2.2 スペクトル値時間積分 \texttt{{[}TINTGR(dintg.F){]}}

  2.3 トレーサー項の時間積分 \texttt{{[}GTRACE(dtrcr.F){]}}

  2.4 時間フィルター \texttt{{[}DADVNC(dadvn.F){]}}

  2.5 スペクトル空間から格子点値のu, v を取得
  \texttt{{[}W2G(xdsphe.F){]}}

  2.6 疑似等p面拡散補正 \texttt{{[}CORDIF(ddifc.F){]}}

  2.7 拡散による摩擦熱の補正 \texttt{{[}CORFRC(ddifc.F){]}}
\end{enumerate}
