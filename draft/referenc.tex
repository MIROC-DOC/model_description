\section{References}
\begin{enumerate}
\setlength{\parskip}{0pt} % 段落の間隔
\setlength{\itemsep}{5pt} % 項目の間隔
\setlength{\leftskip}{0pt} %左からの頭の位置
\item Abdul-Razzak, H., and Ghan, S. J. (2000). A parameterization of aerosol activation: 2. Multiple aerosol types. Journal of Geophysical Research: Atmospheres. \\
\texttt{https://doi.org/10.1029/1999jd901161}
\item Arakawa, A., , and W. H. Schubert, (1974). Interaction of a cumulus cloud ensemble with the large-scale environment, Part I. Journal of Atmospheric Science, 31 , 674–701.
\item Arakawa, A., , and K.-M. Xu, (1990). The macroscopic behavior of simulated cumulus convection and semiprognostic tests of the Arakawa–Schubert cumulus parameterization. Proc. Indo-U.S. Seminar on Parameterization of Subgrid-Scale Processes in Dynamical Models of Medium-Range Prediction and Global Climate, Pune, India, Indian Institute of Tropical Meteorology, 3–18.
\item Arakawa, A., and K.-M. Xu, (1992). The macroscopic behavior of simulated cumulus convection and semiprognostic tests of the Arakawa-Schubert cumulus parameterization. Physical Process in Atmospheric Models., 3–18.
\item Arakawa, A., and Konor, C. S. (1996). Vertical Differencing of the Primitive Equations Based on the Charney–Phillips Grid in Hybrid andsigma–p Vertical Coordinates. Monthly Weather Review, 124(3), 511–528.
\item Asselin, R. (1972). Frequency Filter for Time Integrations. Monthly Weather Review, 100(6), 487–490.
\item Berry, E. X. (1968). Modification of the warm rain process. In Proc. First Conf. on Weather Modification, AMS, 1968, 81-85.
\item Berger, A. (1978). Long-term variations of caloric insolation resulting from the Earth's orbital elements. Quaternary research, 9(2), 139-167.
\item Bourke, W. (1988). Spectral Methods in Global Climate and Weather Prediction Models. In M. E. Schlesinger (Ed.), Physically-Based Modelling and Simulation of Climate and Climatic Change: Part 1 (pp. 169–220). Dordrecht: Springer Netherlands.
\item Bretherton, C. S., McCaa, J. R., and Grenier, H. (2004). A new parameterization for shallow cumulus convection and its application to marine subtropical cloud-topped boundary layers. Part I: Description and 1D results. Monthly Weather Review, 132(4), 864–882.
\item Bushell, A. C., Wilson, D. R., and Gregory, D. (2003). A description of cloud production by non-uniformly distributed processes. Quarterly Journal of the Royal Meteorological Society, 129(590), 1435–1455.
\item Businger, J. A., Wyngaard, J. C., Izumi, Y., and Bradley, E. F. (1971). Flux-Profile Relationships in the Atmospheric Surface Layer. Journal of the Atmospheric Sciences, 28(2), 181–189.
\item Cesana, G., Waliser, D. E., Jiang, X., and Li, J. ‐L F. (2015). Multimodel evaluation of cloud phase transition using satellite and reanalysis data. Journal of Geophysical Research, 120(15), 7871–7892.
\item Chikira, M., and Sugiyama, M. (2010). A Cumulus Parameterization with State-Dependent Entrainment Rate. Part I: Description and Sensitivity to Temperature and Humidity Profiles. Journal of the Atmospheric Sciences, 67(7), 2171–2193.
\item Colella, P., and Woodward, P. R. (1984). The Piecewise Parabolic Method (PPM) for gas-dynamical simulations. Journal of Computational Physics. \\
\texttt{https://doi.org/10.1016/0021-9991(84)90143-8}
\item Diehl, K., Simmel, M., and Wurzler, S. (2006). Numerical sensitivity studies on the impact of aerosol properties and drop freezing modes on the glaciation, microphysics, and dynamics of clouds. Journal of Geophysical Research, 111(D7). \\
\texttt{https://doi.org/10.1029/2005jd005884}
\item Gregory, D. (2001). Estimation of entrainment rate in simple models of convective clouds. Quarterly Journal of the Royal Meteorological Society, 127(571), 53–72.
\item Gregory, D., Kershaw, R., and Inness, P. M. (1997). Parametrization of momentum transport by convection. II: Tests in single-column and general circulation models. Quarterly Journal of the Royal Meteorological Society, 123(541), 1153–1183.
\item Haltiner, G. J., and Williams, R. T. (1980). Numerical prediction and dynamic meteorology (No. 551.5 HAL ). sidalc.net. Retrieved from \\
\texttt{http://www.sidalc.net/cgi-bin/wxis.exe/?IsisScript=FCL.xisandmethod=\\
postandformato=2andcantidad=1andexpresion=mfn=004222}
\item Hasumi, H. (2015). CCSR ocean component model (COCO) (Version 4.0) (p. 68). Atmosphere and Ocean Research Institute, The University of Tokyo. Retrieved from \\
\texttt{https://ccsr.aori.u-tokyo.ac.jp/~hasumi/COCO/coco4.pdf}
\item Helfand, H. M., and Labraga, J. C. (1988). Design of a Nonsingular Level 2.5 Second-Order Closure Model for the Prediction of Atmospheric Turbulence. Journal of the Atmospheric Sciences, 45(2), 113–132.
\item Hines, C. O. (1997) Doppler-spread parameterization of gravity wave momentum deposition in the middle atmosphere, Part 2: Broad and quasi monochromatic spectra, and implementation, Journal of Atmosphere Sol.-Terr. Physics, 59, 387–400.
\item Holtslag, A. A. M., and Boville, B. A. (1993). Local Versus Nonlocal Boundary-Layer Diffusion in a Global Climate Model. Journal of Climate, 6(10), 1825–1842.
\item K-1 model developers (2004). K-1 coupled GCM (MIROC) description, K-1 Tech. Rep., 1, edited by: Hasumi, H. and Emori, S., Center for Climate System Research, the Univ. of Tokyo, Tokyo, 34 pp.
\item Kain, J. S., and Michael Fritsch, J. (1990). A One-Dimensional Entraining/Detraining Plume Model and Its Application in Convective Parameterization. Journal of the Atmospheric Sciences, 47(23), 2784–2802.
\item Lew, J. K., , D. C. Montague, , and H. R. Pruppacher, (1986) A wind tunnel investigation on the riming of snowflakes. Part I: Porous disks and large stellars. Journal of Atmospheric Science, 43, 2392–2409.
\item Lin, S.-J., and Rood, R. B. (1996). Multidimensional Flux-Form Semi-Lagrangian Transport Schemes. Monthly Weather Review, 124(9), 2046–2070.
\item Lohmann, U. (2004). Can anthropogenic aerosols decrease the snowfall rate?. Journal of the atmospheric sciences, 61(20), 2457-2468.
\item Lohmann, U., and Diehl, K. (2006). Sensitivity studies of the importance of dust ice nuclei for the indirect aerosol effect on stratiform mixed-phase clouds. Journal of the Atmospheric Sciences, 63(3), 968–982.
\item Louis, J., Tiedtke, M., and Geleyn, J. (1982). A short history of the PBL parameterization at ECMWF, paper presented at Workshop on Planetary Boundary Layer Parameterization, Eur. Cent. for Medium Range Weather Forecasts. Reading, UK.
\item Louis, J.-F. (1979). A parametric model of vertical eddy fluxes in the atmosphere. Boundary-Layer Meteorology, 17(2), 187–202.
\item McFarlane, N. (1987). The Effect of Orographically Excited Gravity Wave Drag on the General Circulation of the Lower Stratosphere and Troposphere. Journal of the Atmospheric Sciences, 44(14), 1775-1800.
\item Mellor, G. L. (1973). Analytic Prediction of the Properties of Stratified Planetary Surface Layers. Journal of the Atmospheric Sciences, 30(6), 1061–1069.
\item Mellor, G. L., and Yamada, T. (1974). A Hierarchy of Turbulence Closure Models for Planetary Boundary Layers. Journal of the Atmospheric Sciences, 31(7), 1791–1806.
\item Mellor, G. L., and Yamada, T. (1982). Development of a turbulence closure model for geophysical fluid problems. Reviews of Geophysics, 20(4), 851-875.
\item Mesinger, F., and Arakawa, A. (1976). Numerical methods used in atmospheric models (GARP Publications Series) (p. 64). CAU. Retrieved from \\
\texttt{http://eprints.uni-kiel.de/40278/}
\item Miller, M.J., A.C.M. Beljaars and T.N. Palmer, (1992). The sensitivity of the ECMWF model to the parameterization of evaporation from the tropical oceans. Journal of Climate, 5, 418-434.
\item Miura, H. (2002). Vertical differencing of the primitive equations in a $\sigma - p$ hybrid coordinate (For Spectral AGCM)) (Version DRAFT). Center for Climate System Research,University of Tokyo.
\item Nakajima, T., Tsukamoto, M., Tsushima, Y., Numaguti, A., and Kimura, T. (2000). Modeling of the radiative process in an atmospheric general circulation model. Applied Optics, 39(27), 4869–4878.
\item Nakanishi, M. (2001). IMPROVEMENT OF THE MELLOR–YAMADA TURBULENCE CLOSURE MODEL BASED ON LARGE-EDDY SIMULATION DATA. Boundary-Layer Meteorology, 99, 349–378.
\item Nakanishi, M., , and H. Niino, (2004). An improved Mellor–Yamada level-3 model with condensation physics: Its design and verification. Boundary-Layer Meteor, 112 , 1–31.
\item Nitta, T., Yoshimura, K., Takata, K., O’ishi, R., Sueyoshi, T., Kanae, S., et al. (2014). Representing Variability in Subgrid Snow Cover and Snow Depth in a Global Land Model: Offline Validation. Journal of Climate, 27(9), 3318–3330.
\item Ogura, T. (2015). Implementation of shallow cumulus Parameterization. National Institute of Environmental Studies.
\item Ogura, T., Shiogama, H., Watanabe, M., Yoshimori, M., Yokohata, T., Annan, J. D., Hargreaves, J. C., Ushigami, N., Hirota, K., Someya, Y., Kamae, Y., Tatebe, H., and Kimoto, M. (2017). Effectiveness and limitations of parameter tuning in reducing biases of top-of-atmosphere radiation and clouds in MIROC version 5, Geoscience Model Development, 10, 4647–4664
\item Pan, D.-M. (1995, January 1). Development and Application of a Prognostic Cumulus Parametrization. ui.adsabs.harvard.edu. Retrieved from \\
\texttt{https://ui.adsabs.harvard.edu/abs/1995PhDT........74P}
\item Pan, D-M., , and D. A. Randall, 1998: A cumulus parameterization with a prognostic closure. Quarterly Journal of the Royal Meteorological Society, 124 , 949–981.
\item Park, S., and Bretherton, C. S. (2009). The University of Washington Shallow Convection and Moist Turbulence Schemes and Their Impact on Climate Simulations with the Community Atmosphere Model. Journal of Climate, 22(12), 3449–3469.
\item Pruppacher, H. R., , and J. D. Klett, (1997). Microphysics of Clouds and Precipitation. Kluwer Academic, 954 pp.
\item Randall, D. A. (1980). Conditional Instability of the First Kind Upside-Down. Journal of the Atmospheric Sciences, 37(1), 125–130.
\item Randall, D. A., and Pan, D.-M. (1993). Implementation of the Arakawa-Schubert Cumulus Parameterization with a Prognostic Closure. In K. A. Emanuel and D. J. Raymond (Eds.), The Representation of Cumulus Convection in Numerical Models (pp. 137–144). Boston, MA: American Meteorological Society.
\item Randall, D. A., , D-M. Pan, , P. Ding, , and D. G. Cripe, (1997). Quasi-equilibrium. The Physics and Parameterization of Moist Atmospheric Convection, R. K. Smith, Ed., Kluwer Academic, 359–386.
\item Sekiguchi, M., and Nakajima, T. (2008). A k-distribution-based radiation code and its computational optimization for an atmospheric general circulation model. Journal of Quantitative Spectroscopy and Radiative Transfer, 109(17), 2779–2793.
vSmith, R. K. (2013). The Physics and Parameterization of Moist Atmospheric Convection. Springer Science and Business Media.
\item Sommeria, G., and Deardorff, J. W. (1977). Subgrid-Scale Condensation in Models of Nonprecipitating Clouds. Journal of the Atmospheric Sciences, 34(2), 344–355.
\item Stevens, B. (2005). ATMOSPHERIC MOIST CONVECTION. Annual Review of Earth and Planetary Sciences, 33(1), 605–643.
\item Suzuki, T., Saito, F., Nishimura, T., and Ogochi, K. (2009). Coupling procedures of heat and freshwater fluxes in the MIROC (Model for Interdisciplinary Research on Climate) version 4. JAMSTEC report of research and development
\item Takata, K., Emori, S., and Watanabe, T. (2003). Development of the minimal advanced treatments of surface interaction and runoff. Global and Planetary Change, 38(1), 209–222.
\item Takemura, T. (2005). Simulation of climate response to aerosol direct and indirect effects with aerosol transport-radiation model. Journal of Geophysical Research, 110(D2). \\
\texttt{https://doi.org/10.1029/2004jd005029}
\item Takemura, T., Okamoto, H., Maruyama, Y., Numaguti, A., Higurashi, A., and Nakajima, T. (2000). Global three-dimensional simulation of aerosol optical thickness distribution of various origins. Journal of Geophysical Research, 105(D14), 17853–17873.
\item Takemura, T., Nakajima, T., Dubovik, O., Holben, B. N., and Kinne, S. (2002). Single-scattering albedo and radiative forcing of various aerosol species with a global three-dimensional model. Journal of Climate, 15(4), 333–352.
\item Takemura, T., Egashira, M., Matsuzawa, K., Ichijo, H., O’ishi, R., and Abe-Ouchi, A. (2009). A simulation of the global distribution and radiative forcing of soil dust aerosols at the Last Glacial Maximum. Atmospheric Chemistry and Physics, 9(9), 3061–3073.
\item Tatebe, H., Ogura, T., Nitta, T., Komuro, Y., Ogochi, K., Takemura, T., et al. (2019). Description and basic evaluation of simulated mean state, internal variability, and climate sensitivity in MIROC6. Geoscientific Model Development, 12(7), 2727–2765.
\item Tompkins, A. M. (2002). A prognostic parameterization for the subgrid-scale variability of water vapor and clouds in large-scale models and its use to diagnose cloud cover. Journal of the Atmospheric Sciences, 59(12), 1917–1942.
\item Watanabe, M., Emori, S., Satoh, M., and Miura, H. (2009). A PDF-based hybrid prognostic cloud scheme for general circulation models. Climate Dynamics, 33(6), 795–816.
\item Watanabe, S. (2008). Constraints on a Non-orographic Gravity Wave Drag Parameterization Using a Gravity Wave Resolving General Circulation Model, SOLA, 4, 61-64.
   Watanabe, S., Hajima, T., Sudo, K., Nagashima, T., Takemura, T., Okajima, H., Nozawa, T., Kawase, H., Abe, M., Yokohata, T., Ise, T., Sato, H., Kato, E., Takata, K., Emori, S., and Kawamiya, M. (2011) MIROC-ESM 2010: model description and basic results of CMIP5-20c3m experiments, Geoscience Model Development, 4, 845-872.
\item Williams, P. D. (2009). A Proposed Modification to the Robert–Asselin Time Filter. Monthly Weather Review, 137(8), 2538–2546.
\item Wilson, D. R., and Ballard, S. P. (1999). A microphysically based precipitation scheme for the UK meteorological office unified model. Quarterly Journal of the Royal Meteorological Society, 125(557), 1607-1636.
\item Wood, R., and Bretherton, C. S. (2006). On the relationship between stratiform low cloud cover and lower-tropospheric stability. Journal of Climate, 19(24), 6425-6432.
\item Xu, K.-M. (1991). The coupling of cumulus convection with large-scale processes (Ph.D.). University of California. Retrieved from \\
\texttt{https://elibrary.ru/item.asp?id=5851863}
\item Xu, K.-M., (1993). Cumulus ensemble simulation. The Representation of Cumulus Convection in Numerical Models, Meteor. Monogr., American Meteorological Society,46, 221-235.
\end{enumerate}
