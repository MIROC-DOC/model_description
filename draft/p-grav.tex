\hypertarget{gravitational-wave-resistance}{%
\subsection{Gravitational wave
resistance}\label{gravitational-wave-resistance}}

\textbf{NOTE: the descriptions in this section are outdated.}

\hypertarget{gravitational-wave-resistance-scheme-overview}{%
\subsubsection{Gravitational Wave Resistance Scheme
Overview}\label{gravitational-wave-resistance-scheme-overview}}

The gravitational wave resistance scheme represents the upward momentum
flux of the gravitational waves induced by sub-grid scale topography and
calculates the horizontal wind deceleration associated with its
convergence. The main input data are east-west wind (\(u\)), north-south
wind (\(v\)), and temperature (\(T\)), and the output data are the rates
of temporal variation of east-west wind and north-south wind,
\(\partial u/\partial t, \partial v/\partial t\).

The outline of the calculation procedure is as follows.

The momentum flux at the ground surface is calculated from the
dispersion of surface height, the horizontal wind speed at the lowest
level, and the stratification stability.

We consider the upward propagation of gravitational waves with momentum
fluxes. If the momentum flux exceeds the critical fluid number, the
momentum flux is determined by the critical fluid number, then breaking
waves occur and the flux becomes the critical value of the momentum
flux.

\begin{enumerate}
\def\labelenumi{\arabic{enumi}.}
\setcounter{enumi}{2}
\tightlist
\item
  compute the time evolution of the horizontal wind as the momentum flux
  converges in each layer.
\end{enumerate}

\hypertarget{relationship-between-local-fluid-number-and-momentum-flux}{%
\subsubsection{Relationship between local fluid number and momentum
flux}\label{relationship-between-local-fluid-number-and-momentum-flux}}

Considering the vertical flux of horizontal momentum due to
surface-derived gravitational waves, the difference between the flux
(\(\tau\)) and the local fluid number (\(F_L = NH/U\)) at a certain
altitude is

\begin{eqnarray}
   F_L = \left(
            \frac{\tau N}{E_f \rho U^3}
           \right)^{1/2} \; ,
\end{eqnarray}

This relationship holds for the following cases where
\(N = g/\theta \partial \theta/\partial z\) is the Brant-Visala
frequency, \(\rho\) is the density of the atmosphere, \(U\) is the wind
speed, and \(E_f\) is the proportional constant corresponding to the
horizontal scale of the rippling at the surface. From now on,

\begin{eqnarray}
  \tau = \frac{E_f F_L^2 \rho U^3}{N}
\end{eqnarray}

The local fluid number (\(F_L\)) cannot exceed the critical fluid number
(\(F_{c}\)) at a certain value. If the local fluid number calculated
from (589) exceeds the critical fluid number \(F_{c}\), the
gravitational wave becomes supersaturated and the flux decreases to the
momentum flux corresponding to the critical fluid number.

\hypertarget{momentum-fluxes-at-the-surface.}{%
\subsubsection{Momentum fluxes at the
surface.}\label{momentum-fluxes-at-the-surface.}}

The magnitude of the vertical flux of horizontal momentum due to
gravitational waves excited at the earth's surface, \(\tau_{1/2}\), is
calculated by substituting the local fluid number
\((F_L)_{1/2} = N_1 h/U_1\) into (590),

\begin{eqnarray}
  \tau_{1/2} = E_f h^2 \rho_1 N_1 U_1 \; ,
\end{eqnarray}

where \(U_1 = |{\mathbf v}_1| = (u_1^2 + v_1^2)^{1/2}\) is the surface
wind speed, \(N_1, \rho_1\) are estimated to be the stability and
density of the atmosphere near the earth's surface, respectively. where
\(U_1 = |{\mathbf v}_1| = (u_1^2 + v_1^2)^{1/2}\) is the surface wind
speed, and \(N_1, \rho_1\) are the stability and density of the
atmosphere near the earth's surface, respectively. \(h\) is an indicator
of the change in the surface height of the sub-grid and is assumed to be
equal to the standard deviation of the surface height (\(Z_{SD}\)).

Here, when the local fluid number (\((F_L)_{1/2} = N_1 Z_{SD}/U_1\))
exceeds the critical fluid number (\(F_c\)), the momentum flux is
suppressed to the value obtained by substituting (590) for \(F_c\). In
other words,

\begin{eqnarray}
  \tau_{1/2} = \min \left(
                   E_f Z_{SD}^{2} \rho_1 N_1 U_1, \;
                  \frac{E_f F_c^{2} \rho_1 U_1^3}{N_1}
               \right)
\end{eqnarray}

\hypertarget{momentum-fluxes-in-the-upper-levels.}{%
\subsubsection{Momentum fluxes in the upper
levels.}\label{momentum-fluxes-in-the-upper-levels.}}

Suppose that the momentum flux \(\tau_{k-1/2}\) is required for level
\(k-1/2\). When no saturation occurs, \(\tau_{k+1/2}\) is equal to
\(\tau_{k-1/2}\). If the momentum flux (\(\tau_{k-1/2}\)) exceeds the
momentum flux calculated from the critical fluid number at the \(k+1/2\)
level, wave breaking occurs in the \(k\) layer and the momentum flux
decreases to the critical flux.

\begin{eqnarray}
  \tau_{k+1/2} = \min \left(
               \tau_{k-1/2}, \;
               \frac{E_f F_c^2 \rho_{k+1/2} U_{k+1/2}^3}{N_{k+1/2}}
                      \right),
\end{eqnarray}

Note that \(U_{k+1/2}\) is the magnitude of the projective component of
the wind speed vectors for each layer relative to the direction of the
lowest level of the horizontal wind,

\begin{eqnarray}
  U_{k+1/2} = \frac{{\mathbf v}_{k+1/2}
                      \cdot {\mathbf v}_{1}}
                   {|{\mathbf v}_{1}|       }
\end{eqnarray}

\hypertarget{the-magnitude-of-the-time-variation-of-horizontal-wind-due-to-momentum-convergence.}{%
\subsubsection{The magnitude of the time variation of horizontal wind
due to momentum
convergence.}\label{the-magnitude-of-the-time-variation-of-horizontal-wind-due-to-momentum-convergence.}}

The temporal rate of change of the projective component of the
horizontal wind, \(U_{k}\), is ,

\begin{eqnarray}
  \frac{\partial U}{\partial t}
        = - \frac{1}{\rho} \frac{\partial \tau}{\partial z}
        = g  \frac{\partial \tau}{\partial p}
\end{eqnarray}

as determined by i.e.~,

\begin{eqnarray}
  \frac{\partial U_{k}}{\partial t}
        =  g  \frac{\tau_{k+1/2} - \tau{k-1/2}}{\Delta p}.
\end{eqnarray}

Using this, the temporal rates of change for the east-west and
north-south winds are calculated as follows

\begin{eqnarray}
  \frac{\partial u_{k}}{\partial t}  =
           \frac{\partial U_{k}}{\partial t} \frac{u_{1}}{U_{1}} \\
  \frac{\partial v_{k}}{\partial t}  =
           \frac{\partial U_{k}}{\partial t} \frac{v_{1}}{U_{1}}
\end{eqnarray}

\hypertarget{other-notes.}{%
\subsubsection{Other Notes.}\label{other-notes.}}

\begin{enumerate}
\def\labelenumi{\arabic{enumi}.}
\tightlist
\item
  it is assumed that no gravitational waves are excited at the ground
  surface when the wind speed is small (\(U_{1} \le v_{min}\)) and when
  the undulations at the surface are small
  (\(Z_{SD} \le (Z_{SD})_{min}\)).
\end{enumerate}
