\begin{itemize}
\tightlist
\item
  \ref{1-surface-flux}
\item
  \ref{11-sea-surface-flux-ocnflx}

  \begin{itemize}
  \tightlist
  \item
    \ref{12-sea-surface-conditions-ocnbcs}

    \begin{itemize}
    \tightlist
    \item
      \ref{121-sea-surface-albedo-for-visible-seaalb}
    \item
      \ref{122-sea-surface-roughness-seaz0f}
    \end{itemize}
  \item
    \ref{13-sea-surface-flux-sfcflx}

    \begin{itemize}
    \tightlist
    \item
      \ref{131-bulk-factor-blkcof}
    \end{itemize}
  \item
    \ref{14-radiation-flux-at-sea-surface-radsfc}
  \item
    \ref{15-sea-surface-heat-balance-ocnslv}
  \end{itemize}
\end{itemize}

\hypertarget{surface-flux}{%
\section{1 Surface Flux}\label{surface-flux}}

Until CCSR/NIES AGCM, both land surface and sea surface were treated as
one of the atmospheric physical processes, but after MIROC3 (Hasumi and
Emori, 2004), land surface processes became independent as MATSIRO.
However, since MIROC3
(\href{https://ccsr.aori.u-tokyo.ac.jp/~hasumi/miroc_description.pdf}{Hasumi
and Emori, 2004}), land surface processes have been separated into
MATSIRO. In \texttt{SUBROUTINE:{[}SURFCE{]}} in pgsfc.F,
\texttt{ENTRY:{[}OCNFLX{]}} (in \texttt{SUBROUTINE:{[}OCEAN{]}} of
pgocn.F) is called for the sea surface, and \texttt{ENTRY:{[}LNDFLX{]}}
(in \texttt{SUBROUTINE:{[}MATSIRO{]}} of matdrv.F) is called for the
land surface, respectively. This chapter describes sea surface
processes, which are still treated within the framework of atmospheric
physical processes (MIROC6). For the land surface processes, please
refer to
\href{https://github.com/integrated-land-simulator/model_description}{Description
of ILS}.

\href{https://github.com/MIROC-DOC/model_description/blob/coupler_iwakiri/draft/AO-coupler.md}{カップラーのセクション}とmerge予定。

\begin{itemize}
\tightlist
\item
  Outputs
\end{itemize}

\setlength\LTleft{0pt}\setlength\LTright{0pt}\begin{longtable}[]{@{}lllll@{}}
\toprule\relax
Meaning & Presentation & Variable & dimension & unit \\
\midrule\relax
\endhead
upward long wave & & RFLXLG & IJLSDIM & \\
upward short wave & & RFLXSG & IJLSDIM & \\
\bottomrule
\end{longtable}

\begin{itemize}
\tightlist
\item
  Inputs
\end{itemize}

\setlength\LTleft{0pt}\setlength\LTright{0pt}\begin{longtable}[]{@{}lllll@{}}
\toprule\relax
Meaning & Presentation & Variable & dimension & unit \\
\midrule\relax
\endhead
surface downward radiation & & RFSFCD & IJSDIM, NRALB & \\
cos(solar zenith) & \(cos(\theta)\) & RCOSZ & IJSDIM & {[}-{]} \\
rainfall (cumulus convection scheme) & & GPRCC & IJSDIM, NTR & \\
rainfall (Large scale condensation scheme) & & GPRCL & IJSDIM, NTR & \\
snowfall (cumulus convection scheme) & & GSNWC & IJSDIM, NTR & \\
snowfall (cLarge scale condensation scheme) & & GSNWL & IJSDIM, NTR & \\
u wind & \(u\) & GDU & IJSDIM, KMAX & {[}m/s{]} \\
v wind & \(v\) & GDV & IJSDIM, KMAX & {[}m/s{]} \\
temperature & \(T\) & GDT & IJSDIM, KMAX & {[}K{]} \\
humidity & \(q\) & GDQ & IJSDIM, KMAX, NTR & {[}kg/kg{]} \\
pressure & \(P\) & GDP & IJSDIM, KMAX+1 & \\
pressure (half level) & & GDPM & IJSDIM, KMAX+1 & \\
altitude (half level) & & GDZM & IJSDIM, KMAX+1 & \\
time & & TIME & & \\
dt for implicit & & DELTP & & \\
time step (interval) & & DELTI & & \\
\bottomrule
\end{longtable}

The only 1st layer is practically handed to the surface schemes.

\hypertarget{sea-surface-flux-ocnflx}{%
\section{\texorpdfstring{1.1 Sea surface flux
\texttt{{[}OCNFLX{]}}}{1.1 Sea surface flux {[}OCNFLX{]}}}\label{sea-surface-flux-ocnflx}}

Sea surface processes provide the boundary conditions at the lower end
of the atmosphere through the exchange of momentum, heat, and water
fluxes between the atmosphere and the surface. In
\texttt{ENTRY:{[}OCNFLX{]}}, the following procedure is used to deal
with sea surface processes.

\begin{enumerate}
\def\labelenumi{\arabic{enumi}.}
\tightlist
\item
  prepare variables for sea ice extent and no ice extent, respectively,
  using sea ice concentration.
\item
  Determine the surface boundary conditions.
\item
  Calculate the flux balance.
\item
  Calculate the radiation budget at the sea surface.
\item
  Calculate the deposition by CHASER.
\item
  solve the heat balance at the sea surface and update the surface
  temperature and each flux value.
\end{enumerate}

\setlength\LTleft{0pt}\setlength\LTright{0pt}\begin{longtable}[]{@{}lllll@{}}
\toprule\relax
Meaning & Presentation & Variable & dimension & unit \\
\midrule\relax
\endhead
u wind of the 1st layer of the atmosphere & \(u_a\) & GAUA & IJOSDM &
{[}m/s{]} \\
v wind of the 1st layer of the atmosphere & \(v_a\) & GAVA & IJOSDM &
{[}m/s{]} \\
temperature of the 1st layer of the atmosphere & \(T_a\) & GATA & IJOSDM
& {[}K{]} \\
humidity of the 1st layer of the atmosphere & \(q_a\) & GAQA & IJOSDM &
{[}kg/kg{]} \\
pressure of the 1st layer of the atmosphere & \(P_a\) & GAPA & IJOSDM
& \\
surface pressure Ps & \(P_s\) & GAPS & IJOSDM & \\
surface height & & GAZS & IJOSDM & \\
surface radiation fluxes & & RSFCD & IJOSDM & \\
cos(solar zenith) & \(cos(\theta)\) & RCOSZ & IJOSDM & {[}-{]} \\
\bottomrule
\end{longtable}

If use CHASER, variables below are also needed.

\setlength\LTleft{0pt}\setlength\LTright{0pt}\begin{longtable}[]{@{}lllll@{}}
\toprule\relax
Meaning & Presentation & Variable & dimension & unit \\
\midrule\relax
\endhead
Henry const & & EH & IJOSDM & \\
precipitation flux (cumulus convection scheme) & \(Pr_c\) & PFLXC &
IJOSDM & \\
precipitation flux (large scale condensation scheme) & \(Pr_l\) & PFLXL
& IJOSDM & \\
latitude & & LLAT & IJOSDM & \\
\bottomrule
\end{longtable}

Practically, precipitation flux from 2 schemes are treated together.

\begin{eqnarray}
    Pr = Pr_c + Pr_l
\end{eqnarray}

In the sea ice area (\(L=1\)), the surface temperature \(T_s\) is the
sea ice surface temperature \(T_{ice}\). However, if \(T_{ice}\) is
higher than \(T_{melt}=0\), then \(T_{melt}\) is used.

\begin{eqnarray}
    T_s = min(T_{ice},T_{melt})
\end{eqnarray}

The sea ice bottom temperature \(T_b\) is assumed to be the sea surface
temperature \(T_{o(1)}\).

\begin{eqnarray}
    T_b = T_{o(1)}
\end{eqnarray}

The amount of sea ice \(W_{ice}\) and the amount of snow on it
\(W_{snow}\) are converted per unit area by considering \(R_{ice}\) and
used in the calculation. However, a limiter \(\epsilon\) is provided to
prevent the values from becoming too small.

\begin{eqnarray}
R_{ice} =\mathrm{max}( R_{ice,orginal}, \epsilon)
\end{eqnarray}

In the ice-free region (\(L=2\)), the surface temperature \(T_s\) and
sea ice bottom temperature \(T_b\) are assumed to be the sea temperature
temperature \(T_{o(1)}\).

\begin{eqnarray}
    T_s = T_b = T_{o(1)}
\end{eqnarray}

The evaporation efficiency ise set to 1 for both \(L=1, 2\).

If the sea ice concentration \(R_{ice}\) is not given, it can be
diagnosed simply from the sea ice volume \(W_{ice}\) in
\texttt{ENTRY:{[}OCNICR{]}}.

\begin{eqnarray}
R_{ice} = \mathrm{min}\Big(\sqrt{\frac{\mathrm{max}(W_{ice},0)}{W_{ice,c}}},1.0\Big)
\end{eqnarray}

The standard gives the amount of sea ice per area as
\(W_{ice,c}=300 \mathrm{[kg/m^2]}\).

\hypertarget{sea-surface-conditions-ocnbcs}{%
\subsection{\texorpdfstring{1.2 Sea Surface Conditions
\texttt{{[}OCNBCS{]}}}{1.2 Sea Surface Conditions {[}OCNBCS{]}}}\label{sea-surface-conditions-ocnbcs}}

\begin{itemize}
\tightlist
\item
  Output variables
\end{itemize}

\setlength\LTleft{0pt}\setlength\LTright{0pt}\begin{longtable}[]{@{}lllll@{}}
\toprule\relax
Meaning & Presentation & Variable & dimension & unit \\
\midrule\relax
\endhead
surface albedo & \(\alpha\) & GRALB & IJLODM, NRDIR, NRBND & -- \\
surface roughness & \(r_0\) & GRZ0 & IJLODM, NTYZ0 & -- \\
heat flux & \(G\) & FOGFLX & IJLODM & -- \\
heat diffusion coefficient & \(\frac{\partial G}{\partial T}\) & DGFDS &
IJLODM & -- \\
\bottomrule
\end{longtable}

\begin{itemize}
\tightlist
\item
  Input variables
\end{itemize}

\setlength\LTleft{0pt}\setlength\LTright{0pt}\begin{longtable}[]{@{}lllll@{}}
\toprule\relax
Meaning & Presentation & Variable & dimension & unit \\
\midrule\relax
\endhead
surface temperature & \(T_s\) & GRTS & IJLODM & \(\mathrm{[K]}\) \\
ice base temperature & \(T_b\) & GRTB & IJLODM & \(\mathrm{[K]}\) \\
lake ice amount & \(Ic\) & GRICE & IJLODM & \(\mathrm{[kg/m^2]}\) \\
snow amount & \(Sn\) & GRSNW & IJLODM & \\
ice concentration & \(R_{ice}\) & GRICR & IJLODM & {[}-{]} \\
u wind of the 1st layer of the atmosphere & \(u_a\) & GDUA & IJLODM &
\(\mathrm{[m/s]}\) \\
v wind of the 1st layer of the atmosphere & \(v_a\) & GDVA & IJLODM &
\(\mathrm{[m/s]}\) \\
cos(solar zenith) & \(\mathrm{cos}(\theta)\) & RCOSZ & IJLODM &
{[}-{]} \\
\bottomrule
\end{longtable}

In \texttt{ENTRY{[}OCNBCS{]}} in \texttt{SUBROUTINE:{[}OCNSUB{]}},
surface albedo and roughness are calculated. They are calculated
supposing ice-free conditions, then modified.

First, let us consider the sea albedo. The sea level \(\alpha_{(d,b)}\),
\(b=1,2,3\) represent the visible, near-infrared, and infrared
wavelength bands, respectively. Also, \(d=1,2\) represents direct and
scattered light, respectively. The albedo for the visible bands are
calculated in \texttt{SUBROUTINE\ {[}SEAALB{]}}, supposing ice-free
conditions. The albedo for near-infrared is set to same as the visible
one. The albedo for infrared is uniformly set to a constant value.

The grid-averaged albedo, taking into account the sea ice concentration
\(R_{ice}\), is

\begin{eqnarray}
    \alpha = \alpha -R_{ice} \alpha_{ice}
\end{eqnarray}

\(\alpha_{ice}\) is given by the standard as
\(\alpha_{ice,1}=0.5,\alpha_{ice,2}=0.5,\alpha_{ice,3}=0.05\). 4.

In addition, we want to consider the effect of snow cover. Here, we
consider the albedo modification by temperature. The standard threshold
values for snow temperature are \(T_{al,2}=258.15 \mathrm{[K]}\) and
\(T_{al,1}=273.15 \mathrm{[K]}\). The snow albedo changes linearly with
temperature change from \(\alpha_{snow,1}=0.75\) to
\(\alpha_{ snow,2}\). Let the coefficient \(\tau_{snow}\), which is
\(0\le \tau \le 1\).

\begin{eqnarray}
\tau_{snow} = \frac{T_s - T_{al,1}}{T_{al,2}-T_{al,1}}
\end{eqnarray}

Update the snow albedo \(\alpha_{snow}\) as

\begin{eqnarray}
    \alpha_{snow} = \alpha_{snow,0} + \tau_{snow}(\alpha_{snow,2}-\alpha_{snow,1})
\end{eqnarray}

Second, let us consider the sea surface roughness. The roughnesses of
for momentum, heat and vapor are calculated in \texttt{{[}SEAZ0F{]}},
supposing the ice-free conditions.

When the sea ice exists (\(L=1\)), each roughness is modified to take
into account the sea concentration \(R_{ice}\).

\begin{eqnarray}
    z_{0,M} = z_{0,M} + ( z_{0,ice,M} - z_{0,M})  \alpha_{ice}
\end{eqnarray}

\begin{eqnarray}
    z_{0,H} = z_{0,H} + ( z_{0,ice,H} - z_{0,H})  \alpha_{ice}
\end{eqnarray}

\begin{eqnarray}
    z_{0,E} = z_{0,E} + ( z_{0,ice,E} - z_{0,E})  \alpha_{ice}
\end{eqnarray}

Here, \(r_{0,ice,*}\) is roughness of sea ice, \(\alpha_{ice}\) is the
sea ice concentration.

When the snow even exists,

\begin{eqnarray}
    z_{0,M} = z_{0,M} + ( z_{0,snow,M} - z_{0,M})  \alpha_{snow}
\end{eqnarray}

\begin{eqnarray}
    z_{0,H} = z_{0,H} + ( z_{0,snow,H} - z_{0,H})  \alpha_{snow}
\end{eqnarray}

\begin{eqnarray}
    z_{0,E} = z_{0,E} + ( z_{0,snow,E} - z_{0,E})  \alpha_{snow}
\end{eqnarray}

Here, \(r_{0,snow,*}\) is roughness of sea ice, \(\alpha_{snow}\) is the
sea ice concentration.

Third, let us consider the conductivity of ice.

When sea ice exists (\(L=1\)), the thermal conductivity
\(k_{ice}^\star\) of sea ice is obtained by using \(D_{f,ice}\) (thermal
diffusivity of sea ice) and sea ice density \(\sigma_{ice}\).

\begin{eqnarray}
k_{ice}^\star = \frac{D_{f,ice}}{\mathrm{max}(R_{ice}/\sigma_{ice}, \epsilon)}
\end{eqnarray}

The calculated thermal conductivity is modified to \(k_{ice}\) to take
into account that it varies with snow cover.

\begin{eqnarray}
h_{snow} = \mathrm{min}(
    \mathrm{max}(
    R_{snow}/\sigma_{snow}),\epsilon
        ),h_{snow,max}
        )
\end{eqnarray}

\begin{eqnarray}      
k_{ice} = k_{ice}^\star (1-R_{ice}) + \frac{D_{ice}}{1+\| D_{ice}/D_{snow} \cdot h_{snow} \|} R_{ice}
\end{eqnarray}

where \(h_{snow}\) is the snow depth, \(R_{snow}\) is the snow area
fraction, \(\sigma_{snow}\) is the snow density, \(h_{snow,max}\) is the
maximum snow depth, and \(D_{snow}\) is the thermal diffusivity of snow.

Therefore, the heat conduction flux and its derivative are

\begin{eqnarray}
 G = k_{ice} (T_b - T_s)
\end{eqnarray}

\begin{eqnarray}
 \frac{\partial G}{\partial T} = k_{ice}
\end{eqnarray}

Note that in the ice-free region (\(L=2\))

\begin{eqnarray}
G=k_{ocn}
\end{eqnarray}

where \(k_{ocn}\) is the heat flux in the sea temperature layer. Here,
\(k_{ocn}\) is the heat flux in the sea temperature layer.

\hypertarget{sea-surface-albedo-for-visible-seaalb}{%
\subsubsection{\texorpdfstring{1.2.1 Sea Surface Albedo for Visible
\texttt{{[}SEAALB{]}}}{1.2.1 Sea Surface Albedo for Visible {[}SEAALB{]}}}\label{sea-surface-albedo-for-visible-seaalb}}

\begin{itemize}
\tightlist
\item
  Inputs
\end{itemize}

\setlength\LTleft{0pt}\setlength\LTright{0pt}\begin{longtable}[]{@{}lllll@{}}
\toprule\relax
Meaning & Presentation & Variable & dimension & unit \\
\midrule\relax
\endhead
cos(solar zenith) & \(cos(\theta)\) & COSZ & IJLODM & {[}-{]} \\
\bottomrule
\end{longtable}

\begin{itemize}
\tightlist
\item
  Outputs
\end{itemize}

\setlength\LTleft{0pt}\setlength\LTright{0pt}\begin{longtable}[]{@{}lllll@{}}
\toprule\relax
Meaning & Presentation & Variable & dimension & unit \\
\midrule\relax
\endhead
sea surface albedo (direct, diffuse) & \(\alpha_{L(d)}\) & GALB & IJLODM
,2 & {[}-{]} \\
\bottomrule
\end{longtable}

For sea surface level albedo \(\alpha_{L(d)}\), \(d=1,2\) represents
direct and scattered light, respectively.

Using the solar zenith angle at latitude \(\theta\), the albedo for
direct light is presented by

\begin{eqnarray}
    \alpha_{L(1)} = e^{(C_3A^* + C_2) A^* +C_1}
\end{eqnarray}

where

\begin{eqnarray}
    A = \mathrm{min}(\mathrm{max}(\mathrm{cos}(\theta),0.03459),0.961)
\end{eqnarray}

On the other hand, the albedo for scattered light is uniformly set to a
constant parameter.

\begin{eqnarray}
    \alpha_{L(2)} = 0.06
\end{eqnarray}

\hypertarget{sea-surface-roughness-seaz0f}{%
\subsubsection{\texorpdfstring{1.2.2 Sea Surface Roughness
\texttt{{[}SEAZ0F{]}}}{1.2.2 Sea Surface Roughness {[}SEAZ0F{]}}}\label{sea-surface-roughness-seaz0f}}

\begin{itemize}
\tightlist
\item
  Outputs
\end{itemize}

\setlength\LTleft{0pt}\setlength\LTright{0pt}\begin{longtable}[]{@{}lllll@{}}
\toprule\relax
Meaning & Presentation & Variable & dimension & unit \\
\midrule\relax
\endhead
surface roughness for momentum & \(r_{0,M}\) & GRZ0M & IJLODM & -- \\
surface roughness for heat & \(r_{0,H}\) & GRZ0H & IJLODM & -- \\
surface roughness for vapor & \(r_{0,E}\) & GRZ0E & IJLSDIM & -- \\
\bottomrule
\end{longtable}

\begin{itemize}
\tightlist
\item
  Inputs
\end{itemize}

\setlength\LTleft{0pt}\setlength\LTright{0pt}\begin{longtable}[]{@{}lllll@{}}
\toprule\relax
Meaning & Presentation & Variable & dimension & unit \\
\midrule\relax
\endhead
u wind of the 1st layer of the atmosphere & \(u_a\) & GDUA & IJLODM &
{[}m/s{]} \\
v wind of the 1st layer of the atmosphere & \(v_a\) & GDVA & IJLODM &
{[}m/s{]} \\
\bottomrule
\end{longtable}

The roughness variation of the sea surface is determined by the friction
velocity \(u^\star\)

\begin{eqnarray}
u^{\star} = \sqrt{C_{M_0} ({u_a}^2  +{v_a}^2)}
\end{eqnarray}

We perform successive approximation calculation of \({C_{M_0}}\),
because \(F_u,F_v,F_\theta,F_q\) are required.

\begin{eqnarray}
    r_{0,M} = z_{0,M_0} + z_{0,M_R} + \frac{z_{0,M_R} {u^\star }^2 }{g} + \frac{z_{0,M_S}\nu }{u^\star}
\end{eqnarray}

\begin{eqnarray}
    r_{0,H} = z_{0,H_0} + z_{0,H_R} + \frac{z_{0,H_R} {u^\star }^2 }{g} + \frac{z_{0,H_S}\nu }{u^\star}
\end{eqnarray}

\begin{eqnarray}
    r_{0,E} = z_{0,E_0} + z_{0,E_R} + \frac{z_{0,E_R} {u^\star }^2 }{g} + \frac{z_{0,E_S}\nu }{u^\star}
\end{eqnarray}

Here, \(\nu = 1.5 \times 10^{-5} \mathrm{[m^2/s]}\) is the kinetic
viscosity of the atmosphere. \(z_{0,M},z_{0,H}\) and \(z_{0,E}\) are
surface roughness for momentum, heat, and vapor, respectively.
\(z_{0,M_0},z_{0,H_0}\) and \(z_{0,E_0}\) are base, and rough factor
(\(z_{0,M_R},z_{0,M_R}\) and \(z_{0,E_R}\)) and smooth factor
(\(z_{0,M_S},z_{0,M_S}\) and \(z_{0,E_S}\)) are taken into account.

\hypertarget{sea-surface-flux-sfcflx}{%
\subsection{\texorpdfstring{1.3 Sea Surface Flux
\texttt{{[}SFCFLX{]}}}{1.3 Sea Surface Flux {[}SFCFLX{]}}}\label{sea-surface-flux-sfcflx}}

The surface flux scheme evaluates the physical quantity fluxes between
the atmospheric surfaces due to turbulent transport in the boundary
layer. The main input are wind speed (\(u_a, v_a\)), temperature
(\(T_a\)), and specific humidity (\(q_s\)) from the 1st layer of the
atmosphere. The output are the vertical fluxes and the differential
values (for obtaining implicit solutions) of momentum, heat, and water
vapor.

Surface fluxes (\(F_u, F_v, F_\theta, F_q\)) are expressed using the
bulk coefficients (\(C_M, C_H, C_E\)) as follows

\begin{eqnarray}
    F_u  =  - \rho C_M |{\mathbf{v}}| u
\end{eqnarray}

\begin{eqnarray}
    F_v  =  - \rho C_M |{\mathbf{v}}| v
\end{eqnarray}

\begin{eqnarray}
    F_\theta  = \rho c_p C_H |{\mathbf{v}}| ( \theta_g - \theta )
\end{eqnarray}

\begin{eqnarray}
    F_q^P =  \rho C_E |{\mathbf{v}}| ( q_g - q_a )
\end{eqnarray}

Note that \(F_q^P\) is the possible evaporation flux.

The turbulent fluxes at the sea surface are solved by bulk formulae as
follows. Then, by solving the surface energy balance, the ground surface
temperature (\(T_s\)) is updated, and the surface flux values with
respect to those values are also updated. The solutions obtained here
are temporary values. In order to solve the energy balance by
linearizing with respect to \(T_s\), the differential with respect to
\(T_s\) of each flux is calculated beforehand.

\begin{itemize}
\tightlist
\item
  Momentum flux
\end{itemize}

\begin{eqnarray}
 \tau_x = - \rho C_{M}|V_a| u_a
\end{eqnarray}

\begin{eqnarray}
 \tau_y = - \rho C_{M}|V_a| v_a
\end{eqnarray}

where \(\tau_x\) and \(\tau_y\) are the momentum fluxes (surface stress)
of the zonal and meridional directions, respectively.

\begin{itemize}
\tightlist
\item
  Sensible heat flux
\end{itemize}

\begin{eqnarray}
 H_s = c_p \rho C_{Hs}|V_a| (T_s - (P_s/P_a)^{\kappa}T_a)
\end{eqnarray}

where \(H_s\) is the sensible heat flux from the sea surface;
\(\kappa = R_{air} / c_p\) and \(R_{air}\)are the gas constants of air;
and \(c_p\) is the specific heat of air.

\begin{itemize}
\tightlist
\item
  Bare sea surface evaporation flux
\end{itemize}

\begin{eqnarray}
\hat{F}q^P_{1/2} = \rho_{1/2} C_E |{\mathbf{v}}_1| \left( q^\star(T_0) - q_1 \right)
\end{eqnarray}

\hypertarget{bulk-factor-blkcof}{%
\subsubsection{\texorpdfstring{1.3.1 Bulk factor
\texttt{{[}BLKCOF{]}}}{1.3.1 Bulk factor {[}BLKCOF{]}}}\label{bulk-factor-blkcof}}

The bulk Richardson number (\(R_{iB}\)), which is used as a benchmark
for the stability between the atmospheric surfaces, is

\begin{eqnarray}
R_{iB} =
            \frac{ \frac{g}{\theta_s} (\theta_1 - \theta(z_0))/z_1 }
              { (u_1/z_1)^2                                  }
       = \frac{g}{\theta_s}
         \frac{T_1 (p_s/p_1)^\kappa - T_0 }{u_1^2/z_1} f_T
\end{eqnarray}

Here, \(g\) is the gravitational accerelation, \(\theta_s\)
(\(\Theta_0\) in MATSIRO description) is the basic potential
temperature, \(T_1\) is the atmospheric temperature of the 1st layer,
\(T_0\) is the surface surface temperature, \(p_s\) is the surface
pressure, \(p_1\) is the pressure of the 1st layer, \$\kappa \$ is the
Karman constant, and

\begin{eqnarray}
f_T = (\theta_1 - \theta(z_0))/(\theta_1 - \theta_0)
\end{eqnarray}

The bulk coefficients of \(C_M,C_H,C_E\) are calculated according to
\href{./papers/Louis1979_Article_AParametricModelOfVerticalEddy.pdf}{Louis
(1979)} and
\href{./papers/Louis1982_a_short_history_of_the_operational_pbl_parameterization_at_ecmwf.pdf}{Louis
{\emph{et al.}}(1982)}. However, corrections are made to take into
account the difference between momentum and heat roughness. If the
roughnesses for momentum, heat, and water vapor are set to
\(z_{0,M}, z_{0,H}, z_{0,E}\), respectively, the results are generally
\(z_{0,M} > z_{0,H}, z_{0,E}\), but the bulk coefficients for heat and
water vapor for the fluxes from the height of \(z_{0,M}\) are also set
to \(\widetilde{C_H}\), \(\widetilde{C_E}\) first, and then corrected.

\begin{eqnarray}
    C_M = \left\{
      \begin{array}{lr}
      C_{0,M} [ 1 + (b_M/e_M)  R_{iB} ]^{-e_M}
            &,
          R_{iB} \geq 0 \\
      C_{0,M} \left[ 1 - b_M R_{iB} \left( 1+ d_M b_M C_{0,M}
                                  \sqrt{\frac{z_1}{z_{0,M}}| R_{iB}|} \,
                                  \right)^{-1} \right]     
          &,
          R_{iB} < 0 \\
      \end{array} \right.
\end{eqnarray}

\begin{eqnarray}
    \widetilde{C_H} = \left\{
      \begin{array}{lr}
      \widetilde{C_{0,H}} [ 1 + (b_H/e_H) R_{iB} ]^{-e_H}
            &,
          R_{iB} \geq 0 \\
      \widetilde{C_{0,H}} \left[ 1 - b_H R_{iB}
                                  \left( 1+ d_H b_H \widetilde{C_{0,H}}
                                  \sqrt{\frac{z_1}{z_{0,M}}| R_{iB}|} \,
                                  \right)^{-1} \right]
             &,     
          R_{iB} < 0 \\
      \end{array} \right.
\end{eqnarray}

\begin{eqnarray}
    C_H = \widetilde{C_H} f_T
\end{eqnarray}

\begin{eqnarray}
    \widetilde{C_E} = \left\{
      \begin{array}{lr}
      \widetilde{C_{0,E}} [ 1 + (b_E/e_E) R_{iB} ]^{-e_E}
            &,
          R_{iB} \geq 0 \\
      \widetilde{C_{0,E}} \left[ 1 - b_E R_{iB}
                                  \left( 1+ d_E b_E \widetilde{C_{0,E}}
                                  \sqrt{\frac{z_1}{z_{0,M}}| R_{iB}|} \,
                                  \right)^{-1} \right]      
          &,
          R_{iB} < 0 \\
      \end{array} \right.
\end{eqnarray}

\begin{eqnarray}
    C_E = \widetilde{C_E} f_q
\end{eqnarray}

\(C_{0M}, \widetilde{C_{0H}}, \widetilde{C_{0E}}\) is the bulk
coefficient (for fluxes from \(z_{0M}\)) at neutral,

\begin{eqnarray}
    C_{0M}  =  \widetilde{C_{0H}}  =  \widetilde{C_{0E}}  =
       \frac{k^2}{\left[\ln \left(\frac{z_1}{z_{0M}}\right)\right]^2 }
\end{eqnarray}

Correction Factor \(f_q\) is ,

\begin{eqnarray}
  f_q = (q_1 - q(z_0))/(q_1 - q^{\ast}(\theta_0))
\end{eqnarray}

but the method of calculation is omitted. The coefficients of Louis
factors are \(( b_M, d_M, e_M ) = ( 9.4, 7.4, 2.0 )\),
\(( b_H, d_H, e_H ) = ( b_E, d_E, e_E ) = ( 9.4, 5.3, 2.0 )\).

is a correction factor, which is approximated from the uncorrected bulk
Richardson number, but we abbreviate the calculation here.

\hypertarget{radiation-flux-at-sea-surface-radsfc}{%
\subsection{\texorpdfstring{1.4 Radiation Flux at Sea Surface
\texttt{{[}RADSFC{]}}}{1.4 Radiation Flux at Sea Surface {[}RADSFC{]}}}\label{radiation-flux-at-sea-surface-radsfc}}

For the ground surface albedo \(\alpha_{(d,b)}\), \(b=1,2\) represent
the visible and near-infrared wavelength bands, respectively. Also,
\(d=1,2\) are direct and scattered, respectively. For the downward
shortwave radiation \(SW^\downarrow\) and upward shortwave radiation
\(SW^\uparrow\) incident on the earth's surface, the direct and
scattered light together are

\begin{eqnarray}
    SW^\downarrow = SW^\downarrow_{(1,1)}+SW^\downarrow_{(1,2)}+SW^\downarrow_{(2,1)}+SW^\downarrow_{(2,2)} \\
SW^\uparrow = SW^\downarrow_{(1,1)}\cdot\alpha_{(1,1)}+SW^\downarrow_{(1,2)}\cdot\alpha_{(1,2)}+SW^\downarrow_{(2,1)}\cdot\alpha_{(2,1)}+SW^\downarrow_{(2,2)}\cdot\alpha_{(2,2)}
\end{eqnarray}

\hypertarget{sea-surface-heat-balance-ocnslv}{%
\subsection{\texorpdfstring{1.5 Sea Surface Heat Balance
\texttt{{[}OCNSLV{]}}}{1.5 Sea Surface Heat Balance {[}OCNSLV{]}}}\label{sea-surface-heat-balance-ocnslv}}

The comments for some variables say ``soil'', but this is because the
program was adapted from a land surface scheme, and has no particular
meaning.

\begin{itemize}
\tightlist
\item
  Outputs
\end{itemize}

\setlength\LTleft{0pt}\setlength\LTright{0pt}\begin{longtable}[]{@{}lllll@{}}
\toprule\relax
Meaning & Presentation & Variable & dimension & unit \\
\midrule\relax
\endhead
surface water flux & \(W_{free/ice}\) & WFLUXS & IJLODM,2 & -- \\
upward long wave & \(LW^\uparrow\) & RFLXLU & IJLODM & -- \\
flux balance & \(F\) & SFLXBL & IJLODM & -- \\
\bottomrule
\end{longtable}

\begin{itemize}
\tightlist
\item
  Inputs variables
\end{itemize}

\setlength\LTleft{0pt}\setlength\LTright{0pt}\begin{longtable}[]{@{}lll@{}}
\toprule\relax
Meaning & Presentation & Variable \\
\midrule\relax
\endhead
sensible heat flux coefficent & \(\frac{\partial H}{\partial T_s}\) &
DTFDS \\
latent heat flux coefficient & \(\frac{\partial E}{\partial T_s}\) &
DQFDS \\
surface heat flux coefficient & \(\frac{\partial G}{\partial T_s}\) &
DGFDS \\
downward SW radiation & \(SW^\downarrow\) & RFLXSD \\
upward SW radiation & \(SW^\uparrow\) & RFLXLU \\
downward LW radiation & \(LW^\downarrow\) & RFLXLD \\
sea surface albedo & \(\alpha\) & GRALBL \\
sea ice concentration & \(R_{ice}\) & GRICR \\
\bottomrule
\end{longtable}

\begin{itemize}
\tightlist
\item
  Modified in this subroutine
\end{itemize}

\setlength\LTleft{0pt}\setlength\LTright{0pt}\begin{longtable}[]{@{}lllll@{}}
\toprule\relax
Meaning & Presentation & Variable & dimension & unit \\
\midrule\relax
\endhead
surface temperature & \(T_s\) & GDTS & IJLODM & -- \\
surface heat flux from \texttt{seaBC} & \(G\) & GFLUXS & IJLODM & -- \\
sensible heat flux & \(H\) & TFLUXS & IJLODM & -- \\
latent heat flux & \(E\) & QFLUXS & IJLDSM & -- \\
\bottomrule
\end{longtable}

\begin{itemize}
\tightlist
\item
  Others (appeared in texts)
\end{itemize}

\setlength\LTleft{0pt}\setlength\LTright{0pt}\begin{longtable}[]{@{}lllll@{}}
\toprule\relax
Meaning & Presentation & Variable & dimension & unit \\
\midrule\relax
\endhead
sea surface albedo for shortwave radiation (ice-free) & \(\alpha_S\) &
-- & {[}-{]} & -- \\
the Stefan-Boltzmann constant & \(\sigma\) & STB & -- & -- \\
\bottomrule
\end{longtable}

Reference:
\href{https://ccsr.aori.u-tokyo.ac.jp/~hasumi/COCO/coco4.pdf}{Hasumi,
2015, Appendices A}

Downward radiative fluxes are not directly dependent on the condition of
the sea surface, and their observed values are simply specified to drive
the model. Shotwave emission from the sea surface is negligible, so the
upward part of the shortwave radiative flux is accounted for solely by
reflection of the incoming downward flux. Let \(\alpha _S\) be the sea
surface albedo for shortwave radiation. The upward shortwave radiative
flux is represented by

\begin{eqnarray}
    SW^\uparrow = - \alpha_S SW^\downarrow
\end{eqnarray}

On the other hand, the upward longwave radiative flux has both
reflection of the incoming flux and emission from the sea surface. Let
\(\alpha\) be the sea surface albedo for longwave radiation and
\(\epsilon\) be emissivity of the sea surface relative to the black body
radiation. The upward shortwave radiative flux is represented by

\begin{eqnarray}
    LW^\uparrow = - \alpha LW^\downarrow + \epsilon \sigma T_s ^4
\end{eqnarray}

where \(\sigma\) is the Stefan-Boltzmann constant and \(T_s\) is surface
temperature. If sea ice exists, snow or sea ice temperature is
considered by fractions. When radiative equilibrium is assumed,
emissivity becomes identical to co-albedo:

\begin{eqnarray}
    \epsilon = 1 - \alpha
\end{eqnarray}

The net surface flux is presented by

\begin{eqnarray}
    F^*=H + (1-\alpha)\sigma T_s^4 + \alpha LW^\uparrow - LW^\downarrow +SW^\uparrow - SW^\downarrow        
\end{eqnarray}

The heat flux into the sea surface is presented, with the surface heat
flux calculated in \texttt{PSFCM}

\begin{eqnarray}
    G^* = G - F^*
\end{eqnarray}

Note that \(G^*\) is downward positive.

The temperature derivative term is

\begin{eqnarray}
    \frac{\partial G^*}{\partial T_s} = \frac{\partial G}{\partial T_s}+\frac{\partial H}{\partial T_s}+\frac{\partial R}{\partial T_s}
\end{eqnarray}

When the sea ice exists, the sublimation flux is considered

\begin{eqnarray}
    G_{ice} = G^* - l_s E
\end{eqnarray}

The temperature derivative term is

\begin{eqnarray}
    \frac{\partial G_{ice}}{\partial T_s}=\frac{\partial G^*}{\partial T_s} + l_s\frac{\partial E}{\partial T_s}
\end{eqnarray}

Finally, we can update the surface temperature with the sea ice
concentration with
\(\Delta T_s=G_{ice} ( \frac{\partial G_{ice}}{\partial T_s})^{-1}\)

\begin{eqnarray}
    T_s = T_s +R_{ice} \Delta T_s
\end{eqnarray}

Then, the sensible and latent heat flux on the sea ice is updated.

\begin{eqnarray}
    E_{ice} = E + \frac{\partial E}{\partial T_s}\Delta T_s
\end{eqnarray}

\begin{eqnarray}
    H_{ice} = H + \frac{\partial H}{\partial T_s}\Delta T_s
\end{eqnarray}

When the sea ice does not existed, otherwise, the evaporation flux is
added to the net flux.

\begin{eqnarray}
    G_{free}=F^* + l_cE
\end{eqnarray}

Finally each flux is updated.

For the sensible heat flux, the temperature change on the sea ice is
considered.

\begin{eqnarray}
    H=H+ R_{ice}  H_{ice}
\end{eqnarray}

Then, the heat used for the temperature change is saved.

\begin{eqnarray}
    F = R_{ice} H_{ice}
\end{eqnarray}

For the upward longwave radiative flux, the temperature change on the
sea ice is considered.

\begin{eqnarray}
    LW^\uparrow=LW^\uparrow +  4\frac{\sigma}{T_s}R_{ice}  \Delta T_s
\end{eqnarray}

For the surface heat flux, the sea ice concentration is considered.

\begin{eqnarray}
    G=(1-R_{ice})G_{free} + R_{ice}G_{ice}
\end{eqnarray}

For the latent heat flux, the sea ice concentration is considered.

\begin{eqnarray}
    E=(1-R_{ice})E + R_{ice}E_{ice}
\end{eqnarray}

Each term above are saved as freshwater flux.

\begin{eqnarray}
    W_{free} = (1-R_{ice}) E
\end{eqnarray}

\begin{eqnarray}
    W_{ice} = R_{ice} E_{ice}
\end{eqnarray}
