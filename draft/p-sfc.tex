\begin{itemize}
\tightlist
\item
  \protect\hyperlink{1-sea-surface-conditions}{1 Sea Surface Conditions}

  \begin{itemize}
  \tightlist
  \item
    \protect\hyperlink{11-overview-ux6e08-1ux6708}{1.1 Overview {[}済
    1月{]}}
  \item
    \protect\hyperlink{12-passing-variables-between-agcm-and-landsea-level-schemes-pgsfcnot-yet-done-work-to-be-done-in-january}{1.2
    Passing variables between AGCM and land/sea level schemes
    \texttt{{[}PGSFC{]}}{[}not yet done, work to be done in January{]}.}
  \item
    \protect\hyperlink{13-setting-sea-surface-conditions-ux6e08-1ux6708}{1.3
    Setting Sea Surface Conditions {[}済 1月{]}}

    \begin{itemize}
    \tightlist
    \item
      \protect\hyperlink{131-input-variables-from-the-atmosphere}{1.3.1
      Input variables from the atmosphere}
    \item
      \protect\hyperlink{132-ocean-surface-conditions-ocnbcs}{1.3.2
      Ocean Surface Conditions \texttt{{[}OCNBCS{]}}}

      \begin{itemize}
      \tightlist
      \item
        \protect\hyperlink{1321-albedo}{1.3.2.1 Albedo}
      \item
        \protect\hyperlink{1322-roughness}{1.3.2.2 Roughness}
      \item
        \protect\hyperlink{1323-sea-surface-heat-flux}{1.3.2.3 Sea
        Surface heat flux}
      \end{itemize}
    \end{itemize}
  \item
    \protect\hyperlink{14-surface-flux-ux6e08-1ux6708}{1.4 Surface Flux
    {[}済 1月{]}}

    \begin{itemize}
    \tightlist
    \item
      \protect\hyperlink{141-overview}{1.4.1 Overview}

      \begin{itemize}
      \tightlist
      \item
        \protect\hyperlink{142-basic-formula-for-flux-calculations}{1.4.2
        Basic Formula for Flux Calculations}
      \end{itemize}
    \item
      \protect\hyperlink{142-roughness-sea0f}{1.4.2 Roughness
      \texttt{{[}SEA0F{]}}}

      \begin{itemize}
      \tightlist
      \item
        \protect\hyperlink{1421-variables}{1.4.2.1 variables}
      \end{itemize}
    \item
      \protect\hyperlink{143-richardson-number-psfcl}{1.4.3 Richardson
      Number \texttt{{[}PSFCL{]}}}
    \item
      \protect\hyperlink{144-bulk-factor-blkcof}{1.4.4 Bulk factor
      \texttt{{[}BLKCOF{]}}}
    \item
      \protect\hyperlink{14-calculation-of-surface-turbulent-fluxes-psfcm}{1.4.
      Calculation of surface turbulent fluxes \texttt{{[}PSFCM{]}}}
    \end{itemize}
  \item
    \protect\hyperlink{15-radiation-flux-at-sea-surface-radsfc-ux6e08-1ux6708}{1.5
    Radiation Flux at Sea Surface \texttt{{[}RADSFC{]}} {[}済 1月{]}}
  \item
    \protect\hyperlink{16-surface-heat-balance-ocnslv-ux6e08-1ux6708}{1.6
    Surface Heat Balance \texttt{{[}OCNSLV{]}} {[}済 1月{]}}
  \end{itemize}
\end{itemize}

\hypertarget{sea-surface-conditions}{%
\section{1 Sea Surface Conditions}\label{sea-surface-conditions}}

Sea surface processes provide the boundary conditions at the lower end
of the atmosphere through the exchange of momentum, heat, and water
fluxes between the atmosphere and the surface. Until \href{}{CCSR/NIES
AGCM}, both land surface and sea surface were treated as one of the
atmospheric physical processes, but after MIROC3 (Hasumi and Emori,
2004), land surface processes became independent as MATSIRO. However,
since MIROC3 (Hasumi and Emori, 2004), land surface processes have been
separated into MATSIRO. This chapter describes sea surface processes,
which are still treated within the framework of atmospheric physical
processes (MIROC6). For the land surface processes, please refer to
\href{https://github.com/integrated-land-simulator/model_description}{Description
of ILS}.

\hypertarget{overview-ux6e08-1ux6708}{%
\subsection{1.1 Overview {[}済 1月{]}}\label{overview-ux6e08-1ux6708}}

In \texttt{MODULE:\ {[}PGSFC{]}}, \texttt{{[}OCNFLX{]}}
(\texttt{MODULE:\ {[}POCEN{]}}) is called for the sea surface, and
\texttt{{[}LNDFLX{]}} of MATSIRO model is called for the land surface.
In \texttt{{[}OCNFLX{]}}, the following procedure is used to deal with
sea surface processes.

\begin{enumerate}
\def\labelenumi{\arabic{enumi}.}
\tightlist
\item
  prepare variables for sea ice extent and no ice extent, respectively,
  using sea ice concentration. 2. Determine the surface boundary
  conditions. 3. Calculate the flux balance. 4. Calculate the radiation
  budget at the sea surface. 5.
\end{enumerate}

\begin{enumerate}
\def\labelenumi{(\arabic{enumi})}
\setcounter{enumi}{2}
\tightlist
\item
  Calculate the deposition by CHASER.
\end{enumerate}

\begin{enumerate}
\def\labelenumi{\arabic{enumi}.}
\setcounter{enumi}{5}
\tightlist
\item
  solve the heat balance at the sea surface and update the surface
  temperature and each flux value.
\end{enumerate}

The four modules discussed in this chapter are as follows.

\setlength\LTleft{0pt}\setlength\LTright{0pt}\begin{longtable}[]{@{}lll@{}}
\toprule\relax
module name & file name & contents \\ \addlinespace
\midrule\relax
\endhead
\texttt{MODULE:{[}PGSFC{]}} & \texttt{./physics/pgsfc.F} & surface
driver \\ \addlinespace
\texttt{MODULE:{[}PSFCL{]}} & \texttt{./physics/psfcl.F} & surface bulk
transfer coefficient \\ \addlinespace
\texttt{MODULE:{[}PSFCM{]}} & \texttt{./physics/psfcm.F} & surface
fluxes \\ \addlinespace
\texttt{MODULE:{[}PGOCN{]}} & \texttt{./physics/pgocn.F} & mixed
layer/fixed SST ocean \\ \addlinespace
\bottomrule
\end{longtable}

\hypertarget{passing-variables-between-agcm-and-landsea-level-schemes-pgsfcnot-yet-done-work-to-be-done-in-january.}{%
\subsection{\texorpdfstring{1.2 Passing variables between AGCM and
land/sea level schemes \texttt{{[}PGSFC{]}}{[}not yet done, work to be
done in
January{]}.}{1.2 Passing variables between AGCM and land/sea level schemes {[}PGSFC{]}{[}not yet done, work to be done in January{]}.}}\label{passing-variables-between-agcm-and-landsea-level-schemes-pgsfcnot-yet-done-work-to-be-done-in-january.}}

Calling \texttt{OCNFLX} (\texttt{MODULE:\ {[}POCEN{]}}) for sea level
and \texttt{LNDFLX} of the MATSIRO model for land level, respectively.

\href{https://github.com/MIROC-DOC/model_description/blob/coupler_iwakiri/draft/AO-coupler.md}{カップラーのセクション}とmerge予定。

\hypertarget{setting-sea-surface-conditions-ux6e08-1ux6708}{%
\subsection{1.3 Setting Sea Surface Conditions {[}済
1月{]}}\label{setting-sea-surface-conditions-ux6e08-1ux6708}}

\hypertarget{input-variables-from-the-atmosphere}{%
\subsubsection{1.3.1 Input variables from the
atmosphere}\label{input-variables-from-the-atmosphere}}

\setlength\LTleft{0pt}\setlength\LTright{0pt}\begin{longtable}[]{@{}ll@{}}
\toprule\relax
variable & meaning \\ \addlinespace
\midrule\relax
\endhead
GAUA & u-wind \\ \addlinespace
GAVA & v-wind \\ \addlinespace
GATA & temperature T \\ \addlinespace
GAQA & humidity q \\ \addlinespace
GAPA & pressure P \\ \addlinespace
GAPS & surface pressure Ps \\ \addlinespace
GAZS & surface height \\ \addlinespace
RSFCD & surface radiation fluxes \\ \addlinespace
RCOSZ & cos(solar angle) \\ \addlinespace
\bottomrule
\end{longtable}

If use CHASER, variables below are also needed.

\setlength\LTleft{0pt}\setlength\LTright{0pt}\begin{longtable}[]{@{}ll@{}}
\toprule\relax
variable & meaning \\ \addlinespace
\midrule\relax
\endhead
EH & Henry const \\ \addlinespace
PFLXC & precipitation flux (cumulus convection scheme) \\ \addlinespace
PFLXL & precipitation flux (large scale condensation
scheme) \\ \addlinespace
LLAT & latitude \\ \addlinespace
\bottomrule
\end{longtable}

Practically, precipitation flux from 2 schemes are treated together.

\begin{eqnarray}
    Pr = Pr_c + Pr_l
\end{eqnarray}

In the sea ice area (\(L=1\)), the surface temperature \(T_s\) is the
sea ice surface temperature \(T_{ice}\). However, if \(T_{ice}\) is
higher than \(T_{melt}=0\), then \(T_{melt}\) is used.

\begin{eqnarray}
    T_s = min(T_{ice},T_{melt})
\end{eqnarray}

The sea ice bottom temperature \(T_b\) is assumed to be the ocean
surface temperature \(T_{o(1)}\).

\begin{eqnarray}
    T_b = T_{o(1)}
\end{eqnarray}

The amount of sea ice \(W_{ice}\) and the amount of snow on it
\(W_{snow}\) are converted per unit area by considering \(R_{ice}\) and
used in the calculation. However, a limiter \(\epsilon\) is provided to
prevent the values from becoming too small.

\begin{eqnarray}
R_{ice} =\mathrm{max}( R_{ice,orginal}, \epsilon)
\end{eqnarray}

In the ice-free region (\(L=2\)), the surface temperature \(T_s\) and
sea ice bottom temperature \(T_b\) are assumed to be the ocean surface
temperature \(T_{o(1)}\).

\begin{eqnarray}
    T_s = T_b = T_{o(1)}
\end{eqnarray}

The evaporation coefficient is assumed to be \(GRBET=1\) for both
\(L=1 and 2\).

If the sea ice concentration \(R_{ice}\) is not given, it can be
diagnosed simply from the sea ice volume \(W_{ice}\).

\begin{eqnarray}
R_{ice} = \mathrm{min}\Big(\sqrt{\frac{\mathrm{max}(W_{ice},0)}{W_{ice,c}}},1.0\Big)
\end{eqnarray}

The standard gives the amount of sea ice per area as
\(W_{ice,c}=300 \mathrm{[kg/m^2]}\).

\hypertarget{ocean-surface-conditions-ocnbcs}{%
\subsubsection{\texorpdfstring{1.3.2 Ocean Surface Conditions
\texttt{{[}OCNBCS{]}}}{1.3.2 Ocean Surface Conditions {[}OCNBCS{]}}}\label{ocean-surface-conditions-ocnbcs}}

\begin{itemize}
\tightlist
\item
  Output variables
\end{itemize}

\setlength\LTleft{0pt}\setlength\LTright{0pt}\begin{longtable}[]{@{}lll@{}}
\toprule\relax
variable & Presentation & Meaning \\ \addlinespace
\midrule\relax
\endhead
GRALB & \(\alpha\) & surface albedo \\ \addlinespace
GRZ0 & -- & surface roughness \\ \addlinespace
GFLUXS & \(G\) & heat flux \\ \addlinespace
DGFDS & \(\frac{\partial G}{\partial T_s}\) & dG/dTs \\ \addlinespace
\bottomrule
\end{longtable}

\begin{itemize}
\tightlist
\item
  Input variables
\end{itemize}

\setlength\LTleft{0pt}\setlength\LTright{0pt}\begin{longtable}[]{@{}lll@{}}
\toprule\relax
Variable & Presentation & Meaning \\ \addlinespace
\midrule\relax
\endhead
GRTS & \(T_s\) & skin temperature \\ \addlinespace
GRTB & \(T_{b,ice}\) & ice base temp. \\ \addlinespace
GRICE & -- & sea ice \\ \addlinespace
GRSNW & -- & snow smount \\ \addlinespace
GRICR & \(R_{ice}\) & ice fraction \\ \addlinespace
GDUA & \(U_0\) & u sfc wind \\ \addlinespace
GDVA & \(V_0\) & v sfc wind \\ \addlinespace
RCOSZ & -- & cos(sol.zenith) \\ \addlinespace
\bottomrule
\end{longtable}

\hypertarget{albedo}{%
\paragraph{1.3.2.1 Albedo}\label{albedo}}

In this module, surface albedo and roughness are calculated. They are
calculated supposing ice-free conditions, then modified.

First, let us consider the sea albedo. The sea level \(\alpha_{(d,b)}\),
\(b=1,2,3\) represent the visible, near-infrared, and infrared
wavelength bands, respectively. Also, \(d=1,2\) represents direct and
scattered light, respectively.

\begin{enumerate}
\def\labelenumi{\arabic{enumi}.}
\tightlist
\item
  Sea Surface Albedo for Visible \texttt{{[}SEAALB{]}}
\end{enumerate}

\begin{itemize}
\tightlist
\item
  Internal parameters
\end{itemize}

\setlength\LTleft{0pt}\setlength\LTright{0pt}\begin{longtable}[]{@{}lllll@{}}
\toprule\relax
Meaning & Presentation & Variable & unit & value \\ \addlinespace
\midrule\relax
\endhead
-- & \(C_1, C_2, C_3\) & CC & {[}-{]} &
\(-0.7479, -4.677039, 1.583171\) \\ \addlinespace
\bottomrule
\end{longtable}

For sea surface level albedo \(\alpha_{L(d)}\), \(d=1,2\) represents
direct and scattered light, respectively.

Using the solar zenith angle at latitude \(\theta\), the albedo for
direct light is presented by

\begin{eqnarray}
    \alpha_{L(1)} = e^{(C_3A^* + C_2) A^* +C_1}
\end{eqnarray}

where

\begin{eqnarray}
A = \mathrm{min}(\mathrm{max}(\mathrm{cos}(\theta),0.03459),0.961)
\end{eqnarray}

On the other hand, the albedo for scattered light is uniformly set to a
constant parameter.

\begin{eqnarray}
    \alpha_{L(2)} = 0.06
\end{eqnarray}

\begin{enumerate}
\def\labelenumi{\arabic{enumi}.}
\setcounter{enumi}{1}
\tightlist
\item
  Sea Surface Albedo for Near-Infrared and Infrared
\end{enumerate}

The albedo for near-infrared is set to same as the visible one.

\begin{eqnarray}
    \alpha_{1,2} =\alpha_{1,1}
\end{eqnarray}

\begin{eqnarray}
    \alpha_{2,2} =\alpha_{2,1}
\end{eqnarray}

The albedo for infrared is uniformly set to a constant value.

\begin{enumerate}
\def\labelenumi{\arabic{enumi}.}
\setcounter{enumi}{2}
\tightlist
\item
  Albedo modification by ice
\end{enumerate}

The grid-averaged albedo, taking into account the sea ice concentration
\(R_{ice}\), is

\begin{eqnarray}
    \alpha = \alpha -R_{ice} \alpha_{ice}
\end{eqnarray}

\(\alpha_{ice}\) is given by the standard as
\(\alpha_{ice,1}=0.5,\alpha_{ice,2}=0.5,\alpha_{ice,3}=0.05\). 4.

\begin{enumerate}
\def\labelenumi{\arabic{enumi}.}
\setcounter{enumi}{3}
\tightlist
\item
  albedo modification by snow
\end{enumerate}

In addition, we want to consider the effect of snow cover. Here, we
consider the albedo modification by temperature. The standard threshold
values for snow temperature are \(T_{al,2}=258.15 \mathrm{[K]}\) and
\(T_{al,1}=273.15 \mathrm{[K]}\). The snow albedo changes linearly with
temperature change from \(\alpha_{snow,1}=0.75\) to
\(\alpha_{ snow,2}\). Let the coefficient \(\tau_{snow}\), which is
\(0\le \tau \le 1\).

\begin{eqnarray}
\tau_{snow} = \frac{T_s - T_{al,1}}{T_{al,2}-T_{al,1}}
\end{eqnarray}

Update the snow albedo \(\alpha_{snow}\) as

\begin{eqnarray}
    \alpha_{snow} = \alpha_{snow,0} + \tau_{snow}(\alpha_{snow,2}-\alpha_{snow,1})
\end{eqnarray}

\hypertarget{roughness}{%
\paragraph{1.3.2.2 Roughness}\label{roughness}}

\begin{enumerate}
\def\labelenumi{\arabic{enumi}.}
\tightlist
\item
  Sea Surface Roughness \texttt{{[}SEAZ0F{]}}
\end{enumerate}

The roughness variation of the sea surface is determined by the friction
velocity \(u^\star\)

\begin{eqnarray}
u^{\star} = \sqrt{C_{M_0} ({U_0}^2  +{V_0}^2)}
\end{eqnarray}

We perform successive approximation calculation of \({C_{M_0}}\),
because \(F_u,F_v,F_\theta,F_q\) are required.

\begin{eqnarray}
    r_{0,M} = z_{0,M_0} + z_{0,M_R} + \frac{z_{0,M_R} {u^\star }^2 }{g} + \frac{z_{0,M_S}\nu }{u^\star}
\end{eqnarray}

\begin{eqnarray}
    r_{0,H} = z_{0,H_0} + z_{0,H_R} + \frac{z_{0,H_R} {u^\star }^2 }{g} + \frac{z_{0,H_S}\nu }{u^\star}
\end{eqnarray}

\begin{eqnarray}
    r_{0,E} = z_{0,E_0} + z_{0,E_R} + \frac{z_{0,E_R} {u^\star }^2 }{g} + \frac{z_{0,E_S}\nu }{u^\star}
\end{eqnarray}

Here, \(\nu = 1.5 \times 10^{-5} \mathrm{[m^2/s]}\) is the kinetic
viscosity of the atmosphere. \(z_{0,M},z_{0,H}\) and \(z_{0,E}\) are
surface roughness for momentum, heat, and vapor, respectively.
\(z_{0,M_0},z_{0,H_0}\) and \(z_{0,E_0}\) are base, and rough factor
(\(z_{0,M_R},z_{0,M_R}\) and \(z_{0,E_R}\)) and smooth factor
(\(z_{0,M_S},z_{0,M_S}\) and \(z_{0,E_S}\)) are taken into account.

\begin{enumerate}
\def\labelenumi{\arabic{enumi}.}
\setcounter{enumi}{1}
\tightlist
\item
  Roughness modification by ice
\end{enumerate}

When the sea ice exists (\(L=1\)),

\begin{eqnarray}
    z_{0,M} = z_{0,M} + ( z_{0,ice,M} - z_{0,M})  \alpha_{ice}
\end{eqnarray}

\begin{eqnarray}
    z_{0,H} = z_{0,H} + ( z_{0,ice,H} - z_{0,H})  \alpha_{ice}
\end{eqnarray}

\begin{eqnarray}
    z_{0,E} = z_{0,E} + ( z_{0,ice,E} - z_{0,E})  \alpha_{ice}
\end{eqnarray}

Here, \(r_{0,ice,*}\) is roughness of sea ice, \(\alpha_{ice}\) is the
sea ice concentration.

\begin{enumerate}
\def\labelenumi{\arabic{enumi}.}
\setcounter{enumi}{2}
\tightlist
\item
  Roughness modification by snow
\end{enumerate}

When the snow even exists,

\begin{eqnarray}
    z_{0,M} = z_{0,M} + ( z_{0,snow,M} - z_{0,M})  \alpha_{snow}
\end{eqnarray}

\begin{eqnarray}
    z_{0,H} = z_{0,H} + ( z_{0,snow,H} - z_{0,H})  \alpha_{snow}
\end{eqnarray}

\begin{eqnarray}
    z_{0,E} = z_{0,E} + ( z_{0,snow,E} - z_{0,E})  \alpha_{snow}
\end{eqnarray}

Here, \(r_{0,snow,*}\) is roughness of sea ice, \(\alpha_{snow}\) is the
sea ice concentration.

\hypertarget{sea-surface-heat-flux}{%
\paragraph{1.3.2.3 Sea Surface heat flux}\label{sea-surface-heat-flux}}

\begin{enumerate}
\def\labelenumi{\arabic{enumi}.}
\tightlist
\item
  Conductivity of ice
\end{enumerate}

When sea ice exists (\(L=1\)), the thermal conductivity
\(k_{ice}^\star\) of sea ice is obtained by using \(D_{f,ice}\) (thermal
diffusivity of sea ice) and sea ice density \(\sigma_{ice}\).

\begin{eqnarray}
k_{ice}^\star = \frac{D_{f,ice}}{\mathrm{max}(R_{ice}/\sigma_{ice}, \epsilon)}
\end{eqnarray}

\begin{enumerate}
\def\labelenumi{\arabic{enumi}.}
\setcounter{enumi}{1}
\tightlist
\item
  conductivity modification by snow
\end{enumerate}

The calculated thermal conductivity is modified to \(k_{ice}\) to take
into account that it varies with snow cover.

\begin{eqnarray}
h_{snow} = \mathrm{min}(
    \mathrm{max}(
    R_{snow}/\sigma_{snow}),\epsilon
        ),h_{snow,max}
        )
\end{eqnarray}

\begin{eqnarray}      
k_{ice} = k_{ice}^\star (1-R_{ice}) + \frac{D_{ice}}{1+\| D_{ice}/D_{snow} \cdot h_{snow} \|} R_{ice}
\end{eqnarray}

where \(h_{snow}\) is the snow depth, \(R_{snow}\) is the snow area
fraction, \(\sigma_{snow}\) is the snow density, \(h_{snow,max}\) is the
maximum snow depth, and \(D_{snow}\) is the thermal diffusivity of snow.
3.

\begin{enumerate}
\def\labelenumi{\arabic{enumi}.}
\setcounter{enumi}{2}
\tightlist
\item
  calculate flux and its derivative
\end{enumerate}

Therefore, the heat conduction flux and its derivative are

\begin{eqnarray}
 G = k_{ice} (T_b - T_s)
\end{eqnarray}

\begin{eqnarray}
 \frac{\partial G}{\partial T} = k_{ice}
\end{eqnarray}

Note that in the ice-free region (\(L=2\))

\begin{eqnarray}
G=k_{ocn}
\end{eqnarray}

where \(k_{ocn}\) is the heat flux in the ocean surface layer. Here,
\(k_{ocn}\) is the heat flux in the ocean surface layer.

\hypertarget{surface-flux-ux6e08-1ux6708}{%
\subsection{1.4 Surface Flux {[}済
1月{]}}\label{surface-flux-ux6e08-1ux6708}}

\hypertarget{overview}{%
\subsubsection{1.4.1 Overview}\label{overview}}

The surface flux scheme evaluates the physical quantity fluxes between
the atmospheric surfaces due to turbulent transport in the boundary
layer. The main input \textbar{} are wind speed (\(u, v\)), temperature
(\(T\)), and specific humidity (\(q\)), and the output \textbar{} are
the vertical fluxes and the differential values (for obtaining implicit
solutions) of momentum, heat, and water vapor.

The bulk coefficients are obtained according to
\href{./papers/Louis1979_Article_AParametricModelOfVerticalEddy.pdf}{Louis
(1979)} and
\href{./papers/Louis1982_a_short_history_of_the_operational_pbl_parameterization_at_ecmwf.pdf}{Louis
{\emph{et al.}}(1982)}, except for the correction for the difference in
roughness between momentum and heat. However, corrections are made to
take into account the difference between momentum and heat roughness.

The outline of the calculation procedure is as follows.

\begin{enumerate}
\def\labelenumi{\arabic{enumi}.}
\tightlist
\item
  Calculate the roughness including modifications by ice and snow.
  \texttt{MODULE:{[}SEAZ0F{]}}
\item
  Calculate the Richardson number as the stability of the atmosphere.
  \texttt{MODULE:{[}PSFCL{]}}
\item
  Calculate the bulk coefficient from Richardson number.
  \texttt{MODULE:{[}PSFCL{]}}
\item
  Calculate the flux and its derivative from the bulk coefficient.
  \texttt{MODULE:{[}PSFCM{]}}
\item
  If necessary, the calculated fluxes are re-calculated after taking
  into account the roughness effect, the free flow effect, and the wind
  speed correction.
\end{enumerate}

\hypertarget{basic-formula-for-flux-calculations}{%
\paragraph{1.4.2 Basic Formula for Flux
Calculations}\label{basic-formula-for-flux-calculations}}

Surface fluxes (\(F_u, F_v, F_\theta, F_q\)) are expressed using the
bulk coefficients (\(C_M, C_H, C_E\)) as follows

\begin{eqnarray}
    F_u  =  - \rho C_M |{\mathbf{v}}| u
\end{eqnarray}

\begin{eqnarray}
    F_v  =  - \rho C_M |{\mathbf{v}}| v
\end{eqnarray}

\begin{eqnarray}
    F_\theta  = \rho c_p C_H |{\mathbf{v}}| ( \theta_g - \theta )
\end{eqnarray}

\begin{eqnarray}
    F_q^P =  \rho C_E |{\mathbf{v}}| ( q_g - q )
\end{eqnarray}

Note that \(F_q^P\) is the possible evaporation flux.

\hypertarget{roughness-sea0f}{%
\subsubsection{\texorpdfstring{1.4.2 Roughness
\texttt{{[}SEA0F{]}}}{1.4.2 Roughness {[}SEA0F{]}}}\label{roughness-sea0f}}

\hypertarget{variables}{%
\paragraph{1.4.2.1 variables}\label{variables}}

\begin{itemize}
\tightlist
\item
  Output Variables
\end{itemize}

\setlength\LTleft{0pt}\setlength\LTright{0pt}\begin{longtable}[]{@{}lll@{}}
\toprule\relax
Variable & Presentation & Meaning \\ \addlinespace
\midrule\relax
\endhead
GRZ0M & \(r_{0,M}\) & surface roughness (V) \\ \addlinespace
GRZ0H & \(r_{0,H}\) & surface roughness (T) \\ \addlinespace
GRZ0E & \(r_{0,E}\) & surface roughness (q) \\ \addlinespace
\bottomrule
\end{longtable}

\begin{itemize}
\tightlist
\item
  Input variables
\end{itemize}

\setlength\LTleft{0pt}\setlength\LTright{0pt}\begin{longtable}[]{@{}lll@{}}
\toprule\relax
Variable & Presentation & Meaning \\ \addlinespace
\midrule\relax
\endhead
USFC & \(U_0\) & u sfc wind speed \\ \addlinespace
VSFC & \(V_0\) & v sfc wind speed \\ \addlinespace
\bottomrule
\end{longtable}

\begin{itemize}
\tightlist
\item
  Internal variables
\end{itemize}

\setlength\LTleft{0pt}\setlength\LTright{0pt}\begin{longtable}[]{@{}lll@{}}
\toprule\relax
Variable & Presentation & Meaning \\ \addlinespace
\midrule\relax
\endhead
USTAR & \(u^\star\) & friction velocity \\ \addlinespace
\bottomrule
\end{longtable}

\begin{itemize}
\tightlist
\item
  Internal parameters
\end{itemize}

\setlength\LTleft{0pt}\setlength\LTright{0pt}\begin{longtable}[]{@{}lllr@{}}
\toprule\relax
Variable & Presentation & Meaning & Values \\ \addlinespace
\midrule\relax
\endhead
Z0M0 & \(r_{0,M_0}\) & base & \(0\) \\ \addlinespace
Z0MR & \(r_{0,M_R}\) & rough factor & $ 0.18 $ \\ \addlinespace
Z0MS & \(r_{0,M_S}\) & smooth factor & $ 0.11 $ \\ \addlinespace
Z0H0 & \(r_{0,H_0}\) & base & $ 1.4\^{}-5 $ \\ \addlinespace
Z0HR & \(r_{0,H_R}\) & rough factor & $ 0.0 $ \\ \addlinespace
Z0HS & \(r_{0,H_S}\) & smooth factor & $ 0.4 $ \\ \addlinespace
Z0E0 & \(r_{0,E_0}\) & base & $ 1.3\^{}-4 $ \\ \addlinespace
Z0ER & \(r_{0,E_R}\) & rough factor & $ 0.0 $ \\ \addlinespace
Z0ES & \(r_{0,E_S}\) & smooth factor & $ 0.62 $ \\ \addlinespace
VISAIR & \(\nu\) & kinematic viscosity & $ 1.5\^{}-5
$ \\ \addlinespace
CM0 & \(C_{M_0}\) & bulk coef for \(u^\star\) & $ 1.0\^{}-3
$ \\ \addlinespace
USTRMN & -- & min(\(u^\star\)) & $ 1.0\^{}-3 $ \\ \addlinespace
Z0MMIN & -- & minimum & $ 3.0\^{}-5 $ \\ \addlinespace
Z0HMIN & -- & minimum & $ 3.0\^{}-5 $ \\ \addlinespace
Z0EMIN & -- & minimum & $ 3.0\^{}-5 $ \\ \addlinespace
\bottomrule
\end{longtable}

\hypertarget{richardson-number-psfcl}{%
\subsubsection{\texorpdfstring{1.4.3 Richardson Number
\texttt{{[}PSFCL{]}}}{1.4.3 Richardson Number {[}PSFCL{]}}}\label{richardson-number-psfcl}}

The bulk Richardson number (\(R_{iB}\)), which is used as a benchmark
for the stability between the atmospheric surfaces, is

\begin{eqnarray}
R_{iB} =
            \frac{ \frac{g}{\theta_s} (\theta_1 - \theta(z_0))/z_1 }
              { (u_1/z_1)^2                                  }
       = \frac{g}{\theta_s}
         \frac{T_1 (p_s/p_1)^\kappa - T_0 }{u_1^2/z_1} f_T
\end{eqnarray}

Here, \(g\) is the gravitational accerelation, \(\theta_s\)
(\(\Theta_0\) in MATSIRO description) is the basic potential
temperature, \(T_1\) is the atmospheric temperature of the 1st layer,
\(T_0\) is the surface skin temperature, \(p_s\) is the surface
pressure, \(p_1\) is the pressure of the 1st layer, $\kappa $ is the
Karman constant, and

\begin{eqnarray}
f_T = (\theta_1 - \theta(z_0))/(\theta_1 - \theta_0)
\end{eqnarray}

is a correction factor, which is approximated from the uncorrected bulk
Richardson number, but we abbreviate the calculation here.

\hypertarget{bulk-factor-blkcof}{%
\subsubsection{\texorpdfstring{1.4.4 Bulk factor
\texttt{{[}BLKCOF{]}}}{1.4.4 Bulk factor {[}BLKCOF{]}}}\label{bulk-factor-blkcof}}

The bulk coefficients of \(C_M,C_H,C_E\) are calculated according to
\href{./papers/Louis1979_Article_AParametricModelOfVerticalEddy.pdf}{Louis
(1979)} and
\href{./papers/Louis1982_a_short_history_of_the_operational_pbl_parameterization_at_ecmwf.pdf}{Louis
{\emph{et al.}}(1982)}. However, corrections are made to take into
account the difference between momentum and heat roughness. If the
roughnesses for momentum, heat, and water vapor are set to
\(z_{0,M}, z_{0,H}, z_{0,E}\), respectively, the results are generally
\(z_{0,M} > z_{0,H}, z_{0,E}\), but the bulk coefficients for heat and
water vapor for the fluxes from the height of \(z_{0,M}\) are also set
to \(\widetilde{C_H}\), \(\widetilde{C_E}\) first, and then corrected.

\begin{eqnarray}
    C_M = \left\{
      \begin{array}{lr}
      C_{0,M} [ 1 + (b_M/e_M)  R_{iB} ]^{-e_M}
            &,
          R_{iB} \geq 0 \\
      C_{0,M} \left[ 1 - b_M R_{iB} \left( 1+ d_M b_M C_{0,M}
                                  \sqrt{\frac{z_1}{z_{0,M}}| R_{iB}|} \,
                                  \right)^{-1} \right]     
          &,
          R_{iB} < 0 \\
      \end{array} \right.
\end{eqnarray}

\begin{eqnarray}
    \widetilde{C_H} = \left\{
      \begin{array}{lr}
      \widetilde{C_{0,H}} [ 1 + (b_H/e_H) R_{iB} ]^{-e_H}
            &,
          R_{iB} \geq 0 \\
      \widetilde{C_{0,H}} \left[ 1 - b_H R_{iB}
                                  \left( 1+ d_H b_H \widetilde{C_{0,H}}
                                  \sqrt{\frac{z_1}{z_{0,M}}| R_{iB}|} \,
                                  \right)^{-1} \right]
             &,     
          R_{iB} < 0 \\
      \end{array} \right.
\end{eqnarray}

\begin{eqnarray}
    C_H = \widetilde{C_H} f_T
\end{eqnarray}

\begin{eqnarray}
    \widetilde{C_E} = \left\{
      \begin{array}{lr}
      \widetilde{C_{0,E}} [ 1 + (b_E/e_E) R_{iB} ]^{-e_E}
            &,
          R_{iB} \geq 0 \\
      \widetilde{C_{0,E}} \left[ 1 - b_E R_{iB}
                                  \left( 1+ d_E b_E \widetilde{C_{0,E}}
                                  \sqrt{\frac{z_1}{z_{0,M}}| R_{iB}|} \,
                                  \right)^{-1} \right]      
          &,
          R_{iB} < 0 \\
      \end{array} \right.
\end{eqnarray}

\begin{eqnarray}
    C_E = \widetilde{C_E} f_q
\end{eqnarray}

\(C_{0M}, \widetilde{C_{0H}}, \widetilde{C_{0E}}\) is the bulk
coefficient (for fluxes from \(z_{0M}\)) at neutral,

\begin{eqnarray}
    C_{0M}  =  \widetilde{C_{0H}}  =  \widetilde{C_{0E}}  =
       \frac{k^2}{\left[\ln \left(\frac{z_1}{z_{0M}}\right)\right]^2 }
\end{eqnarray}

Correction Factor \(f_q\) is ,

\begin{eqnarray}
  f_q = (q_1 - q(z_0))/(q_1 - q^{\ast}(\theta_0))
\end{eqnarray}

but the method of calculation is omitted. The coefficients of Louis
factors are \(( b_M, d_M, e_M ) = ( 9.4, 7.4, 2.0 )\),
\(( b_H, d_H, e_H ) = ( b_E, d_E, e_E ) = ( 9.4, 5.3, 2.0 )\).

\hypertarget{calculation-of-surface-turbulent-fluxes-psfcm}{%
\subsubsection{\texorpdfstring{1.4. Calculation of surface turbulent
fluxes
\texttt{{[}PSFCM{]}}}{1.4. Calculation of surface turbulent fluxes {[}PSFCM{]}}}\label{calculation-of-surface-turbulent-fluxes-psfcm}}

\begin{itemize}
\tightlist
\item
  Modified variables
\end{itemize}

\setlength\LTleft{0pt}\setlength\LTright{0pt}\begin{longtable}[]{@{}lll@{}}
\toprule\relax
Variable & Presentation & Meaning \\ \addlinespace
\midrule\relax
\endhead
UFLUXS & -- & flux of U \\ \addlinespace
VFLUXS & -- & flux of V \\ \addlinespace
TFLUXS & -- & flux of T \\ \addlinespace
QFLUXS & -- & flux of q \\ \addlinespace
\bottomrule
\end{longtable}

\begin{itemize}
\tightlist
\item
  Output variables
\end{itemize}

\setlength\LTleft{0pt}\setlength\LTright{0pt}\begin{longtable}[]{@{}lll@{}}
\toprule\relax
variable & Presentation & Meaning \\ \addlinespace
\midrule\relax
\endhead
DUFDU & -- & -d(tau)/du \\ \addlinespace
DTFDT & -- & -dH/dTa \\ \addlinespace
DQFDQ & -- & -dLE/dqa \\ \addlinespace
DTFDS & -- & dH/dTs \\ \addlinespace
DQFDS & -- & dLE/dTs \\ \addlinespace
CDVE & -- & bulk coef. \\ \addlinespace
RM10 & -- & coef. for 10m u \\ \addlinespace
RH2 & -- & coef. for 2m T \\ \addlinespace
RE2 & -- & coef. for 2m q \\ \addlinespace
U10 & -- & 10m u \\ \addlinespace
V10 & -- & 10m v \\ \addlinespace
T2 & -- & 2m T \\ \addlinespace
Q2 & -- & 2m q \\ \addlinespace
\bottomrule
\end{longtable}

\begin{itemize}
\tightlist
\item
  Input variables
\end{itemize}

\setlength\LTleft{0pt}\setlength\LTright{0pt}\begin{longtable}[]{@{}lll@{}}
\toprule\relax
Meaning & Presentation & Variable \\ \addlinespace
\midrule\relax
\endhead
westerly u & -- & GDUA \\ \addlinespace
southern wind v & -- & GDVA \\ \addlinespace
temperature T & -- & GDTA \\ \addlinespace
humidity q & -- & GDQA \\ \addlinespace
pressure (lev=1) & -- & GDPA \\ \addlinespace
surface pressure & -- & GDPS \\ \addlinespace
surface skin temperature & -- & GDTS \\ \addlinespace
surface roughness & -- & GRZ0 \\ \addlinespace
soil wetness & -- & GRBET \\ \addlinespace
ocean u & -- & GRUA \\ \addlinespace
ocean v & -- & GRVA \\ \addlinespace
\bottomrule
\end{longtable}

The turbulent fluxes at the sea surface are solved by bulk formulae as
follows. Then, by solving the surface energy balance, the ground surface
temperature (\(T_s\)) is updated, and the surface flux values with
respect to those values are also updated. The solutions obtained here
are temporary values. In order to solve the energy balance by
linearizing with respect to \(T_s\), the differential with respect to
\(T_s\) of each flux is calculated beforehand.

\begin{itemize}
\tightlist
\item
  Momentum flux
\end{itemize}

\begin{eqnarray}
 \tau_x = - \rho C_{M}|V_a| u_a
\end{eqnarray}

\begin{eqnarray}
 \tau_y = - \rho C_{M}|V_a| v_a
\end{eqnarray}

where \(\tau_x\) and \(\tau_y\) are the momentum fluxes (surface stress)
of the zonal and meridional directions, respectively.

\begin{itemize}
\tightlist
\item
  Sensible heat flux
\end{itemize}

\begin{eqnarray}
 H_s = c_p \rho C_{Hs}|V_a| (T_s - (P_s/P_a)^{\kappa}T_a)
\end{eqnarray}

where \(H_s\) is the sensible heat flux from the sea surface;
\(\kappa = R_{air} / c_p\) and \(R_{air}\)are the gas constants of air;
and \(c_p\) is the specific heat of air.

\begin{itemize}
\tightlist
\item
  Bare sea surface evaporation flux
\end{itemize}

\begin{eqnarray}
\hat{F}q^P_{1/2} = \rho_{1/2} C_E |{\mathbf{v}}_1| \left( q^\star(T_0) - q_1 \right)
\end{eqnarray}

\hypertarget{radiation-flux-at-sea-surface-radsfc-ux6e08-1ux6708}{%
\subsection{\texorpdfstring{1.5 Radiation Flux at Sea Surface
\texttt{{[}RADSFC{]}} {[}済
1月{]}}{1.5 Radiation Flux at Sea Surface {[}RADSFC{]} {[}済 1月{]}}}\label{radiation-flux-at-sea-surface-radsfc-ux6e08-1ux6708}}

For the ground surface albedo \(\alpha_{(d,b)}\), \(b=1,2\) represent
the visible and near-infrared wavelength bands, respectively. Also,
\(d=1,2\) are direct and scattered, respectively. For the downward
shortwave radiation \(SW^\downarrow\) and upward shortwave radiation
\(SW^\uparrow\) incident on the earth's surface, the direct and
scattered light together are

\begin{eqnarray}
    SW^\downarrow = SW^\downarrow_{(1,1)}+SW^\downarrow_{(1,2)}+SW^\downarrow_{(2,1)}+SW^\downarrow_{(2,2)}
\end{eqnarray}

\begin{eqnarray}
SW^\uparrow = SW^\downarrow_{(1,1)}\cdot\alpha_{(1,1)}+SW^\downarrow_{(1,2)}\cdot\alpha_{(1,2)}+SW^\downarrow_{(2,1)}\cdot\alpha_{(2,1)}+SW^\downarrow_{(2,2)}\cdot\alpha_{(2,2)}
\end{eqnarray}

\hypertarget{surface-heat-balance-ocnslv-ux6e08-1ux6708}{%
\subsection{\texorpdfstring{1.6 Surface Heat Balance
\texttt{{[}OCNSLV{]}} {[}済
1月{]}}{1.6 Surface Heat Balance {[}OCNSLV{]} {[}済 1月{]}}}\label{surface-heat-balance-ocnslv-ux6e08-1ux6708}}

The comments for some variables say ``soil'', but this is because the
program was adapted from a land surface scheme, and has no particular
meaning.

\begin{itemize}
\tightlist
\item
  Outputs
\end{itemize}

\setlength\LTleft{0pt}\setlength\LTright{0pt}\begin{longtable}[]{@{}lllll@{}}
\toprule\relax
Meaning & Presentation & Variable & dimension & unit \\ \addlinespace
\midrule\relax
\endhead
surface water flux *1 & \(W_{free/ice}\) & WFLUXS & IJLSDM,2 &
-- \\ \addlinespace
upward long wave & \(LW^\uparrow\) & RFLXLU & IJLSDM &
-- \\ \addlinespace
flux balance & \(F\) & SFLXBL & IJLSDM & -- \\ \addlinespace
\bottomrule
\end{longtable}

\begin{itemize}
\tightlist
\item
  Inputs variables
\end{itemize}

\setlength\LTleft{0pt}\setlength\LTright{0pt}\begin{longtable}[]{@{}lll@{}}
\toprule\relax
Meaning & Presentation & Variable \\ \addlinespace
\midrule\relax
\endhead
sensible heat flux coefficent & \(\frac{\partial H}{\partial T_s}\) &
DTFDS \\ \addlinespace
latent heat flux coefficient & \(\frac{\partial E}{\partial T_s}\) &
DQFDS \\ \addlinespace
surface heat flux coefficient & \(\frac{\partial G}{\partial T_s}\) &
DGFDS \\ \addlinespace
downward SW radiation & \(SW^\downarrow\) & RFLXSD \\ \addlinespace
upward SW radiation & \(SW^\uparrow\) & RFLXLU \\ \addlinespace
downward LW radiation & \(LW^\downarrow\) & RFLXLD \\ \addlinespace
sea surface albedo & \(\alpha\) & GRALBL \\ \addlinespace
sea ice concentration & \(R_{ice}\) & GRICR \\ \addlinespace
\bottomrule
\end{longtable}

\begin{itemize}
\tightlist
\item
  Modified in this subroutine
\end{itemize}

\setlength\LTleft{0pt}\setlength\LTright{0pt}\begin{longtable}[]{@{}lllll@{}}
\toprule\relax
Meaning & Presentation & Variable & dimension & unit \\ \addlinespace
\midrule\relax
\endhead
skin temperature & \(T_s\) & GDTS & IJLSDM & -- \\ \addlinespace
surface heat flux from \texttt{seaBC} & \(G\) & GFLUXS & IJLSDM &
-- \\ \addlinespace
sensible heat flux & \(H\) & TFLUXS & IJLSDM & -- \\ \addlinespace
latent heat flux & \(E\) & QFLUXS & IJLDSM & -- \\ \addlinespace
\bottomrule
\end{longtable}

\begin{itemize}
\tightlist
\item
  Others (appeared in texts)
\end{itemize}

\setlength\LTleft{0pt}\setlength\LTright{0pt}\begin{longtable}[]{@{}lllll@{}}
\toprule\relax
Meaning & Presentation & Variable & dimension & unit \\ \addlinespace
\midrule\relax
\endhead
sea surface albedo for shortwave radiation (ice-free) & \(\alpha_S\) &
-- & {[}-{]} & -- \\ \addlinespace
the Stefan-Boltzmann constant & \(\sigma\) & STB & -- &
-- \\ \addlinespace
\bottomrule
\end{longtable}

Reference:
\href{https://ccsr.aori.u-tokyo.ac.jp/~hasumi/COCO/coco4.pdf}{Hasumi,
2015, Appendices A}

Downward radiative fluxes are not directly dependent on the condition of
the sea surface, and their observed values are simply specified to drive
the model. Shotwave emission from the sea surface is negligible, so the
upward part of the shortwave radiative flux is accounted for solely by
reflection of the incoming downward flux. Let \(\alpha _S\) be the sea
surface albedo for shortwave radiation. The upward shortwave radiative
flux is represented by

\begin{eqnarray}
    SW^\uparrow = - \alpha_S SW^\downarrow
\end{eqnarray}

On the other hand, the upward longwave radiative flux has both
reflection of the incoming flux and emission from the sea surface. Let
\(\alpha\) be the sea surface albedo for longwave radiation and
\(\epsilon\) be emissivity of the sea surface relative to the black body
radiation. The upward shortwave radiative flux is represented by

\begin{eqnarray}
    LW^\uparrow = - \alpha LW^\downarrow + \epsilon \sigma T_s ^4
\end{eqnarray}

where \(\sigma\) is the Stefan-Boltzmann constant and \(T_s\) is skin
temperature. If sea ice exists, snow or sea ice temperature is
considered by fractions. When radiative equilibrium is assumed,
emissivity becomes identical to co-albedo:

\begin{eqnarray}
    \epsilon = 1 - \alpha
\end{eqnarray}

The net surface flux is presented by

\begin{eqnarray}
    F^*=H + (1-\alpha)\sigma T_s^4 + \alpha LW^\uparrow - LW^\downarrow +SW^\uparrow - SW^\downarrow        
\end{eqnarray}

The heat flux into the sea surface is presented, with the surface heat
flux calculated in \texttt{PSFCM}

\begin{eqnarray}
    G^* = G - F^*
\end{eqnarray}

Note that \(G^*\) is downward positive.

The temperature derivative term is

\begin{eqnarray}
    \frac{\partial G^*}{\partial T_s} = \frac{\partial G}{\partial T_s}+\frac{\partial H}{\partial T_s}+\frac{\partial R}{\partial T_s}
\end{eqnarray}

When the sea ice exists, the sublimation flux is considered

\begin{eqnarray}
    G_{ice} = G^* - l_s E
\end{eqnarray}

The temperature derivative term is

\begin{eqnarray}
    \frac{\partial G_{ice}}{\partial T_s}=\frac{\partial G^*}{\partial T_s} + l_s\frac{\partial E}{\partial T_s}
\end{eqnarray}

Finally, we can update the surface temperature with the sea ice
concentration with
\(\Delta T_s=G_{ice} ( \frac{\partial G_{ice}}{\partial T_s})^{-1}\)

\begin{eqnarray}
    T_s = T_s +R_{ice} \Delta T_s
\end{eqnarray}

Then, the sensible and latent heat flux on the sea ice is updated.

\begin{eqnarray}
    E_{ice} = E + \frac{\partial E}{\partial T_s}\Delta T_s
\end{eqnarray}

\begin{eqnarray}
    H_{ice} = H + \frac{\partial H}{\partial T_s}\Delta T_s
\end{eqnarray}

When the sea ice does not existed, otherwise, the evaporation flux is
added to the net flux.

\begin{eqnarray}
    G_{free}=F^* + l_cE
\end{eqnarray}

Finally each flux is updated.

For the sensible heat flux, the temperature change on the sea ice is
considered.

\begin{eqnarray}
    H=H+ R_{ice}  H_{ice}
\end{eqnarray}

Then, the heat used for the temperature change is saved.

\begin{eqnarray}
    F = R_{ice} H_{ice}
\end{eqnarray}

For the upward longwave radiative flux, the temperature change on the
sea ice is considered.

\begin{eqnarray}
    LW^\uparrow=LW^\uparrow +  4\frac{\sigma}{T_s}R_{ice}  \Delta T_s
\end{eqnarray}

For the surface heat flux, the sea ice concentration is considered.

\begin{eqnarray}
    G=(1-R_{ice})G_{free} + R_{ice}G_{ice}
\end{eqnarray}

For the latent heat flux, the sea ice concentration is considered.

\begin{eqnarray}
    E=(1-R_{ice})E + R_{ice}E_{ice}
\end{eqnarray}

Each term above are saved as freshwater flux.

\begin{eqnarray}
    W_{free} = (1-R_{ice}) E
\end{eqnarray}

\begin{eqnarray}
    W_{ice} = R_{ice} E_{ice}
\end{eqnarray}
