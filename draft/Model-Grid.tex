\hypertarget{configuration-of-model-grid}{%
\subsection{Configuration of model
grid}\label{configuration-of-model-grid}}

The atmospheric and oceanic models of MIROC are independent and run on
different computational nodes. The node where the atmospheric model is
executed is called the atmospheric node, and the node where the ocean
model is executed is called the ocean node. In the atmospheric node, the
land surface model, sea surface model, and river model are executed as
sub-models. Information exchange between the atmosphere and the oceans
is performed through the sea surface model in the atmosphere node.
Information such as sea surface temperature and sea ice concentration of
the ocean model in the ocean node is converted to the grid of the sea
surface model so that it can be treated as a boundary condition of the
model in the atmosphere node. On the other hand, the heat, freshwater,
and momentum fluxes calculated on the grid of the sea surface model in
the atmospheric node are converted to the grid of the ocean model and
sent to the ocean node. These series of data communication and
conversion are done by the exchanger. The flux coupler stores data such
as boundary conditions, heat and freshwater fluxes calculated by the sea
surface model and land surface model, and distributes them to each model
as needed. In general, the flux coupler also includes the function of
the exchanger, but this document describes it separately.

\hypertarget{horizontal-grid-of-model}{%
\subsection{Horizontal grid of model}\label{horizontal-grid-of-model}}

The horizontal grid of MIROC is defined as the atmospheric grid, the
land grid, the river grid, and the sea surface grid for each model in
the atmospheric node. The sea surface grid in the atmospheric node is
different from the horizontal grid of the ocean model in the ocean node.
The land surface grid and the sea surface grid are the horizontal grid
of the atmospheric model divided equally into north-south and east-west
directions. The number of divisions can be set arbitrarily for each
grid. However, the number of divisions for the sea surface grid must be
divisible by the number of divisions for the land surface model. The
river grid can be the same as the atmospheric grid or an equal
latitude/longitude interval grid. The horizontal grid of the ocean model
uses horizontal general curvilinear Cartesian coordinates, so it is not
necessary to use the same coordinate system as the atmospheric model.
The exchange of data between the atmospheric model and the ocean model
is performed by using an exchanger, which is prepared in advance with
information on the location, number, area, and vector rotation of the
ocean grid that overlaps with the sea surface grid of the atmospheric
model.

\hypertarget{definition-of-land-sea-distribution}{%
\subsection{Definition of land-sea
distribution}\label{definition-of-land-sea-distribution}}

The land-sea distribution in MIROC is prioritized by the land-sea
distribution defined by the ocean model. While one grid in the ocean
model is defined by land or sea only, the land and ocean grids in the
atmospheric model are determined in proportion to the land and sea to be
consistent with the ocean model's land-sea distribution.

\(SA\) : area of the atmospheric grid, \(SL _ {ij}\) : area of the land
grid, \(SO _ {ij}\) : area of the sea surface grid, \(FLND^{atm}\),
\(FLND^{land} _ {ij}\), \(FLND^{oc} _ {ij}\) : percentage of land
surface is occupied by each grid. Then, following equation is satisfied.

\begin{eqnarray} SA*FLND^{atm} = \sum _ {j=1}^{jldiv}\sum _ {i=1}^{ildiv}(SL _ {ij}*FLND^{land} _ {ij}) = \sum _ {j=1}^{jodiv}\sum _ {i=1}^{iodiv}(SO _ {ij}*FLND^{oc } _{ij}) \end{eqnarray}

where, (ildiv,jldiv) is the number of east-west and north-south
divisions of the land surface grid, and (iodiv,jodiv) is the number of
east-west and north-south divisions of the sea surface grid. In the land
surface grid, if even a small amount of land is defined to exist,
boundary values such as land cover are required.
