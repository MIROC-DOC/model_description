\hypertarget{introduction}{%
\section{Introduction}\label{introduction}}


\hypertarget{characteristics-of-ccsrnies-agcm}{%
\subsection{Characteristics of CCSR/NIES
AGCM}\label{characteristics-of-ccsrnies-agcm}}

\textbf{NOTE: the descriptions in this section are outdated.}

AGCM5.4 is a three-dimensional global general circulation model
developed by the Center for Climate System Research (CCSR) of the
University of Tokyo and the National Institute for Environmental Studies
(NIES). The features of the model are described below.

\begin{table}[hp]
\begin{tabular}{ll}
system of equations & hydrostatic primitive equation system \\
area & Global 3D \\
predictive variable & Horizontal wind speed, Temperature, Surface
pressure, Specific humidity, Cloud water content, Land surface
temperature, Soil moisture \\
horizontal discretization & spectral transformation technique \\
vertical discretization & σ-system (Arakawa and Suarez, 1983) \\
emission & 2-stream DOM/adding method (Based on Nakajima and Tanaka, 1986) \\
large-scale cloud process & Scheme with total water mixing ratio as a
forecast variable (Based on Le Treut and Li, 1991) \\
cumulus convection & Simplified Arakawa-Schubert scheme \\
vertical diffusion & Mellor and Yamada(1974) level2 \\
Surface flux & Louis(1979) Bulk type \\
 & Miller et
al. (1992) convection effects of stomatal resistance \\
surface thermal process & multi-layered heat conduction \\
& Surface Hydrological Processes \\
& bucket model \\
& Multilayer Water Transport \\
& Marine Mixed-Layer Coupling Model \\
gravitational wave resistance & Scheme based on McFarlane (1987) \\
\end{tabular}
\end{table}
