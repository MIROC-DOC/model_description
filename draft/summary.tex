\hypertarget{introduction}{%
\section{Introduction}\label{introduction}}

This document describes about the atmospheric general circulation
model(AGCM) of MIROC6 (the sixth version of MIROC or Model for
Interdisciplinary Research on Climate; Tatebe et al., 2019), which has
been cooperatively developed at the Center for Climate System Research
(CCSR; the precursor of a part of the Atmosphere and Ocean Research
Institute), the University of Tokyo, the Japan Agency for Marine-Earth
Science and Technology (JAMSTEC), and the National Institute for
Environmental Studies (NIES). All the descriptions are basically
corresponded with the source codes of MIROC6, specifically the version
used for CMIP6 DECK experiment, whereas the original document written in
Japanese by Dr.~Numaguti in 1995 was corresponded with those of
CCSR/NIES AGCM5.4.

\hypertarget{characteristics-of-miroc6-agcm}{%
\subsection{Characteristics of MIROC6
AGCM}\label{characteristics-of-miroc6-agcm}}

MIROC6 AGCM are summarized below.

\begin{itemize}
\tightlist
\item
  \textbf{System of equations}: Hydrostatic primitive equations
\item
  \textbf{Area}: Global 3D
\item
  \textbf{Prognostic variables}: Horizontal wind speed, temperature,
  surface pressure, specific humidity, cloud water
\item
  \textbf{Horizontal discretization}: Spectral transformation (Bourke,
  1988) method
\item
  \textbf{Vertical discretization}: Hybrid \(\sigma - p\) coordinate,
  based on Arakawa and Konor (1996)
\item
  \textbf{Resolution for default}: T85 (150 km), 81 levels up to 0.004
  hPa
\item
  \textbf{Time integration}: Essentially the leap frog scheme, with a
  time filter (Williams, 2009)
\item
  \textbf{Cumulus}: An entrainment plume model with multiple cloud types
  (Chikira and Sugiyama, 2010)
\item
  \textbf{Shallow convection}: A mass-flux-based single-plume model
  based on Park and Bretherton (2009)
\item
  \textbf{Large scale condensation \& Cloud microphysics}: A prognostic
  large scale condensation scheme (Watanabe et al., 2009) and the
  implementation of a bulk mirco-physical scheme (Wilson and Ballard,
  1999)
\item
  \textbf{Radiation}: k-distribution scheme (Sekiguchi and Nakajima,
  2008) with a hexagonal solid column as ice particle habit and extended
  mode radius of cloud particles
\item
  \textbf{Turbulence}: The Mellor-Yamada-Nakanishi-Niino scheme
  (Nakanishi 2001; Nakanishi and Niino 2004)'s level 2.5 closure scheme
\item
  \textbf{Surface flux}: Bulk coefficients (Louis, 1979; Louis et al.,
  1982) with convection effects at sea surface (Miller et al., 1992)
\item
  \textbf{Gravity wave drag}: An orographic gravity wave
  parameterization (McFarlane, 1987) with a non-orographic gravity wave
  parameterization (Hines, 1997; Watanabe et al., 2011)
\end{itemize}
