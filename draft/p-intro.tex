\hypertarget{physics}{%
\section{Physics}\label{physics}}

\hypertarget{overview-of-physical-processes.}{%
\subsection{Overview of Physical
Processes.}\label{overview-of-physical-processes.}}

\textbf{NOTE: the descriptions in this section are outdated.}

As a physical process, we can consider the following

\begin{itemize}
\item
  cumulus convection process
\item
  large-scale condensation process
\item
  radiation process
\item
  vertical diffusion process
\item
  surface flux
\item
  Surface and underground processes
\item
  gravitational wave resistance
\end{itemize}

We compute the time-varying terms \(F_x, F_y, Q, M, S\) for the forecast
variables from these processes, and perform time integration. In order
to evaluate the atmospheric and surface fluxes, the ground surface
sub-model is used. The ground surface sub-model uses such predictors as
the ground temperature (\(T_g\)), ground moisture (\(W_g\)), and snow
cover (\(W_y\)) as predictor variables.

\hypertarget{fundamental-equations.}{%
\subsubsection{Fundamental Equations.}\label{fundamental-equations.}}

Considering the equation of motion of the atmosphere in \(\sigma\)
coordinate system, thermodynamic equations, and equations for continuity
of materials such as water vapor etc. Considering the vertical fluxes of
momentum, heat, water vapor, etc., the time variation of the fluxes is
determined by their convergence. All the vertical fluxes are assumed to
be positive for the upward direction.

\begin{enumerate}
\def\labelenumi{\arabic{enumi}.}
\tightlist
\item
  equation of motion
\end{enumerate}

\begin{eqnarray}
  \rho \frac{d {u}}{d {t}} = \frac{\partial{Fu}}{\partial {\sigma}}
\end{eqnarray}

\begin{eqnarray}
  \rho \frac{d {v}}{d {t}} = \frac{\partial{Fv}}{\partial {\sigma}}
\end{eqnarray}

\begin{verbatim}
 $u, v$: East-West, North-South Wind; $Fu, Fv$: Their Vertical Flux.
\end{verbatim}

\begin{enumerate}
\def\labelenumi{\arabic{enumi}.}
\setcounter{enumi}{1}
\tightlist
\item
  thermodynamic equation
\end{enumerate}

\begin{eqnarray}
  \rho \frac{d {c_p T}}{d {t}} = \frac{T}{\theta} \frac{\partial{F{\theta}}}{\partial {\sigma}}
                     + \frac{\partial{F{R}}}{\partial {\sigma}}
\end{eqnarray}

\begin{verbatim}
 $T$: Temperature; $c_p$: Constant Pressure Specific Heat; $\theta=T(p/p_0)^{-R/c_p}=T(p/p_0)^{-\kappa}$: Temperature Level; $F\theta$: Vertical Sensible Heat Flux; $FR$: Vertical Radiation Flux.
\end{verbatim}

Here, with \(\theta'=T(p/p_s)^{-\kappa}=T\sigma^{-\kappa}\), this is,

\begin{eqnarray}
  \rho \frac{d {c_p T}}{d {t}} = \sigma^\kappa \frac{\partial{F{\theta'}}}{\partial {\sigma}}
                     + \frac{\partial{F{R}}}{\partial {\sigma}}
\end{eqnarray}

As far as one-dimensional vertical processes are concerned, instead of
\(\theta\), we can consider \(\theta'\). For simplicity, unless there is
a risk of confusion, \(\theta'\) should be written as \(\theta\) below.

\begin{enumerate}
\def\labelenumi{\arabic{enumi}.}
\setcounter{enumi}{2}
\tightlist
\item
  water vapor continuity formula
\end{enumerate}

\begin{eqnarray}
  \rho \frac{d {q}}{d {t}} = \frac{\partial{Fq}}{\partial {\sigma}}
\end{eqnarray}

\begin{verbatim}
 $q$: Specific Humidity; $F{q}$: Vertical Water Vapor Flux.

 ### Fundamental Equations in the Ground
\end{verbatim}

Consider the \(z\) coordinates with the downward direction positive. As
expected, the vertical fluxes are all positive in the upward direction.

\begin{enumerate}
\def\labelenumi{\arabic{enumi}.}
\setcounter{enumi}{3}
\tightlist
\item
  thermal formula
\end{enumerate}

\begin{eqnarray}
  \frac{\partial{C_g G}}{\partial {t}} = \frac{\partial{Fg}}{\partial {z}} + Sg
\end{eqnarray}

\begin{verbatim}
 $G$: Ground Temperature; $C_g$: Constant Pressure Specific Heat; $F{g}$: Vertical Heat Flux; $Sg$; Heating Term (due to phase change etc.).
\end{verbatim}

\begin{enumerate}
\def\labelenumi{\arabic{enumi}.}
\setcounter{enumi}{4}
\tightlist
\item
  formula for ground moisture
\end{enumerate}

\begin{eqnarray}
  C_w \frac{\partial{w}}{\partial {t}} = \frac{\partial{Fw}}{\partial {z}} + Sw
\end{eqnarray}

\begin{verbatim}
 $w$: Ground moisture; $F{w}$: Vertical water flux; $Sw$; Water sources (e.g., runoff).
\end{verbatim}

\begin{enumerate}
\def\labelenumi{\arabic{enumi}.}
\setcounter{enumi}{5}
\tightlist
\item
  energy balance equation
\end{enumerate}

At the surface, an energy balance is established.

\begin{eqnarray}
    F{\theta} + L F{q} + F{R} - F{g} = \Delta s \; \; (\sigma=1, z=0)
\end{eqnarray}

\begin{verbatim}
 $L$: Latent heat of evaporation; $\Delta s$: Surface energy balance (associated with phase change, etc.).
\end{verbatim}

\begin{enumerate}
\def\labelenumi{\arabic{enumi}.}
\setcounter{enumi}{6}
\tightlist
\item
  surface water balance
\end{enumerate}

\begin{eqnarray}
  Pg + Fw - Rg = 0
\end{eqnarray}

\begin{verbatim}
 $Pg$: Precipitation; $Rg$: Surface Runoff.
\end{verbatim}

\begin{enumerate}
\def\labelenumi{\arabic{enumi}.}
\setcounter{enumi}{7}
\tightlist
\item
  the snow balance
\end{enumerate}

\begin{eqnarray}
  \frac{\partial{Wy}}{\partial {t}} = Py - Fy - My
\end{eqnarray}

\begin{verbatim}
 $Wy$: Snow cover (kg/m$^2$); $Py$: Snowfall; $Fy$: Sublimation; $My$: Snowmelt.
\end{verbatim}

\hypertarget{time-integration-of-physical-processes.}{%
\subsubsection{Time integration of physical
processes.}\label{time-integration-of-physical-processes.}}

In terms of time integration of predictors, we can classify the physical
processes in the following three orders of execution.

\begin{enumerate}
\def\labelenumi{\arabic{enumi}.}
\item
  cumulus convection and large-scale condensation
\item
  radiation, vertical diffusion, ground boundary layer and surface
  processes
\item
  gravitational wave resistance, mass regulation, dry convection
  regulation
\end{enumerate}

Cumulus convection and large-scale condensation,

\begin{eqnarray}
  \hat{T}^{t+\Delta t,(1)} = \hat{T}^{t+\Delta t}
                         +  2 \Delta t Q_{CUM}(\hat{T}^{t+\Delta t})
\end{eqnarray}

\begin{eqnarray}
  \hat{T}^{t+\Delta t,(2)} = \hat{T}^{t+\Delta t,(1)}
                         +  2 \Delta t Q_{LSC}(\hat{T}^{t+\Delta t,(1)})
\end{eqnarray}

where the values are updated by the usual Euler difference Note that the
large-scale condensation scheme is updated by the cumulus convection
scheme. In practice, the routines of cumulus convection and large-scale
condensation output the heating rates and so on, and the time
integration is performed immediately afterwards by
\texttt{MODULE:{[}GDINTG{]}}.

The calculations of the radiative, vertical diffusion, ground boundary
layer and surface processes in the following groups are basically
performed with these updated values
(\(\hat{T}^{t+\Delta t,(1)}, \hat{q}^{t+\Delta t,(2)}\), etc.). However,
in order to calculate some of the terms as implicit, we calculate the
heating rates and so on for all of these terms together, and then
perform time integration at the end. In other words, if we write
symbolically

\begin{eqnarray}
  \hat{T}^{t+\Delta t,(3)} = \hat{T}^{t+\Delta t,(2)}
              + 2 \Delta t Q_{RAD,DIF,SFC}
               (\hat{T}^{t+\Delta t,(2)},\hat{T}^{t+\Delta t,(3)})
\end{eqnarray}

That would be.

The gravitational wave resistance, mass modulation, and dry convection
modulation are the same as those for cumulus convection and large-scale
condensation.

\begin{eqnarray}
  \hat{T}^{t+\Delta t,(4)} = \hat{T}^{t+\Delta t,(3)}
              +  2 \Delta t Q_{ADJ}(\hat{T}^{t+\Delta t,(3)})
\end{eqnarray}

\hypertarget{various-physical-quantities.}{%
\subsubsection{Various physical
quantities.}\label{various-physical-quantities.}}

Here are definitions of various geophysical quantities that can be
computed simply from the predictors. Some of them are calculated with
\texttt{MODULE:{[}PSETUP{]}}.

\begin{enumerate}
\def\labelenumi{\arabic{enumi}.}
\tightlist
\item
  temporary temperature
\end{enumerate}

Provisional Temperature \(T_v\) is ,

\begin{eqnarray}
  T_v = T ( 1 + \epsilon_v q - l )
\end{eqnarray}

\begin{enumerate}
\def\labelenumi{\arabic{enumi}.}
\setcounter{enumi}{1}
\tightlist
\item
  air density
\end{enumerate}

The atmospheric density \(\rho\) is calculated as follows

\begin{eqnarray}
  \rho = \frac{p}{RT_v}
\end{eqnarray}

\begin{enumerate}
\def\labelenumi{\arabic{enumi}.}
\setcounter{enumi}{2}
\tightlist
\item
  high degree
\end{enumerate}

The altitude \(z\) is evaluated in the same way as the calculation of
the geopotential for mechanical processes.

\begin{eqnarray}
  z = \frac{\Phi}{g}
\end{eqnarray}

\begin{eqnarray}
 \Phi_{1}  =  \Phi_{s} + C_{p} ( \sigma_{1}^{-\kappa} - 1  ) T_{v,1}
\end{eqnarray}

\begin{eqnarray}
 \Phi_k - \Phi_{k-1}
   =  C_{p}
   \left[ \left( \frac{ \sigma_{k-1/2} }{ \sigma_k } \right)^{\kappa}
          - 1 \right] T_{v,k}
       + C_{p}
   \left[ 1-
         \left( \frac{ \sigma_{k-1/2} }{ \sigma_{k-1} } \right)^{\kappa}
              \right] T_{v,k-1}
\end{eqnarray}

\begin{enumerate}
\def\labelenumi{\arabic{enumi}.}
\setcounter{enumi}{3}
\tightlist
\item
  layer boundary temperature
\end{enumerate}

The temperature at the boundary of the layer is calculated by performing
a linear interpolation on \(\ln p\), i.e., \(\ln \sigma\).

\begin{eqnarray}
  T_{k-1/2} = \frac{\ln \sigma_{k-1} - \ln \sigma_{k-1/2}}
                   {\ln \sigma_{k-1} - \ln \sigma_k      } T_k
            + \frac{\ln \sigma_{k-1/2} - \ln \sigma_k}
                   {\ln \sigma_{k-1} - \ln \sigma_k      } T_{k-1}
\end{eqnarray}

\begin{enumerate}
\def\labelenumi{\arabic{enumi}.}
\setcounter{enumi}{4}
\tightlist
\item
  saturated specific humidity
\end{enumerate}

The saturated specific humidity \(q^*(T,p)\) are approximated using the
saturated vapor pressure \(e^*(T)\),

\begin{eqnarray}
q^*(T,p) = \frac{\epsilon e^*(T)}{p} .
\end{eqnarray}

Here, it is \(\epsilon=0.622\),

\begin{eqnarray}
\frac{1}{e^*_v} \frac{\partial{e^*_v}}{\partial {T}} = \frac{L}{R_v T^2}
\end{eqnarray}

Therefore, if the latent heat of evaporation (\(L\)) and the gas
constant of water vapor (\(R_v\)) are held constant, the number of
vaporized materials will be reduced,

\begin{eqnarray}
  e^*(T) = e^*(T=273{K})
                      \exp \left[ \frac{L}{R_v}
                            \left( \frac{1}{273} - \frac{1}{T} \right)
                       \right] ,
\end{eqnarray}

\begin{verbatim}
 .....

 (199) from ,
\end{verbatim}

\begin{eqnarray}
\frac{\partial{q^*}}{\partial {T}} = \frac{L}{R_v T^2} q^*(T,p) .
\end{eqnarray}

Here, if the temperature is lower than the freezing point 273.15K, the
sublimation latent heat \(L+L_M\) is used as the latent heat \(L\).

\begin{enumerate}
\def\labelenumi{\arabic{enumi}.}
\setcounter{enumi}{5}
\tightlist
\item
  dry static energy, wet static energy
\end{enumerate}

Dry static energy \(s\) is

\begin{eqnarray}
  s = C_p T + g z \; ,
\end{eqnarray}

Wet Static Energy \(h\) is

\begin{eqnarray}
  h = C_p T + g z + L q \; ,
\end{eqnarray}

. defined by .
