\hypertarget{physics}{%
\section{Physics}\label{physics}}

\hypertarget{overview-of-physical-parameterizations}{%
\subsection{Overview of Physical
Parameterizations}\label{overview-of-physical-parameterizations}}

As a physical process, we can consider the following

\begin{itemize}
\tightlist
\item
  Cumulus convection
\item
  Shallow convection
\item
  Large scale condensation
\item
  Cloud microphysics
\item
  Radiation
\item
  Turbulence
\item
  Surface fluxes
\item
  Gravity wave drag
\end{itemize}

We compute the time-varying terms \(F_x, F_y, Q, M, S\) for the
prognostic variables from these processes, and perform time integration.

\hypertarget{time-integration-of-physical-parameterizations}{%
\subsubsection{Time Integration of Physical
Parameterizations}\label{time-integration-of-physical-parameterizations}}

\textbf{NOTE: the descriptions in this section are outdated.}

In terms of time integration of predictors, we can classify the physical
Parameterizations in the following three orders of execution.

\begin{enumerate}
\def\labelenumi{\arabic{enumi}.}
\item
  Cumulus convection, shallow convection, large-scale condensation, and
  cloud microphysics
\item
  Radiation, turbulence and surface fluxes
\item
  Gravitational wave drag
\end{enumerate}

For cumulus convection, shallow convection, large-scale condensation,
and cloud microphysics, the values are updated by the usual Euler
difference as follows.

\begin{eqnarray}
  \hat{T}^{t+\Delta t,(1)} = \hat{T}^{t+\Delta t}
                         +  2 \Delta t Q_{CUM}(\hat{T}^{t+\Delta t})
\end{eqnarray}

\begin{eqnarray}
  \hat{T}^{t+\Delta t,(2)} = \hat{T}^{t+\Delta t,(1)}
                         +  2 \Delta t Q_{LSC}(\hat{T}^{t+\Delta t,(1)})
\end{eqnarray}

Note that the large-scale condensation scheme is updated by the cumulus
convection scheme. In practice, the routines of cumulus convection and
large-scale condensation output the heating rates and so on, and the
time integration is performed immediately afterwards by
\texttt{MODULE:{[}GDINTG{]}}.

The calculations of the radiative, vertical diffusion, ground boundary
layer and surface processes in the following groups are basically
performed with these updated values
(\(\hat{T}^{t+\Delta t,(1)}, \hat{q}^{t+\Delta t,(2)}\), etc.). However,
in order to calculate some of the terms as implicit, we calculate the
heating rates and so on for all of these terms together, and then
perform time integration at the end. In other words, if we write
symbolically

\begin{eqnarray}
  \hat{T}^{t+\Delta t,(3)} = \hat{T}^{t+\Delta t,(2)}
              + 2 \Delta t Q_{RAD,DIF,SFC}
               (\hat{T}^{t+\Delta t,(2)},\hat{T}^{t+\Delta t,(3)})
\end{eqnarray}

That would be the gravitational wave resistance, mass modulation, and
dry convection modulation are the same as those for cumulus convection
and large-scale condensation.

\begin{eqnarray}
  \hat{T}^{t+\Delta t,(4)} = \hat{T}^{t+\Delta t,(3)}
              +  2 \Delta t Q_{ADJ}(\hat{T}^{t+\Delta t,(3)})
\end{eqnarray}

\hypertarget{various-physical-quantities}{%
\subsubsection{Various physical
quantities}\label{various-physical-quantities}}

Here are definitions of various physical quantities that can be computed
simply from the prognostic variables. Some of them are calculated with
\texttt{MODULE:{[}PSETUP{]}}.

\begin{enumerate}
\def\labelenumi{\arabic{enumi}.}
\tightlist
\item
  Virtual temperature
\end{enumerate}

Virtual Temperature \(T_v\) is calculated as follows.

\begin{eqnarray}
  T_v = T ( 1 + \epsilon_v q - l )
\end{eqnarray}

\begin{enumerate}
\def\labelenumi{\arabic{enumi}.}
\setcounter{enumi}{1}
\tightlist
\item
  Air density
\end{enumerate}

The air density \(\rho\) is calculated as follows.

\begin{eqnarray}
  \rho = \frac{p}{RT_v}
\end{eqnarray}

\begin{enumerate}
\def\labelenumi{\arabic{enumi}.}
\setcounter{enumi}{2}
\tightlist
\item
  Altitude
\end{enumerate}

The altitude \(z\) is evaluated in the same way as the calculation of
the geopotential height in the dynamics.

\begin{eqnarray}
  z = \frac{\Phi}{g}
\end{eqnarray}

\begin{eqnarray}
 \Phi_{1}  =  \Phi_{s} + C_{p} ( \sigma_{1}^{-\kappa} - 1  ) T_{v,1}
\end{eqnarray}

\begin{eqnarray}
 \Phi_k - \Phi_{k-1}
   =  C_{p}
   \left[ \left( \frac{ \sigma_{k-1/2} }{ \sigma_k } \right)^{\kappa}
          - 1 \right] T_{v,k}
       + C_{p}
   \left[ 1-
         \left( \frac{ \sigma_{k-1/2} }{ \sigma_{k-1} } \right)^{\kappa}
              \right] T_{v,k-1}
\end{eqnarray}

\begin{enumerate}
\def\labelenumi{\arabic{enumi}.}
\setcounter{enumi}{3}
\tightlist
\item
  Half-level temperature
\end{enumerate}

Half-level temperature is calculated by performing a linear
interpolation on \(\ln p\), i.e., \(\ln \sigma\).

\begin{eqnarray}
  T_{k-1/2} = \frac{\ln \sigma_{k-1} - \ln \sigma_{k-1/2}}
                   {\ln \sigma_{k-1} - \ln \sigma_k      } T_k
            + \frac{\ln \sigma_{k-1/2} - \ln \sigma_k}
                   {\ln \sigma_{k-1} - \ln \sigma_k      } T_{k-1}
\end{eqnarray}

\begin{enumerate}
\def\labelenumi{\arabic{enumi}.}
\setcounter{enumi}{4}
\tightlist
\item
  Saturated specific humidity
\end{enumerate}

The saturated specific humidity \(q^\*(T,p)\) is approximated using the
saturated vapor pressure \(e^\*(T)\),

\begin{eqnarray}
q^\*(T,p) = \frac{\epsilon e^\*(T)}{p} .
\end{eqnarray}

Here, it is \(\epsilon=0.622\),

\begin{eqnarray}
\frac{1}{e^\*_v} \frac{\partial{e^\*_v}}{\partial {T}} = \frac{L}{R_v T^2} \label{p199}
\end{eqnarray}

Therefore, if the latent heat of evaporation (\(L\)) and the gas
constant of water vapor (\(R_v\)) are held constant,

\begin{eqnarray}
  e^\*(T) = e^\*(T=273{K})
                      \exp \left[ \frac{L}{R_v}
                            \left( \frac{1}{273} - \frac{1}{T} \right)
                       \right] ,
\end{eqnarray}

where \(e^\*(T=273 \mathrm{[K]}) = 611 \mathrm{[hPa]}\).

From eq.\ref{p199},

\begin{eqnarray}
\frac{\partial{q^\*}}{\partial {T}} = \frac{L}{R_v T^2} q^\*(T,p) .
\end{eqnarray}

It is noted that if the temperature is lower than the freezing point
\(273.15 \mathrm{[K]}\), the sublimation latent heat \(L+L_M\) is used
as the latent heat \(L\).

\begin{enumerate}
\def\labelenumi{\arabic{enumi}.}
\setcounter{enumi}{5}
\tightlist
\item
  Dry static energy and moisture static energy
\end{enumerate}

The dry static energy \(s\) is defined by

\begin{eqnarray}
  s = C_p T + g z
\end{eqnarray}

The moisture static Energy \(h\) is defined by

\begin{eqnarray}
  h = C_p T + g z + L q
\end{eqnarray}
