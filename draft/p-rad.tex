\hypertarget{radiation-scheme}{%
\subsection{Radiation scheme}\label{radiation-scheme}}

\hypertarget{summary-of-the-radiation-flux-calculation}{%
\subsubsection{Summary of the radiation flux
calculation}\label{summary-of-the-radiation-flux-calculation}}

The radiation scheme in the MIROC was created based on the Discrete
Ordinate Method and the \(k\)-distribution Method (Nakajima et al.,
2000), and updated by Sekiguchi and Nakajima (2008). The scheme
calculates the value of the radiation flux at each level by considering
the absorption, emission, and scattering processes of terrestrial and
solar radiation by gases and clouds/aerosols. The main input data are
temperature \(T\), specific humidity \(q\), cloud water \(l\), and cloud
cover \(C\). The output data are shortwave or longwave upward and
downward radiation fluxes \(F^{\mp}\), and derivative coefficient to
surface temperature \(\mathrm{d}F^{\mp}/dT_{g}\), surface downward
radiation flux \(F_{sf}^{+}\), and 0.5 and 0.67 µm optical thickness
\(\tau^{vis}\).

The calculation is separated for several wavelength bands. It is further
divided into several sub-channels, based on the \(k\)-distribution
method. As for gaseous absorption, the line absorption in
\(\mathrm{H_2O}\), \(\mathrm{CO_2}\), \(\mathrm{O_2}\),
\(\mathrm{O_3}\), \(\mathrm{N_2} \mathrm{O}\), \(\mathrm{CH_4}\), the
continuous absorption in \(\mathrm{H_2} \mathrm{O}\), \(\mathrm{CO_2}\),
\(\mathrm{O_2}\), \(\mathrm{O_3}\), and the CFC absorption are
incorporated. As for scattering, Rayleigh scattering of gases and
scattering by cloud and aerosol particles are considered.

Major subroutines used to calculate the radiation flux in
\texttt{SUBROUTINE:{[}DTRN31{]}} of pradt.F are as follows.

\begin{enumerate}
\def\labelenumi{\arabic{enumi}.}
\tightlist
\item
  Calculate the Planck function from atmospheric temperature
  \texttt{SUBROUTINE:{[}PLANKS,\ PLANKF{]}}
\item
  Calculate the optical thickness to the gas in each sub-channel
  \texttt{SUBROUTINE:{[}PTFIT2{]}}
\item
  Calculate the optical thickness to the CFC absorption
  \texttt{SUBROUTINE:{[}CNTCFC2{]}}
\item
  Calculate the optical thickness to aerosol, Rayleigh scattering, and
  cloud \texttt{SUBROUTINE:{[}SCATAE,\ SCATRY,\ SCATCL{]}}
\item
  Expand the Planck function by optical thickness for each sub-channel
  \texttt{SUBROUTINE:{[}PLKEXP{]}}
\item
  Calculate the transmission coefficient (T), reflection coefficient (R)
  and source function (S) \texttt{SUBROUTINE:{[}TWST{]}}
\item
  Make T, R, and S matrixes for maximal/random approximation
  \texttt{SUBROUTINE:{[}RTSMR{]}}
\item
  Calculate the radiation flux by adding method
  \texttt{SUBROUTINE:{[}ADDMR,\ ADDING{]}}
\end{enumerate}

\begin{figure}
\centering
\includegraphics{Prad_Fig1.png}
\caption{Flowchart of \texttt{SUBROUTINE:{[}DTRN31{]}}}
\end{figure}

To account for the partial coverage of clouds, the transmission and
reflection coefficients and source functions for each layer are
calculated at weighted average of the cloud cover, separately for cloud
cover and clear-sky conditions. The cloud cover of the cumulus is also
considered. In addition, it also performs several adding and calculates
the clear-sky radiation flux.

\hypertarget{wavelength-and-sub-channel}{%
\subsubsection{Wavelength and
Sub-channel}\label{wavelength-and-sub-channel}}

The basics of radiative flux calculations are represented by
Beer-Lambert's Law.

\begin{eqnarray}
  F^\lambda(z) = F^\lambda(0) exp (-k^\lambda z)
\end{eqnarray}

\(F^{\lambda}\) is the radiant flux density at the wavelength of
\(\lambda\) and \(k^{\lambda}\) is the absorption coefficient. In order
to calculate the radiative fluxes related to the heating rate, the
integration operation with respect to the wavelength is required.

\begin{eqnarray}
  F(z) = \int F^\lambda(z) d \lambda= \int F^\lambda(0) exp (-k^\lambda z) d \lambda
\end{eqnarray}

However, it is not easy to calculate this integration precisely because
the absorption and emission of radiation by gas molecules have the
complicated wavelength dependence of the absorption line attributed to
the structure of the molecule. The k-distribution method is a method
designed to make the relatively precise calculation easier. Within a
certain wavelength range, considering the density function \(F(k)\) for
\(\lambda\) of the absorption coefficient of \(k\), the above formula is
approximated as follows,

\begin{eqnarray}
 \int F^\lambda(0) exp (-k^\lambda z) d \lambda
 \simeq \int \bar{F}^k(0) exp (-k z) F(k) dk
\end{eqnarray}

where \(\bar{F}^k(0)\) is the flux averaged over a wavelength having the
absorption coefficient in this wavelength \(k\) in \(z=0\).

If \(\bar{F}^k(0)\) and \(F(k)\) are a relatively smooth functions to
the \(k\),

\begin{eqnarray}
 \int F^\lambda(0) exp (-k^\lambda z) d \lambda
 \simeq \sum \bar{F}^i(0) exp (-k^i z) F^i
\end{eqnarray}

the formula, as such above, can be relatively precisely calculated by
the addition of a finite number (sub-channels) of exponential terms.
This method has furthermore the advantage easy to consider the
absorption and scattering at the same time.

In the MIROC 6.0, by changing the radiation parameter data, the
calculations can be performed at various wavelengths. In the standard
version, the wavelength range is divided into 29 parts. In addition,
each wavelength range is divided into 1 to 6 sub-channels (corresponding
to the \(i\) in the above formula). There are 111 channels in total. The
wavelength range is divided by the wavenumber ( \(\mathrm{cm}^{-1}\) ),
1, 250, 400, 530, 610, 670, 750, 820, 980, 1175, 1225, 1325, 1400, 2000,
2500, 3300, 3800, 4700, 5200, 6000, 10000, 12750, 13250, 14750, 23000,
30000, 33500, 36000, 43500, 50000. Additionally, a chemical version is
also with 37 bands and 126 channels for chemical transport model and the
boundary of the shortwave region is also changed to 54000
\(\mathrm{cm}^{-1}\).

\hypertarget{calculation-of-the-planck-function}{%
\subsubsection{Calculation of the Planck
function}\label{calculation-of-the-planck-function}}

In this section, \texttt{SUBROUTINE:{[}PLANKS,\ PLANKF{]}} in pradt.F is
described.

The Planck function \(\bar{B}^{w}(T)\), integrated in each wavelength
range, is evaluated by the following formula.

\begin{eqnarray}
\bar{B}^{w}(T)=\lambda^{-2}{Texp}\left\{\sum_{n=1}^{5} B_{n}^{w}\left(\bar{\lambda}^{w} T\right)^{-n}\right\}
\end{eqnarray}

where \(\bar{\lambda}^{w}\) is the averaged wavelength of the wavelength
range, \(B_{n}^{w}\) is the parameter determined by function fitting.
This is calculated to the atmospheric temperature of each layer \(T_l\),
and the boundary atmospheric temperature of each layer \(T_{l+1/2}\),
surface temperature \(T_g\) and temperature \(1\mathrm{K}\) higher than
surface temperature \(T_{g+1K}\). The calculations are performed for
each wavelength and each layer. In the following description, the
subscript of the wavelength range \(w\) is omitted.

\hypertarget{calculation-of-the-optical-thickness-to-gas-absorption}{%
\subsubsection{Calculation of the optical thickness to gas
absorption}\label{calculation-of-the-optical-thickness-to-gas-absorption}}

In this section, \texttt{SUBROUTINE:{[}PTFIT2{]}} in pradt.F is
described.

The optical thickness of the gas absorption (the line and continuum
absorption are unified) \(\tau^{K D}\) is expressed as follows by using
the index \(m\) as the type of molecules.

\begin{eqnarray}
\tau^{KD}=\sum_{m=1} k^{(m)} C^{(m)}
\end{eqnarray}

where \(k^{(m)}\) is the absorption coefficient of the molecule \(m\),
which is different for each sub-channel and determined as a function of
temperature \(T\) and atmospheric pressure \(p\). \(C^{(m)}\) represents
the amount of gas in the layer represented by
\(\mathrm{mol} / \mathrm{cm}^{2} / \mathrm{km}\), calculated by using
the gas concentration \(r^{(m)}\) in ppmv (
\(C^{(m)}=10^{-1} r^{(m)} \rho d z\) ). In the MIROC 6.0, the number of
the considered molecule types \(m\) is 6 (1:\(\mathrm{H_2} \mathrm{O}\),
2:\(\mathrm{CO_2}\), 3:\(\mathrm{O_3}\), 4:\(\mathrm{N_2} \mathrm{O}\),
5:\(\mathrm{CH_4}\), 6:\(\mathrm{O_2}\)). Also, \(k^{(m)}\) is
represented as follows (the details are in Sekiguchi and Nakajima,
2008).

\begin{eqnarray}
k^{(m)}=\exp \left(\log 10 k_{2}^{(m)}+(A+B T) \log \left(T / T_{\text {ref2 } }\right)\right)
\end{eqnarray}

\begin{eqnarray}
B=\left[\frac{\log 10\left(k_{3}^{(m)}-k_{2}^{(m)}\right)}{\log \left(\frac{T_{r e f 3}}{T_{r e f 2}}\right)}-\frac{\log 10\left(k_{1}^{(m)}-k_{2}^{(m)}\right)}{\log \left(\frac{T_{\text {ref1} }}{T_{\text {ref2}}}\right)}\right] /\left(T_{\text {ref3}}-T_{\text {ref1 }}\right)
\end{eqnarray}

\begin{eqnarray}
A=\frac{\log 10\left(k_{3}^{(m)}-k_{2}^{(m)}\right)}{\log \left(T_{\text {ref3 } } / T_{\text {ref2 } }\right)}-B T_{\text {ref3 }}
\end{eqnarray}

\(T_{ref1-3}\) are the reference temperatures prepared in advance (200,
260, 320 \(K\)), and \(k_{1-3}^{(m)}\) are the absorption coefficients
when the reference temperatures \(T_{ref1-3}\) is used (also fitted at
26 atmospheric pressure grids).

When considering the absorption of \(\mathrm{H_2O}\), we calculate the
optical thickness of the self-broadening and add \(\tau^{self}\).

\begin{eqnarray}
\tau^{K D\left(\mathrm{H_2O}\right)}=\tau^{K D\left(\mathrm{H_2O}\right)}+\tau^{\text {self }}
\end{eqnarray}

\begin{eqnarray}
\tau^{\text {self }}=\frac{k^{\left(\mathrm{H_2O\_self}\right)} C^{\left(\mathrm{H_2O}\right)^{2}}}{C^{\left(\mathrm{H_2O}\right)}+\rho d z 10^{5}}
\end{eqnarray}

\(k^{(\mathrm{H_2O\_self})}\) is calculated in the same way as
\(k^{(m)}\). The self-broadening absorption coefficients in the
reference temperatures \(T_{ref1-3}\) are prescribed and dependent on
the pressure. In the above formula, \(10^{5}\) is multiplied to convert
the unit from \(\mathrm{km}\) to \(\mathrm{cm}\). This calculation is
done for each sub-channel and each layer.

\hypertarget{calculation-of-the-optical-thickness-to-cfc-absorption}{%
\subsubsection{Calculation of the optical thickness to CFC
absorption}\label{calculation-of-the-optical-thickness-to-cfc-absorption}}

In this section, \texttt{SUBROUTINE:{[}CNTCFC2{]}} in pradt.F is
described.

The optical thickness of the CFC absorption \(\tau^{CFC}\) is considered
for several types of CFCs \(m\).

\begin{eqnarray}
\tau^{C F C}=\sum_{m} 10^{k^{(m)}} r^{(m)} \rho \Delta z 10^{-1}
\end{eqnarray}

In MIROC 6.0, the number of the considered CFCs \(m\) is 28
(1:\(\mathrm{CFC\text{-11}}\), 2:\(\mathrm{CFC\text{-12}}\),
3:\(\mathrm{CFC\text{-13}}\), 4:\(\mathrm{CFC\text{-14}}\),
5:\(\mathrm{CFC\text{-113}}\), 6:\(\mathrm{CFC\text{-114}}\),
7:\(\mathrm{CFC\text{-115}}\), 8:\(\mathrm{HCFC\text{-21}}\),
9:\(\mathrm{HCFC\text{-22}}\), 10:\(\mathrm{HCFC\text{-123}}\),
11:\(\mathrm{HCFC\text{-124}}\), 12:\(\mathrm{HCFC\text{-141b}}\),
13:\(\mathrm{HCFC\text{-142b}}\), 14:\(\mathrm{HCFC\text{-225ca}}\),
15:\(\mathrm{HCFC\text{-225cb}}\), 16:\(\mathrm{HFC\text{-32}}\),
17:\(\mathrm{HFC\text{-125}}\), 18:\(\mathrm{HFC\text{-134}}\),
19:\(\mathrm{HFC\text{-134a}}\), 20:\(\mathrm{HFC\text{-143a}}\),
21:\(\mathrm{HFC\text{-152a}}\), 22:\(\mathrm{SF_6}\),
23:\(\mathrm{ClONO_2}\), 24:\(\mathrm{CCl_4}\), 25:\(\mathrm{N_2O_5}\),
26:\(\mathrm{C_2F_6}\), 27:\(\mathrm{HNO_4}\),
28:\(\mathrm{SF_5CF_3}\)). In the above formula, \(10^{-1}\) is
multiplied to convert from \(\mathrm{km}\) to \(\mathrm{cm}\), and from
ppmv to ratio. This calculation is done for each sub-channel and each
layer. This calculation is performed for each layer and the wavelength
range from about 540 to 1800 \(\mathrm{cm}^{-1}\).

\hypertarget{optical-thickness-to-scattering-and-scattering-moment}{%
\subsubsection{Optical thickness to scattering and scattering
moment}\label{optical-thickness-to-scattering-and-scattering-moment}}

Calculate the optical thickness of scattering and the scattering moment.
These calculations are performed for each wavelength and each layer. The
optical parameters for the particle matter \(q_{m}^{(p)}\) are prepared,
including the extinction coefficient (\(m = 1\)) including the
scattering and absorption process and the absorption coefficient
(\(m = 2\)) the moments of the volume scattering phase function
(\(m=3\text{-}4\): first-second order).

\hypertarget{aerosol}{%
\paragraph{Aerosol}\label{aerosol}}

In this section, \texttt{SUBROUTINE:{[}SCATAE{]}} in pradt.F is
described.

The optical thickness \(\tau^{a e}\), the part of the optical thickness
due to absorption \(\tau_{ab}^{a e}\), the scattering moment
\(Q_{m}^{a e}\) for aerosol are

\begin{eqnarray}
\tau^{a e}=\sum_{p} q_{1, n}^{(p)} r^{(p)} \times \rho \Delta z 10^{-1}
\end{eqnarray}

\begin{eqnarray}
\tau_{ab}^{a e}=\sum_{p} q_{2, n}^{(p)} r^{(p)} \times \rho \Delta z 10^{-1}
\end{eqnarray}

\begin{eqnarray}
Q_{m}^{a e}=\sum_{p} q_{m, n}^{(p)} r^{(p)} \times \rho \Delta z 10^{-1} (\mathrm{~m} \geq 3)
\end{eqnarray}

\(p\) is the aerosol type, and \(r^{(p)}\) is volume mixing ratio of the
particle. The optical parameters for the particle \(q_{m, n}^{(p)}\)
depend on the mode radius index \(n\) prescribed for each particle
(IRA). In the MIROC 6.0, the number of the considered aerosol types
\(p\) 15 (1-6:soil dust (bin1-6), 7:carbonaceous (BC/OC=0.3),
8:carbonaceous (BC/OC=0.15), 9:carbonaceous (BC/OC=0), 10:black carbon
(external mixture), 11:sulfate, 12-15:sea salt (bin 1-4)).

If the aerosol radius is used, the optical thickness \(\tau^{a e}\), the
part of the optical thickness due to absorption \(\tau_{ab}^{a e}\), and
the scattering moment \(Q_{m}^{a e}\) for the hygroscopic aerosols
(e.g., carbonaceous, sulfate, sea salt) are

\begin{eqnarray}
\tau^{a e}=\sum_{p}\left[\left(1-F X_{a e}\right) q_{1, n f i t}^{(p)} r^{(p)}+F X_{a e} q_{1, n f i t+1}^{(p)} r^{(p)}\right] \times \rho \Delta z 10^{-1}
\end{eqnarray}

\begin{eqnarray}
\tau_{ab}^{a e}=\sum_{p}\left[\left(1-F X_{a e}\right) q_{2, n f i t}^{(p)} r^{(p)}+F X_{a e} q_{2, n f i t+1}^{(p)} r^{(p)}\right] \times \rho \Delta z 10^{-1}
\end{eqnarray}

\begin{eqnarray}
Q_{m}^{a e}=\sum_{p}\left[\left(1-F X_{a e}\right) q_{m, n f i t}^{(p)} r^{(p)}+F X_{a e} q_{m, n f i t+1}^{(p)} r^{(p)}\right] \times \rho \Delta z 10^{-1}(\mathrm{~m} \geq 2)
\end{eqnarray}

\begin{eqnarray}
F X_{a e}=\left(R H-R H_{n f i t}^{(r e f)}\right)\left(\frac{1}{R H_{n f i t+1}^{(r e f)}-R H_{n f i t}^{(r e f)}}\right)
\end{eqnarray}

where \(RH\) is the local relative humidity and
\(R H_{n f i t}^{(r e f)}\) is the relative humidity given in the
parameter and \(nfit\) is the number of the prescribed relative humidity
closest to the \(RH\). \(nfit\) and \(FX_{ae}\) are calculated in the
\texttt{SUBROUTINE:{[}RMDIDX{]}} in pradt.F and determined in advance.
In the above formulas, \(10^{-1}\) is multiplied to convert from
\(\mathrm{km}\) to \(\mathrm{cm}\), and from ppmv to ratio.

\hypertarget{rayleigh-scattering}{%
\paragraph{Rayleigh scattering}\label{rayleigh-scattering}}

In this section, \texttt{SUBROUTINE:{[}SCATRY{]}} in pradt.F is
described.

The optical thickness \(\tau^{r}\) of Rayleigh scattering and the part
of the optical thickness due to absorption \(\tau_{ab}^{r}\) are

\begin{eqnarray}
\tau^{r}=\frac{e^{r}qmol_{1}dp}{p_{S T D}}
\end{eqnarray}

\begin{eqnarray}
\tau_{ab}^{r}=\frac{e^{r}qmol_{2}dp}{p_{S T D}}
\end{eqnarray}

\begin{eqnarray}
p_{S T D}=1013.25
\end{eqnarray}

where \(e^{r}\) is the Rayleigh scattering coefficient, \(qmol_m\) is
the moments of the phase function. These calculations are performed up
to \(m=2\). Also, this is added to the optical thickness for the
aerosol.

\begin{eqnarray}
\tau^{a e+r}=\tau^{a e}+\tau^{r}
\end{eqnarray}

\begin{eqnarray}
\tau_{ab}^{a e+r}=\tau_{ab}^{a e}+\tau_{ab}^{r}
\end{eqnarray}

\hypertarget{cloud}{%
\paragraph{Cloud}\label{cloud}}

In this section, \texttt{SUBROUTINE:{[}SCATCL{]}} in pradt.F is
described.

The optical thickness \(\tau^{cl}\), the part of the optical thickness
due to absorption \(\tau_{ab}^{cl}\), and the scattering moment
\(Q_{m}^{c l}\) for cloud are

\begin{eqnarray}
\tau^{c l}=\sum_{c t} q_{1, n}^{(c t)}r^{(c t)}\times \rho \Delta z 10^{-1}
\end{eqnarray}

\begin{eqnarray}
\tau_{ab}^{c l}=\sum_{c t} q_{2, n}^{(c t)}r^{(c t)}\times \rho \Delta z 10^{-1}
\end{eqnarray}

\begin{eqnarray}
Q_{m}^{c l}=\sum_{c t} q_{m, n}^{(c t)} \times r^{(c t)} \rho \Delta z 10^{-1}(\mathrm{~m} \geq 3)
\end{eqnarray}

\(ct\) is the cloud particle type (1:liquid cloud, 2:ice cloud). The
optical parameters for the particle \(q_{m, n}^{(c t)}\) depend on the
mode radius index \(n\) prescribed for each particle (IRC). If the cloud
radius is used, the optical thickness \(\tau^{cl}\), the part of the
optical thickness due to absorption \(\tau_{ab}^{cl}\), and the
scattering moment \(Q_{m}^{c l}\) for cloud are

\begin{eqnarray}
\tau^{c l}=\sum_{c t}\left[\left(1-F X_{c l}\right) q_{1, n f i t}^{(c t)} r^{(c t)}+F X_{c l} q_{1, n f i t+1}^{(c t)} r^{(c t)}\right] \times \rho \Delta z 10^{-1}
\end{eqnarray}

\begin{eqnarray}
\tau_{ab}^{c l}=\sum_{c t}\left[\left(1-F X_{c l}\right) q_{2, n f i t}^{(c t)} r^{(c t)}+F X_{c l} q_{2, n f i t+1}^{(c t)} r^{(c t)}\right] \times \rho \Delta z 10^{-1}
\end{eqnarray}

\begin{eqnarray}
Q_{m}^{c l}=\sum_{c t}\left[\left(1-F X_{c l}\right) q_{m, n f i t}^{(c t)} r^{(c t)}+F X_{c l} q_{m, n f i t+1}^{(c t)} r^{(c t)}\right] \times \rho \Delta z 10^{-1}(\mathrm{~m} \geq 3)
\end{eqnarray}

\begin{eqnarray}
F X_{c l}=\left(R^{(c t)}-R_{n f i t}^{(r e f)}\right)\left(\frac{1}{R_{n f i t+1}^{(r e f)}-R_{n f i t}^{(r e f)}}\right)
\end{eqnarray}

where \(R^{(ct)}\) is the calculated mode radius and
\(R_{n f i t}^{(r e f)}\) is the mode radius given in the parameter and
\(nfit\) is the number of the prescribed mode radius closest to the
\(R^{(ct)}\). \(nfit\) and \(FX_{cl}\) are calculated in the subroutine
\texttt{SUBROUTINE:{[}RMDIDX{]}} in pradt.F and determined in advance.
In the above formulas, \(10^{-1}\) is multiplied to convert from
\(\mathrm{km}\) to \(\mathrm{cm}\), and from ppmv to ratio.

Finally, the total optical thickness for particle scattering, Rayleigh
scattering and absorption \(\tau^p\) and the contribution of scattering
\(\tau^{scat}\) are obtained as follows.

\begin{eqnarray}
\tau^{P}=\tau^{c l}+\tau^{a e+r}
\end{eqnarray}

\begin{eqnarray}
\tau^{s c a t}=\tau^{P}-\left(T_{ab}^{c l}+T_{ab}^{a e+r}\right)
\end{eqnarray}

In addition, the moments of the normalized phase function \(G\) are
calculated up to the three orders. The zeroth moment \(G_1\) is trivial
from the normalization condition of the phase function. The first and
second moments \(G_2\), \(G_3\), are referred as the asymmetry factor
\(g\) and the truncation factor \(f\). \begin{eqnarray}
G_{1}=1.0
\end{eqnarray}

\begin{eqnarray}
G_{m-1}=\frac{Q_{m}^{c l}+Q_{m}^{ae}}{\tau^{s c a t}}(m \geq 3), \quad G_{2}=g, \quad G_{3}=f
\end{eqnarray}

This calculation is divided into the cloudy, clear sky and cumulus
conditions. In the cloudy and cumulus conditions, \(\tau^{cl}\) in the
0.5 and 0.67 \(\mathrm{{\mu}m}\) regions is as recorded as
\(\tau^{vis}\) in subroutine DTRN31.

**\(R^{ct}\) is calculated in \texttt{SUBROUTINE:{[}RADFLX{]}} as
follows.

\begin{eqnarray}
R^{(c t)}=\left(\frac{3}{4 \pi} \frac{\rho r^{(c t)}}{\rho_{w}^{(c t)} n_{c}^{(c t)}}\right)^{1 / 3}
\end{eqnarray}

\(\rho_{w}^{(c t)}\) is the liquid or ice density. \(r^{ct}\) is the
amount of the liquid or ice cloud and calculated as follows.

\begin{eqnarray}
r^{(c t)}=\frac{C_{s t} r_{s t}^{(c t)}+C_{c u} r_{c u}^{(c t)}}{1-\left(1-C_{s t}\right)\left(1-C_{c u}\right)}
\end{eqnarray}

\(C\) is the area of the cloud, and the subscript \(st\) and \(cu\) mean
the stratus and cumulus. When \(r_{s t, c u}^{(c t)}\) is the small
amount in the stratosphere, it is reset to 0. \(n_{c}^{(c t)}\) is the
number density of cloud particles.

\begin{eqnarray}
n_{c}^{(l i q)}=\max \left(\frac{q_{a e}^{l i q} p N_{A}}{R T_{v}\left(18 \times 10^{-3} R_{v} / R\right)}, f_{l i q} n_{\min }^{(l i q)}\right)
\end{eqnarray}

\begin{eqnarray}
n_{c}^{(i c e)}=\max \left(\frac{q_{a e}^{i c e} p N_{A}}{R T_{v}\left(18 \times 10^{-3} R_{v} R_{v} / R\right)},\left(1-f_{l i q}\right) n_{\min }^{(i c e)}\right)
\end{eqnarray}

where \(q_{a e}^{l i q}\) is the mixing ratio of the aerosol particles
calculated by the SPRINTERS and converted to the number concentration,
and \(n_{\min }^{(c t)}\) is the minimum number of the cloud particles.
and \(f_{liq}\) is liquid fraction. Also, \(n_{c}^{(c t)}\) is
calculated as follows when using OPT\_AECL\_SIMPLE.

\begin{eqnarray}
n_{c}^{(c t)}=\frac{\varepsilon n_{a} n_{m a x}^{(c t)}}{\varepsilon n_{a}+n_{\max }^{(c t)}}
\end{eqnarray}

where \(n_a\) is the number density of aerosol particles give as an
external condition, and \(\varepsilon\) and \(n_{\max }^{(c t)}\) are
constants. \(f_{liq}\) is calculated by the following formula using the
amount of cloud water \(w\) (\(0 \leq f_{\text {liq }} \leq 1\)).

\begin{eqnarray}
f_{l i q}=\frac{w_{s t} f_{l i q, s t}+w_{c u} f_{l i q, c u}}{w_{s t}+w_{c u}}
\end{eqnarray}

\hypertarget{total-optical-thickness}{%
\subsubsection{Total optical thickness}\label{total-optical-thickness}}

All optical thickness including gaseous band absorption, and scattering
is,

\begin{eqnarray}
\tau=\tau^{K D}+\tau^{C O N}+\tau^{P}
\end{eqnarray}

where because \(\tau^{K D}\) is different for each subchannel, the
calculation is done for each sub-channel and each layer, and divided
into the cloudy, clear sky, and cumulus conditions.

\hypertarget{expansion-of-the-plank-function}{%
\subsubsection{Expansion of the plank
function}\label{expansion-of-the-plank-function}}

In this section, \texttt{SUBROUTINE:{[}PLKEXP{]}} in pradt.F is
described.

In each layer, the Planck function \(B\) is expanded as follows and the
expansion coefficients \(b_0\), \(b_1\), and \(b_2\), are obtained.

\begin{eqnarray}
{B}\left(\tau^{\prime}\right)=b_{0}+b_{1} \tau^{\prime}+b_{2}\left(\tau^{\prime}\right)^{2}
\end{eqnarray}

Here, as \({B}\left(\tau^{\prime}\right)\), \(B\) at the top of each
layer (the boundary with the top layer) is used, and as \({B}(\tau)\),
\(B\) at the bottom edge of each layer (the boundary with the layer
below), and as \({B}(\tau / 2)\), the \(B\) at the representative level
of each layer.

\begin{eqnarray}
b_{0}=B(0)
\end{eqnarray}

\begin{eqnarray}
b_{1}=(4{~B}(\tau / 2)-{B}(\tau)-3{~B}(0)) / \tau
\end{eqnarray}

\begin{eqnarray}
b_{2}=2({~B}(\tau)-{B}(0)-2{~B}(\tau / 2)) / \tau^{2}
\end{eqnarray}

\begin{figure}
\centering
\includegraphics{Prad_Fig2.png}
\caption{Second-order expansion using the optical thickness of the plank
function}
\end{figure}

This calculation is done for each sub-channel and each layer and divided
into the cloudy, clear sky and cumulus conditions.

\hypertarget{transmission-and-reflection-coefficients-and-source-function}{%
\subsubsection{Transmission and reflection coefficients, and source
function}\label{transmission-and-reflection-coefficients-and-source-function}}

In this section, \texttt{SUBROUTINE:{[}TWST{]}} in pradt.F is described.

Using the obtained optical thickness \(\tau\), optical thickness of
scattering \(\tau^{scat}\), scattering moments \(g\), \(f\), expansion
coefficients for Planck function \(b_n\), and solar incidence angle
factor \(\mu_{0}\), the transmission coefficient \(T\), reflection
coefficient \(R\), downward radiation source function \(\epsilon^{+}\),
and the upward radiation source function \(\epsilon^{-}\) are founded,
by assuming a uniform layer and using the two-stream approximation.

The single-scattering albedo \(\omega\) is,

\begin{eqnarray}
\omega=\tau^{\text{scat}}/\tau
\end{eqnarray}

The optical thickness \(\tau^{*}\), the single-scattering albedo
\(\omega^{*}\), and asymmetric factor \(g^{*}\), corrected by the
contribution from the forward scattering factor \(f\) are,

\begin{eqnarray}
\tau^{*}=\tau(1-\omega f)
\end{eqnarray}

\begin{eqnarray}
\omega^{*}=\frac{(1-f) \omega}{1-\omega f}
\end{eqnarray}

\begin{eqnarray}
g^{*}=\frac{g-f}{1-f}
\end{eqnarray}

\(\mu\) is a two-stream directional cosine.

\begin{eqnarray}
\mu \equiv\left(\frac{1}{\sqrt{3}}, \frac{1}{1.66}\right) \text { (shortwave, longwave) }
\end{eqnarray}

\begin{eqnarray}
W^{-} \equiv \mu^{-1 / 2}
\end{eqnarray}

Furthermore, \begin{eqnarray}
\hat{P}^{\pm}=\omega^{*} W^{-2}\left(1 \pm 3 g^{*} \mu^{2}\right) / 2
\end{eqnarray}

\begin{eqnarray}
\hat{S}_{s}^{\pm}=\omega^{*} W^{-}\left(1 \pm 3 g^{*} \mu_{0} \mu\right)
\end{eqnarray}

can be determined as above as a normalized scattering phase function.

\begin{eqnarray}
\begin{aligned}
X &=\mu^{-1}-\left(\hat{P}^{+}-\hat{P}^{+}\right) \\
&=\mu^{-1}-3 \omega^{*} W^{-2} \mu^{2} g^{*} \\
\end{aligned}
\end{eqnarray}

\begin{eqnarray}
\begin{aligned}
Y &=\mu^{-1}-\left(\hat{P}^{+}+\hat{P}^{+}\right) \\
&=\mu^{-1}-\omega^{*} W^{-2} \\
\end{aligned}
\end{eqnarray}

\begin{eqnarray}
\begin{aligned}
\hat{\sigma}_{S}^{+} &=\hat{S}_{S}^{+}+\hat{S}_{S}^{-} \\
&=\omega^{*} W^{-} \\
\end{aligned}
\end{eqnarray}

\begin{eqnarray}
\begin{aligned}
\hat{\sigma}_{S}^{-} &=\hat{S}_{s}^{+}-\hat{S}_{S}^{-} \\
&=3 \omega^{*} \mu_{0} W^{-} \mu g^{*}
\end{aligned}
\end{eqnarray}

Using the above formula, the reflectance \(R\) and transmission \(T\)
become

\begin{eqnarray}
\begin{array}{c}
A A=\frac{X\left(1+e^{-\lambda \tau^{*}}\right)-\lambda\left(1-e^{-\lambda \tau^{*}}\right)}{X\left(1+e^{-\lambda \tau^{*}}\right)+\lambda\left(1-e^{-\lambda \tau^{*}}\right)}
\end{array}
\end{eqnarray} \begin{eqnarray}
\begin{array}{c}
B B=\frac{X\left(1-e^{-\lambda \tau^{*}}\right)-\lambda\left(1+e^{-\lambda \tau^{*}}\right)}{X\left(1-e^{-\lambda \tau^{*}}\right)+\lambda\left(1+e^{-\lambda \tau^{*}}\right)}
\end{array}
\end{eqnarray}

\begin{eqnarray}
\begin{array}{c}
\lambda=\sqrt{X Y}
\end{array}
\end{eqnarray}

\begin{eqnarray}
\begin{array}{c}
R=\frac{1}{2}(A A+B B)
\end{array}
\end{eqnarray}

\begin{eqnarray}
\begin{array}{c}
T=\frac{1}{2}(A A-B B)
\end{array}
\end{eqnarray}

Next, we find the source function derived from the Planck function.

\begin{eqnarray}
\hat{b}_{n}=2 \pi\left(1-\omega^{*}\right) W^{-}\left(\frac{1}{1-\omega f}\right)^{n} b_{n}
\end{eqnarray}

The expansion coefficients of the radiation source function can be found
from the above formulas.

\begin{eqnarray}
\begin{array}{l}
D_{2}^{\pm}=\frac{\hat{b}_{2}}{Y}
\end{array}
\end{eqnarray}

\begin{eqnarray}
\begin{array}{l}
D_{1}^{\pm}=\frac{\hat{b}_{1}}{Y} \mp \frac{2 \hat{b}_{2}}{X Y}
\end{array}
\end{eqnarray}

\begin{eqnarray}
\begin{array}{l}
D_{0}^{\pm}=\frac{\hat{b}_{0}}{Y}+\frac{2 \hat{b}_{2}}{X Y^{2}} \mp \frac{\hat{b}_{1}}{X Y}
\end{array}
\end{eqnarray}

\begin{eqnarray}
\begin{array}{l}
D^{\pm}\left(\tau^{*}\right)=D_{0}^{+}+D_{1}^{+} \tau^{*}+D_{2}^{+} \tau^{* 2}
\end{array}
\end{eqnarray}

The radiation source function derived from the Planck function
\(\hat{\epsilon}_{A}^{\pm}\) is

\begin{eqnarray}
\begin{array}{l}
\hat{\epsilon}_{A}^{-}=D_{0}^{-}-R D_{0}^{-}-T D^{-}\left(\tau^{*}\right)
\end{array}
\end{eqnarray}

\begin{eqnarray}
\begin{array}{l}
\hat{\epsilon}_{A}^{+}=D^{+}\left(\tau^{*}\right)-T D_{0}^{+}-R D^{-}\left(\tau^{*}\right)
\end{array}
\end{eqnarray}

On the other hand, the radiation source function of the solar-induced
radiation is

\begin{eqnarray}
\begin{array}{l}
\hat{\epsilon}_{S}^{+}=F_{\text {sol }}\left(V_{s}^{+} e^{-\frac{\tau^{*}}{\mu_{0}}}-T V_{s}^{+}-R V_{s}^{-} e^{-\frac{\tau^{*}}{\mu_{0}}}\right)
\end{array}
\end{eqnarray}

\begin{eqnarray}
\begin{array}{l}
\hat{\epsilon}_{S}^{+}=F_{\text {sol }}\left(V_{s}^{+} e^{-\frac{\tau^{*}}{\mu_{0}}}-T V_{s}^{+}-R V_{s}^{-} e^{-\frac{\tau^{*}}{\mu_{0}}}\right)
\end{array}
\end{eqnarray}

Here, \(Q \gamma\) and \(V_{s}^{\pm}\) are expressed by the following
formulas, and \(F_{sol}\) is solar irradiance.

\begin{eqnarray}
\begin{array}{c}
V_{s}^{\pm}=\frac{1}{2}\left[Q \gamma \pm\left(\frac{Q \gamma}{\mu_{0}}+\frac{\hat{\sigma}_{S}^{-}}{X}\right)\right]
\end{array}
\end{eqnarray}

\begin{eqnarray}
\begin{array}{c}
Q \gamma=\frac{X \hat{\sigma}_{S}^{+} \mu_{0}+\mu_{0}^{-1} \hat{\sigma}_{S}^{-}}{X Y \mu_{0}+\mu_{0}^{-1}}
\end{array}
\end{eqnarray}

The direct solar transmission is also calculated in this subroutine.

\begin{eqnarray}
E x^{d i r}=e^{-\tau^{*}/ \mu_{0}}
\end{eqnarray}

This calculation is done for each sub-channel and each layer and divided
into the cloudy, clear sky, and cumulus conditions.

\hypertarget{t-r-s-matrixes-for-maximalrandom-approximation}{%
\subsubsection{T, R, S matrixes for maximal/random
approximation}\label{t-r-s-matrixes-for-maximalrandom-approximation}}

In this section, \texttt{SUBROUTINE:{[}RTSMR{]}} in pradt.F is
described.

In this subroutine, T, R, S matrixes for maximal/random approximation is
made. The radiation source function, which is the sum of both the plank
function and the solar incident origins, is

\begin{eqnarray}
\begin{array}{l}
\epsilon^{-(\text {cloud})}=\hat{\epsilon}_{S}^{-(\text {cloud})} {Tr}^{(\text {cloud})}+\hat{\epsilon}_{A}^{-(\text {cloud})} C
\end{array}
\end{eqnarray}

\begin{eqnarray}
\begin{array}{l}
\epsilon^{-(\text {clear})}=\hat{\epsilon}_{S}^{-(\text {clear})}{Tr}^{(\text {clear})}+\hat{\epsilon}_{A}^{-(\text {clear})}(1-C)
\end{array}
\end{eqnarray}

\begin{eqnarray}
\begin{array}{l}
\epsilon^{+(\text {cloud})}=\hat{\epsilon}_{S}^{+(\text {cloud})}{Tr}^{(\text {cloud})}+\hat{\epsilon}_{A}^{-(\text {cloud})} C
\end{array}
\end{eqnarray}

\begin{eqnarray}
\begin{array}{l}
\epsilon^{+(\text {clear})}=\hat{\epsilon}_{S}^{+(\text {clear})}{Tr}^{(\text {clear})}+\hat{\epsilon}_{A}^{-(\text {clear})}(1-C)
\end{array}
\end{eqnarray}

\(Tr\) is the direct solar transmission for maximal/random approximation
and calculated as follows in \texttt{SUBROUTINE:{[}DTRN31{]}}.

\begin{eqnarray}
\begin{aligned}
\operatorname{Tr}_{n}^{(\text {cloud})} &=E x_{n}^{(\text {cloud})} B_{n}^{(3)}+E x_{n}^{(\text {clear})}\left(1-B_{n}^{(1)}\right)
\end{aligned}
\end{eqnarray}

\begin{eqnarray}
\begin{aligned}
E x_{n+1}^{(\text {cloud})} &={Tr}_{n}^{(\text {cloud})} E x_{n}^{\operatorname{dir}(\text {cloud})}
\end{aligned}
\end{eqnarray}

\begin{eqnarray}
\begin{aligned}
T r_{n}^{(\text {clear})} &=E x_{n}^{(\text {cloud})}\left(1-B_{n}^{(3)}\right)+E x_{n}^{(\text {clear})} B_{n}^{(1)}
\end{aligned}
\end{eqnarray}

\begin{eqnarray}
\begin{aligned}
E x_{n+1}^{(\text {clear)} }&={Tr}_{n}^{(\text {clear})} E x_{n}^{\text {dir}(\text {clear})}
\end{aligned}
\end{eqnarray}

\(Ex\) is the cumulative direct solar transmission. \(B_{n}^{(1-4)}\) is
the cloud cover interaction index and calculated in
\texttt{SUBROUTINE:{[}BCVR{]}}in pradt.F.

\begin{eqnarray}
\begin{array}{l}
B_{n}^{(1)}=\frac{1-\max \left(C_{n-1}, C_{n}\right)}{1-C_{n-1}}
\end{array}
\end{eqnarray} \begin{eqnarray}
\begin{array}{l}
B_{n}^{(2)}=\frac{1-\max \left(C_{n+1}, C_{n}\right)}{1-C_{n+1}}
\end{array}
\end{eqnarray}

\begin{eqnarray}
\begin{array}{l}
B_{n}^{(3)}=\frac{\min \left(C_{n-1}, C_{n}\right)}{C_{n-1}}
\end{array}
\end{eqnarray}

\begin{eqnarray}
\begin{array}{l}
B_{n}^{(4)}=\frac{\min \left(C_{n+1}, C_{n}\right)}{C_{n+1}}
\end{array}
\end{eqnarray}

Next, T matrixes for maximal/random approximation are calculated.

\begin{eqnarray}
\begin{array}{l}
T^{+(cloud, 1)}=T^{(\text {cloud})} B^{(3)}
\end{array}
\end{eqnarray}

\begin{eqnarray}
\begin{array}{l}
T^{+(\text {cloud}, 2)}=T^{(\text {cloud})}\left(1-B^{(1)}\right)
\end{array}
\end{eqnarray}

\begin{eqnarray}
\begin{array}{l}
T^{+(\text {clear}, 1)}=T^{(\text {clear})}\left(1-B^{(3)}\right)
\end{array}
\end{eqnarray}

\begin{eqnarray}
\begin{array}{l}
T^{+(\text {clear}, 2)}=T^{(\text {clear})} B^{(1)}
\end{array}
\end{eqnarray}

\begin{eqnarray}
\begin{array}{l}
T^{-(\text {cloud}, 1)}=T^{(\text {cloud})} B^{(4)}
\end{array}
\end{eqnarray}

\begin{eqnarray}
\begin{array}{l}
T^{-(\text {cloud}, 2)}=T^{(\text {cloud})}\left(1-B^{(2)}\right)
\end{array}
\end{eqnarray}

\begin{eqnarray}
\begin{array}{l}
T^{-(\text {clear}, 1)}=T^{(\text {clear})}\left(1-B^{(4)}\right)
\end{array}
\end{eqnarray}

\begin{eqnarray}
\begin{array}{l}
T^{-(\text {clear}, 2)}=T^{(\text {clear})} B^{(2)}
\end{array}
\end{eqnarray}

Also, R matrixes for maximal/random approximation are calculated.

\begin{eqnarray}
\begin{array}{l}
R^{+(\text {cloud}, 1)}=R^{(\text {cloud})} B^{(3)}
\end{array}
\end{eqnarray}

\begin{eqnarray}
\begin{array}{l}
R^{+(\text {cloud}, 2)}=R^{(\text {cloud})}\left(1-B^{(1)}\right)
\end{array}
\end{eqnarray}

\begin{eqnarray}
\begin{array}{l}
R^{+(\text {clear}, 1)}=R^{(\text {clear})}\left(1-B^{(3)}\right)
\end{array}
\end{eqnarray}

\begin{eqnarray}
\begin{array}{l}
R^{+(\text {clear}, 2)}=R^{(\text {clear})} B^{(1)}
\end{array}
\end{eqnarray}

\begin{eqnarray}
\begin{array}{l}
R^{-(\text {cloud}, 1)}=R^{(\text {cloud})} B^{(4)}
\end{array}
\end{eqnarray}

\begin{eqnarray}
\begin{array}{l}
R^{-(\text {cloud,2})}=R^{(\text {cloud})}\left(1-B^{(2)}\right)
\end{array}
\end{eqnarray}

\begin{eqnarray}
\begin{array}{l}
R^{-(\text {clear}, 1)}=R^{(\text {clear})}\left(1-B^{(4)}\right) \\
\end{array}
\end{eqnarray}

\begin{eqnarray}
\begin{array}{l}
R^{-(\text {clear}, 2)}=R^{(\text {clear})} B^{(2)}
\end{array}
\end{eqnarray}

This calculation is done for each sub-channel and each layer.

\hypertarget{adding-of-source-functions-for-each-layer}{%
\subsubsection{Adding of source functions for each
layer}\label{adding-of-source-functions-for-each-layer}}

In this section, \texttt{SUBROUTINE:{[}ADDMR{]}} and
\texttt{SUBROTINE:{[}ADDING{]}} in pradt.F is described.

By using transmission coefficient \(T\), reflection coefficient \(R\),
and radiation source function \(\varepsilon\) in all layers, the
radiation fluxes \(u\) at each layer boundary can be obtained by using
the adding method. This means that the when two layers of \(T\), \(R\),
\(\varepsilon\) are known, the \(T\), \(R\), \(\varepsilon\) of the
whole combined layer of the two layers can be easily calculated.

\begin{figure}
\centering
\includegraphics{Prad_Fig3.png}
\caption{Schematic illustration of the adding method}
\end{figure}

\hypertarget{subroutineaddmr}{%
\paragraph{\texorpdfstring{\texttt{SUBROUTINE:{[}ADDMR{]}}}{SUBROUTINE:{[}ADDMR{]}}}\label{subroutineaddmr}}

In this subroutine, the maximal/random flux in cloudy conditions is
calculated by the adding method and the T, R, and S matrixes are used
for calculations.

First, the upward radiation source function the bottom layer is
calculated.

In the shortwave region,

\begin{eqnarray}
\begin{array}{l}
\epsilon_{N}^{-(\text {cloud})}=W^{+} \alpha_{s} \mu_{0}\left(\frac{1}{\mu}\right) F_{0} e_{N}^{-\left\langle\tau^{*}\right\rangle / \mu_{0}(\text{cloud})}
\end{array}
\end{eqnarray}

\begin{eqnarray}
\begin{array}{l}
\epsilon_{N}^{-(\text {clear})}=W^{+} \alpha_{s} \mu_{0}\left(\frac{1}{\mu}\right) F_{0} e_{N}^{-\left\langle\tau^{*}\right\rangle / \mu_{0} \text {(clear) }}
\end{array}
\end{eqnarray}

\(\left\langle\tau^{*}\right\rangle\) indicates the total optical
thickness \(\tau^{*}\) of from the upper end of the atmosphere to the
upper end of the layer currently being considered and
\(e^{-\left\langle\tau^{*}\right\rangle / \mu_{0}}\) is calculated in
\texttt{SUBROUTINE:{[}ADDMR{]}} of pradt.F.

In the longwave region,

\begin{eqnarray}
\begin{array}{l}
\epsilon_{N}^{-(\text {cloud})}=\epsilon_{N}^{-(\text {cloud})}+W^{+} 2 \pi\left(1-\alpha_{s}\right) B_{N} C_{N}
\end{array}
\end{eqnarray}

\begin{eqnarray}
\begin{array}{l}
\epsilon_{N}^{-(\text {clear})}=\epsilon_{N}^{-(\text {clear})}+W^{+} 2 \pi\left(1-\alpha_{s}\right) B_{N}\left(1-C_{N}\right)
\end{array}
\end{eqnarray}

Here,

\begin{eqnarray}
W^{+} \equiv \mu^{1 / 2}
\end{eqnarray}

The reflectance \(R_{1, n}^{-}\) and source function
\(\epsilon_{1, n}^{+}\) regarded from the first to the \(n\) layers as a
single layer are

\begin{eqnarray}
\begin{array}{l}
\epsilon_{1, n}^{+}=\epsilon_{n}^{+}+T_{n}^{+}\left(1-R_{n}^{+} R_{1, n-1}^{-}\right)^{-1}\left(R_{1, n-1}^{-} \epsilon_{n}^{-}+\epsilon_{1, n-1}^{+}\right)
\end{array}
\end{eqnarray}

\begin{eqnarray}
\begin{array}{l}
R_{1, n}^{-}=R_{n}^{-}+T_{n}^{+}\left(1-R_{n}^{+} R_{1, n-1}^{-}\right)^{-1} R_{1, n-1}^{-} T_{n}^{-}
\end{array}
\end{eqnarray}

The upward and downward fluxes at the bottom of the atmosphere are

\begin{eqnarray}
\begin{array}{l}
u_{N+1 / 2}^{+}=\left(1-R_{1, N-1}^{-} R_{N}^{+}\right)^{-1}\left(\epsilon_{1, N-1}^{+}+R_{1, N-1}^{-} \epsilon_{N}^{-}\right)
\end{array}
\end{eqnarray}

\begin{eqnarray}
\begin{array}{l}
u_{N+1 / 2}^{-}=\left(1-R_{N}^{+} R_{1, N-1}^{-}\right)^{-1}\left(\epsilon_{N}^{-}+R_{N}^{+} \epsilon_{1, N-1}^{+}\right)
\end{array}
\end{eqnarray}

Also, upward and downward fluxes at the boundary between layers \(n\)
and \(n+1\) are \begin{eqnarray}
\begin{array}{l}
u_{n+1 / 2}^{+}=\left(1-R_{1, n-1}^{-} R_{n}^{+}\right)^{-1}\left(R_{1, n-1}^{-} T_{n}^{-} u_{n+1 / 2}^{-}+R_{1, n-1}^{-} \epsilon_{n}^{-}+\epsilon_{1, n}^{+}\right)
\end{array}
\end{eqnarray} \begin{eqnarray}
\begin{array}{l}
u_{n+1 / 2}^{-}=\left(1-R_{n}^{+} R_{1, n-1}^{-}\right)^{-1}\left(T_{n}^{-} u_{n+1 / 2}^{-}+R_{n}^{+} \epsilon_{1, n-1}^{+}+\epsilon_{n}^{-}\right)
\end{array}
\end{eqnarray}

However, the upward and downward flux at the upper end of the atmosphere
is as follows.

\begin{eqnarray}
\begin{array}{l}
u_{1 / 2}^{+}=0
\end{array}
\end{eqnarray}

\begin{eqnarray}
\begin{array}{l}
u_{1 / 2}^{-}=T_{1}^{-} u_{1+1 / 2}^{-}+\epsilon_{1}^{-}
\end{array}
\end{eqnarray}

Finally, since this flux is scaled, we rescaled and added the direct
solar incidence to the find the radiation flux. Furthermore, the flux in
the cloudy area and the clear sky area is added.

\begin{eqnarray}
\begin{array}{c}
F_{n+1 / 2}^{+}=\frac{W^{+}}{W}\left(u_{n+1 / 2}^{+(\text{cloud})}+u_{n+1 / 2}^{+(\text{clear})}\right)+\mu_{0} F_{0}\left(e_{n+1 / 2}^{-\left\langle\tau^{*}\right\rangle / \mu_{0}(\text {cloud})}\right)
\end{array}
\end{eqnarray}

\begin{eqnarray}
\begin{array}{c}
F_{n+1 / 2}^{-}=\frac{W^{+}}{W}\left(u_{n+1 / 2}^{-(\text{cloud})}+u_{n+1 / 2}^{-(\text {clear})}\right)
\end{array}
\end{eqnarray}

Also, surface direct downward radiation flux \(F_{s f}^{+}\) is

\begin{eqnarray}
F_{s f}^{+}=\mu_{0} F_{0}\left(e_{N}^{-\left\langle\tau^{*}\right\rangle / \mu_{0}(\text {cloud})}+e_{N}^{-\left\langle\tau^{*}\right\rangle / \mu_{0} \text{(clear)}}\right)
\end{eqnarray}

This calculation is done for each sub-channel.

\hypertarget{subroutineadding}{%
\paragraph{\texorpdfstring{\texttt{SUBROUTINE:{[}ADDING{]}}}{SUBROUTINE:{[}ADDING{]}}}\label{subroutineadding}}

Since the maximal/random approximation cannot be used under the clear
sky condition, this subroutine is used to calculate the flux.

First, the radiation source function, which is the sum of both the plank
function origin and the solar incident origin, is \begin{eqnarray}
\epsilon^{\pm}=\hat{\epsilon}_{S}^{\pm} e^{-\left\langle\tau^{*}\right\rangle / \mu_{0}}+\hat{\epsilon}_{A}^{\pm}
\end{eqnarray} There are layers \(1, 2,\dots, N\) from the top. The surface layer is
considered to be a single layer and the \(N\) layer. Given the
reflectance and source function of the layers from the n to the \(N\)
layer as a single layer \(R_{n, N}\), \(\epsilon_{n, N}^{-}\), \begin{eqnarray}
\begin{array}{c}
R_{n, N}=R_{n, N}+T_{n}\left(1-R_{n+1, N} R_{n}\right)^{-1} R_{n+1} T_{n} \\
\end{array}
\end{eqnarray} \begin{eqnarray}
\begin{array}{c}
\epsilon_{n, N}^{-}=\epsilon_{n}^{-}+T_{n}\left(1-R_{n+1, N} R_{n}\right)^{-1}\left(R_{n+1, N} \epsilon_{n}^{+}+\epsilon_{n, N}^{-}\right)
\end{array}
\end{eqnarray}

\begin{eqnarray}
\begin{array}{c}
R_{n, N}=R_{n, N}+T_{n}\left(1-R_{n+1, N} R_{n}\right)^{-1} R_{n+1} T_{n} \\
\end{array}
\end{eqnarray}

\begin{eqnarray}
\begin{array}{c}
\epsilon_{n, N}^{-}=\epsilon_{n}^{-}+T_{n}\left(1-R_{n+1, N} R_{n}\right)^{-1}\left(R_{n+1, N} \epsilon_{n}^{+}+\epsilon_{n, N}^{-}\right)
\end{array}
\end{eqnarray}

This can be solved by \(n=N-1,\dots,1\) in sequence, starting from the
values at the surface \(R_{N, N}\), \(\epsilon_{N, N}^{-}\).

\begin{eqnarray}
\begin{array}{c}
R_{N, N}=R_{N}=\alpha_{s} \\
\end{array}
\end{eqnarray}

\begin{eqnarray}
\begin{array}{c}
\epsilon_{N, N}=W^{+}\left(\alpha_{s} \mu_{0}\left(\frac{1}{\mu}\right) e^{-\left\langle\tau^{*}\right\rangle / \mu_{0}} F_{0}+2 \pi\left(1-\alpha_{s}\right) B_{N}\right)
\end{array}
\end{eqnarray}

The reflectance \(R_{1, n}\) and source function \(\epsilon_{1, n}^{+}\)
regarded from the first to the \(n\) layers as a single layer are

\begin{eqnarray}
\begin{array}{c}
R_{1, n}=R_{n}+T_{n}\left(1-R_{1, n-1} R_{n}\right)^{-1} R_{1, n-1} T_{n}
\end{array}
\end{eqnarray}

\begin{eqnarray}
\begin{array}{c}
\epsilon_{1, N}^{+}=\epsilon_{n}^{+}+T_{n}\left(1-R_{1, n-1} R_{n}\right)^{-1}\left(R_{1, n-1} \epsilon_{n}^{-}+\epsilon_{1, n-1}^{+}\right)
\end{array}
\end{eqnarray}

It can be solved by \(n=2,\dots, N\), starting from
\(R_{1,1}=R_{1}, \epsilon_{1,1}^{+}=\epsilon_{1}^{+}\).

With these, downward flux at the boundary between layers \(n\) and
\(n+1\), the downward and upward flux are came back to a problem between
two layers of combined layer, the combinations of layers \(1-n\) and
\(n+1-N\).

\begin{eqnarray}
\begin{array}{c}
u_{n+1 / 2}^{+}=\left(1-R_{1, n} R_{n+1, N}\right)^{-1}\left(\epsilon_{1, n}^{+}+R_{1, n} \epsilon_{n+1, N}^{-}\right)
\end{array}
\end{eqnarray}

\begin{eqnarray}
\begin{array}{c}
u_{n+1 / 2}^{-}=R_{n+1, N} u_{n, n+1}^{+}+\epsilon_{n+1, N}^{-}
\end{array}
\end{eqnarray}

can be written as above. However, the flux at the top of the atmosphere
is

\begin{eqnarray}
\begin{array}{c}
u_{1 / 2}^{+}=0
\end{array}
\end{eqnarray}

\begin{eqnarray}
\begin{array}{c}
u_{1 / 2}^{-}=\epsilon_{1, N}^{-}
\end{array}
\end{eqnarray}

Finally, since this flux is scaled, we rescaled and added the direct
solar incidence to the find the radiation flux.

\begin{eqnarray}
\begin{array}{c}
F_{n+1 / 2}^{+}=\frac{W^{+}}{W} u_{n+1 / 2}^{+}+\mu_{0} e^{-\left\langle\tau^{*}\right\rangle / \mu_{0}} F_{0}
\end{array}
\end{eqnarray}

\begin{eqnarray}
\begin{array}{c}
F_{n+1 / 2}^{-}=\frac{W^{+}}{W} u_{n+1 / 2}^{-}
\end{array}
\end{eqnarray}

Also, surface direct downward radiation flux \(F_{s f}^{+}\) is

\begin{eqnarray}
F_{s f}^{+}=\mu_{0} F_{0} e^{-\left\langle\tau^{*}\rangle\ \mu_{0}\right.}
\end{eqnarray}

\hypertarget{adding-in-the-flux}{%
\subsubsection{Adding in the flux}\label{adding-in-the-flux}}

\begin{eqnarray}
F^{\pm}=\sum_{c} w_{c}\left(1-C_{c u}\right) \bar{F}^{\pm}+\sum_{c} w_{c} C_{c u} F^{c \pm}
\end{eqnarray}

\begin{eqnarray}
F^{\circ \pm}=\sum_{c} w_{c} F^{\circ \pm}
\end{eqnarray}

If the radiation flux \(F_{C}^{\pm}\) is found for each sub-channel in
each layer, the wavelength-integrated flux is found by multiplying a
weight \(w_c\) correspondingly to a wavelength representative of the
sub-channel and adding. \(C_{cu}\) is the horizontal coverage of the
cumulus cloud. It is divided into the short wavelength range and long
wavelength range and added together. In addition, the downward flux of a
part of the short wavelength region (shorter than the wavelength of
0.7-0.8 \(\mathrm{{\mu}m}\)) at the surface is obtained as PAR
(photosynthetically active radiation). Also, the radiation flux in the
clear-sky condition is calculated (\(F^{\circ \pm}\)).

Also, in the shortwave region, we find the downward radiation at the
lower end of the atmosphere and the difference between the surface
direct downward radiation flux.

\begin{eqnarray}
F_{s f}^{+}=\sum_{c} w_{c}\left(1-C_{u}\right) \bar{F}_{N+1 / 2}^{+}+\sum_{c} w_{c} C_{c u} F_{N+1 / 2}^{c}
\end{eqnarray}

\begin{eqnarray}
F_{s f, d i f}^{+}=\sum_{c} w_{c}\left(1-C_{c u}\right)\left(\bar{F}_{N+1 / 2}^{+}-\bar{F}_{s f}^{+}\right)+\sum_{c} w_{c} C_{c u}\left(F_{N+1 / 2}^{c}-F_{s f}^{c}\right)
\end{eqnarray}

This calculation is done in \texttt{SUBROUTINE:{[}DTRN31{]}}

\hypertarget{calculation-of-the-temperature-derivative-of-the-flux}{%
\subsubsection{Calculation of the temperature derivative of the
flux}\label{calculation-of-the-temperature-derivative-of-the-flux}}

To implicitly solve for surface temperature, calculate differential term
of upward flux with respect to surface temperature
\(\mathrm{d}F^{\mp}/dT_{g}\). Therefore, we obtained the value for
temperatures \(1\text{K}\) higher than \(T_g\)
\(\bar{B}^{w}\left(T_{g}+1\right)\) and used it to redo the flux
calculation using the addition method, and the difference from the
original value is set to \(\mathrm{d}F^{\mp}/dT_{g}\). This is a
meaningful value only in the longwave region (earth radiation region).
This calculation is done in \texttt{SUBROUTINE:{[}RADFLX{]}} of pradt.F.

\hypertarget{calculation-of-the-heating-rate}{%
\subsubsection{Calculation of the heating
rate}\label{calculation-of-the-heating-rate}}

The heating rate of the nth layer \(H_n\) is calculated by using the
radiation flux obtained so far. It is calculated separately for
shortwave and longwave ranges, and finally add together
(\texttt{SUBROUTINE:{[}RDTND{]}} in pradm.F).

\begin{eqnarray}
H_{n}=-\frac{\left(F_{n}^{-}-F_{n}^{-}\right)-\left(F_{n}^{+}-F_{n+1}^{+}\right)}{g C_{p} d p}
\end{eqnarray}

\hypertarget{flux-of-incidence-and-incident-angle}{%
\subsubsection{Flux of incidence and incident
angle}\label{flux-of-incidence-and-incident-angle}}

In this section, \texttt{SUBROUTINE:{[}SHTINS{]}} in pradi.F is
described.

The following parameters are determined using the orbital eccentricity
\(e\), with reference to Berger (1978).

\begin{eqnarray}
\begin{array}{l}
\beta=\sqrt{1-e^{2}}
\end{array}
\end{eqnarray}

\begin{eqnarray}
\begin{array}{l}
a_{1}=-2\left(\frac{1}{2} e+\frac{1}{8} e^{3}\right)(1+\beta)
\end{array}
\end{eqnarray}

\begin{eqnarray}
\begin{array}{l}
a_{2}=-2\left(-\frac{1}{4} e^{2}\right)\left(\frac{1}{2}+\beta\right)
\end{array}
\end{eqnarray}

\begin{eqnarray}
\begin{array}{l}
a_{3}=-2\left(\frac{1}{8} e^{3}\right)\left(\frac{1}{2}+\beta\right) \\
\end{array}
\end{eqnarray}

\begin{eqnarray}
\begin{array}{l}
b_{1}=2 e-\frac{1}{4} e^{3}
\end{array}
\end{eqnarray}

\begin{eqnarray}
\begin{array}{l}
b_{2}=\frac{5}{4} e^{2}
\end{array}
\end{eqnarray}

\begin{eqnarray}
\begin{array}{l}
b_{3}=\frac{13}{12} e^{3}
\end{array}
\end{eqnarray}

Additionally,

\begin{eqnarray}
\begin{array}{l}
\epsilon=\frac{e p s d}{180} \pi
\end{array}
\end{eqnarray}

\begin{eqnarray}
\begin{array}{l}
\varpi=\frac{v p i d+180}{180} \pi
\end{array}
\end{eqnarray}

where \(epsd\) and \(vpid\) are the angle of the obliquity, and the
precession represented by the true longitude of the perihelion measured
from the vernal equinox.

The mean longitude of the vernal equinox \(\lambda_{0}\) is computed.

\begin{eqnarray}
\begin{array}{c}
\lambda_{0}=a_{1} \sin (-\varpi)+a_{2} \sin (-2 \varpi)+a_{3} \sin (-3 \varpi)
\end{array}
\end{eqnarray}

The mean longitude of the Earth position \(\lambda_{m}\) at time \(t_m\)
is represented by using the position of the vernal equinox
\(\lambda_{0}\).

\begin{eqnarray}
\begin{array}{c}
\lambda_{m}=\frac{t_{m}-t_{0}}{2 \pi \times t_{year}}+\lambda_{0}
\end{array}
\end{eqnarray}

where \(t_{year}\) is the number of seconds in year, and the origin of
the time \(t_0\) is defined as the time of the vernal (march) equinox.

The true longitude of the Earth position \(V\) at time \(t_m\) is
calculated as below.

\begin{eqnarray}
\\V=\lambda_{m}-\varpi+b_{1} \sin \left(\lambda_{m}-\varpi\right)+b_{2} \sin 2\left(\lambda_{m}-\varpi\right)+b_{3} \sin 3\left(\lambda_{m}-\varpi\right)
\end{eqnarray}

The solar declination \(\delta\) is

\begin{eqnarray}
\delta=\arcsin (\sin \epsilon \sin (V+\varpi))
\end{eqnarray}

The incident angle \(\cos \zeta\) is founded by using the latitude
\(\varphi\), the solar declination \(\delta\), and the hour angle at a
point of longitude \(h\).

\begin{eqnarray}
\mu_{0}=\cos \zeta=\sin \varphi \sin \delta+\cos \varphi \cos \delta \cosh
\end{eqnarray}

Incident Flux \(F_0\) is represented as follows,

\begin{eqnarray}
\begin{array}{c}
F_{0}=F_{00} r^{-2}
\end{array}
\end{eqnarray}

\begin{eqnarray}
\begin{array}{c}
r=\frac{1-e^{2}}{1+e(\cos V+\varpi)}
\end{array}
\end{eqnarray}

where \(F_{00}\) is the solar constant and is the ratio to the time of
the distance between the sun and the earth. The number of times when
\(\cos \zeta \geq 0\) (in the daytime) in time increments (set in
NHSUB), is counted, and \(F_{0}\) and \(\cos \zeta\) are finally
averaged.

It is also possible to give average annual insolation. In this case, the
annual and day mean incidence and angle of incidence are approximated as
follows.

\begin{eqnarray}
\begin{array}{c}
\bar{F}=F_{00} / \pi
\end{array}
\end{eqnarray}

\begin{eqnarray}
\begin{array}{c}
\bar{\mu}_{0} \simeq 0.410+0.590 \cos ^{2} \varphi
\end{array}
\end{eqnarray}

\hypertarget{reading-the-each-parameter}{%
\subsubsection{Reading the each
parameter}\label{reading-the-each-parameter}}

In \texttt{SUBROUTINE:{[}OPPARM2{]}}of pradt.F, various parameters used
for radiation calculation are read. The outline of the procedure is
shown below.

\begin{enumerate}
\def\labelenumi{\arabic{enumi}.}
\item
  Read the numbers of bands, the radiances representative of upward and
  downward radiation, the grids of the pressure \(\log (\mathrm{p})\),
  grid of the temperatures, the optical flag, and CFCs.
\item
  Read the band boundaries and the information of the pressure grid and
  temperature grid.
\item
  First, the optical property flag, the number of channels, the weights
  for channels and the number of the molecules including in a waveband
  are read. The optical property flag is modified in order to
  distinguish the PAR, 0.67, 0.5, and 10.5 \(\mathrm{{\mu}m}\).
  Additionally, molecule ID and \(k\)-width are read for each molecule.
  Finally, the absorption coefficient for the self-broadening and CFC
  are also done only when the optical property flag in the band is
  positive. The k-width and the absorption coefficient for the
  self-broadening are arranged in the order of the grid of the
  temperatures, the grids of the pressures, and the channels. Step 3 is
  performed for each wavelength band.
\item
  First, the number of particles is read. Next, the numbers of the modes
  and the mode radius (or relative humidity) are read for each particle.
  Using the mode radius (or relative humidity), the following parameter
  required to interpolate the calculated values is calculated for each
  mode number.
\end{enumerate}

\begin{eqnarray}
   \frac{1}{R_{n+1}^{(r e f)}-R_{n}^{(r e f)}}
\end{eqnarray}

\begin{enumerate}
\def\labelenumi{\arabic{enumi}.}
\setcounter{enumi}{4}
\item
  Read the band boundaries again.
\item
  Read the Plank function coefficient, solar insolation, surface
  properties (not output), Rayleigh scattering coefficient, Rayleigh
  scattering phase function. The moment for particle scattering phase
  function is read in the order of the particle and the optical number
  and read up to the second moment. Step 6 is performed for each
  wavelength band.
\end{enumerate}

\hypertarget{other-notes}{%
\subsubsection{Other notes}\label{other-notes}}

The calculation of the radiation is usually not done at every step.
Thus, the radiation flux is saved, and it is used if the time is not
used for radiation calculation. As for the shortwave radiation, using
the percentage of time (time is \(\mu_{0}>0\)) between the next
calculation time \(f\) and the solar incidence angle factor averaged
over the daylight hours \(\bar{\mu}_{0}\), seek the Flux \(\bar{F}\),

\begin{eqnarray}
{F}=f \frac{\mu_{0}}{\bar{\mu}_{0}} \bar{F}
\end{eqnarray}
