\hypertarget{surface-flux.}{%
\subsection{Surface Flux.}\label{surface-flux.}}

\textbf{NOTE: the descriptions in this section are outdated.}

\hypertarget{overview-of-the-surface-flux-scheme}{%
\subsubsection{Overview of the Surface Flux
Scheme}\label{overview-of-the-surface-flux-scheme}}

The surface flux scheme evaluates the physical quantity fluxes between
the atmospheric surfaces due to turbulent transport in the boundary
layer. The main input data are wind speed (\(u, v\)), temperature
(\(T\)), and specific humidity (\(q\)), and the output data are the
vertical fluxes and the differential values (for obtaining implicit
solutions) of momentum, heat, and water vapor.

The bulk coefficients are obtained according to Louis (1979) and Louis
et al.~(1982), except for the correction for the difference in roughness
between momentum and heat. However, corrections are made to take into
account the difference between momentum and heat roughness.

The outline of the calculation procedure is as follows.

Calculate the Richardson number as the stability of the atmosphere.

\begin{enumerate}
\def\labelenumi{\arabic{enumi}.}
\item
  calculate the bulk coefficient from Richardson number
  \texttt{MODULE:{[}PSFCL{]}}.
\item
  calculate the flux and its derivative from the bulk coefficient.
\end{enumerate}

If necessary, the calculated fluxes are re-calculated after taking into
account the roughness effect, the free flow effect, and the wind speed
correction.

\hypertarget{basic-formula-for-flux-calculations}{%
\subsubsection{Basic Formula for Flux
Calculations}\label{basic-formula-for-flux-calculations}}

Surface fluxes (\(F_u, F_v, F_\theta, F_q\)) are expressed using the
bulk coefficients (\(C_M, C_H, C_E\)) as follows

\begin{eqnarray}
Fu  =  - \rho C_M |{\mathbf{v}}| u
\end{eqnarray}

\begin{eqnarray}
Fv  =  - \rho C_M |{\mathbf{v}}| v
\end{eqnarray}

\begin{eqnarray}
F\theta  = \rho c_p C_H |{\mathbf{v}}| ( \theta_g - \theta )
\end{eqnarray}

\begin{eqnarray}
Fq^P =  \rho C_E |{\mathbf{v}}| ( q_g - q )
\end{eqnarray}

Note that \(F_q^P\) is the possible evaporation rate. The calculation of
actual evaporation is described in the sections ``Surface processes''
and ``Solution of diffuse-type balance equations for atmospheric surface
systems''.

\hypertarget{richardson-number}{%
\subsubsection{Richardson Number}\label{richardson-number}}

The bulk Richardson number (\(R_{iB}\)), which is used as a benchmark
for the stability between the atmospheric surfaces, is

\begin{eqnarray}
R_{iB} = \frac{ \frac{g}{\theta_s} (\theta_1 - \theta(z_0))/z_1 }
              { (u_1/z_1)^2                                  }
       = \frac{g}{\theta_s}
         \frac{T_1 (p_s/p_1)^\kappa - T_0 }{u_1^2/z_1} f_T .
\end{eqnarray}

Here,

\begin{eqnarray}
f_T = (\theta_1 - \theta(z_0))/(\theta_1 - \theta_0)
\end{eqnarray}

is a correction factor, which is approximated from the uncorrected bulk
Richardson number, but we abbreviate the calculation here.

\hypertarget{bulk-factor}{%
\subsubsection{Bulk factor}\label{bulk-factor}}

The bulk coefficients of \(C_M,C_H,C_E\) are calculated according to
Louis (1979) and Louis et al.~(1982). However, corrections are made to
take into account the difference between momentum and heat roughness. If
the roughnesses for momentum, heat, and water vapor are set to
\(z_{0M}, z_{0H}, z_{0E}\), respectively, the results are generally
\(z_{0M} > z_{0H}, z_{0E}\), but the bulk coefficients for heat and
water vapor for the fluxes from the height of \(z_{0M}\) are also set to
\(\widetilde{C_H}\), \(\widetilde{C_E}\) first, and then corrected.

\begin{eqnarray}
C_M = \left\{
      \begin{array}{lr}
      C_{0M} [ 1 + (b_M/e_M)  R_{iB} ]^{-e_M}
            &,
          R_{iB} \geq 0 \\
      C_{0M} \left[ 1 - b_M R_{iB} \left( 1+ d_M b_M C_{0M}
                                  \sqrt{\frac{z_1}{z_{0M}}| R_{iB}|} \,
                                  \right)^{-1} \right]
          &,
          R_{iB} < 0 \\
      \end{array} \right.
\end{eqnarray}

\begin{eqnarray}
\widetilde{C_H} = \left\{
      \begin{array}{lr}
      \widetilde{C_{0H}} [ 1 + (b_H/e_H) R_{iB} ]^{-e_H}
            &,
          R_{iB} \geq 0 \\
      \widetilde{C_{0H}} \left[ 1 - b_H R_{iB}
                                  \left( 1+ d_H b_H \widetilde{C_{0H}}
                                  \sqrt{\frac{z_1}{z_{0M}}| R_{iB}|} \,
                                  \right)^{-1} \right]
             &,
          R_{iB} < 0 \\
      \end{array} \right.
\end{eqnarray}

\begin{eqnarray}
C_H = \widetilde{C_H} f_T
\end{eqnarray}

\begin{eqnarray}
\widetilde{C_E} = \left\{
      \begin{array}{lr}
      \widetilde{C_{0E}} [ 1 + (b_E/e_E) R_{iB} ]^{-e_E}
            &,
          R_{iB} \geq 0 \\
      \widetilde{C_{0E}} \left[ 1 - b_E R_{iB}
                                  \left( 1+ d_E b_E \widetilde{C_{0E}}
                                  \sqrt{\frac{z_1}{z_{0M}}| R_{iB}|} \,
                                  \right)^{-1} \right]
          &,
          R_{iB} < 0 \\
      \end{array} \right.
\end{eqnarray}

\begin{eqnarray}
C_E = \widetilde{C_E} f_q
\end{eqnarray}

\(C_{0M}, \widetilde{C_{0H}}, \widetilde{C_{0E}}\) is the bulk
coefficient (for fluxes from \(z_{0M}\)) at neutral,

\begin{eqnarray}
C_{0M}  =  \widetilde{C_{0H}}  =  \widetilde{C_{0E}}  =
       \frac{k^2}{\left[\ln \left(\frac{z_1}{z_{0M}}\right)\right]^2 } .
\end{eqnarray}

Correction Factor \(f_q\) is ,

\begin{eqnarray}
  f_q = (q_1 - q(z_0))/(q_1 - q^{\ast}(\theta_0))
\end{eqnarray}

but the method of calculation is omitted. The coefficients are
\(( b_M, d_M, e_M ) = ( 9.4, 7.4, 2.0 )\),
\(( b_H, d_H, e_H ) = ( b_E, d_E, e_E ) = ( 9.4, 5.3, 2.0 )\).

\hypertarget{calculating-flux}{%
\subsubsection{Calculating Flux}\label{calculating-flux}}

This will calculate the flux.

\begin{eqnarray}
\hat{F}_{u,1/2}  =  - \rho_{1/2} C_M |{\mathbf{v}}_1| u_1
\end{eqnarray}

\begin{eqnarray}
\hat{F}_{v,1/2}  =  - \rho_{1/2} C_M |{\mathbf{v}}_1| v_1
\end{eqnarray}

\begin{eqnarray}
\hat{F}_{\theta,1/2}  = \rho_{1/2} c_p C_H |{\mathbf{v}}_1|
                    \left( T_0 - \sigma_1^{-\kappa} T_1 \right)
\end{eqnarray}

\begin{eqnarray}
\hat{F}_{q,1/2}^P  =  \rho_{1/2} C_E |{\mathbf{v}}_1|
                    \left( q^*(T_0) - q_1 \right)
\end{eqnarray}

The differential term is as follows

\begin{eqnarray}
\frac{\partial{F_{u,1/2}}}{\partial {u_1}} = \frac{\partial{F_{v,1/2}}}{\partial {v_1}}
= - \rho_{1/2} C_M |{\mathbf{v}}_1|
\end{eqnarray}

\begin{eqnarray}
\frac{\partial{F_{\theta,1/2}}}{\partial {T_1}}
= - \rho_{1/2} c_p C_H |{\mathbf{v}}_1| \sigma_1^{-\kappa}
\end{eqnarray}

\begin{eqnarray}
\frac{\partial{F_{\theta,1/2}}}{\partial {T_0}}
= \rho_{1/2} c_p C_H |{\mathbf{v}}_1|
\end{eqnarray}

\begin{eqnarray}
\frac{\partial{F_{q,1/2}}}{\partial {q_1}}
 =  - \beta \rho_{1/2} C_E |{\mathbf{v}}_1|
\end{eqnarray}

\begin{eqnarray}
\frac{\partial{F_{q,1/2}^P}}{\partial {T_0}}
 =  \beta \rho_{1/2} C_E |{\mathbf{v}}_1| \left( \frac{d {q^*}}{d {T}} \right)_{1/2}
\end{eqnarray}

Here, it is important to note that \(T_0\) is a quantity that is not
required at this point in time. The surface temperature is the condition
of the surface heat balance

\begin{eqnarray}
   F_\theta(T_0,T_1) + L \beta F_q^P(T_0,q_1) + F_R(T_0) - F_g(T_0,G_1) = 0
\end{eqnarray}

is determined to satisfy. At this point, \(T_0\) evaluates the fluxes in
the previous time step. The true value of the flux that satisfies the
surface balance is determined by solving this equation in conjunction
with surface processes. In this sense, we have marked the above fluxes
with \(\hat{{}}\).

\hypertarget{handling-at-sea-level}{%
\subsubsection{handling at sea level}\label{handling-at-sea-level}}

At sea level, we follow
\href{papers/Millers1992_Measuring_dynamic_surface\%20and_interfacial_tensions.pdf}{Miller
et al.~(1992)} and consider the following two effects.

\begin{itemize}
\item
  Free convection is preeminent when the wind speed is low
\item
  The roughness of the sea surface varies with the wind speed.
\end{itemize}

The effect of free convective motion is estimated by calculating the
buoyancy flux \(F_B\),

\begin{eqnarray}
  F_B = F_\theta/c_p + \epsilon T_0 F_q^P
\end{eqnarray}

When it was \(F_B >0\),

\begin{eqnarray}
  w^* = ( H_{B} F_B )^{1/3}
\end{eqnarray}

\begin{eqnarray}
  |{\mathbf{v}}_1| = \left( u_1^2 + v_1^2 + (w^*)^2 \right)^{1/2}
\end{eqnarray}

The \(H_B\) corresponds to the mixed layer thickness scale. \(H_B\)
corresponds to the thickness scale of the mixing layer. The current
standard value is \(H_B=2000\) m.

The roughness variation of the sea surface is determined by the friction
velocity (\(u^*\))

\begin{eqnarray}
  u^* = \left( \sqrt{Fu^2 + Fv^2}/\rho \right)^{1/2}
\end{eqnarray}

with ,

\begin{eqnarray}
  Z_{0M}  =  A_M + B_M (u^*)^2/g + C_M \nu/u^* \\
  Z_{0H}  =  A_H + B_H (u^*)^2/g + C_H \nu/u^* \\
  Z_{0E}  =  A_E + B_E (u^*)^2/g + C_E \nu/u^*
\end{eqnarray}

The evaluation is performed as follows. \(\nu=1.5\times10^{-5}\) m\(^2\)
s\(^{-1}\) is the kinematic viscosity coefficient of the atmosphere and
the other standard values of the coefficients are
\((A_M, B_M, C_M) = (0, 0.018, 0.11)\),
\((A_H, B_H, C_H) = (1.4\times10^{-5}, 0, 0.4)\),
\((A_E, B_E, C_E) = (1.3\times10^{-4}, 0, 0.62)\).

In the above calculations, we perform successive approximation
calculations because \(F_u, F_v, F_\theta, F_q\) are required.

\hypertarget{wind-speed-correction}{%
\subsubsection{Wind Speed Correction}\label{wind-speed-correction}}

In general, the downward transport of momentum is more efficient on a
large rough surface than on a small rough surface, which results in a
weaker wind over the surface, and the difference in wind speed can
cancel out the difference in \(C_D\) due to the difference in roughness.

In the model, the wind speed passed to the surface flux calculation is
the value calculated by the time integration of the dynamic process and
smoothed by the spectral expansion. Therefore, this compensation effect
cannot be well represented in a region where the land and sea surfaces
with widely different roughnesses coexist at small scales, for example,
where the sea surface and the land surface. The momentum fluxes are
calculated once, and then the wind speed in the lowermost layer of the
atmosphere is corrected by the fluxes and the momentum, heat and water
fluxes are recalculated once again.

\hypertarget{minimum-wind-speed.}{%
\subsubsection{Minimum wind speed.}\label{minimum-wind-speed.}}

The minimum value of the surface wind speed (\(|{\mathbf{v}}_1|\)) for
calculating the surface fluxes is set to take into account the effects
of small-scale motions. The current standard value is
\(3 \mathrm{m/s}\), which is common to all fluxes.
