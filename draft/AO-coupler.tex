\hypertarget{ux30abux30c3ux30d7ux30e9ux30fc}{%
\section{カップラー}\label{ux30abux30c3ux30d7ux30e9ux30fc}}

\begin{itemize}
\tightlist
\item
  大気モデルへのフラックス

  \begin{itemize}
  \tightlist
  \item
    大気海洋間のフラックス
  \item
    大気陸面間のフラックス
  \item
    大気へのフラックスの合計
  \end{itemize}
\item
  陸面モデルと河川モデル間のフラックス

  \begin{itemize}
  \tightlist
  \item
    河川陸面間のフラックスと河川モデル
  \item
    陸面からの水の流出
  \item
    河川から湖への水の流入
  \end{itemize}
\item
  海洋モデルへのフラックス

  \begin{itemize}
  \tightlist
  \item
    海面グリッドでの海洋の境界条件
  \item
    海面グリッドで計算した大気海洋間のフラックスの海洋グリッドへの変換
  \item
    海洋モデルでのフラックスの再配分
  \item
    河川から海洋への水の流出
  \item
    海面グリッドの分割個数と海洋モデルへの解像度
  \end{itemize}
\end{itemize}

\hypertarget{ux5927ux6c17ux30e2ux30c7ux30ebux3078ux306eux30d5ux30e9ux30c3ux30afux30b9}{%
\subsection{大気モデルへのフラックス}\label{ux5927ux6c17ux30e2ux30c7ux30ebux3078ux306eux30d5ux30e9ux30c3ux30afux30b9}}

\hypertarget{ux5927ux6c17ux6d77ux6d0bux9593ux306eux30d5ux30e9ux30c3ux30afux30b9}{%
\subsubsection{大気海洋間のフラックス}\label{ux5927ux6c17ux6d77ux6d0bux9593ux306eux30d5ux30e9ux30c3ux30afux30b9}}

海面から大気へのフラックス(\(FLXO\))は大気モデルの海面グリッドで計算する.
境界条件としての海面水温,海氷密接度,海氷厚さ,海氷上の積雪深,海氷内部温度,海洋表層流速は海洋モデルからエクスチェンジャーを通して取得する(海氷面温度は海氷内部温度と海氷の厚さおよび海氷上の大気の状態から決定される.現在のところMIROCでは海氷速度はフラックス計算には用いていない).
大気の境界条件としての海上での風速,気温,比湿等は大気のグリッドから,線形もしくは3次スプライン補完を用いて,海面のグリッドに変換する.
海面からのフラックスは海水面と海氷面でそれぞれ別々に計算され,面積の重みで平均し大気に渡される.
(海氷厚さでカテゴリー分けされた海氷モデルを使用する場合はそれぞれの海氷厚のカテゴリーでフラックスを計算する必要があるかも知れないが,現在のモデルの仕様では平均の海氷厚さでフラックスを計算している.)
エクスチェンジャーによる大気海洋間のフラックス,境界条件の変換に関しては後で詳しく記述する.

\hypertarget{ux5927ux6c17ux9678ux9762ux9593ux306eux30d5ux30e9ux30c3ux30afux30b9}{%
\subsubsection{大気陸面間のフラックス}\label{ux5927ux6c17ux9678ux9762ux9593ux306eux30d5ux30e9ux30c3ux30afux30b9}}

陸面から大気へのフラックス(\(FLXL\))は陸面グリッドで計算する。
1つの陸面グリッドは複数の土壌被覆と湖から構成される。
湖の凍結・融解および積雪は鉛直1次元の氷モデル(0-layer
model)によって考慮されている。
ここで陸面グリッドの面積を\(SL\)とすると湖とそれぞれの土壌被覆の占める面積はそれぞれ

\begin{eqnarray} SL^{lake}=SL * LKFRC * FLND \end{eqnarray}

\begin{eqnarray} SL^{grd}_k = SL * GRFRC_k * (1-LKFRC) * FLND \end{eqnarray}

となる。ここで\(LKFRC\)は陸に占める湖の割合、\(k\)は土壌被覆の種類、\(GRFRC_k\)は湖を除く陸に占める土壌被覆\(k\)の割合で示す。
陸面からのフラックスはこれらの土壌被覆、湖の上でそれぞれ別々に計算され、面積の重さで平均し、大気に渡される。

\begin{eqnarray} FLXL = LKFRC * FLXL^{lake} + (1-LKFRC) * \sum_{k=1}^{km} (GRFRC_k * FLXL_k^{grd}) \end{eqnarray}

ここで、\(FLXL^{lake}\) は湖面でのフラックス、\(FLXL_{k}^{grd}\)
は土壌被覆\(k\)でのフラックス、\(km\) は土壌被覆の種類の数。

\hypertarget{ux5927ux6c17ux3078ux306eux30d5ux30e9ux30c3ux30afux30b9ux306eux5408ux8a08}{%
\subsubsection{大気へのフラックスの合計}\label{ux5927ux6c17ux3078ux306eux30d5ux30e9ux30c3ux30afux30b9ux306eux5408ux8a08}}

河川モデルは面積を持たないので、大気へのフラックス(\(FLXA\))は、陸面グリッドのフラックス(\(FLXL\))・海面グリッドでのフラックス(\(FLXO\))の海陸分布の重み付き平均として次式のように求まる。

\begin{eqnarray} FLXA = \frac{1}{SA} * [ \sum _ {j=1}^{jldiv} \sum_{i=1}^{ildiv}(SL _ {ij} * FLND^{land} _ {ij}*FLXL_{ij}) + \sum _ {j=1}^{jodiv}\sum _ {i=1}^{iodiv}(SO _ {ij} * (1-FLND^{oc} _ {ij}) * FLXO _ {ij})] \end{eqnarray}

降水のように大気モデルで計算されたフラックスも\(FLXL\)と\(FLXO\)に含まれる。
このようなフラックスの場合、分割された陸面・海面グリッドのフラックスはすべて対応するグリッドと同じ値になる。

\hypertarget{ux9678ux9762ux30e2ux30c7ux30ebux3068ux6cb3ux5dddux30e2ux30c7ux30ebux9593ux306eux30d5ux30e9ux30c3ux30afux30b9}{%
\subsection{陸面モデルと河川モデル間のフラックス}\label{ux9678ux9762ux30e2ux30c7ux30ebux3068ux6cb3ux5dddux30e2ux30c7ux30ebux9593ux306eux30d5ux30e9ux30c3ux30afux30b9}}

\hypertarget{ux6cb3ux5dddux9678ux9762ux9593ux306eux30d5ux30e9ux30c3ux30afux30b9ux3068ux6cb3ux5dddux30e2ux30c7ux30eb}{%
\subsubsection{河川陸面間のフラックスと河川モデル}\label{ux6cb3ux5dddux9678ux9762ux9593ux306eux30d5ux30e9ux30c3ux30afux30b9ux3068ux6cb3ux5dddux30e2ux30c7ux30eb}}

現在のモデルの仕様では、河川陸面間の水のフラックスは、河川から湖への水の流入(\(RUNIN\))・湖から河川への流出(\(RUNOFF\))、内陸消失点での陸面への水の流入(\(RUNIN\))、土壌であふれた水の河川への流出(\(RUNOFF\))のみを取り扱っている。
ここで、内陸消失点とは砂漠などで河川の終点が消滅している地点を示している。
河川モデルでの水収支は氷と水に分けて取り扱っている。
河川モデルで扱う氷は疑似的な氷河に対応する。
ここで、融解熱の保存を保証するため、河川モデル内での相変化は考慮していない。
また、河川の流量は河川グリッドに存在する水量をその面積で割ったものとして定義されている。
河川モデル内で水と氷は河川流路網データに従って下流に運ばれる。
MIROC6における河川モデルでは河川の内陸消失点での河川流量は全球海洋に散布することで水収支をとっている.

\hypertarget{ux9678ux9762ux304bux3089ux306eux6c34ux306eux6d41ux51fa}{%
\subsubsection{陸面からの水の流出}\label{ux9678ux9762ux304bux3089ux306eux6c34ux306eux6d41ux51fa}}

陸面グリッドにおけるそれぞれの土壌被覆において水もしくは雪氷を保持できなくなった場合、各土壌モデルからカップラーを通して河川モデルに水もしくは氷が渡される。

\begin{eqnarray} RUNOFF^{grd}_{all} = 
    (1-LKFRC) * \sum_{k=1}^{km}(GFLRC_{k} * RUNOFF^{grd}_{k}) \end{eqnarray}

各土壌被覆からの流出量の詳細は陸面モデルMATSIROの資料を参照されたい。
湖モデルにおいては湖の水位または雪氷の厚さ(\(H\))が一定値(\(H_c\))を超えた場合、時定数
\(\tau_h\) で河川に流出する。

\begin{eqnarray} RUNOFF^{lake} = LKFRC * \frac{(H-H_c)}{\tau_h},~~~~~~ (H>H_c) \end{eqnarray}

\begin{eqnarray} RUNOFF^{lake} = 0,~~~~~~~~~~ (H<H_c) \end{eqnarray}

陸面からの平均流出量は次のようになる。

\begin{eqnarray} RUNOFF^{land}_{all} = RUNOFF^{lake} + RUNOFF^{grd}_{all} \end{eqnarray}

陸面グリッドの平均流出量で考えた場合上記の式に陸面の占める割合 \(FLND\)
を乗じる必要がある。 河川モデルでは \(RUNOFF^{land}_{all}\)
を海陸分布の重みを付けて河川グリッドに変換した流出量 \(RUNOFF^{riv}\)
を用いて計算を行う。

\hypertarget{ux6cb3ux5dddux304bux3089ux6e56ux3078ux306eux6c34ux306eux6d41ux5165}{%
\subsubsection{河川から湖への水の流入}\label{ux6cb3ux5dddux304bux3089ux6e56ux3078ux306eux6c34ux306eux6d41ux5165}}

河川流路の途中に湖が存在する場合、河川流量に応じた水が湖に流入する。
湖へ流入する水量を計算するため、カップラーを通して、河川グリッドの河川流量
\(GDRIV\) を陸面グリッドにおける河川流量 \(GDRIVL\) に変換する。
ここで、 \(GDRIVL\) は、陸面グリッドの面積で規格化した量である。
陸面グリッドにおいて、湖への河川からの流入量 \(RUNINN\) は河川流量
(\(GDRIVL\)) と時定数 \(\tau\) で次式のように定義する。

\begin{eqnarray} RUNINN^{lake}=GDRIVL/\tau \end{eqnarray}

河川から陸面への流入は内陸消失点を除き、現在の仕様では湖への流入しか考慮していないので、陸面での平均の流入は

\begin{eqnarray} RUNINN^{land}=RUNINN^{lake}*LKFRC \end{eqnarray}

となる。
陸面グリッドに対して河川グリッドが複数対応している場合、陸面グリッドで平均した陸面への河川水の流入を,
\(RUNOFF\)
と同様に面積の重みのみで河川グリッドに戻すと、河川グリッドに存在する流量以上の水が河川から流出してしまうということが起こり得る。
そこで河川流量に対する流出量の比を陸面グリッドから河川グリッドへ変換し、それぞれの河川グリッドで河川流出量(陸面への流入量)を見積る。河川流量の陸面グリッドでの流出率は

\begin{eqnarray} RINN^{land}=RUNINN^{land}/GDRIVL \end{eqnarray}

となり河川グリッドに変換した流出率を \(RINN^{riv}\)
とすると河川グリッドでの流出量(陸面への流入量)は

\begin{eqnarray} RUNINN^{riv}=RINN^{riv}*GDRIV \end{eqnarray}

\hypertarget{ux6d77ux6d0bux30e2ux30c7ux30ebux3078ux306eux30d5ux30e9ux30c3ux30afux30b9}{%
\subsection{海洋モデルへのフラックス}\label{ux6d77ux6d0bux30e2ux30c7ux30ebux3078ux306eux30d5ux30e9ux30c3ux30afux30b9}}

\hypertarget{ux6d77ux9762ux30b0ux30eaux30c3ux30c9ux3067ux306eux6d77ux6d0bux306eux5883ux754cux6761ux4ef6}{%
\subsubsection{海面グリッドでの海洋の境界条件}\label{ux6d77ux9762ux30b0ux30eaux30c3ux30c9ux3067ux306eux6d77ux6d0bux306eux5883ux754cux6761ux4ef6}}

上述の通り、大気海洋間のフラックスは海面グリッドで計算される。ここでは海洋モデルのグリッドから海面グリッドへの変換について記述する。
海洋モデルから大気の海面グリッドに変換する変数は標準では海面水温(\(SST\))、海氷密接度(\(AI\))、海氷厚さ(\(HI\))、海氷上の積雪深(\(HSN\))、海氷内部温度(\(TI\))、海洋表層流速(\(UO,VO\))である。
今後、どのグリッドでの変数を扱っているか明らかにするため、海洋モデルのグリッドでの変数には上付きで\(OGCM\)、海面グリッドでの変数には上付きで\(oc\)と表記する。また、海洋グリッドでの位置を\(LO\)、海面グリッドの位置を\(LC\)とする。
海面グリッドでの海洋の境界条件は以下のように定義される。

\begin{eqnarray} SST^{oc}(LC) = \sum_{N=1}^{IJO(LC)}[SST^{OGCM}(IJO2C(LC,N))*SOCN(LC,N)]/SOCNG(LC) \end{eqnarray}

\begin{eqnarray} AI^{oc}(LC) = \sum_{N=1}^{IJO(LC)}[AI^{OGCM}(IJO2C(LC,N))*SOCN(LC,N)]/SOCNG(LC) \end{eqnarray}

\begin{eqnarray} HI^{oc}(LC) = \frac{\sum_{N=1}^{IJO(LC)}[HI^{OGCM}(IJO2C(LC,N))*AI^{OGCM}(IJO2C(LC,N))*SOCN(LC,N)]/SOCNG(LC)} {SOCNG(LC)*AI^{oc}(LC)} \end{eqnarray}

\begin{eqnarray} HSN^{oc}(LC) = \frac{\sum_{N=1}^{IJO(LC)}[HSN^{OGCM}(IJO2C(LC,N))*AI^{OGCM}(IJO2C(LC,N))*SOCN(LC,N)]/SOCNG(LC)} {SOCNG(LC)*AI^{oc}(LC)} \end{eqnarray}

\begin{eqnarray} TI^{oc}(LC) = \frac{\sum_{N=1}^{IJO(LC)}[TI^{OGCM}(IJO2C(LC,N))*HI^{OGCM}(IJO2C(LC,N))*AI^{OGCM}(IJO2C(LC,N))*SOCN(LC,N)]/SOCNG(LC)} {SOCNG(LC)*HI^{oc}(LC)*AI^{oc}(LC)} \end{eqnarray}

\begin{eqnarray} UO^{oc}(LC)=RUO(LC)* \frac{\sum_{N=1}^{IJO(LC)}[UO^{OGCM}(IJO2C(LC,N))*SOCN(LC,N)]}{SOCNG(LC)}+RVO(LC)* \frac{\sum_{N=1}^{IJO(LC)}[VO^{OGCM}(IJO2C(LC,N))*SOCN(LC,N)]}{SOCNG(LC)} \end{eqnarray}

\begin{eqnarray} VO^{oc}(LC)=-RVO(LC)* \frac{\sum_{N=1}^{IJO(LC)}[UO^{OGCM}(IJO2C(LC,N))*SOCN(LC,N)]}{SOCNG(LC)}+RUO(LC)* \frac{\sum_{N=1}^{IJO(LC)}[VO^{OGCM}(IJO2C(LC,N))*SOCN(LC,N)]}{SOCNG(LC)} \end{eqnarray}

\begin{eqnarray} SOCNG(LC)= \sum_{N=1}^{IJO(LC)}SOCN(LC,N) \end{eqnarray}

ここで、
\(IJO(LC)\):大気ノード内の海面グリッド(\(LC\))に対応する海洋グリッドの数

\(IJO2C(LC,N)\):大気ノード内の海面グリッドに対応する海洋グリッドの位置

\(SOCN(LC,N)\):大気ノード内の海面グリッドに対応する海洋グリッドの面積

\(RUO(LC)\):ベクトルの回転角の余弦

\(RVO(LC)\):ベクトルの回転角の正弦

\(SOCNG(LC)\):海面グリッドに海洋が占める面積

である。 海面グリッドに占める陸面の割合もここで定義され

\(FLND^{oc}=(1-SOCNG)/SO\)

となる。 \(SO\)は海面グリッドの面積。

また、海面グリッドに変換する海氷に関する変数は海氷層厚でカテゴリー分けした変数(\(AIM,HIM,HSM,TIM\))の平均値として以下のように計算されている。

\begin{eqnarray} AI^{OGCM} = \sum_{L=1}^{NIC} AIM^{OGCM}(L) \end{eqnarray}

\begin{eqnarray} HI^{OGCM} = \sum_{L=1}^{NIC} HIM^{OGCM}(L)*AIM^{OGCM}(L)/AI^{OGCM} \end{eqnarray}

\begin{eqnarray} HSN^{OGCM} = \sum_{L=1}^{NIC} HSM^{OGCM}(L)*AIM^{OGCM}(L)/AI^{OGCM} \end{eqnarray}

\begin{eqnarray} TI^{OGCM} = \sum_{L=1}^{NIC} TIM^{OGCM}(L)*AIM^{OGCM}(L)/(AI^{OGCM}*HI^{OGCM}) \end{eqnarray}

ここで \(NIC\) は海氷カテゴリーの数。

\hypertarget{ux6d77ux9762ux30b0ux30eaux30c3ux30c9ux3067ux8a08ux7b97ux3057ux305fux5927ux6c17ux6d77ux6d0bux9593ux306eux30d5ux30e9ux30c3ux30afux30b9ux306eux6d77ux6d0bux30b0ux30eaux30c3ux30c9ux3078ux306eux5909ux63db}{%
\subsubsection{海面グリッドで計算した大気海洋間のフラックスの海洋グリッドへの変換}\label{ux6d77ux9762ux30b0ux30eaux30c3ux30c9ux3067ux8a08ux7b97ux3057ux305fux5927ux6c17ux6d77ux6d0bux9593ux306eux30d5ux30e9ux30c3ux30afux30b9ux306eux6d77ux6d0bux30b0ux30eaux30c3ux30c9ux3078ux306eux5909ux63db}}

海面グリッドで計算されるフラックスは海水面と海氷面でそれぞれ計算され、大気へのフラックスは

\begin{eqnarray} FLXO=(1-AI)*FLUXO+AI*FLUXI \end{eqnarray}

となる。
これらのフラックスは大気モデル内のフラックスカップラーで海水面、海氷面積の重みを付けて時間積算した後で、大気海洋の結合時間ステップで海洋グリッドに変換し、海洋モデルに渡される。

\begin{eqnarray} FLUXOA^{OGCM}(LO) = ROCN(LO)*\sum_{N=1}^{IJA(LO)} [FLUXOA^{oc}(IJC2O(LO,N))*SATM(LO,N)]/SATMG(LO) \end{eqnarray}

\begin{eqnarray} FLUXIA^{OGCM}(LO) = ROCN(LO)*\sum_{N=1}^{IJA(LO)} [FLUXIA^{oc}(IJC2O(LO,N))*SATM(LO,N)]/SATMG(LO) \end{eqnarray}

\begin{eqnarray} SATMG(LO)=ROCN(LO)*\sum_{N=1}^{IJA(LO)} SATM(LO,N) \end{eqnarray}

\begin{eqnarray} ROCN(LO)=SATMG(LO)/S^{OGCM}(LO) \end{eqnarray}

\begin{eqnarray} FLUXOA^{oc}=(1-AI^{oc})*FLUXO^{oc} \end{eqnarray}

\begin{eqnarray} FLUXIA^{oc}=AI^{oc}*FLUXI^{oc} \end{eqnarray}

ここで

\(IJA(LO)\):海洋グリッド(\(LO\))に対応する大気モデルの海面グリッドの数

\(IJC2O(LO,N)\):海洋グリッドに対応する大気モデルの海面グリッドの位置

\(SATM(LO,N)\):海洋グリッドに対応する大気モデルの海面グリッドの面積
\(SATM(LO,N)=SOCN(LC,L),LC=IJC2O(LO,N),LO=IJO2C(LC,L)\)

\(S^{OGCM}(LO)\):海洋グリッドの面積

である。
海洋グリッドの面積と対応する海面グリッドの面積の総和は一致するはずであるが、大気モデルと海洋モデルの座標系の違い(大気モデルと海洋モデルの地球の表面積が厳密に一致しない)。
変換ファイル作成時に、大気モデルのグリッドを微小面積に分割し、その積算から対応する海洋グリッドの面積を見積っていることから厳密には一致しない。
このため、その比(ROCN)を乗じることによって、大気海洋間のフラックスの収支を合わせている。
海洋への風応力も同様に海水面上えの風応力(TXO,TYO),海氷面上での風応力(TXI,TYI)としてそれぞれ計算するが、積算時に海水面、海氷面積の重みを乗じない。

\begin{eqnarray} TXO^{OGCM}(LO)=RU(LO)*ROCN(LO)*\sum_{N=1}^{IJA(LO)} \frac{[TXO^{oc}(IJC2O(LO,N))*SATM(LO,N)]}{SATMG(LO)} + RV(LO)*ROCN(LO)*\sum_{N=1}^{IJA(LO)}\frac{[TYO^{oc}(IJC2O(LO,N))*SATM(LO,N)]}{SATMG(LO)} \end{eqnarray}

\begin{eqnarray} TYO^{OGCM}(LO)=-RV(LO)*ROCN(LO)*\sum_{N=1}^{IJA(LO)} \frac{[TXO^{oc}(IJC2O(LO,N))*SATM(LO,N)]}{SATMG(LO)} + RU(LO)*ROCN(LO)*\sum_{N=1}^{IJA(LO)}\frac{[TYO^{oc}(IJC2O(LO,N))*SATM(LO,N)]}{SATMG(LO)} \end{eqnarray}

ここで

\(RU(LO)\):ベクトルの回転角の余弦

\(RV(LO)\):ベクトルの回転角の正弦

である。

\hypertarget{ux6d77ux6d0bux30e2ux30c7ux30ebux3067ux306eux30d5ux30e9ux30c3ux30afux30b9ux306eux518dux914dux5206}{%
\subsubsection{海洋モデルでのフラックスの再配分}\label{ux6d77ux6d0bux30e2ux30c7ux30ebux3067ux306eux30d5ux30e9ux30c3ux30afux30b9ux306eux518dux914dux5206}}

海洋のグリッドに変換されたフラックスは結合の時間ステップごとに更新される。
結合の時間ステップは海洋モデルの時間ステップより長いため、海洋モデルの海水面・海氷面積の比はフラックスの計算に用いた値とは異なる値に更新されている。
このため、正確に熱・水収支をとるためには更新された海水面・海氷面積比に応じたフラックスの分配が必要になる。
海水面、海氷面でのフラックス
\(FLUXOA\),\(FLUXIA\)は海洋グリッド平均値である。
各海氷カテゴリーへのフラックスは現在のところ海氷の厚さに依存せず均等に分配している。

\begin{eqnarray} FLUXIAM(L)=FLUXIA*AIM(L)\end{eqnarray}

\begin{eqnarray} FLUXOA=FLUXOA+FLUXIA*[1.0-\sum_{L=1}^{LMAX}AIM(L)]\end{eqnarray}

\(AIM\) はグリッドに海氷の占める面積の割合(海氷密接度)、\(L\)
は海氷厚さのカテゴリーの種類、\(LMAX\) は厚さカテゴリーの数を示す。
海氷面が存在しなければ全て海水面にフラックスが入る。
海氷が結合の時間ステップの途中に消滅した場合、昇華によるフラックスは、海氷面を仮定した熱フラックスと淡水フラックス(海氷の減少量)で分けて考える。
熱フラックスはそのまま海洋1層目の温度変化に反映させる。
一方、昇華による淡水フラックスは昇華分の海氷を生成したと仮定し、熱フラックス、淡水フラックスに換算し海洋1層目に与える。
風応力に関しては、グリッド変換前に海水面・海氷面積の重みを付けていないので、海洋モデル内の各海氷厚さカテゴリーでそれぞれの面積の重みを付けて駆動することになる。
このため、運動量は保存していない。

\hypertarget{ux6cb3ux5dddux304bux3089ux6d77ux6d0bux3078ux306eux6c34ux306eux6d41ux51fa}{%
\subsubsection{河川から海洋への水の流出}\label{ux6cb3ux5dddux304bux3089ux6d77ux6d0bux3078ux306eux6c34ux306eux6d41ux51fa}}

河川モデルの最後で加工から海に流れる水を計算する。
河川グリッドの河口にきた水をまず、大気の海面グリッドに変換されフラックスカップラーで時間積算される。
その後、大気の降水データと同様にエクスチェンジャー経由で海洋グリッドに変換し海洋モデルに渡される。
このとき河川水の温度は降水などと同様に海面水温と同じとして取り扱う。
このため、厳密には熱は保存しない。
氷の流出に関しては降雪と同様に処理する。

\hypertarget{ux6d77ux9762ux30b0ux30eaux30c3ux30c9ux306eux5206ux5272ux500bux6570ux3068ux6d77ux6d0bux30e2ux30c7ux30ebux306eux89e3ux50cfux5ea6}{%
\subsubsection{海面グリッドの分割個数と海洋モデルの解像度}\label{ux6d77ux9762ux30b0ux30eaux30c3ux30c9ux306eux5206ux5272ux500bux6570ux3068ux6d77ux6d0bux30e2ux30c7ux30ebux306eux89e3ux50cfux5ea6}}

海面グリッドは大気のグリッドの緯度経度を分割して作成しているが、分割個数が十分ではなく、海洋モデルのグリッドが大気の海面グリッドと比べて高解像度の場合、エクスチェンジャーを通してフラックスを海洋のグリッドに変換した際に大気のグリッドサイズの構造が残る場合がある。
また、大気からの降水などのデータは大気グリッドから海面グリッドに変換する際に補間を行わないため、これらのフラックスについては海洋グリッドで大気グリッドサイズの構造が残る。
また、海面グリッドに変換する際に、3次スプライン補間はなく線形補間を用いた場合は風応力カールのような微分量で大気グリッドサイズの構造が残る場合がある。

\hypertarget{reference}{%
\section{Reference}\label{reference}}

Suzuki, T., Saito, F., Nishimura, T., and Ogochi, K., 2009: Coupling
procedures of heat and freshwater fluxes in the MIROC (Model for
Interdiciplinary Research on Climate) version 4. JAMSTEC Report of
Research and Development, 9, 1-9 (in Japanese)
